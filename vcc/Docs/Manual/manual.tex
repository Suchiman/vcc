 \ifdense
\documentclass[preprint,nocopyrightspace]{sigplanconf}
\else
\documentclass{article}
\usepackage{fancyhdr}
%\pagestyle{fancy}
\fi
\bibliographystyle{plain}

% on-screen reading version
\ifwide
\usepackage[margin=3mm, paperwidth=7in, paperheight=9in]{geometry}
%\usepackage[margin=3mm, paperwidth=5in, paperheight=4in]{geometry}
\else
\ifdense
\else
\usepackage[margin=3mm, paperwidth=8.5in, paperheight=11in]{geometry}
\fi
\fi

\usepackage{amsmath}
\usepackage{stmaryrd}
\usepackage{color}
\usepackage{graphicx}
%\usepackage{marginnote}
\usepackage{hyperref}
\usepackage{amsmath, amssymb}
\usepackage{url, listings}
%\usepackage{msrtr}
\usepackage{vcc}
\usepackage{boogie}
\usepackage{xspace}
\usepackage{paralist}
% default to roman numerals:
\let\oldinpaxraenum=\inparaenum
\def\inparaenum{\oldinparaenum[(i)]}
\definecolor{bgcode}{rgb}{0.92,0.92,0.92}

\definecolor{todocolor}{rgb}{0.9,0.0,0.0}
\definecolor{noteframe}{rgb}{0.5,0.5,0.5}
%\ifdense
\newcommand{\todo}[1]{[\textcolor{red}{\textbf{TODO:} {#1}}]}
%\else
%\newcommand{\todo}[1]{\marginnote{\scriptsize {[\textcolor{red}{\textbf{TODO:} {#1}}]}}}
%\fi
\newcommand{\itodo}[1]{{[\textcolor{red}{\textbf{TODO:} {#1}}]}}
%\renewcommand{\todo}[1]{}
%\renewcommand{\itodo}[1]{}
\newcommand{\lines}[1]{\begin{array}{l}#1\end{array}}
\newcommand{\linesi}[1]{\;\;\;\begin{array}{l}#1\end{array}}
\newcommand{\tuple}[1]{{\langle}{#1}\rangle}
\newcommand{\subsubsubsection}[1]{\paragraph{#1}}
\ifdense
\usepackage[T1]{fontenc}
\usepackage{epigrafica}
%\usepackage{iwona}
%\usepackage[scaled]{berasans}
%\usepackage{cmbright}
%\usepackage{lxfonts}
%\usepackage{kurier}
\usepackage{times}
\else
\usepackage[OT1]{fontenc}
\usepackage{courier}
%\usepackage[T1]{fontenc}
%\usepackage[scaled=0.81]{luximono}
\fi

\newcommand{\secref}[1]{\hyperref[sect:#1]{\textsection~\ref{sect:#1}}}
\newcommand{\figref}[1]{Figure~\ref{fig:#1}}
\newcommand{\Secref}[1]{Section~\ref{sect:#1}}
\newcommand{\Figref}[1]{Figure~\ref{fig:#1}}

\newcommand{\lemmaref}[1]{Lemma~\ref{lemma:#1}}
\newcommand{\thmref}[1]{Theorem~\ref{thm:#1}}
\newcommand{\lineref}[1]{line~\ref{line:#1}}

\newcommand{\ie}[0]{i.e.,{ }}
\newcommand{\eg}[0]{e.g.,{ }}
\newcommand{\cf}[0]{cf.{ }}


\ifdense
\newenvironment{note}{%
\vspace{-3mm}
\begin{list}{}%
    {\setlength{\leftmargin}{0.03\textwidth}}%
    \item[]%
  \textcolor{noteframe}{\rule{0.444\textwidth}{1pt}} \\
  %\small
  %\sffamily
  \selectfont
  %\textbf{Note:}
}{%
  \vspace{-2mm} \\ 
  \textcolor{noteframe}{\rule{0.444\textwidth}{1pt}} 
  \end{list}
\vspace{-2mm}
}
\else
\newenvironment{note}{%
\vspace{-3mm}
\begin{list}{}%
    {\setlength{\leftmargin}{0.1\textwidth}}%
    \item[]%
  \textcolor{noteframe}{\rule{0.9\textwidth}{1pt}} \\
  \small
  %\textbf{Note:}
}{%
  \vspace{-2mm} \\ 
  \textcolor{noteframe}{\rule{0.9\textwidth}{1pt}} 
  \end{list}
\vspace{-2mm}
}
\fi
\newcommand{\notehd}[1]{\textbf{#1} \\}

\newcommand{\eqspc}[1]{\;\;{#1}\;\;}
\setlength{\arraycolsep}{0mm}

%%% the model

%%% general
\newcommand{\MathOp}[2]{{}\mathbin{\hbox{$\mkern#2mu#1\mkern#2mu$}}{}}
\newcommand{\Iff}{\MathOp{\Leftrightarrow}{6}}
\newcommand{\Equal}{\MathOp{=}{6}}
\newcommand{\Xor}{\MathOp{\not\equiv}{6}}
\newcommand{\Implies}{\MathOp{\Rightarrow}{4}}
\renewcommand{\And}{\MathOp{\wedge}{2}}
\newcommand{\Or}{\MathOp{\vee}{2}}
\newcommand{\Neg}{\neg}
\newcommand{\dor}[0]{\MathOp{|}{2}}

\newcommand{\ON}[1]{\operatorname{#1}}
\newcommand{\Version}[0]{0.2}

% -----------------------------------------------------

\newcommand{\Def}[1]{\textit{\textbf{#1}}}
\newcommand{\entry}[2]{\noindent #1 #2}

\begin{document}

\title{The VCC Manual\\
\vspace{2mm}
\Large  Working draft, version \Version, \today}
%\title{Developing VCC Specifications \\ for Concurrent C Programs}

\ifdense
\authorinfo{}
%\authorinfo{Ernie Cohen, Mark A. Hillebrand, \\ Stephan Tobies}{European Microsoft Innovation Center}{\{ecohen,mahilleb,stobies\}@microsoft.com}
%\authorinfo{Micha{\l} Moskal, Wolfram Schulte}{Microsoft Research Redmond}{\{micmo,schulte\}@microsoft.com}
\preprintfooter{VCC User Manual (working draft, ver. \Version)}
\else
\author{
%Micha{\l} Moskal, Wolfram Schulte \\
%\normalsize Microsoft Research Redmond \\
%\and Ernie Cohen, Mark A. Hillebrand, Stephan Tobies \\
%\normalsize European Microsoft Innovation Center
}
\date{}
\pagestyle{fancy}
\fancyhf{}
\renewcommand{\headrulewidth}{0pt}
\renewcommand{\footrulewidth}{0.3pt}
\fancyfoot[LO,RE]{\footnotesize VCC Tutorial (working draft, ver. \Version, \today)}
\fancyfoot[RO,LE]{\thepage}
\fi


%\msrtrno{MSR-TR-2010-9}
%{\def\@titletext{foo}}
%\msrtrmaketitle

%\pagebreak
%\begin{figure*}
%\vspace{3in}
%\begin{center}
%This page intentionally left blank.

%\end{center}
%\end{figure*}

%\setcounter{page}{0}

\maketitle


\begin{abstract}
This manual presents the annotation language of VCC,
a deductive verifier for concurrent C code. 
\end{abstract}

%\tableofcontents
\ifdense 
\lstset{
  basicstyle=\small\sffamily,
  columns=fullflexible,
}
\else
\lstset{
  basicstyle=\small\ttfamily,
}
\fi

\definecolor{kwColor}{rgb}{0.2,0.2,0.8}

\lstset{
  keywordstyle=\bfseries, %\textcolor{kwColor},
  breaklines=true,
  breakatwhitespace=true,
  numberstyle=\tiny\sf,
  escapeinside={/*-}{*/},
  numbers=none,
  emptylines=1,
  rangeprefix=\/\*\{,
  rangesuffix=\}\*\/,
  includerangemarker=false,  
%  aboveskip=2mm,
%  belowskip=2mm,
%  xleftmargin=2mm,
%  xrightmargin=2mm,
}

\ifdense
\renewcommand{\labelitemi}{{\footnotesize \centeroncapheight{$\bullet$}}}
\fi

\section{Introduction}
VCC is a deductive verifier for concurrent C code. VCC takes C code,
annotated with function contracts, loop invariants, data invariants,
ghost data, and ghost code, and tries to verify that the code respects
these annotations. This manual describes these annotations in detail,
and what constitutes correct verification behavior of VCC. It does not
describe in detail the actual implementation of VCC.

\subsection{Terminology}
\Def{VCC} means either the VCC tool or the language in
which the annotations are written (which is an extension of C),
depending on context.
Text in \vcc{this typeface} is written in this language, with the
exception of text enclosed within angle brackets which is used to
give an English description of a command or value or values within
such a text, e.g. \vcc{x == <the absolute value of y>}. Mathematical
formulas are written as VCC expressions, and models of portions of the
state are written as VCC types (with occasional liberties as noted in
the text). 

In this document, we consider the meaning of a fixed, annotated
program, which we refer to as \Def{the program}. \Def{The unannotated
program} means the annotated program stripped of all VCC annotations
(i.e., the program as seen by the C compiler).

Due to an unfortunate historical legacy, the VCC type \vcc{\object}
corresponds to what are in this document called pointers, whereas the
word ``object'' means a particular subset of these pointers.

\section{Overview}

The program operates on a fixed set of typed \Def{objects}; the
identity of an object is given by its \Def{address} and
its \Def{type}.  (For example, for each user-defined \vcc{struct} type
there is an object of that type for each address properly aligned for
objects of that type.)  Each object has a collection of \Def{fields}
determined by its type.  A \Def{state} of the program is a function
from objects and field names to values. One field of each object is
a \vcc{Boolean} indicating whether the object is \Def{valid}; the
valid objects represent those objects that actually exist in a given
state, so object creation/destruction corresponds to object
validation/invalidation.

A \Def{transition} is an ordered pair of states, a \Def{prestate} and
a \Def{poststate}.  Each object has a two-state \Def{invariant} (a
predicate on transitions); these invariants can mention arbitrary
parts of the state, and are mostly generated directly from the
annotations on the program. The invariant of a type/object is the
conjunction of all invariants mandated in this manual.  An invariant
can be interpreted as an invariant on a single state by applying it to
the transition from the state to itself (stuttering transition). A state
is \Def{good} if all object invariants hold in that state; a
transition is good if it satisfies the invariant of each object. A
sequence of states (finite or infinite) is good iff all of its states
are good and the transitions between successive states of the sequence
are good.

A transition \Def{updates} an object if some field of the object differs
between the prestate and poststate of the transition. A transition
is \Def{legal} if its prestate is not good or if it satisfies the
invariants of all updated objects. A sequence of states is an
\Def{legal execution} iff the initial state
is good and the transition from each nonterminal state of the sequence
to its successor is legal. The program is good iff every legal execution of
the program is good; successful verification of the program shows that
it is good.

A type is \Def{admissible} iff, for every object \vcc{o} of the type,
(1) every legal transition with a good prestate satisfies the
invariant of \vcc{o}, and (2) the poststate of every good transition
from a good state satisfies the invariant of \vcc{o}. It is easy to
prove by induction on legal executions that if all object types are
admissible, the program is good. The program is verified by proving
that every type of the program is admissible. (There is a type
corresponding to each function; admissibility if this type shows that 
changing the state of the system according to the operational
semantics of the function is legal.)

Each valid object has a Boolean ghost field that says whether it
is \Def{closed}, and a ghost field that gives its \Def{owner} (which
is also an object).  User-defined object invariants are guaranteed to
hold only when the object is closed. Threads are also modelled as
objects; in the context of a thread, an object owned by the thread is
said to be \Def{wrapped} if it is closed, and \Def{mutable} if it is
open (not closed). Only threads can own open objects.  The owner of an
open object has ``exclusive'' use of the object, and so can operate on
it sequentially. An implicit invariant is that nonvolatile fields of
objects don't change while the object is closed, so such fields can
also be read sequentially if the object is known to be closed.
Fields of closed objects can be updated only if they
are marked as \Def{volatile}; such fields can be accessed only within
explicitly marked \Def{atomic action}s.

Each object field is either \Def{concrete} or \Def{ghost}; concrete
fields correspond to data present in a running program.  Each concrete
field of each object has a fixed address and size.  Each object has an
invariant saying that concrete fields of valid objects don't
overlap, and verification of a function is that it accesses only
fields of valid objects; thus, every legal execution of good
program is simulated by a legal execution in which concrete field
accesses are replaced by memory accesses to shared C heap.  The
program can include ghost code not included in the unannotated
program; however, verification guarantees that all such code
terminates, and its execution does not change the concrete state.
This implies that every legal execution of the unannotated program is
the projection of a legal execution of the program, which allows
properties of the program to be projected to properties of the
unannotated program.

Each thread has a field that gives its ``local copy'' of the global state.
When a thread updates an object, it also updates the local copy;
when it reads an object, it reads from the local copy. 
Just before each atomic update, the thread updates its local copy to
the global state. It is an invariant of each thread that for any
object that it owns, the local copy agrees with the actual object on
all fields if the object is open and on all nonvolatile fields if the
object is closed. However, assertions that mention objects not
directly readable by the thread (without using an atomic action) might
reference fields where the local and global copies disagree; thus,
assertions that appear in the annotation of a thread are guaranteed to
correspond to global assertions only at the beginning of an atomic
action. This machinery allows us to ``pretend'' that threads are
interrupted by other threads only just before entering explicit atomic
actions. 

\section{Preprocessing}

The program must \vcc{#include <vcc.h>}; this inclusion must
precede any annotations (after expansion of \vcc{#include}
directives). 
\\\\
VCC reserves (in both the program and the unannotated program) the
preprocessor macro names \vcc{VERIFY}\footnote{ The explicit use
of \vcc{#ifdef VERIFY} is deprecated.  }, \vcc{_VCC_H}, and the
macro \vcc{_(...)}\footnote{ Caveat: VCC currently makes use of a
number of other preprocessor macro names, all of which are defined
in \vcc{<vccp.h>} in the VCC inclusion directory. This should be
fixed.  }.  All VCC annotations occur within this
macro. If \vcc{VERIFY} is not set, \vcc{_()} is defined as whitespace,
resulting in the unannotated program. VCC assumes that the unannotated
program is a legal C program, i.e. one that a conformant C compiler
would accept.
\\\\
VCC uses the C preprocessor, so comments and macro expansion follow
the rules of C. Preprocessor commands are permitted inside of annotations.

\section{Syntax}

VCC reserves identifiers starting with \vcc{\}, using them as function
names, field names, or operators. 
\\\\
VCC adds the following infix operators. Each has the given associativity.
The specified Unicode characters can also be used.

\begin{tabbing}
operator \= unicode \= associativity \= description\\
\\
\vcc{==>}    \>\>right\> implies \\
\vcc{<==}    \>\>left \> explies \\
\vcc{<==>}   \>\>left \> iff (if and only if) \\
\vcc{\in}    \>$\in$\>left \> set membership\\
\vcc{\is}    \>\>left \> type test\\
\vcc{\inter} \>$\cap$\>left \> set intersection\\
\vcc{\union} \>$\cup$\>left \> set union\\
\vcc{\diff}  \>\>left \> set difference\\
\vcc{\subset}\>$\subseteq$\>left \> subset\\
\end{tabbing}

The precedence of operators, in relation to standard C operators in show below:

\begin{VCC}
/ * %
+ -
<< >>
<= => < > \is \inter
\diff
\union
\in
== !=
&
^
|
&&
||
<==
==>
<==>
?:
= += *= etc.
\end{VCC}

In addition, there is an additional syntactic productions for
expressions:
\begin{VCC}
  <expression> ::= <quantifier> <variable declarator> ; <triggers> <expression>
  <quantifier> ::= \forall | \exists | \lambda
  <triggers>   ::= <empty> | <trigger> <triggers>
  <trigger>    ::= {<terms>}
\end{VCC}
The scope of the variables declared in the declarators in the first production 
extends to the end of the expression. The unicode characters $\lambda$, $\forall$, and
$\exists$ can be used as well, in place of \vcc{\lambda}, \vcc{\forall}, and \vcc{\exists}
respectively.

The syntax of types is extended with the type constructions defined in
section \ref{sec:types}.

\subsection{Annotations}

VCC annotations are of the form \vcc{_(tag stuff)}, where \vcc{tag} is
a \Def{annotation tag} and \vcc{stuff} has balanced parentheses (after
preprocessing). Each VCC annotation is described here with an entry of the form:
\noindent\vcc{_(tag args)} (class)\\
where args is a sequence of 0 or more parameter declarations
(including types) and ``class'' describes the syntactic role of the
annotation, which is one of the following:
\begin{description}
\item[statement] an annotation that acts syntactically as a statement;
\item[cast] an annotation that acts syntactically as a (type) cast in
expressions (in concrete or ghost code);
\item[contract] an annotation that appears either between a function
declaration and the following body or semicolon, or between a loop
construct and the body of the loop;
of a function or a block (in concrete or ghost code);
\item[specifier] an annotation that appears in the syntactic role of a
function specifier (just before a function declaration or definition);
\item[compound] an annotation that can appear only in the annotation
of a compound type definition, just before the \vcc{struct} or \vcc{union}
keyword;
\item[special] an annotation whose role is described in the text.
\end{description}
Several VCC annotations allow a variable number of arguments of the
same type. In such cases, we write the argument as a single typed
argument followed by an ellipsis; the actual parameters are to be 
comma separated iff the ellipsis is preceded by a comma.
\\\\
In addition to annotations, VCC also provides a number of additional
types and functions that can be used inside of annotations. These
types and functions all have names starting with \vcc{\}. Their syntax
is given using conventional C declarations. 
\\\\
Certain annotations create a \Def{pure context} where state changes
are not permitted, certain constructs are not allowed, and type
promotions are somewhat different.  Such contexts are
marked \vcc{:pure} in this manual; this keyword does not apepar in
annotations.
\\\\
Certain annotations, constants, and functions can appear only in
inside of certain contexts: 
\begin{description}
\item[pure] can appear only in a pure context;
\item[thread] can appear only within the body of a function
declaration;
\item[member] can only appear in the syntactic position of a member
declaration of a compound data type;
\item[invariant] can only appear inside an \vcc{_(invariant)} annotation
inside the declaration of a compound type definition.
\end{description}

\section{Types}
\label{sec:types}
A type is said to be concrete if it is a type of the unannotated program,
and is otherwise said to be ghost.
Each type is classified by the number of possible values of that type, which is
either finite, small (countable), or large (equinumerous with the
reals). Concrete types are all finite; for each kind of ghost type 
defined below, we give the class of that type. 
The class of a struct or union type is
the class of its largest member (including ghost fields), so while a C
compound type without ghost fields is finite, one with ghost fields
might be small or large. The class of an array type is the class of
its base type.
\\\\
All types can be classified as either \Def{value} types
or \Def{object} types. Object types are compound types (i.e., \vcc{struct}
or \vcc{union} types), array object types, \vcc{\thread}s, 
\vcc{\claim}s, and \vcc{\blob}s. All other types are value
types. The objects of a program are of object types, while fields of
objects are of value types.

\subsection{Integral Types}

\vcc{\bool} (finite)\\
The Boolean types, with values \vcc{\true} and \vcc{\false}. These
values are equal to the \vcc{\natural} numbers \vcc{0} and \vcc{1}, respectively.
\\\\
\vcc{\natural} (small)\\
The type of (unbounded) natural numbers. 
\\\\
\vcc{\integer} (small)\\
Mathematical (unbounded) integers. 
\\\\
C integral types can be cast to \vcc{\integer},
or unsigned integral types cast to \vcc{\natural}, in the obvious
way. When a signed value is cast to \vcc{\natural} in an impure
context, there is an implicit assertion that the value cast is
nonnegative. In a pure context, the casting of a negative value to 
a \vcc{\natural} is equivalent to applying some unknown (but fixed)
function from \vcc{integer} to \vcc{natural}. In any context, casting
any integral type to \vcc{\bool} yields \vcc{\true} if the argument
is equal to \vcc{0} and \vcc{\false} otherwise.
\\\\
The C arithmetic operators are extended to the types \vcc{\natural}
and \vcc{\integer} with the obvious interpretation\footnote{
Caveat: Because of compiler limitations, VCC currently does not allow
\vcc{<<} or \vcc{>>} operators with the second argument a literal over \vcc{63}.
}. 

An arithmetic operator instance is said to occur in
an \vcc{\unchecked} context if it is is a subterm of an expression
marked \vcc{_(unchecked)}; otherwise, it is said to be checked. 
On any checked arithmetic operation, VCC asserts that the operation
does not overflow or underflow\footnote{
  VCC should not allow unchecked signed overflows or underflows (at
  least in concrete code),
  because the C standard specifies these as resulting in undefined
  behavior; however currently VCC assumes that these produce some
  arbitrary but fixed function of the inputs.
}. On any checked division, VCC asserts
that the divisor is nonzero. On any checked cast between arithmetic
types, VCC asserts that the cast value fits in the range of the target
arithmetic type\footnote{
VCC currently does not support floating point arithmetic, in that it
does not know anything about the semantics of various floating point
operations, and does not check for floating point overflow, underflow,
or division by zero. However, on platforms where floating point
exceptions do not affect control flow, verification remains sound for
programs that uses \vcc{float}s or \vcc{double}s.
}.
The
result of an expression of arithmetic type is implicitly cast
to \vcc{\integer} if (1) the context is pure and checked, or (2) it is
subject to a binary arithmetic operator and the other operand
is \vcc{\natural} or \vcc{\integer}.

\subsection{Record Types}
\vcc{_(record)}\\
A record is like a C struct or union, except that it is treated as a
single abstract value.  Record types are declared using the same syntax as
the declaration of a \vcc{struct} type, but with the following
differences:
\begin{itemize}
\item The keyword \vcc{struct} is immediately followed by
 \vcc{_(record)}.
\item The fields of a record type cannot be marked as volatile.
\item Record types have no invariants.
\item Fields of a record cannot be of struct types or array types (but
can be of map type or record type). 
\item If a field of a record is of union type, it must have
a \vcc{_(backing_member)}.
\end{itemize}

If \vcc{T} is a record type, the expression \vcc{(struct T)} produces an arbitrary
value of type \vcc{T}.
\\\\
If \vcc{e} is an expression of record type with
fields \vcc{f1,f2,...}, and \vcc{e1,e2,...} are expressions that are
assignment compatible with the types of fields \vcc{f1,f2,...}
respectively, then
\begin{VCC}
  e / { .f1 = e1, .f2 = e2, ...}
\end{VCC}
is equal to the record \vcc{e} with the fields \vcc{f1,f2,...} set
to the values \vcc{e1,e2,...} respectively.
\\\\
If \vcc{v} is a variable of a record type with a field \vcc{f}, then
\vcc{v.f = e} translates to \vcc@v = v / {.f = e}@; if \vcc{v.f} is
itself a record with field \vcc{g}, then \vcc{v.f.g = e} translates to 
\vcc@v = v / {.f = v.f / {.g = e}}@, and so on.
\\\\
\vcc@_(record T { ... })@ (macro)\\
Expands to \vcc@_(ghost typedef struct _(record) T { ... } T)@

\subsection{Map Types}
\vcc{T1[T2]}\\
The type of maps (i.e., mathematical functions) from type \vcc{T2} to type \vcc{T1}. 
\vcc{T1} and \vcc{T2} must be value types, and \vcc{T2} must
not be a large type. If \vcc{T2} is finite, this has the same class
as \vcc{T1}. If \vcc{T2} is small, the resulting type is large.
\\\\
Map types can also appear in
declarations and typedefs, and there the syntax matches the syntax of
C arrays, except that a type appears in between the
array brackets instead of a size.
\\\\
\vcc{\lambda T v; :pure e} (expression)\\
Here, the scope of \vcc{v} continues to the end of \vcc{e}; it
introduces \vcc{v} as a variable of type \vcc{T}. 
If \vcc{T1} is the
type of \vcc{e}, this expression has type \vcc{T1[T]}; the resulting
value maps each \vcc{v} to the corresponding value of \vcc{e}, as
evaluated in the current state.
\\\\
If \vcc{e} is a map of
type \vcc{T1[T2]} and \vcc{x} is a value of type \vcc{T2},
then \vcc{e[x]} is an expression of type \vcc{T1}, whose value is
given by the value of \vcc{e} at the point \vcc{x}. If \vcc{v} is a
lvalue of type \vcc{T1[T2]} and \vcc{e1} is an expression of
type \vcc{T2}, then the expression \vcc{v[e] =  e1} is equivalent to
the assignment \vcc{v = (\lambda T2 u; (u == e) ? e1 : v[u])}
where \vcc{u} is a fresh variable. This is extended to maps of maps in
the obvious way.

\subsection{Inductive Types}
An inductive datatype is introduced by a top-level annotation
\begin{VCC}
  _(datatype T {
    case c1(<args1>);
    case c2(<args2>);
    ...
  })
\end{VCC}
where \vcc{<argsi>} is a comma-separated list, each element of which is
a type name optionally followed by a parameter name. This
declares each \vcc{ctor} to be a constructor of values of
type \vcc{T}. Types can be mutually recursive, but a type cannot be referenced
before being declared. The form \vcc{_(type T)} declares \vcc{T} as an
abstract type, which can (but doesn't have to) be later redefined as
a record or datatype.
\\\\
The \vcc{==}, \vcc{!=}, and \vcc{=} operators are extended to \vcc{T}, but
it is an error to apply them to compare an argument of type \vcc{T}
and an argument of another type, or to assign between such mismatched types.
\\\\
The C \vcc{switch} statement is extended to inductive types as
follows: if \vcc{v} is an expression of type, VCC allows the program statement 
\begin{VCC}
switch (v) {
  case c1(t11,t12,...): P1;
  case c2(t21,t22,...): P2;
  ...
}
\end{VCC}
where each \vcc{tij} is a variable name not in scope, no name occurs
as more than one of the \vcc{tij}s, each constructor of type \vcc{T}
occurs exactly once among \vcc{T1, T2, ...}, and each of the cases has
proper arity for the given constructor. Is is asserted that there is
no fallthrough between cases (i.e., a \vcc{break} or \vcc{return} at the end of each one
is mandatory). The switch statement above translates to
the following:
\begin{VCC}
{
  unsigned x;
  switch (x) {
    case (1) : {
      T11 t11; T12 t12; ...
      _(assume v == c1(t11,t12,...))
      { P1 }
      break;
    } case (2) : {
      T21 t21, T22 t22; ...
      _(assume v == c2(t21,t22,...))
      { P2 }
      break;
    } ...
    default : { _(assume \false) }
  }
}
\end{VCC}

\subsection{Pointers}

\vcc{\object} (type) (small)\\
The type of pointers.
\\\\
\vcc{^T} (small)\\
The type of pointers to ghost objects of type \vcc{T}. (\vcc{T} can be a concrete or ghost type.)
\\\\
The program determines a fixed set of \Def{pointers}.  Pointers to
instaces of object types are called object pointers; pointers to instances of
value types are called value pointers. We typically identify an object
and the (unique) pointer that points to it, and so when we talk about
an object \vcc{o}, \vcc{o} is really an object pointer.

A pointer is characterized by the following (state-independent) parameters:
\\\\
\vcc{\natural \addr(\object p)}\\
The address of \vcc{p}. If \vcc{!\ghost(p)}, 
\vcc{\addr(p) + \sizeof_object(p) < UINT_PTR_MAX}.
\\\\
\vcc{\type \typeof(\object p)}\\
The type of \vcc{p}. In C terms, this can be thought of as the type of
object to which \vcc{p} points.
\\\\
\vcc{\bool \ghost(p)}\\
True if \vcc{p} is a ghost pointer (i.e., points to the ghost
heap). If not, we say \vcc{p} is concrete. 
\\\\
\vcc{\non_primitive_ptr(\object p)}\\
True if \vcc{p} is an object. 
\\\\
\vcc{\object \embedding(\object p)}\\
The object of which \vcc{p} is a part of. If \vcc{p} is an object, the
result is undefined. If \vcc{p} is a concrete value pointer,
\vcc{\embedding(p)} is concrete.
When the \vcc{&} operator is applied to an expression of the
form \vcc{o->f}, where \vcc{f} is of value type, the
aforementioned object is \vcc{o}. When it is applied to a local
variable, it is the object containing the variable.
\\\\
\vcc{size_t \sizeof_object(\object p)}\\
Equivalent to \vcc{\sizeof(T)}, where \vcc{T} is the type
of \vcc{*p}. This depends only on \vcc{\typeof(p)}. 
\\\\
In a pure context, for pointers \vcc{p} and \vcc{q}, \vcc{p == q}
iff \vcc{p} and \vcc{q} agree on all the parameters above.
\\\\
Every pointer of the program has a footprint (a set of addresses)
defined as follows.  If \vcc{p} is a concrete value pointer, its footprint
is the set of addresses
\vcc{[\addr(p),\addr(p) + \sizeof_object(p))}.
If \vcc{p} is an object, its footprint is the union of
the footprints of all concrete value pointers 
\vcc{q} such that \vcc{\embedding(q) == p}. 
Footprints of distinct concrete value pointers with the same
embedding are disjoint.  If \vcc{p} is a concrete object, its concrete
footprint is a subset of 
\vcc{[\addr(p),\addr(p) + \sizeof_object(p))}.
\\\\
\vcc{p \is T} (expr)\footnote{Deprecated}\\
\vcc{p} must be a pointer, and \vcc{T} a value type. 
This is equivalent to 
\vcc{(p == (T *) p) || (p == (T ^) p)}.
\\\\
\vcc{_(retype)} (cast)\\
The argument to this case must be a value pointer \vcc{p}.
If there is a \vcc{\valid} object \vcc{o} with a field \vcc{f}
such that \vcc{&o->f} and \vcc{p} point have the same base type and
the same address, then this operator returns \vcc{&o->f}. Otherwise,
the result of the operation is undefined.
This operator is typically used when \vcc{p} points to a value
field of struct that is a member of a union, and some other union
member is currently \vcc{\valid}.  
\\\\
VCC does not allow C's silent promotion of an expression of
type \vcc{(void *)} to another pointer type; such a promotion requires
an explicit cast\footnote{This is justified on the grounds that it avoids
potential errors without changing the compiled code.}.

\subsubsection{Pointer Sets}

\vcc{typedef \bool \objset[\object];} \\
\vcc{\objset} is the type of sets of pointers. While the definition
above will be its eventual semantics, it is currently treated 
as a distinct type (because it is triggerred differently from other
maps). 
\\\\
\vcc{_(pure) \bool \in(\object o, \objset s)} (infix operator)\\
\vcc{_(ensures \result <==> s[o])}\footnote{
Caveat: this is not its actual definition, for type reasons, but
conveys the meaning.}
\\\\
\vcc{_(pure) \objset +(\objset s, \object o)} (infix operator)\\
\vcc{_(ensures \forall \object o1; o1 \in \result <==> o1 == o || o1 \in s)}
\\\\
\vcc{_(pure) \objset -(\objset s, \object o)} (infix operator)\\
\vcc{_(ensures \forall \object o1; o1 \in \result <==> o1 != o && o1 \in s)}\\
(\vcc{+=} and \vcc{-=} are extended analogously.)
\\\\
\vcc{_(pure) \objset \universe()}\\
\vcc{_(ensures \forall \object o; o \in \result)}
\\\\
\vcc{_(pure) \objset \everything()}\\
\vcc{_(ensures \result == \universe())}
\\\\
\vcc{_(pure) \bool \disjoint(\objset o1, \objset o2))} (infix operator)\\
\vcc{_(ensures \result <==> \forall \object o; !(o \in s1) || !(o \in s2))}
\\\\
\vcc{_(pure) \bool \subset(\objset s1, \objset s2)} (infix operator)\\
\vcc{_(ensures \result <==> \forall \object o; o \in s1 ==> o \in s2)}
\\\\
\vcc{_(pure) \objset \diff(\objset s1, \objset s2)} (infix operator)\\
\vcc{_(ensures \forall \object o; o \in \result <==> o \in s1 && !(o \in s2))}
\\\\
\vcc{_(pure) \objset \inter(\objset s1, \objset s2)} (infix operator) \\
\vcc{_(ensures \forall \object o; o \in \result <==> o \in s1 || o \in s2)}

\subsubsection{Objects}

Each object has a set of named, typed fields, 
each of which is either ghost or concrete. Each
concrete field also has an offset (of type \vcc{size_t}). 
The fields of an object are determined by its type. 
\\\\
For object \vcc{o} with field \vcc{f} of value type \vcc{T}, 
\vcc{&o->f} is a pointer of type \vcc{T}, with 
embedding \vcc{o}, which is
ghost iff \vcc{o} is ghost or \vcc{f} is ghost. If \vcc{f} is a
concrete field, \vcc{\addr(&o->f) == \addr(o) + offset},
where \vcc{offset} is the offset of \vcc{f}. 
\\\\
If a concrete object type has padding as part of its representation, the
padding is essentially treated as concrete (nonvolatile) fields of
value types with unknown field names.
\\\\
Every object\footnote{This does not include the \vcc{:ownerOb} groups defined
below.} has the following fields and invariants. Note that these
are true invariants, holding in all states, not only when the object
is \vcc{\closed}. The fields \vcc{\version}, \vcc{\volatile_version},
and \vcc{\blobifiable} below, and the
group \vcc{\ownerOb}, are not allowed in annotations. Also, in
annotations, fields beginning with \vcc{\} can appear only following
the \vcc{->} operator (i.e., \vcc{p->\owner} is allowed, but \vcc{(*p).\owner} is not)\footnote{This restriction should be
eliminated.}. 

\begin{VCC}
struct <ObjectType> {
  _(ghost volatile \bool \closed)
  _(ghost volatile \objset \owns)
  _(ghost \bool \valid)  
  _(ghost \natural \version)
  _(ghost volatile \natural \volatile_version)
  _(ghost volatile \natural \claim_count)
  _(ghost \bool \blobifiable)

  _(group \ownerOb)
  _(:\ownerOb) volatile \bool \owner;

  _(invariant :\ownerOb \unchanged(\owner) 
      || ((\inv2(\old(\owner)) || <\old(\owner) not compound>)
         && (\inv2(\owner) || <\owner not compound>)))
  _(invariant :\ownerOb \owner && \owner->\closed)
  _(invariant :\ownerOb \this \in \owner->\owns)
  _(invariant :\ownerOb \closed || ((\thread) \owner == \owner))
 
  _(invariant \forall \object o1; \this->\closed && o1 \in \this->\owns 
                                  ==> o1->\owner == \this)
  _(invariant \closed ==> \valid)
  _(invariant \closed && (\forall \object s; s \in \owns 
              ==> s->\closed && s->\owner == \this))
  _(invariant (\closed || \old(\this->\closed)) && <\this compound> 
              ==> \inv2(\this))

  _(invariant \old(\version) <= \version)
  _(invariant \unchanged(\version) ==> \unchanged(\closed))
  _(invariant \old(\volatile_version)  <= \this->\volatile_version)
  _(invariant \approves(\owner,\volatile_version))
  _(invariant \forall <field f>; <f nonvolatile> && \closed && \old(\closed) 
                                 ==> \unchanged(f))

  _(ghost volatile \bool \used)
  _(ghost volatile \objset \subjects)
  _(invariant \closed ==> \used && \unchanged(\subjects))
  _(invariant \this \is \claim && \closed ==> 
      \forall \object o; o \in \subjects ==> 
              \claimable(o) && o->\closed)
  _(invariant \this \is \claim && \old(\used) ==> \used)
  _(invariant !\old(\closed) && \closed ==> !\old(\used))  
  _(invariant \claimable(\this) && \claim_count > 0 ==> \closed)
  _(invariant \approves(\owner,\claim_count))
  _(invariant \claimable(\this) && (\closed || \old(\closed)) ==>
     \claim_count == 
     <cardinality of>(\lambda \claim c; c->\closed && \this \in c->\subjects))
  _(invariant <\this of compound type> ==>
        (<type not marked _(volatile_owns)> && \old(\closed) && \closed 
         ==> \unchanged(\owns))
     && (<type not marked _(dynamic_owns)> &&  \old(!\closed) && \closed 
         ==> (\forall \object o; o \in \this->\owns 
              <==> <there is an invariant of this type with a top-level
                    conjunct of the form \mine(e), and \old(e) == o>)
     && (\forall <field f>;
          (<\approves(\this->\owner,f) is a top-level conjunct 
            of an invariant of this type>
          ==> \unchanged(\volatile_version) || !\closed(\this) 
              || \unchanged(f))))
  _(invariant \old(\valid) && \valid ==> \unchanged(\blobifiable))
}
\end{VCC}

\subsection{Special Forms in Invariants}

\vcc{\bool \wrapped(\object o)} (thread)\\
\vcc{_(ensures \result <==> o->\owner == \me && o->\closed)}
\\\\
\vcc{_(dynamic_owns)} (compound)\\
\vcc{_(volatile_owns)} (compound)\\
These cannot both appear on the same type definition. See the
invariants above and the definition of \vcc{_(wrap)} for their meaning.
\\\\
\noindent\vcc{\bool \mine(\object o1)} (invariant)\\
\vcc{_(ensures \result <==> o1 \in \this->\owns)}
\\\\
If the object type is not marked \vcc{_(dynamic_owns)}
or \vcc{_(volatile_owns)}, this function can only be used as the
outermost function of a top-level conjunct of an invariant.
\\\\
\vcc{\bool \approves(\object o, f ))} (special)\\
This function can appear only as a top-level conjunct of an object
invariant of a type \vcc{T}, \vcc{f} must be a field of \vcc{T},
and \vcc{o} must be a field of \vcc{T} or the expression 
\vcc{\this->\owner}. If \vcc{o} is a field other than \vcc{\owner}, 
\vcc{\approves(o,o)} must also be a top-level conjunt of an invariant
of \vcc{T}. \vcc{\approves(o,f)} translates to
\vcc{\unchanged(f) || !o || \inv2(f) || (<o is a thread> && !\unchanged(\volatile_version))}. 

\subsection{States}
\vcc{\state} (large)\\
The type of states. 
\\\\
\vcc{T \at(\state s, T e)}\\
The evaluation of expression \vcc{e} in the state \vcc{s} (with all
free variables replaced with their value in the current state).  
\\\\
When \vcc{\at} occurs (implicitly or explicitly) in the declaration of
a function, purely local variables are implicitly replaced with their
actual values at the state in which the form is evaluated. (For
function preconditions and postconditions, this state is function entry.)
\\\\
\vcc{\state \now()}\\
The current state. 
\\\\
\vcc{\bool \unchanged(\object o)}\\
For value pointer \vcc{o}, this translates to \vcc{*o == \old(*o)} 
For object \vcc{o}, this holds iff, for every field \vcc{f}
of \vcc{o}, \vcc{\unchanged(o->f)}.

\subsubsection{Threads}
\vcc{\thread} (small) \\
The type of threads. 
\\\\
\vcc{const \thread \me} (thread\footnote{Caveat: currently, VCC does
not check this, and should be fixed.})\\
This is a pure local variable that gives the current
thread. 

\subsection{Compound Types}
 
\subsubsection{Ghost Fields}
Within a compound type, a field is ghost is it occurs inside the scope
of a \vcc{_(ghost)}  annotation. Ghost fields may be of any concrete
or ghost type. 
\\\\
In annotations of a compound type, the scope of a member name includes
the entire type definition.

\subsubsection{Invariants}

\vcc{_(pure) \bool \inv2s(:pure \state s1, :pure \state s2, :pure \object o)}\\
If \vcc{o} is not of a user-defined compound type, the result is 
undefined. Otherwise, it returns \vcc{\at(s2,p)}, where \vcc{p}
is the conjunction of all explicit 
type invariants in the definition of the type of \vcc{o},
with \vcc{\old(e)} replaced by \vcc{\at(s1,e)} and
with \vcc{\this} replaced with \vcc{o}.

\begin{VCC}
_(def \bool \inv2(:pure \object o)
{ \return \inv2s(\old(\now()),\now(),o); })
\end{VCC}

\begin{VCC}
_(def \bool \inv(\object o)
{ return \inv2s(\now(), \now(), o); })
\end{VCC}

\noindent\vcc{_(invariant \bool :pure e)} (member)\\
This defines an explicit invariant for the type in which it occurs.
\vcc{e} can mention any part of the state, including
static global variables. However, as there is no thread context, it
cannot mention \vcc{\me}.\footnote{This not currently checked.}
\\\\
An invariant in the declaration of type \vcc{T} forbids unwrapping if
it can be violated in an object \vcc{o} of type \vcc{T} by
unwrapping \vcc{o}. An invariant that forbids unwrapping must be
written in the form \vcc{_(invariant \on_unwrap(\this,p))}. 
\vcc{\on_unwrap(o,p)} translates to 
\vcc{(\old(\this->\closed) && !\this->\closed ==> p)}.
\\\\
Only compound types can have explicit object invariants. Other object
types have additional implicit object invariants.
\\\\
\vcc{:lemma} (special)\\
If the \vcc{invariant} tag is followed by \vcc{:lemma}, then it is
asserted that the invariant given follows from the other (non-lemma)
invariants of the type. When wrapping, upwrapping, or updating an
object of this type, invariants so marked do not have to be checked.
Caveat: currently, \vcc{:lemma} invariants of the form
\vcc{\on_unwrap(...)} are ignored.
\\\\
\vcc{const \object \this}\\
\vcc{\this} can be used only in object invariants, where it refers to the
object whose invariant is being defined.

\subsubsection{Structs}

The infix \vcc{.} operator (selecting a field of a \vcc{struct}
or \vcc{union}) is extended in both arguments, to ghost objects and to
records (on the left) and to ghost field names (on the right)\footnote{
Caveat: VCC currently does not support this operator when the name begins
with \vcc{\ }; the \vcc{->} operator must be used instead.}.
\\\\
The infix \vcc{->} operator (selecting a member of a struct or union
pointed to by the first argument) is extended to ghost pointers and
pointers to ghost types and to pointers to records on the left, and to
ghost fields on the right.
\\\\
Ghost struct declarations need not declare any members (unlike in C).
\\\\
VCC currently does not allow arrays of size 0 at the end of a struct
(these are allowed in C99).
\\\\
\vcc{_(inline)} (member)\\
This annotation must come directly before declaration of a member with
a \vcc{struct} type. The effect is that the inner object is not considered to be an
true object; instead, its fields are semantically fields of the
parent struct. Invariants of the inner struct are ignored.
\\\\
A type definition of structure type defines two types, the usual one
and a volatile type. The volatile type differs from the nonvolatile
type only in that all of its explicit fields are volatile and it has
none of the explicit invariants of the original type. Instead, the
type gets an implicit invariant that each of its fields is
owner-approved. 
\\\\
When a field of struct type is marked as \vcc{volatile}, and that
field is of a struct type, the instance is of the volatile variant of
that type. If the type is inlined, the individual fields are inlined
as usual, but each of the fields is as if it was declared to be
volatile.\footnote{
This is standard C behavior. It is most often seen when a field is
declared as \vcc{volatile T *p}, which is a pointer to volatile \vcc{T},
not a volatile pointer to \vcc{T}. This is usually a mistake.
To get a volatile field, one needs to declare \vcc{typedef T *PT}
and then use \vcc{volatile PT p}.
}

\subsubsubsection{Groups}
A group object is an artificial object that lies inside a
concrete \vcc{struct}. The physical address and size of a group
are equal to those of the containing struct. The fields of a group
need not be contiguous. (This means that the fields left to the
containing struct need not be contiguous either.)
\\\\
\noindent\vcc{_(group G)} (member)\\
\vcc{_(group G, <forms>)} (member)\\
These forms declares the identifier \vcc{G} to be the name of a nested
ghost \vcc{struct}. The scope of \vcc{G} runs from the point of this
declaration to the end of the declaration of the compound type in
which it appears.
If the second form is used,
the \vcc{<forms>} are a comma-separated list of modifiers,
each one of \vcc{_(dynamic_owns)}, \vcc{_(volatile_owns)}, \vcc{_(record)},
or \vcc{_(claimable)}; these attributes are used as attributes of the
newly defined struct type.
The type of the declared group is written as \vcc{T::G}, where \vcc{T}
is the containing \vcc{struct} type. If \vcc{e} is an expression of
type \vcc{T *}, \vcc{(e :: G *)} abbreviates \vcc{((T ::G *) e)}.
\\\\
\vcc{_(:G)} (member)\\
It declares that the immediately
following fields (until semicolon) are semantically fields of \vcc{G}, rather than fields of
the containing struct. The explicit labelling allows the fields of the
group to not be contiguous.
\\\\
\vcc{_(invariant :G p)} (member)\\
This form declares an invariant for group \vcc{G} defined earlier in
the current \vcc{struct}. It follows the same rules as ordinary object
invariants; in particular, it can mention members of \vcc{G} that are
not yet declared.

\subsubsection{Unions}
If a union includes a member of value type, exactly one of
these members must be be marked \vcc{_(backing_member)}. The size of
the backing member must be at least as large as the size of the other
members of the union. It is a VCC error to apply \vcc{&} to a 
member of a union with a backing member. However, all concrete members of a
union have the same address.
\\\\
Members of ghost union types do not necessarily alias, so assignment
through one member need not be reflected in the value of another
member.  
\\\\
In any state, if a union is \vcc{\valid}, exactly one of its members
is \vcc{\valid}. When a union is created (by an explicit or
implicit \vcc{\unblobify}), this member is the backing member (if
there is one), and is otherwise the first member of the \vcc{union}.

\subsection{Array Object Types}
An array object is an object that serves as a container for an array
(whose elements are typically of value type). They are typically used to allow
ownership transfer of the elements of the array.
\\\\
An array object type is defined by the base type and length of the
array. If the base type of the array is a value type, the elements
of the array are effectively fields of the array. If the base type of
the array is an object type, the array object has no fields (other than
the implicit ones
present in all objects), and so is not very useful. Note that
array object types are distinct from C array
types; array object types cannot be defined via \vcc{typedef}, nor can
a variable be declared to have an array object type. Expressions of
array object type can only be created by casting: if \vcc{T} is a type
and \vcc{e} is an expression of integral type, the
expression \vcc{(T[e]) v} casts \vcc{v} to a pointer to an array
object with base type \vcc{T} and length \vcc{e} (where \vcc{e} is
evaluated in the current state)\footnote{Note that this syntax doesn't
create a conflict with ordinary C array types, because C does not
allow casts to array types.}.
If the base type of an array is volatile, the array object has
an implicit invariant that all changes to the array are
owner-approved.
\\\\
\vcc{_(root_index \type T size_t s)}\\
This macro should be applied only to a pointer \vcc{o} of type \vcc{T*}, and
expands to \vcc{_(T *)_(T[s])}. This has the effect of giving
back \vcc{o} but, if \vcc{o} is of value type, changing its
embedding to the array object \vcc{_(T[s])o}. 
\\\\
\vcc{\is_array(T *p, size_t s)} (expr)\\
True iff \vcc{p} is a pointer to an array object of size \vcc{s}.

\subsection{Blobs}
\vcc{\blob} (small)\\
The type of (pointers to) blobs. 
\\\\
A blob is an uninterpreted contiguous portion of the concrete address
space, much like a struct of a given size with nothing but padding. 
A chunk of memory starts out as a blob. This blob can be broken into
smaller blobs, joined into larger contiguous blobs, or turned into a
collection of objects (essentially, the \vcc{\extent} of the outermost
object). The outermost object made from the blob is said to be
``blobifiable''; only blobifiable objects can be blobified. (This is
to avoid accidentally making multiple objects of the same type with
the same address.) 
\\\\
When a new object is allocated (either through a call to \vcc{malloc} or
entry to a block with with a local variable that is not purely local),
it is implicitly created as a blob of appropriate size and alignment,
then implicitly unblobified to the object of specified type.
\\\\
To change a piece of memory from holding one type of object to holding
another, the first object must turned into a blob, this block might be 
merged with other blobs or split into smaller blocks, and finally a blob
is turned into a new object. 
\\\\
\vcc{_(blob size_t sz))} (cast)\\
This returns a pointer to a \vcc{blob} of size \vcc{s} with the same
address as \vcc{ptr}. It has no effect on the state. 
\\\\
\vcc{_(blobify \object o)} (statement)\\
This \vcc{requires}  \vcc{\extent_mutable(o))} and
\vcc{o} is blobifiable. It has the effect of making all
objects in \vcc{\extent(o)} invalid and making valid a blob of
size \vcc{sizeof(o)} with the address of \vcc{o}.
\\\\ 
\vcc{_(join_blobs \blob a, \blob b)} (statement)\\
This joins \vcc{a} and \vcc{b} into a single blob. This requires that
the address of \vcc{b} is the address of \vcc{a} plus the size
of \vcc{a}, and that \vcc{a} and \vcc{b} are mutable.
\\\\
\vcc{_(split_blob \blob a, \natural s)} (statement)\\
This requires \vcc{s} is positive, \vcc{a} is \vcc{\mutable} and of size greater
than \vcc{s}. The result is to split \vcc{a} into two contiguous
blobs, the first of size \vcc{s} and the second of size the original
size of \vcc{a} minus \vcc{s}.
\\\\
\vcc{_(blob_of)} (cast)\\
The argument must be an object.
Applying it to an \vcc{\object o} yields
the same result as \vcc{_(blob \sizeof(o)) o}.
\\\\
\vcc{_(unblobify)} (cast)\footnote{
This is provided as a cast rather than a statement, because
legacy code often requires unblobification in the middle of an expression.}\\
Applying this to object \vcc{o}
\vcc{requires} that there is a \vcc{\wrapped} blob with the same
address and size as \vcc{o}. The effect is to make all of the objects
in \vcc{\extent(o)} \vcc{\mutable}, marking \vcc{o} as blobifiable and
all other objects in \vcc{\extent(o)} as not blobifiable. If
the type of \vcc{o} involves unions, the backing member is chosen as
the ``active'' member of the union; if the union has no backing
member, the first union member is chosen.
\\\\
\vcc{_(union_reinterpret \object o)}\\
This requires \vcc{o} to be of the form \vcc{&(x->f)}, where \vcc{x} is
a pointer to a union type and \vcc{f} is a member of the type. It also
requires that the union is mutable, and that the extent of the currently active
member of the union (if that memver is of object type) is mutable. The effect
is to make \vcc{f} the currently active member of \vcc{x} i.e., to make
its extent mutable (choosing union members of nested unions as
in \vcc{_(unblobify)}. 

\subsection{Claims}
\vcc{\claim} (small)\\
The type of (pointers to) claims. 
\\\\
A compound type (concrete or ghost) can be marked a \vcc{_(claimable)} (just before
the \vcc{struct} keyword in its type definition). Claimable objects
cannot be unwrapped when their \vcc{\claim_count}s are nonzero.
\\\\
A \vcc{\claim} has no data beyond those in every object, but can have
an additional invariant. For any predicate \vcc{p} that does not
use \vcc{\old()}, there is a type of \vcc{\claim}s with
invariant \vcc{p}, as long as the resulting type is admissible.
\\\\
Conceptually, when there is an atomic update, the invariants of
updated objects are checked first, assuming only the 1-state invariants of
all objects in the prestate. (This is the legality check.)
Next the invariants of all unupdated non-\vcc{\claim} objects are
checked (this is admissibility check, in reality performed per-type, not per-update);
this check assumes the 1-state of all objects in the
prestate, and the invariants of all updated objects. Finally, the
invariants of all claims are checked; this proof gets to additionally
assume that all objects (including unmodified ones) satisfy their invariants.
This final check is, like the admissibility check, performed once
for claim type.
\\\\
\noindent\vcc{_(pure) \bool \claims(\claim c, :pure p)}\\
This holds iff, in every \vcc{good} state in which \vcc{c} is
closed, \vcc{p} holds. 
\\\\
\noindent\vcc{_(pure) \bool \claims_object(\claim c, \object o)}\\
True iff \vcc{o} is one of the claimed objects of \vcc{c}.
\\\\
\noindent\vcc{_(pure) \bool \active_claim(\claim c)}\\
True iff \vcc{c} is closed. However, it has the effect of also
asserting the invariant of \vcc{c}. Asserting or assuming \vcc{\active_claim(c)}
in a state (perhaps implicitly) is the only way to trigger
instantiation of the invariant of \vcc{c} in that state.
\\\\
\vcc{\bool \claims_claim(\claim c, \claim c1)}\\
(Deprecated) Equivalent to 
\vcc{\claims(c, c1->\closed && \valid_claim(c1))}, 
but triggers somewhat differently.
\\\\
\vcc{\bool \claimable(\object o)}\\
True iff \vcc{o} is of an object type that was declared as \vcc{_(claimable)}.
\\\\
\vcc{\claim \make_claim(\objset s, :pure \bool p)} (special)\\
The set \vcc{s} is evaluated in the current state, while \vcc{p} is
not. \vcc{p} cannot use \vcc{\old()}. This translates roughly to the following:
\begin{VCC}
_(assert p)
_(assert <the claim type with invariant (p && \subjects == s) is admissible>
\claim c;
_(assume <s is the set of claimed objects of c and \vcc{p} is the invariant of s>)
_(assume !c->\used && !c->\closed)
_(ghost_atomic c,s {
    c->\subjects = s;
    _(wrap c)
    foreach \object o \in s;
      _(assert \non_primitive_ptr(o))
      o->\claim_count ++;
      o->claimants += c;
})
\end{VCC}

\noindent\vcc{\claim \upgrade_claim(\claim c1,..., bool p))} (statement) \\
This creates a new claim that claims the invariants
of \vcc{c1,c2,...} conjoined with \vcc{p}, while
unwrapping \vcc{c1,c2,...}. (It thus writes \vcc{c1,...}, but does
not write the subjects of \vcc{c1,...}.)
The subjects of the result claim are the sum of the subjects of \vcc{c1, c2, ...}.
Note that if the input
claims claim a common object, the number of objects claiming that
object will fall below the \vcc{\claim_count} of the object, making it
unable to open again, so this should normally not be the case for
concrete objects.
\\\\
\vcc{_(pure) \bool \account_claim(:pure \claim c, :pure \object o)}\\
This is semantically equivalent to \vcc{(\wrapped(c) && \claims_object(c,o))}.
It exists to trigger the following axioms: 
\begin{VCC}
  (\forall \claim c1,c2; \forall \object o; {\account_claim(c1,o), \account_claim(c2,o)}
     \account_claim(c1,o) && \account_claim(c2,o) && c1 != c2 ==> o->\claim_count >= 2)
\end{VCC}
and similarly for up to 4 distinct claims.
\\\\
\noindent\vcc{_(pure) _(by_claim :pure \claim c)} (cast)(thread)\\
The target of the cast must be of the form \vcc{o->f}, where \vcc{f} must be
a nonvolatile\footnote{
Caveat: this should be generalized to volatile fields, where \vcc{c}
guarantees that the field has the same value in all states
where \vcc{c} is valid, but this extension is not yet supported.
} field of \vcc{o}. It evaluates
to \vcc{o->f}, but also asserts 
\vcc{\active_claim(c) && \claims(c,o->\closed)}.
\\\\
\noindent\vcc{\bool \always_by_claim(\claim c, \object o)}\\
\vcc{o} must be an identifier. This expression evaluates
to \vcc{\true}. Asserting this as a top-level conjunct causes each 
subsequent occurrence of \vcc{o->f} (until the end of the current
block), where \vcc{f} is a nonvolatile field of \vcc{o}, to be
replaced with \vcc{(_(by_claim c)(o->f))}.
\\\\
\noindent\vcc{void \destroy_claim(\claim c, \objset s)}\\
This asserts that \vcc{c} is \vcc{\wrapped} and \vcc{\writable}, and
that it claims every object in \vcc{s} (though it might claim
others). It has the effect of opening \vcc{c} and decrementing
the \vcc{\claim_cnt} of each object in \vcc{s}.
\\\\
\vcc{_(pure) T \when_claimed(T :pure e)} (special)\\
This macro can appear only inside of \vcc{\claims} or in the second
argument to \vcc{\make_claim}. It translates to \vcc{\at(s,e)}, where
\vcc{s} is the state in which the surrounding \vcc{\make_claim} is
being executed, or the state in which the surrounding \vcc{\claims} is
being evaluated. (If inside an \vcc{_(ensures)}, \vcc{s} is the state
in which the function returned.)
\\\\
Within a program block, local variables (purely local or not) occurring
inside the second argument of \vcc{\claims} or \vcc{\make_claim} are
implicitly wrapped in \vcc{\when_claimed}.

\section{Declarations}
In general, declarations follow the C conventions regarding
declarations and scoping, with the following exceptions. 
\\\\
Ghost declarations are enclosed in \vcc{_(ghost)}, or in ghost code
blocks.  Top-level ghost declarations (global variables, typedefs, and
function declarations) can be added anywhere a top-level C declaration
can appear. Local ghost variable declarations can appear wherever a C
statement can appear.  Ghost identifiers use the same namespace as
ordinary C identifiers; is an error to shadow one with the other.
\\\\
The scope of a struct or union declaration, as well as the fields of
such declarations, is extended to include all annotations within the
entire compilation unit in which it appears. (However, a typedef name
declared along with it has the usual scope, from point of declaration
forward.) This exception to the scoping rules of C is needed to allow
type invariants to mention the fields of types that have not yet been
declared (since C provides no mechanism for forward declaration of
fields).
\\\\
If the identifier \vcc{T} is not in use as an annotation tag, the
top-level form \vcc{_(T <decl>)} abbreviates \vcc{_(ghost T <decl>)}.

\subsection{Purely Local Variables}
A purely local variable is a local variable of a function (or a
function parameter) such that neither the variable nor any nested
member of the variable (if it is of a compound type) is subject to
the \vcc{&} operator\footnote{Caveat:
a variable should remain purely local if its
address is taken only inside of \vcc{_(writes)}, \vcc{_(wrap)}, and \vcc{_(unwrap)}.
Finer-grained escape analysis might also be possible, to allow
a function spec to declare that its use of a variable ends on function return.
}. (That is, a variable is purely local if its
address is never taken.) Such variables are given special treatment
by VCC, because they can be modified by a program fragment only if it
mentions the variable by name. (In particular, it cannot be modified
by a function call.) Formal parameters of a function (because they are not
lvalues) are always purely local. 
\\\\
In a loop, purely local variables that are not updated are
automatically asserted to not change during the loop, without
requiring an explicit invariant. Conversely, purely local variables
that are updated are implicitly included in the writes clause for the
loop, and so do not have to be explicitly listed \footnote{Currently,
mentioning the variable explicitly would ruin its pure locality.
}.
\\\\
Purely local variables are always considered writable, and so do not
have to be mentioned in writes clauses for blocks\footnote{This
feature is depricated, and will likely be removed in future.}. 
\\\\
A purely local variable of \vcc{struct} type is treated
as a collection of purely local variables, one per field, and so each
field gets separately the treatment described above.

\subsection{Global Variable Declarations}
For every global variable \vcc{v} of value type, there is a dummy object
of struct type of which \vcc{v} is the only field (outside those of
all objects). If \vcc{v} is declared as volatile, this dummy object has the implicit invariant
\vcc{_(invariant \approves(\this->\owner,v))}
footnote{Without this invariant, there would be no way to admissibly
maintain any information about the value of the variable, making it
useless for verified programming.
}.
\\\\
\vcc{\bool \program_entry_point()}\\
True iff all global variables of object types, and the dummy
objects owning global variables of value types,
are \vcc{\mutable}. This function is allowed only in a thread
context\footnote{
Caveat: currently VCC does not check this. This should be fixed.
}.

\subsection{Function Declarations}

A function declaration, with or without a body, 
can include a specification. If multiple declarations of the same
function include specifications, their specifications must be identical.
A specification consists of a nonempty sequence of annotations of the
following forms:
\\\\
\vcc{_(requires :pure b)} (contract)\\
This puts  an obligation on  all callers of  the function that  at the
point of the  call (after formal parameters have  been bound), \vcc{b}
should evaluate to a nonzero  value (i.e., \vcc{b} is asserted at each
call site). This also introduces  the assumption\vcc{b} to be added at
the beginning of the function body. 
\\\\
\vcc{b} (or a top-level conjuncts of \vcc{b}) can be of the
form \vcc{_(assume b)}. The effect is to assume \vcc{b} on every call
to the function (i.e., to omit the \vcc{_(assert b)} that would
otherwise be generated.
\\\\
\vcc{_(ensures :pure b)} (contract)\\
This asserts \vcc{b} at each return point from the function. This also
allows callers of the function to assume \vcc{b} when the function
returns. Within \vcc{b}, \vcc{\old(e)}, where \vcc{e} is an
expression, means the value \vcc{e} evaluated to on function entry
(after binding of parameters).
\\\\
\vcc{_(writes :pure \object o1, o2, ...)} (contract) \\
\vcc{_(writes :pure \objset s)} (contract)\\
The first form is shorthand for \vcc@_(writes {o1,o2,...})@. 
If no such forms appear, there is an implicit form \vcc@_(writes {})@.
Multiple forms are tacitly combined by rewriting 
\vcc{_(writes s1) _(writes s2)} to \vcc{_(writes s1 \union s2)}. 
\\\\
For a function with exactly one form \vcc{_(writes s)} of this type,
the form is, for callers of the function, semantically equivalent to 
\begin{VCC}
_(requires \forall \object o; o \in s ==> \writable(o))
_(ensures \forall \object o; !(o \in s) ==> 
   (\non_primitive_ptr(o) && \old(o->\owner == \me) 
    ==> \unchanged(o->\version) 
        && \unchanged(o->\volatile_version))
   && (!\non_primitive_ptr(o) && \old(\mutable(o)) ==> \unchanged(o)))
\end{VCC}

\noindent However, within the body of \vcc{f}, a stronger condition is used: a
sequential write to \vcc{o} requires \vcc{\writable(o)}.
\\\\
\vcc{_(reads :pure \object o1,...)} (contract)\\
This annotation can appear only on \vcc{_:pure} functions. If
nultiple reads annotations are on the same function, their lists are
concatenated to form a single list. This annotation introduces
an additional proof obligation that, for any two states that differ only on the values
of fields of \vcc{o1,...}, the function returns the same result. It also
allows generation an axiom to that effect (to be used in the verification
of other functions), unless the function is also annotated with \vcc{_(no_frameaxiom)}
annotation. 
\\\\
\vcc{_(maintains \bool p)} (contract)\\
Equivalent to \vcc{_(requires p) _(ensures p)}.
\\\\
\vcc{_(always \claim c, \bool p)} (contract)\\
Equivalent to 
\begin{VCC}
  _(requires \wrapped(c) && \claims(c, cond))
  _(requires \assume(\active_claim(c)))
  _(ensures \wrapped(c))
  _(ensures \assume(\active_claim(c)))
\end{VCC}

\noindent\vcc{_(updates \object o)} (contract)\\
Equivalent to
\begin{VCC}
  _(requires \wrapped(o))
  _(ensures \wrapped(o))
  _(writes o)
\end{VCC}

\noindent\vcc{_(out_param(\object p)} (contract)\\
Equivalent to
\begin{VCC}
  _(writes p)
  _(maintains \mutable(p))
\end{VCC}

\noindent\vcc{_(returns e)}\\
Equivalent to \vcc{_(ensures \result == e)}

\subsection{Function Declarations}
A function specification consists of a nonempty sequence of
annotations with the tag \vcc{requires}, \vcc{ensures},
or \vcc{writes} (or macros that translate to these annotations). A
function specification can appear immediately after a function
declaration (before the ending semicolon), in a function definition
(between the declaration and the body), or immediately before a
program block. All specifications on function declarations or
definitions of the same function must have the identical
specifications.
\\\\
A typedef of a type that includes a function application can include
specifications for the function(s) specified in the defined type;
these specifications are given just before the semicolon terminating
the typedef. For example:
\begin{VCC}
typedef void *Sort(size_t size, int *b)
_(requires \mutable_array(b,size))
_(writes \array_range(b,size))
_(ensures sorted(b,size))
;
\end{VCC}

\subsubsection{Ghost Parameters}

Before the closing parenthesis of the parameter list, ghost parameters
to the function can be declared using the two forms below. Note that
such parameters are not comma-separated.
These parameters obey the usual rules of function
parameters (their names cannot conflict with names of other
parameters, their scopes are the entire function declaration, etc.).
\\\\
\vcc{_(ghost decl)} (special)\\
This declares a ghost parameter of the function. Such parameters
behave just as ordinary function parameters, e.g. they are passed by value.
\\\\
\vcc{_(out decl)} (special)\\
Here, \vcc{decl} is an ordinary parameter declaration (i.e., one that
could appear as a parameter of a ghost function). This 
deckares an out parameter of the function. The parameter
cannot be mentioned in \vcc{_(requires)} or \vcc{_(writes)}
specifications, and is considered writable in the body of the
function. On function return, the value of the formal parameter is
copied to the variable passed as an actual parameter.

\subsubsection{Atomic Inline Functions}

\vcc{_(atomic_inline)} (specifier)\\
If a function is marked \vcc{_(atomic_inline)}, (1)calls to the
function are replaced by the body (with function parameters assigned
as local variables), and (2) the body of the function is treated as an
atomic action (which means that it can appear inside an atomic
action). Such functions cannot have specifications.

\subsubsection{Pure Functions}
A pure function is one that doesn't modify the state (as seen my its
caller and other threads), and whose return value can be treated as a
mathematical function of the state. Only pure functions can be used in
specifications and assertions. This mathematical function is defined
to be an arbitrary one such that, for any state satisfying its
preconditions, the result satisfies its postconditions. For states not
satisfying its preconditions, the mathematical function takes on an
arbitrary value of the result type\footnote{Such a value always exists,
because all value types are nonempty.}.

\vcc{_:pure} (specifier)\\
This declaration can appear only after the return type declaration of
a function declaration or definition. A function marked in this way is
subject to the following restrictions:
\begin{itemize}
\item the function can assign only to local variables;
\item its local variables must be purely local;
\item it can use \vcc{_(ghost_atomic)} actions, but not other atomic actions;
\item it can call only pure functions;
\item if it is not a ghost function, its local variables must be
initialized before they are read\footnote{Caveat: this check is not
currently done.}.
\end{itemize}

A warning is issued for a pure function that is not given a
body\footnote{
VCC currently supports pure functions without function
bodies, but with postconditions of particular forms. However, the
restrictions on such functions are complex, and their use is
deprecated.
}.
\\\\
\vcc{_(def <decl>)}\\
This is an abbreviation for 
\vcc{_(ghost _:pure _(inline) <decl>)}.
\\\\
\vcc{_(pure <decl>)}\\
\vcc{_(abstract <decl>)}\footnote{Depricated}\\
These are abbreviations for \vcc{_(ghost _:pure <decl>)}. 

\subsubsection{Custom Admissibility Checks}
\vcc{_(admissibility)} (specifier)\\
\vcc{_(admissibility)} declares a function as a custom admissibility check. The function
function must take a single argument having type a pointer to a \vcc{struct} type;
verification of the function is used instead of an admissibility check
of the \vcc{struct} type. The function adds an explicit postcondition
that  \vcc{_(havoc_others)} is called exactly once, and that the
function does not otherwise modify the state. It should modify only
local variables, and should not take the addresses of any of these
variables to assure that they are not heapified. Finally, there is an
implicit postcondition that \vcc{\inv2s(s1,s2,o)}, where \vcc{s1} is
the state on entry, \vcc{s2} is the state on exit, and \vcc{o} is the
function parameter.
\\\\
\vcc{_(havoc_others \object o)} (statement)\\
This statement can appear only in functions
marked \vcc{_(admissibility)}. It changes the state to an arbitrary
\vcc{\legal} state, without changing any fields of \vcc{o}.

\subsubsection{Termination}
If a function, block, or loop is annotated with a \vcc{_(decreases)}
annotation, it is guaranteed to terminate (in any infinite execution
that includes an infinite number of steps by the thread entering the
function, block, or loop).

\vcc{_(recursive_with f1,f2,...)} (contract)\\
\\ 
This annotation can appear only on the specification of a
function \vcc{f}; we say that \vcc{f} ``publicly calls'' each of the
functions \vcc{f1,f2,...}. 
\\\\
A function \vcc{f} ``calls'' function \vcc{g} if the body of \vcc{f}
includes a call to \vcc{g}; if the body of \vcc{f} is visible to VCC,
\vcc{f} ``visibly calls'' \vcc{g}. Let ``apparently calls'' by the
union of ``visibly calls'' and ``publicly calls''. 
Let ``transitively calls'' be the
transitive closure of the ``calls'' relation, and let Let ``visibly
transitively calls'' be the transitive closure of the ``publicly
calls'' relation. 
\\\\
For termination verification to be sound, VCC requires, but does not
check, the following condition: if function \vcc{f} in the
verification unit calls function \vcc{g} that is not in the
verification unit, but does not publicly call \vcc{g}, \vcc{g}
transitively calls \vcc{f}, and the call to \vcc{g} does not
terminate, then \vcc{g} visibly transitively calls \vcc{f}.
\\\\
\vcc{_(decreases :pure \natural t1, t2, ...)} (contract)\\
\\
A \vcc{_(decreases)} annotation can appear only in the specification of a function or
a block. Within the
function body or block, this defines a function from states of the
body or block to finite tuples of \vcc{\natural}s, given by the value
of the tuple \vcc{<f1,f2,...>}. If the annotation is on a function,
this measure is extended to calls of the function by evaluating this
tuple after the binding of formal parameters. 
\\\\
Any function declared \vcc{_:pure} that does not have
a \vcc{_(decreases)} clause in its specification is given implicitly
the annotation \vcc{_(decreases p1, p2, ...)} where the
sequence \vcc{p1,p2,...} is obtained from the sequence of function
parameters by removing all parameters that are of neither 
integral nor record type, removing all parameters of type \vcc{\bool},
and applying \vcc{size()} to any arguments of record type.
\\\\
In the body of a function \vcc{f} with a \vcc{_(decreases)} clause, on
any call to a function \vcc{g} that visibly transitively calls \vcc{f},
it is asserted that the measure of the call to \vcc{g} is smaller
than the measure of \vcc{f} on entry to \vcc{f}, where measures are
ordered as follows:
\begin{VCC}
<x1,x2,...> < <y1,y2,...> <==> x1 < y1 || (x1 == y1 && <x2,...> < <y2,...>)
<x1,...> = <0,x1,...>
\end{VCC}
the latter equation used to compare tuples of unequal length.
\\\\
If a \vcc{_(decreases)} clause is put on a loop, VCC asserts on any
return to the head of the loop from within the loop that the measure
on loop is smaller than the value of that measure on loop entry. If a
\vcc{while} or \vcc{for} loop without a \vcc{_(decreases)} clause appears in the body of a function
or loop with a \vcc{_(decreases)} clause, and the loop test is of the
form \vcc{e1 < e2}, \vcc{e1 <= e2}, \vcc{e1 > e2}, \vcc{e1 >= e2},
or \vcc{e1 != e2}, the loop is given a default clause 
\vcc{_(decreases abs(e1 - e2))}, where \vcc{abs} is the absolute value function
(on \vcc{\integer}). 


\section{Expressions}
This section covers only expression forms not in C and not covered
earlier in section \ref{sec:types}.

\subsection{Logical Operators}

\vcc{\bool <==>(T1 op1, T2 op2)} (infix operator) \\
\vcc{\bool ==>(T1 op1, T2 op2)} (infix operator)\\
\vcc{\bool <==(T1 op1, T2 op2)} (infix operator)\\
\vcc{T1} and \vcc{T2} must be integral types. 
\vcc{(x ==> y)} is \vcc{\true} iff \vcc{x == \false} or \vcc{y != \false}\\
\vcc{(x <== y)} is equivalent to \vcc{(y ==> x)}.
\vcc{(x <==> y)} is \vcc{\true} iff \vcc{(x ==> y)} and \vcc{(y ==> x)}.

\subsubsection{Quantifications}

\vcc{\forall <decl>; :pure \bool p} (expression) \\
\vcc{\exists <decl>; :pure \bool p} (expression)

A quantification has one of the forms
where \vcc{<decl>} is a variable declaration. In each variable
declarations, the scope of the declared variable is to the end of the
expression. Optional patterns may appear after the semicolon
(\secref{triggers}). 
The forall/exists evaluates to 
\vcc{\true} iff \vcc{p} evaluates to a non-\vcc{\false} value for
each/some possible instantiation of the quantified variables.

\subsection{Function Calls}
In a call to a function with declared \vcc{_(ghost)} or \vcc{_(out)}
parameters, the corresponding arguments of a call must be filled with
parameters of the form \vcc{_(ghost e)} or \vcc{_(out v)},
where \vcc{e/v} is an expression/variable of suitable type. As in the
declaration, these parameters must not be comma-separated.

\section{Statements}

\subsection{Wrapping and Unwrapping}

\vcc{_(wrap :pure \object o1, ...) (statement)}\\
This translates roughly as follows\footnote{Caveat:
currently, \vcc{_(wrap)} and \vcc{_(unwrap)} cannot be used in atomic
actions (except before \vcc{_(begin_update)}. This restriction should
be lifted.  }:
\begin{VCC}
foreach (\object o \in o1,...) {
  _(assert \mutable(o) && \writable(o))
  _(assert \forall \object o1; o1 \in o->\owns ==> \wrapped(o1) && \writable(o1))
  if <the type of o is not marked _(dynamic_owns) or _(volatile_owns_)>
    o->\owns = {};  
    foreach <top level conjunct \mine(t) in the invariant of o>
      o->\owns += {t};
}
_(ghost_atomic o1, ... {
   foreach (\object o \in o1, ...) {
     foreach (\object x \in o->\owns) x->\owner = o;
     o->\closed = \true;
     _(bump_volatile_version o)
   }
})
\end{VCC}

\noindent\vcc{_(unwrap :pure \object o, ...)} (statement)\\
This translates roughly to
\begin{VCC}
  _(assert \wrapped(o) && \writable(o))
  _(ghost_atomic o {
    o->\closed = \false;
    _(bump_volatile_version o)
    foreach (\object o1; o1 \in o->\owns) 
      o1->\owner = \me;
  })
\end{VCC}

\subsubsection{Unwrapping Blocks}
An unwrapping block has the form
\vcc{_(unwrapping \object o1, \object o2, ...) W B} (statement)\\
where \vcc{writes} is a sequence of \vcc{_(writes)} annotations
and \vcc{block} is a program block without a
specification. Let \vcc{S} be the set of objects
listed in \vcc{writes}, interpreted at the entry to the unwrapping
block. The block then translates as follows:

\begin{VCC}
_(ghost \state st = \now())
_(ghost \objset W = \at(st, <set of objects listed in writes clauses>))
_(ghost \object o1 = \at(st,o1))
_(ghost \object o2 = \at(st,o2))
...
_(unwrap o1)
_(unwrap o2)
... 
block
...
_(wrap \at(st,o2)) _(assert \domain(o2) == \at(st,\domain(o2)))
_(wrap \at(st,o1)) _(assert \domain(o1) == \at(st,\domain(o1)))
_(assert \forall \object s,x; s \in {o1,o2,...} && x \in \domain(s)
  ==> (*x == \at(st,*x) || x \in W))
\end{VCC}  
where \vcc{o1,o2,...} are fresh variable names. If the the block has
no writes clauses, the last assersion is omitted. If
the \vcc{_(unwrapping)} is not followed by a program block, it is
equivalent to a program block without writes clauses.
\\\\
\vcc{\bool \domain_updated_at(\object d, \objset s)}\\
This macro translates to 
\begin{VCC}
_(unchanged \domain(d))
&& (\forall \object o; 
      \old(o \in \domain(d) && (\forall \object c; c \in s ==> !(o \in \domain(s))))
      ==> \unchanged(o))
\end{VCC}

Currently, VCC will verify such a postcondition only under some
very specific conditions. First, \vcc{s} must be syntactically
presented as an explicit set of explicit paths through the domain
of \vcc{d} (e.g., \vcc{d->a, d->b->c->e}). Second, the verification 
will fail if \vcc{d} is unwrapped with \vcc{_(unwrap d)}. The only
modifications allowed to \vcc{\domain(d)} are through 
\begin{itemize}
\item a direct assignment
\item 
a function call or block with a contract
that writes \vcc{a}, such that the function or block specification
implies (under the conditions in which it is called) 
\vcc{_(ensures domain_updated_at(d, s1))} where 
\vcc{\subset(s1,\at(st,s))) and where \vcc{st} is the state at entry to
the containing function body
\item an unwrapping block whose writes within \vcc{\domain(d)}
\end{itemize}

\subsection{Expression Statements}
As in C, an expression, followed by a semicolon, is a statement; it is
executed by evaluating the expression. In a ghost context, VCC does
not allow expression statements without side effects; an expression
statement must either contain an assignment or a call to a
non-\vcc{\pure} function.

If the expression statement reads a field of an object, there is an
implicit assertion that the object is \vcc{\thread_local}.
If the expression writes a field of an object, VCC asserts that the
field is \vcc{\writable}.  

For each read of a variable \vcc{v}, VCC asserts that \vcc{v}
is either \vcc{\thread_local} or is a volatile field of an object in
the read set of an enclosing atomic action. 

For each write of a variable \vcc{v}, VCC asserts that \vcc{v} is
either \vcc{mutable} and \vcc{\writable}, or is a volatile
field of an object in the writes set of an enclosing atomic action.

\subsection{Block Contracts}
If a program block immediately preceded by one or more function
annotations with tag \vcc{requires}, \vcc{ensures}, \vcc{writes},
or \vcc{pure}, these annotations are used as specifications for the
block, with \vcc{\old(e)} meaning the value of \vcc{e} at entry to the
block. Any postconditions are enforced for the block, even if control
exits via a goto to an outside location.

If a block has among its
preconditions \vcc{_(requires \full_context())}, the block is verified
using full context from the preceding code of the function. Otherwise,
the block is verified using only those preconditions listed explicitly
in the block contract.

If a block has among its
postconditions \vcc{_(ensures \full_context())}, then the full context
of the block is available on exit from the block. Otherwise, only
those postconditions explicitly given to the block are exposed on
exit. In this case, if the block is marked \vcc{_:pure}, then the
block is treated as a pure function call; otherwise, it is treated as
an impure function call (wrt. the havocing of state). 

\subsection{Assertions and assumptions}
\vcc{_(assume :pure b)}\\
This is equivalent to a call to a pure function with the contract
\begin{VCC}
_(ensures b)
\end{VCC}
\footnote{
A common idiom is to use \vcc{_(assume 0)} to prevent
verifications of assertions that follow. It should be noted that VCC
can often tell that the following code is unreachable, causing later
jumps to be ignored when constructing the control flow graph of the
function. In a loop body, this has the effect of causing VCC to verify
the body only for the first iteration through the loop, rather than an
arbitrary one. This can be avoided by assuming something like 
\vcc{x!= x}.}

\noindent\vcc{_(assert :pure b)} (statement)\\
This is equivalent to a call to a pure function with the contract
\vcc{_(requires b)}. 
\\\\
\vcc|_(assert {:bv} :pure b)|\\
\vcc{b} cannot include free variables or reference the heap, and its
quantifications must be over finite types.
This is equivalent to \vcc{_(assert b)}, but
requires \vcc{b} to be proved using bitwise reasoning. 

\subsection{Ghost Statements}
\vcc{_(ghost stuff)} (statement)\\
This introduces a ghost statement.
Here, \vcc{stuff} is a declaration or a statement (which might be a
block statement). If a declaration,
it does not have to precede all statements in the current block.
\vcc{stuff} follows the syntax of C statements, but can use the
operators and functions described in this document. \vcc{stuff} does
not have to be terminateed by a semicolon.

\subsection{Iterative Statements}
The keywords \vcc{while}, \vcc{do}, and \vcc{for}) 
can be succeeded by a sequence of annotations of the form 
\vcc{_(invariant e)}, \vcc{_(writes)} clauses, or 
\vcc{_(ensures e)}. Such annotations belong to the loop itself, and
not to the program block (if any) that follows the keyword; the
following block cannot have a contract, though blocks nested inside
can. If the keyword is not annotated by any \vcc{_(writes)} annotations,
it is implicitly so annotated with the \vcc{_(writes)} set of the
immediately surrounding loop if it is nested inside another loop, and
otherwise with the \vcc{_(writes)} set of the function. For every
purely local variable \vcc{v} that is not explicitly update in the
loop, there is an implicit loop invariant \vcc{_(invariant v == \at(s,v))},
where \vcc{s} is the state at loop entry.
\\\\
VCC does not allow control transfer into the body of a loop, but does
allow a \vcc{goto} to transfer control out of loops.
Gotos within ghost code cannot use a target outside of
ghost block in which the goto occurs.
\\\\
VCC requires that the control flow graph of a function body is
reducible. This allows it to translate the function into properly
nested blocks, with only forward \vcc{goto}s. In doing this, a label
within the body may become a head of a loop of the form of 
\vcc@while (1) {...}@. If this is done, the invariant associated with that loop is
given by any explicit assertions that immediately follow the label.
However, there is no way to put an explicit writes clause on such a
loop.

\subsection{Atomic Actions}

\subsubsection{Closed Object Lists}
The atomic actions all make use of \Def{closed object lists}. A closed
object list is a nonempty, comma-separated list of objects
each possibly preceded by \vcc{_(read_only)}. Those objects in the list
that are not marked \vcc{_(read_only)} are called the write set of the list.
\\\\
Intuitively, a closed object list represents a set of objects that are
known by this thread to be closed, and are guaranteed to remain closed
despite legal actions by other threads. Each object of the closed
object list must either be \vcc{\wrapped}, or its closedness must
follow from the invariants of objects earlier in the list (along with
thread-local information). The list included as part of an atomic
action is required to include all objects that are read or written in
the atomic action (other than those that are \vcc{\mutable}). 
\\\\
To validate an object \vcc{o} in the list is to 
\vcc{_(assert \active_claim(o))} if \vcc{o} is a \vcc{\claim} and
\vcc{_(assert o->\closed && \inv(o))} otherwise.
To validate an object list is to validate its elements in
left-to-right order. A validation of the list consists of this
sequence of assertions performed in a particular state. The read
objects of a validation is the set of objects ocurring in the list,
along with the union of sets ocurring in the list, as evaluated in the
validation state. The set of write objects of a validation is defined
the same way, but omitting those elements annotated
as \vcc{_(read_only)}.
\\\\
Execution of an atomic block translates approximately as follows:
\begin{VCC}
\state s = \now();
<set the current state to an arbitrary good state>
_(assert \me == \at(s,\me))
_(assert \forall \object o; \at(s, \wrapped(o) || \mutable(o))
    ==> \unchanged(o->\version) && \unchanged(o->\volatile_version))
s = \stutter(\now());
<translation code block before _(begin_update)>
<validate obs>
s = \now();
<translation of code block after _(begin_update)>
foreach (\object o \in obs not marked _(read_only))
  _(assert \inv2s(o, s,\now()))
\end{VCC}


\subsubsection{Ghost Atomic Actions}
\vcc{_(ghost_atomic obs block)} (statement)\\
Here, \vcc{obs} is a closed object list and \vcc{block} is a code
block. The block is considered to be in a ghost context. 
Operationally, this executes \vcc{block} atomically.
\\\\
The ghost block is subject to the following restructions (beyond those
normally in effect for ghost code). Only pure function calls are
allowed, and all such calls must occur before the first state update
(i.e., an update not to a purely local variable). 
\vcc{block} cannot include atomic actions, ghost atomic
actions, \vcc{_(wrap)} or \vcc{_(unwrap)}. Each update object
must be \vcc{\mutable} or in the write set of the closed object
list \vcc{obs}. 
\\\\
Ghost atomic actions do not introduce an implicit scheduler
boundary, unlike non-ghost atomic actions. This is because the
scheduling of ghost actions can be viewed as angelic, rather than
demonic. 

\subsubsection{Atomic Blocks}

\vcc{_(atomic obs) <statement>} (statement)\\
\vcc{_(begin_update)}\\
\vcc{obs} is a closed object list. If the statement following
the \vcc{_(atomic)} is not a block, or if it does not include
an occurrance of \vcc{_(begin_update)}, the annotation says that the
following statement is executed atomically. Otherwise, it says that
that part of the block that follows the \vcc{_(begin_update)} is to be
executed atomically. 
\\\\
An atomic block can contain at most one occurrance of \vcc{_(begin_update)};
if it does occur, it must occur as a top-level statement of the block
(i.e., not inside a conditional or loop). 
\\\\
Before the \vcc{_(begin_update)}, the only operations allowed
are \vcc{_(wrap)}, \vcc{_(unwrap)}, memory operations on mutable data,
\vcc{_(ghost_atomic)} operations. 
\\\\
Inside an atomic block, on any atomic access to a field of an object,
VCC asserts that the object is either \vcc{\mutable} or is one of the
read objects of the validation. If the access is a write, it asserts
that it is one of the write objects of the validation.
\\\\
If the body contains more than one memory access of a concrete object
other than stack variables, at at least one such access along with a
call to a \vcc{_(atomic_inline)} function, VCC will warn that the block might
not appear atomic. Beyond this, it is up to the user to make sure that the
compiler and hardware platform on which the program executes guarantees that 
the accesses to volatile concrete memory within the
atomic action in a way that appears to be atomic. 
\\\\
In addition to \vcc{_(atomic_inline)} functions, the atomic block can
include calls to \vcc{_:pure} functions. (Note, however, that such a
function call after a write is likely to fail to establish the
precondition that the state is \vcc{full_stop()}.) \vcc{_(wrap)}
and \vcc{_(unwrap)} cannot be used after the \vcc{_(begin_update)}.

\subsubsection{Atomic Operations}

\noindent\vcc{_(atomic_op obs, e)} (cast)\\
This cast cannot occur in a pure context or within an atomic action.
It has the effect of turning the evaluation of the following
expression \vcc{exp} into approximately the following:
\begin{VCC}
_(atomic obs) {
  T \result = exp;
  e;
}
\end{VCC}
with \vcc{\result} being the result returned from the execution\footnote{
It is unfortunate that \vcc{e} can only be an expression, rather
than an arbitrary code block; this restriction is in place because C
does not allow code blocks to appear inside expressions.}.

\subsubsubsection{Atomic Reads}

\vcc{_(a_atomic_read obs)} (cast)\\
An atomic read acts just like an atomic action, with all of its
objects implicitly marked \vcc{_(read_only)}, except that it applies
to the implicit sequentialization of the compuation of the term to
which it is applied.

\section{Verifying Functions}
For this section, consider a fixed function body, where all local variables
of the function have been lifted to topmost scope. Define
a \Def{predicate} to be a boolean expression whose free variables
are local variables of the function or parameters of the
function. Define a \Def{label function} to be a function from program
labels to predicates. Given two label functions \vcc{l1} and \vcc{l2},
define \vcc{or(l1,l2)(l) == l1(l) || l2(l)}. 
\\\\
Let \vcc{S} and \vcc{S'} be two states, let \vcc{w} be a pointer
set, and let \vcc{t} be a thread. Define
\begin{VCC}
agree(S,S',E) == (\at(S,E) == \at(S',E))

differOnlyAt(S,S',w,t) <==>
\forall \object o; !(o \in w) && \non_primitive_ptr(o) && \at(S,o->\owner == t) 
         ==> (\at(S,o->\closed) ==> agree(S,S',o->\version) 
                                    && agree(S,S',o->\volatile_version))
             && (\forall <field> f; !(&(o->f) \in w) ==> agree(S,S',o->f))

havocOnlyUnowned(S,S',t) <==>
\forall \object o; \non_primitive_ptr(o) && \at(S,o->\owner == t)
  ==> agree(S,S',o->\version) && \agree(S,S',o->\volatile_version)
\end{VCC}

Each function body is implicitly converted to one of the following
restricted forms, where each statement or block below is also in one
of these restricted forms, and all local variable declarations are at
the outermost level. We define three functions:
\begin{itemize}
\item
\vcc{Oblig(B,P,L)}, where \vcc{B} is a restricted statement, \vcc{P}
is a predicate, and \vcc{L} is a label function. 
This gives a proof obligation sufficient to verify that executing \vcc{B}
from a state satisfying \vcc{P} (or from a jump to label \vcc{l}
satisfying \vcc{L(l)}) generates only legal state transitions and only
returns satisfying the postcondition of the function.
\item
\vcc{Post(B,P,L)}, which gives what is known after normal exit from
\vcc{B} starting from a state satisfying \vcc{P} (or from a
jump to label \vcc{l} satisfying \vcc{L(l)}).
\item
\vcc{Exit(B,P,L)}, which gives a label function which, when applied to a
given label, gives the condition known to hold 
if \vcc{B} exits with a \vcc{goto} to the given label.
\end{itemize}

Succesful verification that the function meets its specification
follows from \vcc{Oblig(B,P,(\lambda l; \false))}, where \vcc{B} is the body of the
function (not including the local variable declarations) \vcc{P} 
is the precondition of the function, conjoined with the condition that 
all object invariants hold.
We define these functions as follows. In each case, \vcc{Exit(B,P,L)}
is \vcc{L} unless defined otherwise.
\begin{itemize}
\item
\begin{VCC}
Oblig(B;C, P, L) == 
  (Oblig(B,P,L) && Oblig(C,Post(B,P,L), or(L,Exit(B,P,L))))
Post(B;C, P, L) ==
  Post(C,Post(B,P,L), or(L,Exit(B,P,L)))
Exit(B;C,P,L) ==
  Exit(C,Post(B,P,L), or(L,Exit(B,P,L)))
\end{VCC}

\item
If \vcc{B} is \vcc{x = e}, where \vcc{e} is the application of a single
operator (other than \vcc{++} or \vcc{--}) to a
suitable number of purely local variables other than \vcc{x}, or 
application of \vcc{&} to a local variable, then
\begin{VCC}
Oblig(x = y op z,P) == (P ==> <y op z doesn't overflow or underflow>)
Oblig(x = *y,P) == 
  (P ==> \non_primitive_ptr(y) ? \mutable(y) : \mutable(\embedding(y))
Post(B,P,L) == ((\exists x; P) && x == e)
\end{VCC}

\item
If \vcc{B} is \vcc{x = f(<args>)} where \vcc{args} is  a list of purely local
variables not including \vcc{x}, and \vcc{f} has the contract
\vcc{_(requires pre) _(writes w) _(ensures post)}:
\begin{VCC}
Oblig(B,P,L) == 
  <pre after parameter substitution>
  && (\forall \object o; o \in w ==> \writable(o))
Post(B,P,L) == 
  \exists \state S; \at(s,\me) == \me && differOnlyAt(s,\now(),w,\me) && post
\end{VCC}

\item
If \vcc{B} is \vcc{l:} (declaring a label), then
\begin{VCC}
Oblig(B,P,L) == \true
Post(B,P,L) == P || L(l)
Exit(B,P,L) == \lambda \label x; x == l ? \false : L(x)
\end{VCC}

\item 
If \vcc{B} is \vcc{goto l} where \vcc{l} is a label that occurs syntactically
later in the function body, then
\begin{VCC}
Oblig(B,P,L) == \true;
Post(B,P,L) == \false;
Exit(B,P,L) == \lambda \label x; x == l ? P : l(x)
\end{VCC}

\item
If \vcc{B} is \vcc@if (v) C@, 
where \vcc{v} is a purely local variable and \vcc{C} is a simplified
block, then
\begin{VCC}
Oblig(B,P,L) == Oblig(C,P && v, L)
Post(B,P,L) == v ? Post(C,P && v, L) : P
Exit(B,P,L) == \lambda \label l; v ? Exit(C,P && v, L)(l) : L(l)
\end{VCC}

\item
If \vcc{B} is \vcc@while (v) _(invariant p) _(writes s) C @
where \vcc{v} is a purely local variable and \vcc{C} is a simplified
block, then
\begin{VCC}
q == \exists \state S; \at(s,\me) == \me && \at(s, P) && p 
                       && differOnlyAt(S,\now(),s,\me)
Oblig(B,P,L) == [P ==> p] 
     && Oblig(C,q,\lambda \label l; \false)
     && [Post(C,q,\lambda \label l; \false) ==> p]
Post(B,P,L) == q && !v
Exit(B,P,L) == or(L,Exit(C,q,\lambda \label l; \false))
\end{VCC}

\item If \vcc{B} is \vcc{_(assert p)}, then
\begin{VCC}
Oblig(B,P,L) == [P ==> p]
Post(B,P,L) == P && p
\end{VCC}

\item Uf \vcc{B} is \vcc{_(assume p)}, then
\begin{VCC}
Oblig(B,P,L) == \true
Post(B,P,L) == P && p
\end{VCC}
\end{itemize}

Conversion into this normalized form can involve changing the
semantics of the C program, because of C's liberality in order of
evaluation. Additional assertions are added to allow the program to be
converted into such a form, e.g. that the order in which parameters to
a function are evaluated doesn't matter.


\section{Utility Functions}

\begin{VCC}
_(bump_volatile_version o)} (statement)
{
  \natural \n;
  _(assume n > o->\volatile_version)    
  _(assert \wrapped(o) ==> \writable(o))
  _(assert o->\owner == \me)
  o->\volatile_version = n;
}
\end{VCC}
This is the only operation that is allowed to update \vcc{o->\volatile_version}
while  \vcc{o->\closed} \footnote{Caveat: Volatile versions 
should be exposed directly.}. 
\\\\
\vcc{\bool \fresh(\object o)}\\
This predicate can appear only in a thread context or in a function or block
\vcc{_(ensures)} annotation; it means that the object was not owned by this thread
when the function was called.
\\\\
\vcc{\bool \nested(\object o)}\\
True iff \vcc{o->\owner} is not a thread.
\\\\
\vcc{\objset \span(\object o)}\\
The set consisting of \vcc{o}, along with a pointer to each primative field of \vcc{o}.
\\\\
\vcc{\objset \extent(\object o)}\\
This returns \vcc{\span(o)}, along with \vcc{\extent} of any struct fields of \vcc{o}.
\\\\
\vcc{\objset \full_extent(\object)}\\
The set consisting of \vcc{\extent(o)}, unioned with the \vcc{\full_extent} of any members of any union field of \vcc{o}.
\\\\
\vcc{\bool \extent_mutable(\object o)}\\
Translates to \vcc{\forall \object o1; o1 \in \extent(o) ==> \mutable(o)}.
\\\\
\vcc{\bool \extent_zero(\object)}\\
True if every all bytes in the value fields of \vcc{o} or in padding between fields of \vcc{o} are \vcc{0}.
\\\\
\vcc{\bool \extent_fresh(\object)}\\
True if every object in the extent of \vcc{\extent(o)} is \vcc{\fresh}.
\\\\
\vcc{\bool \thread_local(\object o)}\\
True iff it is in the \vcc{\domain} of some object 
\vcc{o1} that is \vcc{\wrapped} or \vcc{\mutable}.
\\\\
\vcc{\bool \thread_local_array(\object o, size_t s)}\\
True iff \vcc{\forall size_t i; i < s ==> \thread_local(o + s)}.
\\\\
\vcc{\bool \mutable(\object o)}\\
True iff \vcc{o->\closed && o->\owner == \me}.
\\\\
\vcc{\bool \mutable_array(\object o, size_t s)}\\
True iff \vcc{\forall size_t i; i < s ==> \mutable(o + s)}.
\\\\
\vcc{\bool \in_array(\object o, \object o1, size_t s))}\\
True iff \vcc{\exists size_t i; i < s && o == o1 + s}. 
\\\\
\vcc{\objset \array_range(\object o, size_t s)}\\
The set of all objects \vcc{o+i} where \vcc{size_t i < s}.
\\\\
\vcc{\objset \domain(\object o)}\\
\vcc{_(\objset \vdomain(\object o))}\\
VCC includes the following axioms:

\begin{VCC}
  _(axiom \forall \object o; \forall \state s; {\at(s, o \in \domain(o))}
            o->\closed && !\nested(o) && \non_primitive_ptr(o) 
            ==> \at(s, o \in \domain(o)))

  _(axiom \forall \object o,x,y; \forall \state s; 
            {\at(s, x \in \domain(o)), \at(s,y \in \domain(o))}
            \at(s, x \in \domain(o)) && \at(s, y->\owner == x)
            && <x is of a type that is not declared as _(volatile_owns)>
            ==> \at(s, y \in \domain(o)))

  _(axiom \forall \object o,x,y; \forall \state s;
            {\at(s, x \in \domain(o)), \at(s,y \in \vdomain(o))}
            (\at(s,x \in \domain(o)) && \at(s, y->\owner == x)
              && (\forall \state t; \at(s,x->\version) == \at(t,x->\version)
                                  ==> \at(t, y->\owner == x)))
            ==> y \in \vdomain{o} && y \in \domain(o))

  _(axiom \forall \object o,x; \forall \state s,t; 
             {\at(s,x \in \domain(o))}
             \at(s, x \in \domain(o)) && \at(t,o->version) == \at(s,o->version) 
             ==> \at(t,x->version) == \at(s,x->version)
\end{VCC}

\vcc{T ^\alloc_array<T>(\size_t s))}\\
\vcc{T ^\alloc<T>()}\\
Here, \vcc{T} is a type name. \vcc{\alloc<T>()} allocates a fresh object
of type \vcc{T} on the ghost heap, and returns a pointer to the new
object. \vcc{\alloc_array<T>(s)} allocates an array of \vcc{s} such
ghost objects.
\\\\
\vcc{_(bool \not_shared(\object o))}\\
True iff \vcc{o->\claim_cnt == 0} or if \vcc{o} is of an unclaimable type.
\\\\
\vcc{\bool \malloc_root(\object o))}\\
True iff \vcc{o} is the root object of a memory allocation.
\\
\vcc{\bool \object_root(\object o))}\\
True iff \vcc{o} was directly created from a blob. In particular,
if \vcc{o} is a stack variable, \vcc{\object_root(&o)}. 
\\
\vcc{\bool \union_active(\object o)}\\
True if \vcc{o} is the currently active member of a valid union, i.e.,
if \vcc{o} is of union type, 
\vcc{\valid(o) ==> (\valid(&o->f) <==> \union_active(o->f))}
\\\\
\vcc{\bool \addr_eq(\object o1,\object o2))}\\
True iff \vcc{\addr(o1) == \addr(o2)}.
\\\\
\vcc{\bool \arrays_disjoint(\object o1, size_t s1, \object o2, \size_t s2)}\\
\vcc{o1} and \vcc{o2} must be arrays with the same basetype.
True iff \vcc{(\forall size_t i1,i2; i1 < s1 && i2 < s2 ==> o1[i1] != o2[i2])}.
\\\\
\vcc{\bool \wrapped_with_deep_domain(\object o)}\\
This is semantically equivalent to \vcc{\wrapped(o)}. However, it
causes VCC to assert that, for any object \vcc{o1} in the sequential
domain of \vcc{o}, \vcc{o1 \in \domain(o)}.
\\\\
\vcc{\object \domain_root(\object o)}\\
If there is an \vcc{\object o1} such that \vcc{o} is transitively
owned by \vcc{o1}, \vcc{o1} is not a thread, and \vcc{o1->\owner} is a
thread, then this function returns \vcc{o1}. Otherwise, it returns an
arbitrary object.
\\\\
\vcc{\integer \index_within(\object o1,\object o2)}\\
Semantically equivalent to \vcc{\addr(o1) - \addr(o2) / sizeof(\type(o2))}, but
uses a more efficient representation internally.
\\\\
\vcc{\bool \writable(\object o)}\\
This predicate is allowed only in function bodies, not in function
contracts. Let \vcc{w} be the set of objects in the writes clause of the 
function (instantiated according to the call of the function). 
Then 
\begin{VCC}
\writable(o) <==>
  (\non_primitive_ptr(o) && (o \in w || (o->\owner == \me && 
    <o->\valid or o->\closed changed after the current function was called>)))
  || (!\non_primitive_ptr(o) && (o \in w || \writable(\embedding(o))))  
\end{VCC}

\noindent \vcc{\natural \size(e)}\\
\vcc{e} must be an expression of integral or \vcc{record}
type, all of whose fields are of such type. If \vcc{e} is of integral
type, this returns the absolute value of \vcc{e}. If \vcc{e} is a
record type, this returns 1 plus the sum of the sizes of the fields
of \vcc{e}. 
\\\\
\vcc{\bool \shallow_eq(T s, T t)}\\
\vcc{s} and \vcc{t} must be of the same structured type.
This returns \vcc{\true} iff \vcc{s} and \vcc{t} have the same
values for all explicit fields of non-structured type.
\\\\
\vcc{\bool \deep_eq(T s, T t)}\\
\vcc{s} and \vcc{t} must be of the same structured type. 
This returns \vcc{\true} iff \vcc{\shallow_eq(s,t)} and, for each explicit
structured field \vcc{f} of type \vcc{T}, 
\\\\
\vcc{\bool \wrapped0(\object o)}
Equivalent to \vcc{\wrapped(o) && o->\claim_count == 0)}.
\\\\
\vcc{T \old(T e)}\\
Within an \vcc{\ensures} clause or the body of a
function, \vcc{\old(e)} gives the
value of \vcc{e} at function entry, after replacing any ocurrances of
local variables in \vcc{e} with their current values.
Within an object invariant, this expands to the prestate of the
transition over which the invariant is evaluated. 
\\
\vcc{\bool \unchanged(e)}\\
\vcc{\bool \same(e)}\\
Equivalent to \vcc{(\old(e) == e)}. Note that \vcc{\old} has
different meanings depending on context.\\

\section{Macros}
A new annotation tag can be defined with an annotation of the form
\begin{VCC}
_(\bool \macro_<Name>(args) {
  <body>
}
\end{VCC}
where \vcc{<Name>} is the name of the new annotation tag,
and \vcc{<args>} gives the remaining arguments that occur within the
annotation. This defines the annotation
\vcc{_(<Name> args)} to expand to \vcc{<body>}.

\section{Inference}

If \vcc{\wrapped(e)} is a top-level conjunct of a precondition of a
function, then  \vcc{_(requires e _(assume) \in \domain(e))} is implicitly added
to the function contract. If \vcc{e} is of \vcc{\claim}
type, then \vcc{_(requires _(assume) \valid_claim(e))} is implicitly added to
the contract.
\\\\
If \vcc{\claims(c, \closed(o))} is a top-level conjunct of a
precondition of a function, then the
preconditions \vcc{_(requires _(assume) \always_by_claim(c,o))}
and \vcc{_(requires _(assume) \inv(o)} are added to the
function.
\\\\
\vcc{_(isolate_proof)} (function attribute)\\
This causes VCC to verify this function in a prover session that is
isolated from the rest of the verification session. (That is, the
function should verify if and only if it was the only function being verified.)
\\\\
\vcc{_(frameaxiom)} (specifier)\\
This forms can only be used as a specifier for a  \vcc{pure} function
with a \vcc{reads} clause. It tells VCC to generate an axiom that says
that for any two states that agree on the values of the objects in the
\vcc{_(reads)} clause of the function, the funciton returns the same
value.
\\\\
\vcc{_(no_frameaxiom)}\\
This annotation can appear only on a \vcc{_:pure} function with
a \vcc{_(reads)} clause. It says not to generate a frame axiom for the
function. 

%% #define vcc_force_splits(n) \
%%   __declspec(System.Diagnostics.Contracts.CodeContract.IntBoogieAttr, "vcs_max_splits", n) \
%%   __declspec(System.Diagnostics.Contracts.CodeContract.IntBoogieAttr, "vcs_max_cost", 1)
%% #define vcc_keep_going(n) \
%%   __declspec(System.Diagnostics.Contracts.CodeContract.IntBoogieAttr, "vcs_max_keep_going_splits", n)
%% #endif


\subsection{Triggering}
\label{sect::triggers}
VCC uses an SMT solver as its reasoning backend. SMT solvers typically
work at the level of ground terms, and handle quantifiers by means of
triggers - patterns that, when matched by terms arising in deduction,
cause instances of quantified formulas to be generated. In each
annotation containing a quantification, VCC allows optionally the
inclusion of explicit triggers. If these are not included, VCC
generates triggers automatically. In most cases, it is best to allow
VCC to generate triggers on its own, since it avoids the possibility
of instantiation loops. However, on rare occasions, you may want to
provide your own. See the section on triggers in the tutorial for more
information. 
\\\\
\vcc{_(bool _:pure \match_long(__int64 ) _(ensures \result == \true);)}\\
\vcc{_(bool _:pure \match_ulong(unsigned __int64) _(ensures \result == \true);)}\\
These functions are declared with the contracts given above. They are provided as a
convenience for writing triggers. 

\subsection{Debugging}
The features of this section are included only to aid in the debugging
of verifications; they should never appear in a final program. Use of
any of these annotation may render verification unsound.
\\\\
\vcc{\bool \start_here()}\\
This predicate can appear only in the
annotation \vcc{_(assume \start_here())}; it tells VCC to not verify
any of the assertions in the current function that lexically precede
this statement. At most one such annotation should appear within any
function. 
\\\\
\vcc{_(assume_correct)} (specifier)\\
This annotation can appear only immediately preceding a function
definition. It tells VCC to not verify this function, i.e. to consider
all of the assertions generated inside the body of the function
definition to be assumptions.
\\\\
\vcc{_(no_reads_check)} (specifier)\\
This  can only be used as an attribute of a \vcc{pure} function
with a \vcc{reads} clause. It causes VCC to not perform the reads
check for the function, i.e. to not check that the function depends
only on the objects listed in the reads clause.
\\\\
\vcc{_(no_admissibility)} (specifier)\\
This annotation can appear only immediately before the definition of
a \vcc{struct}, \vcc{union}, or \vcc{_(record)} type. It tells VCC to
omit checking the admissibility of the following type.

\subsection{Smoke Tests}
Smoke testing (which is not done by default) causes VCC to check that no control
location of the program is provably unreachable (provable in the sense
of VCC's treatment of what is known where). This amounts to checking
that an \vcc{\assert(\false)} added to any location would fail (with
some reasonably small timeout). Because deduction is incomplete, smoke
testing is not sound (though it does not interfere with the soundness
of the proofs of other properties being verified).

An explicit \vcc{_(assert \false)} at the beginning of a block causes
smoke testing to be omitted for that block. 
\\\\
\vcc{_(known \bool val)} (cast)\\
\vcc{val} must be the literal \vcc{\true} or the
literal \vcc{\false}. It asserts that the cast expression has
value \vcc{val}. If the expression is the test of a conditional or loop, or
the first argument to \vcc{||} or \vcc{&&} or \vcc{?:}, it disables
smoke testing for the branch that is known not to be taken.
\\\\
\vcc{_(skip_smoke)} (specifier)\\
This annotation causes VCC to skip smoke testing on the following function.

\subsection{Verification Switches}
VCC has a number of parameters that effect how verification is
performed. These can be adjusted using arguments to the statement used
to run VCC (either on the command line or through the Visual Studio
pluggin). 

Some of these can be overridden for individual functions by
means of attributes put on individual function definitions. These
attributes are currently expressed with the special top-level form 
\vcc{vcc_attr(<property>, <value>)} just before a function
definition. Here, \vcc{<property>} is a string giving the name
of the property and \vcc{<value>} is a string giving the value for
that property. 


\section{Caveats}
VCC is intended to be \emph{sound}, which is to say that if VCC
verifies your program without giving any errors or warnings, the
program is correct (the meaning of correctness is defined in section
??). It is not intended to be complete, which is to say that there are
programs that are correct that you will not be able to verify with
VCC, perhaps even some programs that might be verified with other
tools. There are also some programs that require minor syntactic 
changes to be verified; while we can (and have) extend the annotation
syntax to handle such cases, such activity eventually meets with
diminishing returns.

When reasoning about concurrency, VCC assumes that programs are
running under the standard model of shared variable concurrency,
namely sequentially consistent memory. Most architectures, including
x86 and ARM, provide somewhat weaker memory models. VCC as described
here is not sound for these models in general. However, it is sound if
further conditions on the program are met. One simple restriction that
suffices (for x86 and other TSO architectures) is that all writes are
interlocked (i.e., all volatile atomic updates flush the store
buffers). VCC will hopefully support checking weaker conditions in the
future.


\end{document}


To modularize the verification of the functions, each function has a
specification that defines its preconditions (what the function
expects when it is called), postconditions (what it guarantees when it
returns), framing (what parts of the state it is allowed to modify),
and possibly termination guarantees. When verifying code, each
function call is replaced with the most general state change that
satisfies the specification of the called function.  Similar sorts of
specifications can be put on code blocks, and blocks so annotated are
treated much like functions.  Loops are given similar contracts (but
using loop invariants in place of preconditions and postconditions),
and verified by proving that the invariant holds on loop entry and is
maintained by an arbitrary iteration of the loop. An atomic action is
translated into a scheduler boundary (where the thread ``forgets''
everything about objects it doesn't own), followed by the atomic
action (which is checked for legality). In this way, verification of
concurrent programs is reduced to the verification of branching
programs without loops or function calls.

\subsection{Modular Verification}

A program consists of a number of verification units, some part of
which constitutes its public specification (spec). The spec of
unit \vcc{a} depends on the spec of unit \vcc{b} iff \vcc{a}
and \vcc{b} are the same unit or \vcc{a} mentions some unit that
depends on \vcc{b}. Each unit \vcc{a} is verified using its own text
and the specs of those units on which units mentioned by \vcc{a}
depend\
\footnote{Caveat: VCC currently violates this
modularity condition, by making use of the program call graph to
detect cycles when verifying program termination.}
.

