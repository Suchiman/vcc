\section{Loop invariants}

Loops in VCC usually need to be annotated with invariants.
A loop invariant is a condition that needs to hold before
the loop starts executing, and then if the invariant holds
at the beginning of the loop, it should also hold at the
end of the loop.
Moreover the invariant is the only thing the verifier ``knows''
at the beginning of a loop. 
Let's then have a look at an example.


\vccInput[]{c/02_div.c}

\noindent
The \vcc{mod()} function computes a reminder of a integer division
using iterated subtraction.
\todo{do we need to claim it makes sense?}
The invariant is that the \vcc{res % a} stays unchanged,
and equal to \vcc{a % b}.
To make the invariant semantics clearer below
you can find the translation in terms of asserts
and assumes.

\vccInput[]{c/02_div_assert.c}

\noindent
We can see that the invariant is first checked in the state before the
loop iteration.
Then we simulate an arbitrary loop iteration.
Because the loop assigns to the variable \vcc{res},
VCC doesn't know what value it might have.
Thus it conservatively assigns it an arbitrary value.
Should the loop assign more variables, they would all be assigned
arbitrary values.
Then it assumes the invariant, this places some constraints on
the \vcc{res}.
What follows is the loop body, where the \vcc{break} statement
has been replaced by a \vcc{goto}.
Then VCC checks the invariant after the arbitrary iteration.
Finally, the iteration ends with an assumption that will never be satisfied.
The idea is that we are verifying an arbitrary loop iteration,
so verifying just one is enough, we do not need to have a jump
at the end of the loop.
On the other hand, the only way to exit the loop is through the
\vcc{break} statement, and this is why we assume the control never falls over the end of the loop.
Thus, for the verifier, the only way to reach the \vcc{return} statement and 
the associated assertion is through the \vcc{goto}. However, the \vcc{goto}
only happens after the invariant was assumed and from the \vcc{if} statement
we also know \vcc{res < b}.
This is enough to show the postcondition assertion.

If the loop was of the form:
\begin{VCC}
while (res >= b) 
  spec( invariant a % b == res % b )
{
  res -= b;
}
\end{VCC}
the assert/assume translation would be exactly the same: the invariant would be assumed before
checking the guard.

\subsection{Arrays}

Sometimes invariants are not required, in particular when the loop
doesn't modify any interesting locations. For example, consider
the following function looking linearly for an integer in an array.

\vccInput{c/03_lsearch.c}

\noindent
The requirement \itodo{decide what do we put there}
makes sure the array is readable.
The \vcc{P ==> Q} formally means \vcc{!P || Q}, and is usually 
read as ``\vcc{P} implies \vcc{Q}''.
In particular note that \vcc{\false ==> P} is true for any \vcc{P}.
However this specification is not full, what about the case when the result is not found?
To express that an element is not in an array we will use a universal quantifier:

\vccInput{c/03_lsearch_full.c}

\noindent
The syntax \vcc{\forall unsigned j; j < i; ar[j] != elt} means that 
\vcc{ar[j] != elt} for all \vcc{j} between \vcc{0} and \vcc{i - 1}.
The \vcc{0} bound comes from the fact that we're quantifying over an unsigned
integer.
Thus, the invariant says that the element was not found between \vcc{0}
and \vcc{i - 1}.
If the loop terminates with \vcc{i == sz}, then the postcondition follows.
