\section{Loop invariants}

Loops in VCC usually need to be annotated with invariants.
A loop invariant is a condition that needs to hold before
the loop starts executing, and then if the invariant holds
at the beginning of the loop, it should also hold at the
end of the loop.
Moreover the invariant is the only thing the verifier ``knows''
at the beginning of a loop. 
More precisely the way VCC reasons about the loop is that
it assigns an arbitrary value to all the variables that the loop body might modify,
assumes the invariant

\vccInput[]{c/02_sqrt_ovf.c}

\noindent
The \vcc{isqrt()} function finds an integer approximation of a square root.
The loop guard makes sure we do not exceed the upper bound on the squere.
The invariant makes sure we did not yet exceeded the lower bound.

