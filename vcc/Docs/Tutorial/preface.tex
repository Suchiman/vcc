\section{Introduction}
This tutorial is an introduction to verifying C code with VCC. Our
primary audience is programmers who want to write correct code.
The only prerequisite is a working knowledge of C.

To use VCC, you first \emph{annotate} your code to specify 
what your code does (e.g., that your sorting function sorts its input),
and why it works (e.g., suitable invariants for your loops and 
data structures).
VCC then tries to \emph{prove} (mathematically) that
your program meets these specifications.  Unlike most program
analyzers, VCC doesn't look for bugs, or analyze an abstraction of
your program; if VCC certifies that your program is correct, then your
program really is correct%
\footnote{
  In reality, this isn't necessarily true, for two reasons. 
  First, VCC itself might have bugs; in practice, these are unlikely to cause you to accidentally verify an
  incorrect program, unless you find and intentionally exploit such a
  bug. Second, there are a few checks needed for soundness that haven't yet
  been added to VCC, such as checking that ghost code terminates;
  these issues are listed in section \secref{caveats}.
  }. 

To check your program, VCC uses the \emph{deductive verification} paradigm:
it generates a number of mathematical
statements (called \Def{verification conditions}), the validity of
which suffice to guarantee the program's correctness, and tries to
prove these statements using an automatic theorem prover. If any of
these proofs fail, VCC reflects these failures back to you in terms of
the program itself (as opposed to the formulas seen by the theorem
prover). Thus, you normally interact with VCC entirely at the level of
code and program states; you can usually ignore the mathematical
reasoning going on ``under the hood''.
For example, if your program uses division
somewhere, and VCC is unable to prove that the divisor is nonzero, it
will report this to you as a (potential) program error at that point in the
program. This doesn't mean that your program is necessarily incorrect;
most of the time, especially when verifying code that is already well-tested,
it is because you haven't provided enough information to allow VCC
to deduce that the suspected error doesn't occur.
(For example, you might have failed to specify that some function
parameter is required to be nonzero.)
Typically, you fix this ``error'' by strengthening your
annotations. This might lead to other error reports, forcing you to 
add additional annotations, so verification is in
practice an iterative process.  
Sometimes, this process will reveal a genuine programming error.
But even if it doesn't, you will have not only proved your code free from 
such errors, but you will have produced the precise specifications for your
code -- a very useful form of documentation.

This tutorial covers basics of VCC annotation language. By the time
you have finished working through it, you should be able to use VCC to
verify some nontrivial programs. It doesn't cover the theoretical
background of VCC \cite{lci}, implementation details \cite{vcc}
or advanced topics;
information on these can be found on the VCC
homepage\footnote{\url{http://vcc.codeplex.com/}}.
The examples in this tutorial are currently distributed with the VCC sources.%
\footnote{
Available from \url{http://vcc.codeplex.com/SourceControl/list/changesets}: click Download on the right,
get the zip file and navigate to \lstinline|vcc/Docs/Tutorial/c|.}

You can use VCC either from the command line or from Visual Studio
2008/2010 (VS); the VS interface offers easy access to different components of
VCC tool chain and is thus generally recommended.
VCC can be downloaded from the VCC homepage; 
be sure to read the installation instructions\footnote{\url{http://vcc.codeplex.com/wikipage?title=Install}},
which provide important information about installation prerequisites 
and how to set up tool paths.

\subsection{A Note on VCC Invocation}
%\todo{eliminate this section and move it into the instructions for running VCC.}
\label{sect:todo-invoke}
This tutorial assumes you're using VCC version 3. 
Currently, it is necessary to supply the flag \texttt{/3}
(or enable V3 language in Visual Studio verification settings)
to VCC to enable this.
This will soon be the default.
% -3 now implies -2 and -it. --MM

