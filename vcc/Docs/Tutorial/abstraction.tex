\section{Ghosts}
\label{sect:ghosts}
Usually there are many ways of implementing a given data structure.
For example, a set might be implemented as a linked list, an array, or a hash table.

When reasoning about a program which uses a data structure we don't want to be
concerned with implementation details of the data structure.
We should reason at a somewhat higher, abstract level.
For example, when we use a linked list as a representation of a set, we should not be concerned
with how the list nodes are laid out in memory.
You would then need to include \Def{ghost data} to store this set and write small bits of \Def{ghost code} to
update it.  
Ghost code is seen by VCC but not by the C compiler, and
so introduces no runtime overhead. Part of the VCC philosophy is that
programmers would rather do extra programming than drive interactive
theorem provers, so ghost code is the preferred way to help VCC
understand why your program works.  
%Thus, you interact with VCC entirely at the level of code and program states;
%usually, you can safely ignore the mathematical reasoning going on ``under the
%hood''.

In fact we have already used ghost data, which VCC introduced: \vcc{\closed},
\vcc{\owner}, and friends are all ghost fields, not seen by the C compiler.
The \vcc{_(wrap ...)} and \vcc{_(unwrap ...)} operations are ghost code.
In this section we're just going to introduce some ghost data and code ourselves.

It is only rarely the case that a simple C type, say \vcc{unsigned int}, would be suitable
to store such a data-structure abstraction (how would one store an unbounded set in a primitive C type?).
To that end, VCC provides \Def{map types}.
The syntax is similar to syntax of array types, \vcc{int m[T*]} defines a map \vcc{m} from \vcc{T*}
to \vcc{int}.
That is the type of expression \vcc{m[p]} is \vcc{int}, provided that \vcc{p} is a pointer
of type \vcc{T*}.
A map \vcc{T* a[unsigned]} is similar to an array of pointers of length $2^{32}$.%
\footnote{
  Because a map can be used only in ghost code, 
  the issue of runtime memory consumption does not apply to it.}
A map \vcc{bool s[int]} can be thought of as a set of \vcc{int}s: the operation
\vcc{s[k]} will return true if and only if the element \vcc{k} is in the set \vcc{s}.

Let's then have a look at an example of a list abstracted as a set:

\vccInput[linerange={types-init}]{c/07_list0.c}

\noindent
The invariant states that:
\begin{itemize}
\item the list owns the head node (if it's non-null)
\item if the list owns a node, it also owns the next node (provided it's non-null)
\item if the list owns a node, then its data is present in the set \vcc{val};
      this binds the values stored in the implementation to the abstract representation
\end{itemize}
You may note that the set \vcc{val} is under-specified: 
it might be that it has some elements not stored in the list.
We'll get back to this issue later.
Now let's have a look at the specification of a function adding a
node to the list:

\vccInput[linerange={add-endspec}]{c/07_list0.c}

\noindent
The writes-clause and contracts about the list being wrapped are similar to what
we've seen before.
Then, there are the contracts talking about the result value.
This function might fail because there is not enough memory to allocate list node,
in such case it will return a non-zero value (an error code perhaps),
and the contracts guarantee that the set represented by the list will not be changed.
However, if the function succeeds (and thus returns zero), the contract specifies
that if we take an arbitrary integer \vcc{p}, then it is a member of the new abstract
value if and only if it was already a member before or it is \vcc{k}.

In other words, the new value of \vcc{l->val} will be the union of the old
abstract value and the element \vcc{k}.
Ideally, this contract is all that the caller will need to know about that function:
what kind of effect does it have on the abstract state. 
It doesn't specify if the node will be appended at the beginning, or in the middle
of the list.
It doesn't talk about possible duplicates or memory allocation.
Everything about implementation is completely abstracted away.
Still, we need a concrete implementation, and here it goes:

\vccInput[linerange={endspec-member}]{c/07_list0.c}

\noindent
We allocate the node, unwrap the list, initialize the new node
and wrap it (we want to wrap the list, and thus everything it is going to own will
need to be wrapped beforehand; the list is wrapped at the end of the 
unwrapping block), and
prepend the node at the beginning of the list.
Then we update the owns set (we've also already seen that).
Finally, we update the abstract value using a \Def{lambda expression}.
%\subsection{Lambda expressions}
The expression \vcc{\lambda T x; E} returns a map, which for
any \vcc{x} returns the value of expression \vcc{E},
which can reference \vcc{x}.
If the type of \vcc{E} is \vcc{S}, then the type of map, returned by the
lambda expression, is \vcc{S[T]}.
An assignment \vcc{m = \lambda T x; E} has a similar effect to
the following assumption (note that \vcc{E} will most likely reference \vcc{x}):
\begin{VCC}
_(assume \forall T x; m[x] == (E))
\end{VCC}
\noindent
Unlike assumptions, lambda expressions do not compromise 
soundness of the verifier.
Just like for assumptions,
the expression is always evaluated in the state as it was
when the lambda was first defined, for example:
\begin{VCC}
int x = 1;
int m[int] = \lambda int y; y + x;
_(assert m[0] == 1) // succeeds
x = 2;
_(assert m[0] == 1) // still succeeds
\end{VCC}
One can imagine, that when this lambda expression is defined,
VCC will iterate over all possible values of \vcc{x},
and store the value of \vcc{E} in \vcc{m[x]}.
Lambda expressions are much like function values in functional
languages or delegates in C\#.

\begin{note}
The body of our lambda expression shows similarity to the body
of the quantifier we have used in specification.
It doesn't, however, have to be the same:
\begin{VCC}
int[int] foo(int v)
  _(ensures \forall int x; x >= 7 ==> \result[x] >= v)
{
  return \lambda int y; (y&1) == 0 ? INTMAX : v;
}
\end{VCC}
Thus, the specifications for lambda expressions can hide information.
\end{note}


\subsection{Expressing reachability}

\begin{note}
The following subsection, till the beginning of \secref{concurrency} (which is
about concurrency), might be somewhat difficult upon first reading of this
tutorial.
It is not required to understand \secref{concurrency}.
\end{note}

At minimum the list should support adding elements and checking
for membership. For example, we would expect:

\vccInput[linerange={member-endspec}]{c/07_list.c}

\noindent
Our current list invariant is strong enough only to show
\vcc{\result != 0 ==> l->val[k]}, because it only says
that if the list owns something, then it's in the \vcc{val}.
It also says that if something can be reached by following
the \vcc{next} field from the \vcc{head}, then it is owned.
What we want to additionally say is that if something is in the \vcc{val}
set, then it can be reached from the \vcc{head}.
Unfortunately, such property is not directly expressible in first-order
logic (which is the underlying logic of VCC specifications).
To work around this problem we associate with each node
the set of values stored in all the following nodes and
the current node.
Additionally we say that the set for \vcc|NULL| node is empty.
This way, as we walk down the \vcc|next| pointers we can keep
track of all the elements that can be still reached.
Once we reach the \vcc|NULL| pointer, we know that nothing
more can be reached.
The set of reachable nodes are stored as maps from
\vcc|int| to \vcc|bool|.
We need one such map per each node, so we just
put a ghost map from \vcc|struct Node*| to the sets.
Alternatively, we could store these sets as a field inside of each node,
but maps gives more flexibility in updating it using lambda expressions.

\vccInput[linerange={type-init}]{c/07_list.c}

\noindent
All these changes in the invariant do not affect
the contract of \vcc{add()} function, and the only change in the body
is that we need to replace the update of \vcc{l->val} with the following:

\vccInput[linerange={specupdate-updateend}]{c/07_list.c}

\noindent
That is adding a node at the head only affect the followers set of the new head,
and the followers sets of all the other nodes remain unchanged.
Now let us have a look at the \vcc{member()} function:

\vccInput[linerange={member-out}]{c/07_list.c}

\noindent
The invariants of the \vcc{for} loop state that we only iterate over
nodes owned by the list, and that at each iteration \vcc{k} is in the
set of values represented by the list if and only if it is in the followers
set of the current node.
Both are trivially true for the head of the list, for the first iteration
of the loop.
For each next iteration, the invariant of the list tells us that by following
the \vcc|next| pointer we stay in the owns set.
It also tells us, that the \vcc|followers[n->next]| differs
from \vcc|followers[n]| only by \vcc|n->data|.
Thus, if \vcc|n->data| is not \vcc|val|, then the element,
if it's in \vcc|followers[n]| must be also in \vcc|followers[n->next]|.

\subsection{Sorting revisited}
\label{sect:sorting-perm}

In \secref{sorting} we have considered the bozo-sort algorithm. 
We have verified that the array after it returns is sorted.
But we would also like to know that it's a permutation of the input array.
To do that we will return a ghost map, which states the exact permutation
that the sorting algorithm produced.

\vccInput[linerange={begin-out}]{c/04_bozosort_perm.c}

This sample introduces two new features.
The first is the output ghost parameter \vcc{_(out perm_t perm)}.
We use it when we need a function to return something in addition to what it normally returns.
To call \vcc{bozo_sort()} you need to supply a local variable to hold
the permutation when the function exits, as in:
\begin{VCC}
void f(int *buf, unsigned len)
  // ...
{
  _(ghost perm_t myperm; )
  // ...
  bozo_sort(buf, len _(out myperm));
}
\end{VCC}
The value is only copied on exit of \vcc{bozo_sort()}, though during its execution 
it has its own copy.
It is thus different than passing a pointer to the local.
It is also more efficient for the verifier.

The second, somewhat more advanced, feature is explicit state manipulation.
The function \vcc{\now()} returns the current state of the heap (\ie dynamically
allocated memory; in future it will also work for locals, but for now it only applied
to memory location, address of which was taken). 
The state is encapsulated in a value of type \vcc{\state}.
The expression \vcc{\at(s, E)} returns the value of expression
\vcc{E} as evaluated in state \vcc{s}. 
You can see  \vcc{\old(...)} as a special case of this.

Thus, the algorithm maintains the map containing the current permutation of
the data, with respect to the initial data (we store the initial state in
\vcc{s0}).
The initial permutation is just the identity, and whenever we swap elements of
the array, we also swap elements of the permutation.

\subsection*{Exercises}
\begin{enumerate}
\item
Write and verify an iterative program that sorts an array of ints using
bubblesort. The specification should be the same as for bozo-sort above.
\end{enumerate}

