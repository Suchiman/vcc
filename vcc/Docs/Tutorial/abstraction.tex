\section{Ghosts}
\label{sect:ghosts}

VCC methodology makes heavy use of \Def{ghost} data and code - data
and code that are not seen by the C compiler (and therefore are not
part of the executable), but are included to aid reasoning about the
program. Part of the VCC philosophy is that programmers would rather
writing extra code than have to drive interactive theorem provers, so
ghosts are the preferred way to explain to VCC why your program works.

We have already seen some ghost fields of ordinary data structures
(e.g. \vcc{\closed}, \vcc{\owner}, \vcc{\owns}m \vcc{\valid}, etc.) as
well as built-in pieces of ghost code (e.g. \vcc{_(wrap ...)} and
\vcc{_(unwrap ...)}). This section is about how to write your own
ghost code. 

Our first tyical use of ghost data is to express data abstaction. If
you implement an abstract set as a list, it is good practice to expose
to the client only the set abstraction, while keeping the list
implementation private to the implementation of the data type. One way
to do this is to store the abstract value in a ghost field and update
it explicitly in ghost code when operating on the data
structure\footnote{ An alternative approach is to write a pure ghost
  function that takes a concrete data structure and returns its
  abstract value.  The disadvantage of this approach is that for
  recursive structures like lists, the abstraction function is
  likewise recursive, and so reasoning with it requires substantial
  guidance from the user.}. Functions operating on the data structure
are specified in terms of their effect on the abstract value only; the
connection between the abstract and concrete values is written as an
invariant of the data structure.

VCC's mathematical types are usually better suited to representing these
abstract values than C's built-in types. Here, we will use VCC maps.
Recal that a declaration \vcc{int m[T*]} defines a map \vcc{m} from
\vcc{T*} to \vcc{int}; for any pointer \vcc{p} of type \vcc{T*}
\vcc{m[p]} gives the \vcc{int} to which \vcc{m} maps \vcc{p}.
A map \vcc{bool s[int]} can be thought of as a set of \vcc{int}s: the operation
\vcc{s[k]} will return true if and only if the element \vcc{k} is in the set \vcc{s}.

For example, here is a simple example of a set of \vcc{int}s implemented with an array:
\vccInput[linerange={types-init}]{c/7.0.arraySet.c}

\noindent 
The map \vcc{mem} gives the abstract value of the set. The map
\vcc{idx} says where to find each abstract member of the set. We could
have eliminated \vcc{idx} and instead used existential quantification
to say that every member of the abstract set occurs somewhere in the
list. The disadvantage of using an explicit witness like \vcc{idx} is
that we have to update it appropriately. The advantage is that the
prover doesn't have to guess how to instantiate such existential
quantifiers, which makes the verification more robust.

Here is the initializer for these sets:
\vccInput[linerange={init-mem}]{c/7.0.arraySet.c} 
\noindent
The standard form of a constructor is to take some raw
memory\footnote{
In C, it is normal for constructors to take a pointer to raw memory,
to that they can be used to make objects that are embedded within data
structures. 
}, and
wrap it, while establishing some condition on its abstract value.
Values of maps are constructed using \Def{lambda expressions}.  The
expression \vcc{\lambda T x; E} returns a map, which for any \vcc{x}
returns the value of expression \vcc{E} (which can mention \vcc{x}).
If \vcc{S} is the type of \vcc{E}, then this map is of type
\vcc{S[T]}.

Here is the membership test:
\vccInput[linerange={mem-add}]{c/7.0.arraySet.c} 
\noindent
As usual, an accessor is marked as \vcc{_(pure)}, and reads only the
abstract value of the set. Finally, here is a function that adds a new
element to the set:
\vccInput[linerange={add-del}]{c/7.0.arraySet.c} 
\noindent
Note that in addition to updating the concrete representation, we also
update the abstract value and the witness. This example shows another
way to update a map, using array syntax; if \vcc{m} is a variable of
type \vcc{S[T]}, \vcc{e} is of type \vcc{T}, and \vcc{e2} is of
type \vcc{S},  then the statement \vcc{m[e1] = e2} abbreviates the
statement 
\vcc{m = \lambda T v; v == e1 ? e2 : m[v]}.

\subsection*{Exercises}
\begin{enumerate}
\item Extend \vcc{ArraySet} with a function that deletes a value from
  the set. 
\item Modify \vcc{ArraySet} to keep the set without duplication.
\item Modify \vcc{ArraySet} to keep the elements ordered. Use binary
  search to check for membership and for insertion of a new element.
\item Extend \vcc{ArraySet} with a function that adds the contents of
  one set into another. (Try to calculate the size of the combined list
  before modifying the target, so that you can fail gracefully.)
\end{enumerate}

\subsection{Linked Data Structures}
As an example of a more typical dynamic data structure, consider 
the following alternative implementation of \vcc{int} sets as lists:

\vccInput[linerange={types-init}]{c/7.1.list0.c}

\noindent
The invariant states that:
\begin{itemize}
\item the list owns the head node (if it's non-null)
\item if the list owns a node, it also owns the next node (provided it's non-null)
\item if the list owns a node \vcc{n} then \vcc{n->data} is in \vcc{val};
\item if \vcc{v} is in \vcc{val}, then it is the \vcc{data} for some
  node (\vcc{find[v]}) owned by the list.
\end{itemize}

Note that we have chosen to put all of the list
invariants in the \vcc{List} data structure, rather than in the nodes
themselves (which would also work). A disadvantage of putting all of
the invariants in the \vcc{List} type is that when you modify one of
the nodes, you have to check these invariants for all of the nodes
(although the invariants are easy to discharge for nodes that are not
immediate neighbors). Some advantages of this choice is that it is
easier to modify all of the nodes as a group, and that the same
\vcc{Node} type can be used for data structures with different
invariants (e.g., cyclic lists).

Here is the implementation of the \vcc{add} function:

\vccInput[linerange={endspec-member}]{c/7.1.list0.c}

\noindent
We allocate the node, unwrap the list, initialize the new node and
wrap it, and prepend the node at the beginning of the list.  Then we
update the owns set to include the new node, update the abstract value
\vcc{val} and the witness \vcc{find}, and finally wrap the list up
again (when exiting the \vcc{_(unwrapping)} block). 

The invariants of our list say that the abstract value contains
exactly the \vcc{data} fields of nodes owned by the list. It also said
that pointers from list nodes take you to list nodes. But it doesn't
say that every node of the list can be reached from the first node;
the invariants would hold if, in addition to those nodes, the list
also owned some unrelated cycle of nodes. As a result, the natural
algorithm for checking if a value is in the set (by walking down the
list) won't verify; if it finds a value, its data is guaranteed to be
in the set, but not vice-versa. Moreover, the invariants aren't strong
enough to guarantee that the list itself is acyclic.

Thus, we need to strengthen the invariant of the list to guarantee
that every node of the list is reachable from the head. This cannot be
expressed directly with first-order logic, but there are several ways
to express this using VCC using ghost data:
\begin{itemize}
\item you can keep track of the ``depth'' of each list node (i.e.,
  how far down the list it appears);
\item you can maintain the abstract sequence of list nodes (i.e., a
  map from \vcc{\natural} to nodes, along with a \vcc{\natural} giving
  the length of the sequence);
\item you can maintain the ``reachability'' relationship between
  ordered pairs of nodes.
\end{itemize}

For this example, we'll use the first approach. We add the following 
to the definition of the \vcc{List} type:

\vccInput[linerange={moreList-noMoreList}]{c/7.2.list.c}

The new invariants say that the head (if it exists) is at depth 0 (and
is the only node at depth 0), that depth increases by 1 when following
list pointers, and that you can never ``miss'' a node by following the
list. (These are similar to the invariants you would use if each node
had a key and you were maintaining an ordered list.)

The only change to the code we've previously seen is that when adding
a node to the list, we have to also update the node indices:
\begin{VCC}
l->idx = (\lambda Node *m; m == n ? 0 : l->idx[m] + 1);
\end{VCC}

We can now write and verify the membership test:
\vccInput[linerange={member-out}]{c/7.2.list.c}

\noindent Note that the second invariant of the loop is analogous to the
invariant we would use for linear search in an array.

The \vcc{_(assert)} is an example of a situation where VCC needs a
little bit to see why what you think is true really is true. The loop
invariant says that there are no nodes in the list with key \vcc{k},
but VCC on its own will fail to make the appropriate connection to
\vcc{l->val[k]} via \vcc{l->find[k]} without this hint (which is just
giving an instantiation for the last list invariant).

\subsection*{Exercises}
\begin{enumerate}
\item
Modify the list implementation so that on a successful membership
test, the node that is found is moved to the front of the list. (Note
that the resulting function is no longer pure.)
\item 
Implement sets using sorted lists.
\item
Implement sets using binary search trees.
\end{enumerate}


\subsection{Sorting revisited}
\label{sect:sorting-perm}

In \secref{sorting} we verified that bubblesort returns a sorted array.
But we didn't prove that it returned a permutation of the input
array\footnote{For arrays in which no value occurs more than once,
  this property can be expressed that every value in the output array
  is in the input array. But with multiple occurrances, this would
  require stating that the multiplicity of each value is the same, a
  property that isn't first-order.
}. To express this postcondition, we return a ghost map, which states the
exact permutation that the sorting algorithm produced:

\vccInput[linerange={begin-out}]{c/7.3.sort.c}

This sample introduces two new features.
The first is the output ghost parameter \vcc{_(out Perm p)}.
An \vcc{out} parameter is used to return data from the function to the
caller. (One could also do it with a pointer to ghost memory, but
using an \vcc{out} parameter is simpler and more efficient.)

To call \vcc{sort()} you need to supply a local variable to hold
the permutation when the function exits, as in:
\begin{VCC}
void f(int *buf, unsigned len)
  // ...
{
  _(ghost Perm myperm; )
  // ...
  sort(buf, len _(out myperm));
}
\end{VCC}
The effect is to copy the value of the local variable \vcc{p} of the
function to the variable \vcc{myperm} on exit from the function.

The second feature is the use of \Def{record} types. A record type is
introduced by putting \vcc{_(record)} before the definition of a 
like a \vcc{struct} type. They are mathematically cleaner than
\vcc{structs}, in several ways:
\begin{itemize}
\item A record is a single, indivisible value (like a map), so you
  don't have to worry about aliasing individual fields of a record.
\item Two records are equal iff their corresponding fields are
  equal. Conversely, C doesn't allow \vcc{==} on struct values
  (primarily because padding makes the meaning problematic).
\item Because \vcc{==} makes sense on records, a record type can be
  used as the domain of a map, whereas a \vcc{struct} type cannot.
\end{itemize}
However, records also have some limitations relative to \vcc{struct}s:
\begin{itemize}
\item Records are values, not objects, so records cannot have
  invariants.
\item Record fields can be of record type, but cannot be of compound
  (\vcc{struct} or \vcc{union}) type.
\item You can't take the address of a field of a record (and so cannot
  pass it to a function that updates it). However, you can use record
  fields as \vcc{_(out)} parameters.
\end{itemize}

\subsection{Inductive Proofs}
Sometimes, you will want to verify programs that depend on
mathematics that is too hard for VCC to do on its own, typically
because they require guidance. In VCC, you do this by writing ghost
code. In particular, you can do inductive proofs by writing loops or
by writing recursive functions, with the loop invariant or function
spec serving as the inductive hypothesis. Because C doesn't allow the
definition of anonymous recursive functions, loops are usually more
convenient. 

Here is a small example of using ghost code to prove the formula for
triangular numbers:

\vccInput[linerange={begin-end}]{c/7.4.math.c}

Sometimes, you might want to use inductively defined types other than
\vcc{\natural} or \vcc{\integer}. VCC lets you define your own
(ghost) inductive types, much like modern functional languages, but
using the following C-like syntax:

\vccInput[linerange={begin-app}]{c/7.5.lists.c}

This defines an inductive datatype of lists, where a list is either of
the empty list \vcc{nil()} or an \vcc{int} followed by a
list. Values of abstract types are deconstructed using the following
\vcc{switch} construct (corresponding to matching expressions in
functional languages):

\vccInput[linerange={app-appAssoc}]{c/7.5.lists.c}

Note that unlike the usual C \vcc{switch} statement, fallthrough from
one case to the next is not allowed (since the scope of variables like
\vcc{v} and \vcc{l} introduced by the \vcc{case} construct only go to
the end of the case). Note that VCC automatically chooses the
termination measure \vcc{_(decreases size(x), size(y))}, which
suffices for typical functions where termination follows by structural
induction. 

We can now prove something about the function we defined:

\vccInput[linerange={appAssoc-rev}]{c/7.5.lists.c}

Note that the recursive call provides the needed inductive case of the
theorem we're trying to prove. We can similarly use the resulting
function as a lemma in proving other theorems. 


\subsection*{Exercises}
\begin{enumerate}

\item 
Verify your favorite sorting functions (quicksort, heapsort,
mergesort, etc.).
\item
Using \vcc{app}, define a recursive function that reverses a
\vcc{List}. Prove that reversing a list twice gives you the same list
back. (Hint: you will need some additional lemmas along the way; one
solution can be found in \vcc{7.5.lists.c}.)
\end{enumerate}
