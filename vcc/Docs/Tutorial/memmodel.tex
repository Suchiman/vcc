\section{Objects and Memory}
\label{sect:memmodel}
\label{sect:reint}

In VCC, the state of the world is given by the state of a set of
completely independent \Def{objects}, each with a set of fields and a value
for each of these fields. (That's right, the objects are completely
separate.) For a given program, the set of objects is fixed. Each
field of each object is either ghost or concrete; each concrete field
of each object has a fixed address and size. Each object has a
(volatile, ghost) Boolean field \vcc{\valid} that says whether the
object is currently ``active''; a basic invariant guaranteed by
succesful verification is that concrete fields of valid objects do not
overlap. Moreover, another basic invariant guarantees that threads
read and write only through valid objects. These invariants guarantee
that reads and writes of concrete fields can be implemented by reads
and writes in the concrete address space.


The objects mostly correspond to instances of user-defined compound
(i.e., \vcc{struct} or \vcc{union}) types. In addition, there are a few
special objects; the most important of these are the following:
\begin{itemize}
\item 
For every variable of primitive type, there is a dummy object of
which the variable is the only field.
\item 
Each thread is an object. The thread running the current function
activation is called \vcc{\me}, and it owns all of the \vcc{\wrapped}
and \vcc{\mutable} variables.
\end{itemize}

In particular, there are no objects of primitive types (integral types
or pointer types); a C ``object'' of primitive type is always
a field of a a VCC object. If \vcc{p}
is a pointer to a primitive lvalue, the object of which \vcc{p} is a
field is denoted \vcc{\embedding(p)}. The embedding of a field is
considered part of its type, so pointers to two fields of different
objects that happen to have the same address are not considered equal
to VCC, even though they ``test'' as equal in C. For example,
\begin{VCC}
void test(int *x, int *y) {
  if (x == y) {
    _(assert \addr(x) == \addr(y)) // succeeds
    _(assert x == y)               // fails
  }
}
\end{VCC}

\noindent Note that for pointers to fields of valid objects, this
discrepency disappears (since such fields cannot alias), so the
discrepency in the interpreation of pointer equality rarely comes up
in practice. Note also that this discrepency doesn't effect the
soundness of verification, just how assertions are interpreted. If for
some reason you really need to talk about raw addresses, you can
always use the \vcc{\addr} operator.

\subsection{Unions}
A C \vcc{union} defines a group of objects that share memory addresses. Thus,
the members of a union cannot in general all be valid at the same
time. When VCC allocates a union, it chooses an arbitrary member to be
the valid one. You can invalidate the active one and validate another
using the operator
\vcc{union_reinterpret}, as in the following example:
\begin{VCC}
typedef struct S {
	int x;
} S;

typedef struct T {
	int y,z;
} T;

typedef union U {
	S s;
	T t;
} U;

void test() {
  U u;
  //u.s.x = 1;              // fails
  _(union_reinterpret &u.s) 
  u.s.x = 1;
  //u.t.y == 1;             // fails
  _(union_reinterpret &u.t)
  u.t.y = 2;
}
\end{VCC}

\subsection{Blobs}

In addition to unions, on rare occasions one needs a chunk of memory
that can be used to make objects of more arbitrary types. In practice,
this arises primarily when you are implementing your own memory
manager (typically to more efficiently manage the allocation of small
objects within a page of raw memory).

In VCC, a chunk of truly ``raw'' memory is called a \Def{blob}. A blob
of appropriate size and alignment can be turned into an object (along
with its extent), making the blob invalid and the objects valid (and
mutable). The top object of such a hierarchy is said to be
blobifiable, and can be later turned back into a blob (again, this
requires that its extent is mutable).  Contiguous blobs can be
combined into a single blob or vice versa.

When a typed object is allocated (on the heap of the stack), what is
actually allocated is a blob, and this blob is unblobified into an
object of the appropriate type. This means that you can blobify
top-level local objects, as well as an object created from
\vcc{malloc}'d memory. On the other hand, you cannot blobify a field
of a \vcc{struct}.


