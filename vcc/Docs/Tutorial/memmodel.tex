\section{Memory model}
\label{sect:memmodel}

In most situations in C the type of a pointer is statically known:
while at the machine code level the pointer is passed around as a type-less
word, at the C level, in places where it is used, we know its type.
VCC memory model makes this explicit: pointers are understood as pairs
of their type and address (an word or integer representing location in memory
understood as an array of bytes).
For any state of program execution, VCC maintains the set of \Def{proper pointers}.
\todo{we might want a better name}
Only proper pointers can be accessed (read or written).
There are rules on changing the proper pointer set --- \eg one can remove
a pointer \vcc{(T*)a}, and add pointers \vcc{(char*)a}, \vcc{(char*)(a+1)},
\dots, \vcc{(char*)(a+sizeof(T)-1)}, or \textit{vice versa}.
These rules make sure that at any given time, representations of two
unrelated proper pointers do not overlap, which greatly simplifies reasoning.
Note that given a \vcc{struct SafeString *p}, when \vcc{\proper(p)}
we will also expect \vcc{\proper(&p->len)}.
That is, when a structure is proper, and thus safe to access, so should
be all its fields.
This is what ``unrelated'' means in the sentence above:
the representations overlap if and only if they pointer refer to a struct
and fields of that struct.
It is OK that fields overlap with their containing struct, but that
structs overlap each other.

\subsection{Reinterpretation}
\label{sect:reint}
