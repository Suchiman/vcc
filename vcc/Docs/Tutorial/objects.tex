\section{Object invariants}
\label{sect:invariants}

%% Verification in VCC, especially when it comes to concurrent programs,
%% it not so much about loop invariants or functions contracts,
%% as about object invariants. 

Pre- and postconditions allow for associating consistency conditions
with the code.
However, fairly often it is also possible to associate such consistency
conditions with the data and require all the code operating on such data
to obey the conditions.
As we will learn in \secref{concurrency} this is particularly important for
data accessed concurrently by multiple threads,
but even for sequential programs enforcing consistency conditions
on data reduces annotation clutter and allows for introduction of abstraction
boundaries.

% This is not a section about concurrency, thus we need a sequential
% rationale. --M
%When verifying a function, we are usually reasoning about only a very
%small part of the program state. This is particularly true for a
%concurrent program, where concurrency depends on minimizing the amount
%of state that a thread ``owns''. Since a thread only knows about data
%that it owns (and perhaps read-only data that it shares), another
%mechanism is needed to regulate data that is shared between
%threads. But these same mechanisms provide a way to minimize the
%amount of knowledge a thread needs at any program point.

In VCC, the mechanism for enforcing data consistency is \Def{object invariants}, which are conditions associated
with compound C types (\vcc{struct}s and \vcc{union}s).
The invariants of a type describe how ``proper'' objects of that type
behave. 
In this and the following section, we consider only the static aspects of this
behavior, namely what the ``consistent'' states of an object are. 
Dynamic aspects, \ie how objects can change, are covered in \secref{concurrency}.
For example, consider the following type definition of \vcc{'\0'}-terminated
safe strings implemented with statically allocated arrays (we'll see
dynamic allocation later).

\vccInput[linerange={obj-init}]{c/6.1.safestring.c}

\noindent
The invariant of \vcc{SafeString} states that consistent
\vcc{SafeString}s have length not more than \vcc{SSTR_MAXLEN} and are
\vcc{'\0'}-terminated.  Within a type invariant, \vcc{\this} refers to
(the address of) the current instance of the type (as in the first
invariant), but fields can also be referred to directly (as in the
second invariant). 

Because memory in C is allocated without initialization, no nontrivial
object invariant could be enforced to hold at all times
(they would not hold right after allocation).
\Def{Wrapped} objects are ones for which the invariant holds
and which the current thread directly owns (that is they are not
part of representation of some higher-level objects).
After allocating an object we would usually wrap it to make sure its invariant holds
and prepare it for later use:

\vccInput[linerange={init-append}]{c/6.1.safestring.c}

\noindent
For a pointer \vcc{p} of structured type, \vcc{\span(p)} returns the
set of pointers to members of \vcc{p}. Arrays of base types produce
one pointer for each base type component, so in this example,
\vcc{\span(s)} abbreviates the set
\begin{VCC}
  { s, &s->len, &s->content[0], &s->content[1], ..., &s->content[SSTR_MAXLEN] }
\end{VCC}
%pointers to all fields of \vcc{s}.%
%\footnote{
%  This is a bite more complicated when embedded structs are involved,
%  see \secref{TODO}.
%}
Thus, the writes clause says that the function 
%not only can wrap \vcc{s} but can also 
can write the fields of \vcc{s}. 
The postcondition says that the function returns with \vcc{s} wrapped,
which implies also that the invariant of \vcc{s} holds; this invariant
is checked when the object is wrapped. (You can see this check fail by
commenting any of the assignment statements.)

A function that modifies a wrapped object will first unwrap it, make
the necessary updates, and wrap the object again (which causes another
check of the object invariant). Unwrapping an object adds all of its
members to the writes set of a function, so such a function has to
report that it writes the object, but does not have to report writing
the fields of the object.

\vccInput[linerange={append-index}]{c/6.1.safestring.c}

\noindent
Finally, a function that only reads an object need not unwrap, and so
will not list it in its writes clause. For example:

\vccInput[linerange={index-out}]{c/6.1.safestring.c}

The following subsection explains this wrap/unwrap protocol in more details.

\subsection{Wrap/unwrap protocol}
\label{sect:wrap-unwrap}

Because invariants do not always hold,
in VCC one needs to explicitly state which objects are consistent,
using a field \vcc{\closed} which is defined on every object.
A \Def{closed object} is one for which the \vcc{\closed} field
is true, and an \Def{open object} is one where it is false.
The invariants have to (VCC enforces them) to hold only when for closed objects, but
can also hold for open objects.
Newly allocated objects are open, and you need to make them open before disposing them.

In addition to the \vcc{\closed} field each object has an \Def{owner field}.
The owner of \vcc{p} is \vcc{p->\owner}.
%While the consistency flag is always written to by the wrap
%and unwrap operations, the owner field can be under some
%conditions (\secref{dynamic-claims}) written to directly.
%Therefore instead of a function it is referred to with
%a field: the owner of \vcc{p} is \vcc{p->\owner}.
This field is of pointer (object) type, but
VCC provides objects, of \vcc{\thread}
type, to represent threads of execution, so that threads can also own objects.
The idea is that the owner of \vcc{p} should have some special rights to \vcc{p} that others do not.
In particular, the owner of \vcc{p} can transfer ownership of \vcc{p} to
another object (\eg a thread can transfer ownership of \vcc{p} from itself to the memory allocator, 
in order to dispose of \vcc{p}).

When verifying a body of a function VCC assumes that it is being executed by some
particular thread.
The \vcc{\thread} object representing it is referred to as \vcc{\me}.

(Some of) the rules of ownership and consistency are
\begin{enumerate}
\item on every atomic step of the program the invariants of all the closed objects have to hold,
\item only the owning thread can modify fields of an open object,
\item each thread owns itself, and
\item only threads can own open objects.
\end{enumerate}
Thus, by the first two rules, VCC allows updates of objects in the following two situations:
\begin{enumerate}
\item the updated object is closed, the update is atomic, and the update preserves the invariant of the object,
\item or the updated object is open and the update is performed by the owning thread.
\end{enumerate}
In the first case to ensure that an update is atomic, VCC requires that the
updated field has a \vcc{volatile} modifier.
There is a lot to be said about atomic updates in VCC, and we shall do
that in \secref{concurrency}, but for now we're only considering sequentially
accessed objects, with no \vcc{volatile} modifiers on fields.
For such objects we can assume that they \emph{do not change}
when they are closed, so the only way to change their fields is to
first make them open, \ie via method~2 above.

A thread needs to make the object open to update it.
Because making it open counts as an update, the thread needs
to own it first.
This is performed by the unwrap operation, which translates to the following steps:
\begin{enumerate}
\item assert that the object is in the writes set,
\item assert that the object is wrapped (closed and owned by the current thread), 
\item assume the invariant (as a consequence of rule~1, the invariant holds for every closed object),
\item set the \vcc{\closed} field to false, and
\item add the span of the object (\ie all its fields) to the writes set
\end{enumerate}
The wrap operation does just the reverse:
\begin{enumerate}
\item
assert that the object is mutable (open and owned by the current thread),
\item assert the invariant, and
\item set the \vcc{\closed} field to true (this implicitly prevents further writes to the fields of the object).
\end{enumerate}
Let's then have a look at the definitions of \vcc{\wrapped(...)} and \vcc{\mutable(...)}.

\begin{VCC}
logic bool \wrapped(\object o) =
  o->\closed && o->\owner == \me;
logic bool \mutable(\object o) =
  !o->\closed && o->\owner == \me;
\end{VCC}

The definitions of \vcc{\wrapped(...)} and \vcc{\mutable(...)}
use the \vcc{\object} type.
It is much like \vcc{void*}, in the sense that it is a wildcard for any pointer type.
However, unlike \vcc{void*}, it also carries the dynamic information about the type of the pointer.
It can be only used in specifications.

The assert\slash assume desugaring of the \vcc{sstr_append_char()} function looks as follows:

\vccInput[linerange={assert-out}]{c/6.2.safestring_assert.c}

\subsection{Ownership trees}
\label{sect:ownership}

Objects often stand for abstractions that are implemented with
more than just one physical object.
As a simple example, consider our \vcc{SafeString}, changed to have a dynamically
allocated buffer.
The logical string object consists of the control object holding the length
and the array of bytes holding the content.
In real programs such abstraction become hierarchical, \eg an address book might consists of a few hash tables, each
of which consists of a control object, an array of buckets,
and the attached linked lists.

\vccInput[linerange={obj-append}]{c/6.3.safestring_dynamic.c}

\noindent
In C the type \vcc{char[10]} denotes an array with exactly 10 elements.
VCC extends that location to allow
the type \vcc{char[capacity]} denoting an array with \vcc{capacity} elements
(where \vcc{capacity} is a variable).
Such types can be only used in casts. For example, \vcc{(char[capacity])content}
means to take the pointer \vcc{content} and interpret it as an array
of \vcc{capacity} elements of type \vcc{char}.
This notation is used so we can think of arrays as objects (of a special type).
The other way to think about it is that \vcc{content} represents just
one object of type \vcc{char}, whereas \vcc{(char[capacity])content}
is an object representing the array.

The invariant of \vcc{SafeString} specifies that it \Def{owns} the
array object.
The syntax \vcc{\mine(o1, ..., oN)} is roughly equivalent
(we'll get into details later) to:
\begin{VCC}
o1->\owner == \this && ... && oN->\owner == \this
\end{VCC}
Conceptually there isn't much difference between having the \vcc{char}
array embedded and owning a pointer to it.
In particular, the functions operating
on some \vcc{s} of type \vcc{SafeString}
should still list only \vcc{s} in their writes clauses,
and not also \vcc{(char[s->capacity])s->content},
or any other objects the string might comprise of.
To achieve that VCC performs \Def{ownership transfers},
\ie assignments to the \vcc{\owner} field.
Specifically, there is another step when unwrapping an object \vcc{p}:
\begin{enumerate}
\setcounter{enumi}{5}
\item
for each object \vcc{o} owned by \vcc{p},
set \vcc{o->\owner} to \vcc{\me} and add \vcc{o} to the writes set
\end{enumerate}
Similarly, when wrapping \vcc{p}, VCC additionally does:
\begin{enumerate}
\setcounter{enumi}{3}
\item
for each object \vcc{o} that needs to be owned by \vcc{p}
(which is determined by \vcc{p}'s invariant, as you'll see in the next section),
assert that \vcc{o} is wrapped and writable and set \vcc{o->\owner} to \vcc{p}.
\end{enumerate}
Let's have a look at an example:

\vccInput[linerange={append-alloc}]{c/6.3.safestring_dynamic.c}

\noindent
First, let's explain the syntax:
\begin{VCC}
_(unwrapping o) { ... }
\end{VCC}
is equivalent to:
\begin{VCC}
_(unwrap o) { ... } _(wrap o)
\end{VCC}
Let \vcc{cont = (char[s->capacity]) s->content}. 
At the beginning of the function, \vcc{s} is owned by the
current thread (\vcc{\me}) and closed (\ie \vcc{\wrapped}), whereas
(by the string invariant) \vcc{cont} is owned by \vcc{s} (and
therefore closed).  Unwrapping \vcc{s} transfers ownership of 
\vcc{cont} to \vcc{\me}, but \vcc{cont} remains closed.
Thus, unwrapping \vcc{s} makes the string mutable, and \vcc{cont}
wrapped.  Then we unwrap \vcc{cont} (which doesn't own anything, so
the thread gets no new wrapped objects), perform the changes, and wrap
\vcc{cont}.  Finally, we wrap \vcc{s}.  This transfers ownership
of \vcc{cont} from the current thread to \vcc{s}, so \vcc{cont}
is no longer wrapped (but still closed).  Here is
the assert\slash assume translation:

\vccInput[linerange={append-out}]{c/6.4.safestring_dynamic_assert.c}

\noindent
Here, \vcc{\inv(p)} means the (user-defined) invariant of object \vcc{p}.
There are two ownership transfers
of \vcc{cont} to and from \vcc{\me} because \vcc{s} owns \vcc{cont} beforehand,
as specified in its invariant.
However, suppose we had an invariant like the following:
\begin{VCC}
struct S {
  struct T *a, *b;
  _(invariant \mine(a) || \mine(b))
};
\end{VCC}
When wrapping an instance of \vcc{struct S}, VCC wouldn't know which object
to transfer ownership of to the instance. Therefore, 
VCC rejects such invariants, and only allow \vcc{\mine(...)}
as a top-level conjunct in an invariant, unless further annotation is given;
see \secref{dynamic-ownership}.

\subsection{Dynamic ownership}
\label{sect:dynamic-ownership}

When a struct is annotated with \vcc{_(dynamic_owns)} the ownership transfers
during wrapping need to performed explicitly, but \vcc{\mine(...)} can
be freely used in its invariant, including using it under a universal
quantifier.

\vccInput[linerange={obj-set}]{c/6.5.table.c}

\noindent
The invariant of \vcc{struct SafeContainer} states that it owns its underlying array,
as well as all elements pointed to from it.
It also states that there are no duplicates in that array.
Let's now say we want to change a pointer in that array,
from \vcc{x} to \vcc{y}.
After such an operation, the container should own whatever it used
to own minus \vcc{x} plus \vcc{y}.
To facilitate such transfers VCC introduces the \Def{owns set}.
It is essentially the inverse of the owner field.
It is defined on every object \vcc{p} and referred to as \vcc{p->\owns}.
VCC maintains that:
\begin{VCC}
\forall \object p, q; p->\closed ==> 
  (q \in p->\owns <==> q->\owner == p)
\end{VCC}
The operator \vcc{<==>} reads ``if and only if'', and is simply boolean
equality (or implication both ways), with a binding priority lower than implication.
That is, for closed \vcc{p}, the set \vcc{p->\owns} contains exactly
the objects that have \vcc{p} as their owner.
Additionally, the unwrap operation does not touch the owns set,
that is after unwrapping \vcc{p}, the \vcc{p->\owns} still contains
all that objects that \vcc{p} used to own.
Finally, the wrap operation will attempt to transfer ownership
of everything in the owns set to the object being wrapped.
This requires that the current thread has write access to these objects
and that they are wrapped.

Thus, the usual pattern is to unwrap the object, potentially modify the owns
set, and wrap the object.
Note that when no ownership transfers are needed, one can just unwrap
and wrap the object, without worrying about ownership.
Let's have a look at an example, which does perform an ownership transfer:

\vccInput[linerange={set-use}]{c/6.5.table.c}

\noindent
The \vcc{sc_set()} function transfers ownership of \vcc{s} to \vcc{c},
and additionally leaves object initially pointed to by \vcc{s->strings[idx]}
wrapped, \ie owned by the current thread.
\begin{note}
\todo{We should have entire section about BVD. --MM}
VCC needs a help in form of an assertion statement at the beginning:
\vcc{sc_set} gets a wrapped \vcc{c} and \vcc{s}, so it cannot be the
case that \vcc{c} owns \vcc{s}. This is what the assertion says.
Without spelling it out explicitly, VCC thinks that \vcc{s} might be 
somewhere in the \vcc{strings} array beforehand, and
thus after inserting it at \vcc{idx} the distinctness invariant could be violated.
If you look at the error model in that case, you can
see that VCC knows nothing about the truth value of \vcc{s \in c->\owns},
and thus adding an explicit assertion about it helps.
\end{note}
Moreover, it promises that this object is \Def{fresh}, \ie the thread did not own
it directly before.
This can be used at a call site:

\vccInput[linerange={use-out}]{c/6.5.table.c}

\noindent
In the contract of \vcc{sc_add} the string \vcc{s} is mentioned
in the writes clause, but in the postcondition we do not say it's wrapped.
Thus, asserting \vcc{\wrapped(s)} after the call fails.
On the other hand, asserting \vcc{\wrapped(o)} fails before the call,
but succeeds afterwards.
Additionally, \vcc{\wrapped(c)} holds before and after as expected.

\begin{note}
\textbf{How is the write set updated?} \\
Before allowing a write to \vcc{*p} VCC will assert \vcc{\mutable(p)}.
Additionally, it will assert that either \vcc{p} is in the writes
clause, or the consistency or ownership of \vcc{p} was updated after the current
function started executing.
Thus, after you unwrap an object, you modify consistency of all its fields,
which provides the write access to them.
Also, you modify ownership of all the objects that it used to own, providing
write access to unwrap these objects.
In case a write clause is specified on a loop, think of an implicit function
definition around the loop.
\end{note}

%This effectively tells the call site that it no longer has ownership of \vcc{s}.
%Additionally, when we look at the invariant of \vcc{c}, we can even figure out
%that \vcc{s} is indeed no longer wrapped.
%
%VCC does know that the invariant of \vcc{s} holds (because the object is closed),
%but we need to explicitly assert it to bring it into theorem prover scope.
%Normally, this is done by \vcc{unwrap}, or \vcc{requires \wrapped(...)}.
%
%Note the distinction between not being able to prove \vcc{P} and 
%being able to prove \vcc{!P}.
%

\subsection{Ownership domains}

The \Def{sequential ownership domain} of an object \vcc{o} (written
\vcc{\domain(o)}) consists of \vcc{o} along with\footnote{ The domains
  of the objects owned by \vcc{o} are included only if \vcc{o} is not
  declared as \vcc{_(volatile_owns)}; see \secref{concurrency}.} the
union of the ownership domains of all objects owned by \vcc{o}.  In
other words, it's the set of objects that are transitively owned by
\vcc{o}. For most purposes, it is useful to think of \vcc{o} as
``containing'' all of \vcc{\domain(o)}; indeed, if \vcc{o1 != o2} and
neither \vcc{o1} nor \vcc{o2} are in the other's \vcc{\domain}, their
\vcc{\domain}s are necessarily disjoint. In particular, if \vcc{o1}
and \vcc{o2} are owned by threads then (because threads own
themselves) \vcc{o1} and \vcc{o2} are necessarily disjoint.

Writability of \vcc{o} gives a thread potential access to all of
\vcc{\domain(o)}: writability allows the thread to unwrap \vcc{o},
which makes writable both the fields of \vcc{o} and any objects that were
owned by \vcc{o}. Conversely, a call to a function that does not list
a wrapped object \vcc{o} in its writes clause is guaranteed to leave 
all of \vcc{\domain(o)} unchanged\footnote{
  This applies to nonvolatile fields of objects in
  the domain; volatile fields might change silently (see section \secref{concurrency}).
}. However, VCC will only reason about the unchangedness of
\vcc{\domain(o)} if it is explicitly brought to its attention, as in
the following example:

\begin{VCC}
void f(T *p) 
  _(writes p) { ... }
...
T *p, *q, *r;
_(assert \wrapped(q) && q != p)
_(assert q \in \domain(q))
_(assert r \in \domain(q))
f(p);
\end{VCC}
\noindent
The second and third assertions bring to VCC's attention that as long
as \vcc{q} is not explicitly unwrapped or included in the writes
clause of a function call, \vcc{r} and its fields will not change.

\subsection{Simple sequential admissibility}
\label{sect:admissibility0}

Until now we've ignored the issue of constraints on the form of object invariants.
The basic rule is that a state change that preserves the invariants of
all updated objects should also preserve the invariants of any closed,
unupdated object \vcc{o}. If an invariant of \vcc{o} satisfies this
criterion, we say that the invariant is \Def{admissible}. VCC requires
all object invariants to be admissible; this admissibility check is
performed when verifying the declaration of \vcc{o}'s type.
The admissibility check allows VCC to check an update to the state by
just checking the invariants of objects that are actually updated;
admissibility of the remaining objects guarantees that all of their
invariants are preserved also.

Fortunately, the most common case is trivial: if an invariant of
\vcc{o} mentions only (nonvolatile) fields of objects in
\vcc{\domain(o)}, the invariant is necessarily admissible.
More sophisticated kinds of invariants are discussed in \secref{inv2}.

\subsection{Type safety}
\label{sect:type-safety}

In modern languages like Java, C\#, and ML, where memory consists of a
collection of typed objects. Programs in these languages allocate
objects (as opposed to memory), and the type of an object remains
fixed until the object is destroyed. Moreover, a non-null reference to
an object is guaranteed to point to a ``valid'' object. But in C, a
type simply provides a way to interpret a sequence of bytes; nothing
prevents a program from having multiple pointers of different types
pointing into the same physical memory, or having a non-null
pointer point into an invalid region of memory.

That said, most C programs really do access memory using a strict type
discipline and tacitly assume that their callers do
so also. For example, if the parameters of a function are a pointer to
an \vcc{int} and a pointer to a \vcc{char}, we shouldn't have to worry
about crazy possibilities like the \vcc{char} aliasing with the second
half of the \vcc{int}. (Without such assumptions, we would have to
provide explicit preconditions to this effect.)  On the other hand, if
the second parameter is a pointer to an \vcc{int}, we do consider the
possibility of aliasing (as we would in a strongly typed language).
Moreover, since in C objects of structured types literally contain
fields of other types, if the second argument were a struct that had
a member of type \vcc{int}, we would have to consider the possibility
of the first parameter aliasing that member. 

To support this kind of antialiasing by default, VCC essentially
maintains a typed view of memory; in any state, \vcc{p->\valid} means
that \vcc{p} points to memory that is currently ``has'' type
\vcc{p}. The rules governing validity guarantee that in any state, the
valid pointers constitute a typesafe view of memory.  In particular,
valid objects never alias, and valid fields are always fields of valid
objects. Only valid fields (and purely local variables) can be
accessed by a program.

There are rare situations where a program needs to change the type of
memory, i.e., make one object invalid while making valid an object
that aliases with it. The most common example is in the memory
allocator, which needs to create and destroy objects of arbitrary
types from arrays of bytes in its memory pool. Therefore, VCC includes
annotations (explained in \secref{reint}) that explicitly change
object validity (and are in fact the only means to do so).  Thus,
while your program can access memory using pretty much arbitrary types
and typecasting, doing so is likely to require additional
annotations. But for most programs, checking type safety is completely
transparent, so you don't have to worry about it.

