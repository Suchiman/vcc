\section{Object invariants}
\label{sect:invariants}

%% Verification in VCC, especially when it comes to concurrent programs,
%% it not so much about loop invariants or functions contracts,
%% as about object invariants. 

Pre- and postconditions allow for associating consistency conditions
with the code.
However, fairly often it is also possible to associate such consistency
conditions with the data and require all the code operating on such data
to obey the conditions.
As we will learn in \secref{concurrency} this is particularly important for
data accessed concurrently by multiple threads,
but even for sequential programs enforcing consistency conditions
on data reduces annotation clutter and allows for introduction of abstraction
boundaries.

% This is not a section about concurrency, thus we need a sequential
% rationale. --M
%When verifying a function, we are usually reasoning about only a very
%small part of the program state. This is particularly true for a
%concurrent program, where concurrency depends on minimizing the amount
%of state that a thread ``owns''. Since a thread only knows about data
%that it owns (and perhaps read-only data that it shares), another
%mechanism is needed to regulate data that is shared between
%threads. But these same mechanisms provide a way to minimize the
%amount of knowledge a thread needs at any program point.

In VCC, the mechanism for enforcing data consistency is \Def{object invariants}, which are conditions associated
with compound C types (\vcc{struct}s and \vcc{union}s).
The invariants of a type describe how ``proper'' objects of that type
behave. 
In this and the following section, we consider only the static aspects of this
behavior, namely what the ``consistent'' states of an object are. 
Dynamic aspects, \ie how objects can change, are covered in \secref{concurrency}.
For example, consider the following type definition of \vcc{'\0'}-terminated
safe strings implemented with statically allocated arrays (we'll see
dynamic allocation later).

\vccInput[linerange={obj-init}]{c/05_safestring.c}

\noindent
The invariant of \vcc{SafeString} states that consistent
\vcc{SafeString}s have length not more than \vcc{SSTR_MAXLEN} and are
\vcc{'\0'}-terminated.  Within a type invariant, \vcc{\this} refers to
(the address of) the current instance of the type (as in the first
invariant), but fields can also be referred to directly (as in the
second invariant). 

Because memory in C is allocated without initialization, no nontrivial
object invariant could be enforced to hold at all times
(they would not hold right after allocation).
% Consistency is explained later. --M
\Def{Wrapped} objects are ones for which the invariant holds
and which the current thread directly owns (that is they are not
part of representation of some higher-level objects).
After allocating an object we would usually wrap it to make sure its invariant holds
and prepare it for later use:

\vccInput[linerange={init-append}]{c/05_safestring.c}

\noindent
For a pointer \vcc{p} of structured type, \vcc{\span(p)} returns the
set of pointers to members of \vcc{p}. Arrays of base types produce
one pointer for each base type component, so in this example,
\vcc{\span(s)} abbreviates the set
\begin{VCC}
  { s, &s->len, &s->content[0], &s->content[1], ..., &s->content[SSTR_MAXLEN] }
\end{VCC}
%pointers to all fields of \vcc{s}.%
%\footnote{
%  This is a bite more complicated when embedded structs are involved,
%  see \secref{TODO}.
%}
Thus, the writes clause says that the function 
%not only can wrap \vcc{s} but can also 
can write the fields of \vcc{s}. 
The postcondition says that the function returns with \vcc{s} wrapped,
which implies also that the invariant of \vcc{s} holds; this invariant
is checked when the object is wrapped. (You can see this check fail by
commenting any of the assignment statements.)

A function that modifies a wrapped object will first unwrap it, make
the necessary updates, and wrap the object again (which causes another
check of the object invariant). Unwrapping an object adds all of its
members to the writes set of a function, so such a function has to
report that it writes the object, but does not have to report writing
the fields of the object.

\vccInput[linerange={append-index}]{c/05_safestring.c}

\noindent
Finally, a function that only reads an object need not unwrap, and so
will not list it in its writes clause. For example:

\vccInput[linerange={index-out}]{c/05_safestring.c}

The following subsection explains this wrap/unwrap protocol in more details.

\subsection{Wrap/unwrap protocol}
\label{sect:wrap-unwrap}

Because invariants do not always hold,
in VCC one needs to explicitly state which objects are consistent,
using a field \vcc{\consistent} which is defined on every object.
The invariants need to hold only when the \vcc{\consistent} field is true.
When the field is false they need not hold (but of course still can).
The field is initially (on newly allocated objects) false,
and you need to set it to false before disposing objects.

In addition to the \vcc{\consistent} field each object has an \Def{owner field}.
The owner of \vcc{p} is \vcc{p->\owner}.
%While the consistency flag is always written to by the wrap
%and unwrap operations, the owner field can be under some
%conditions (\secref{dynamic-claims}) written to directly.
%Therefore instead of a function it is referred to with
%a field: the owner of \vcc{p} is \vcc{p->\owner}.
This field is of pointer (object) type, but
VCC provides objects, of \vcc{\thread}
type, to represent threads of execution, so that threads can also own objects.
The idea is that the owner of \vcc{p} should have some special rights to \vcc{p} that others do not.
In particular, the owner of \vcc{p} can transfer ownership of \vcc{p} to
another object (\eg a thread can transfer ownership of \vcc{p} from itself to the memory allocator, 
in order to dispose of \vcc{p}).

When verifying a body of a function VCC assumes that it is being executed by some
particular thread.
The \vcc{\thread} object representing it is referred to as \vcc{\me}.

(Some of) the rules of ownership and consistency are
\begin{enumerate}
\item on every atomic step of the program the invariants of all the consistent objects have to hold,
\item only the owning thread can modify fields of an inconsistent object,
\item each thread owns itself, and
\item only threads can own inconsistent objects.
\end{enumerate}
Thus, by the first two rules, VCC allows updates of objects in the following two situations:
\begin{enumerate}
\item the updated object is consistent, the update is atomic, and the update preserves the invariant of the object,
\item or the updated object is inconsistent and the update is performed by the owning thread.
\end{enumerate}
In the first case to ensure that an update is atomic, VCC requires that the
updated field has a \vcc{volatile} modifier.
There is a lot to be said about atomic updates in VCC, and we shall do
that in \secref{concurrency}, but for now we're only considering sequentially
accessed objects, with no \vcc{volatile} modifiers on fields.
For such objects we can assume that they \emph{do not change}
when they are consistent, so the only way to change their fields is to
first make them inconsistent, \ie via method~2 above.

A thread needs to make the object inconsistent to update it.
Because making it inconsistent counts as an update, the thread needs
to own it first.
This is performed by the unwrap operation, which translates to the following steps:
\begin{enumerate}
\item assert that the object is in the writes set,
\item assert that the object is wrapped (consistent and owned by the current thread), 
\item assume the invariant (as a consequence of rule~1, the invariant holds for every consistent object),
\item set the \vcc{\consistent} field to false, and
\item add the span of the object (\ie all its fields) to the writes set
\end{enumerate}
The wrap operation does just the reverse:
\begin{enumerate}
\item
assert that the object is mutable (inconsistent and owned by the current thread),
\item assert the invariant, and
\item set the \vcc{\consistent} field to true (this implicitly prevents further writes to the fields of the object).
\end{enumerate}
Let's then have a look at the definitions of \vcc{\wrapped(...)} and \vcc{\mutable(...)}.

\begin{VCC}
logic bool \wrapped(\object o) =
  o->\consistent && o->\owner == \me;
logic bool \mutable(\object o) =
  !o->\consistent && o->\owner == \me;
\end{VCC}

The definitions of \vcc{\wrapped(...)} and \vcc{\mutable(...)}
use the \vcc{\object} type.
It is much like \vcc{void*}, in the sense that it is a wildcard for any pointer type.
However, unlike \vcc{void*}, it also carries the dynamic information about the type of the pointer.
It can be only used in specifications.

The assert\slash assume desugaring of the \vcc{sstr_append_char()} function looks as follows:

\vccInput[linerange={assert-out}]{c/05_safestring_assert.c}

\subsection{Ownership trees}
\label{sect:ownership}

Objects often stand for abstractions that are implemented with
more than just one physical object.
As a simple example, consider our \vcc{SafeString}, changed to have a dynamically
allocated buffer.
The logical string object consists of the control object holding the length
and the array of bytes holding the content.
In real programs such abstraction become hierarchical, \eg an address book might consists of a few hash tables, each
of which consists of a control object, an array of buckets,
and the attached linked lists.

\vccInput[linerange={obj-append}]{c/06_safestring_dynamic.c}

\noindent
In C the type \vcc{char[10]} denotes an array with exactly 10 elements.
VCC extends that location to allow
the type \vcc{char[capacity]} denoting an array with \vcc{capacity} elements
(where \vcc{capacity} is a variable).
Such types can be only used in casts. For example, \vcc{(char[capacity])content}
means to take the pointer \vcc{content} and interpret it as an array
of \vcc{capacity} elements of type \vcc{char}.
This notation is used so we can think of arrays as objects (of a special type).
The other way to think about it is that \vcc{content} represents just
one object of type \vcc{char}, whereas \vcc{(char[capacity])content}
is an object representing the array.

The invariant of \vcc{SafeString} specifies that it \Def{owns} the
array object.
The syntax \vcc{\mine(o1, ..., oN)} is roughly equivalent
(we'll get into details later) to:
\begin{VCC}
o1->\owner == \this && ... && oN->\owner == \this
\end{VCC}
Conceptually there isn't much difference between having the \vcc{char}
array embedded and owning a pointer to it.
In particular, the functions operating
on some \vcc{s} of type \vcc{SafeString}
should still list only \vcc{s} in their writes clauses,
and not also \vcc{(char[s->capacity])s->content},
or any other objects the string might comprise of.
To achieve that VCC performs \Def{ownership transfers},
\ie assignments to the \vcc{\owner} field.
Specifically, there is another step when unwrapping an object \vcc{p}:
\begin{enumerate}
\setcounter{enumi}{5}
\item
for each object \vcc{o} owned by \vcc{p},
set \vcc{o->\owner} to \vcc{\me} and add \vcc{o} to the writes set
\end{enumerate}
Similarly, when wrapping \vcc{p}, VCC additionally does:
\begin{enumerate}
\setcounter{enumi}{3}
\item
for each object \vcc{o} that needs to be owned by \vcc{p}
(which is determined by \vcc{p}'s invariant, as you'll see in the next section),
assert that \vcc{o} is wrapped and writable and set \vcc{o->\owner} to \vcc{p}.
\end{enumerate}
Let's have a look at an example:

\vccInput[linerange={append-alloc}]{c/06_safestring_dynamic.c}

\noindent
First, let's explain the syntax:
\begin{VCC}
_(unwrapping o) { ... }
\end{VCC}
is equivalent to:
\begin{VCC}
_(unwrap o) { ... } _(wrap o)
\end{VCC}
\itodo{should we use ``s'' instead of ``the string'', and similarly for content?} % mah: <- I guess so...
Thus, at the beginning of the function the string is owned by the current thread and consistent (\ie wrapped),
whereas, by the string invariant, the content is owned by the string and consistent.
After unwrapping the string, the ownership of the content goes to the current thread,
but the content remains consistent.
Thus, unwrapping the string makes the string mutable, and the content wrapped.
Then we unwrap the content (which doesn't own anything, so the thread gets no new wrapped objects), perform the changes,
and wrap the content.
Finally, we wrap the string.
This transfers ownership of the content from the current thread to the string, so the content is no longer wrapped (but still consistent).
Second, let's have a look at the assert\slash assume translation:

\vccInput[linerange={append-out}]{c/06_safestring_dynamic_assert.c}

\noindent
To make it easier to read, we made it store the \vcc{s->content} pointer
casted to an array into a temporary variable.
Also, an invariant of \vcc{p} is referred to as \vcc{\inv(p)}.
As you can see there are two ownership transfers
of \vcc{cont} to and from \vcc{\me}.
This happens because \vcc{s} owns \vcc{cont} beforehand,
as specified in its invariant.
However, let's say we had an invariant like the following:
\begin{VCC}
struct S {
  struct T *a, *b;
  _(invariant \mine(a) || \mine(b))
};
\end{VCC}
When wrapping an instance of \vcc{struct S}, should we transfer ownership of \vcc{a}, \vcc{b}, or both?
By default VCC will reject such invariants, and only allow \vcc{\mine(...)}
as a top-level conjunct in an invariant.
Invariants like the ones above are supported, but need additional annotation
and manual ownership transfer when wrapping, see \secref{dynamic-ownership}.


\subsection{Dynamic ownership}
\label{sect:dynamic-ownership}

When a struct is annotated with \vcc{_(dynamic_owns)} the ownership transfers
during wrapping need to performed explicitly, but \vcc{\mine(...)} can
be freely used in its invariant, including using it under a universal
quantifier.

\vccInput[linerange={obj-set}]{c/07_table.c}

\noindent
The invariant of \vcc{struct SafeContainer} states that it owns its underlying array,
as well as all elements pointed to from it.
It also states that there are no duplicates in that array.
Let's now say we want to change a pointer in that array,
from \vcc{x} to \vcc{y}.
After such an operation, the container should own whatever it used
to own minus \vcc{x} plus \vcc{y}.
To facilitate such transfers VCC introduces the \Def{owns set}.
It is essentially the inverse of the owner field.
It is defined on every object \vcc{p} and referred to as \vcc{p->\owns}.
VCC maintains that:
\begin{VCC}
\forall \object p, q; p->\consistent ==> 
  (q \in p->\owns <==> q->\owner == p)
\end{VCC}
The operator \vcc{<==>} reads ``if and only if'', and is simply boolean
equality (or implication both ways), with a binding priority lower than implication.
That is, for consistent \vcc{p}, the set \vcc{p->\owns} contains exactly
the objects that have \vcc{p} as their owner.
Additionally, the unwrap operation does not touch the owns set,
that is after unwrapping \vcc{p}, the \vcc{p->\owns} still contains
all that objects that \vcc{p} used to own.
Finally, the wrap operation will attempt to transfer ownership
of everything in the owns set to the object being wrapped.
This requires that the current thread has write access to these objects
and that they are wrapped.

Thus, the usual pattern is to unwrap the object, potentially modify the owns
set, and wrap the object.
Note that when no ownership transfers are needed, one can just unwrap
and wrap the object, without worrying about ownership.
Let's have a look at an example, which does perform an ownership transfer:

\vccInput[linerange={set-use}]{c/07_table.c}

\noindent
The \vcc{sc_set()} function transfers ownership of \vcc{s} to \vcc{c},
and additionally leaves object initially pointed to by \vcc{s->strings[idx]}
wrapped, \ie owned by the current thread.
Moreover, it promises that this object is \Def{fresh}, \ie the thread did not own
it directly before.
This can be used at a call site:

\vccInput[linerange={use-out}]{c/07_table.c}

\noindent
In the contract of \vcc{sc_add} the string \vcc{s} is mentioned
in the writes clause, but in the postcondition we do not say it's wrapped.
Thus, asserting \vcc{\wrapped(s)} after the call fails.
On the other hand, asserting \vcc{\wrapped(o)} fails before the call,
but succeeds afterwards.
Additionally, \vcc{\wrapped(c)} holds before and after as expected.

\begin{note}
\textbf{How is the write set updated?} \\
Before allowing a write to \vcc{*p} VCC will assert \vcc{\mutable(p)}.
Additionally, it will assert that either \vcc{p} is in the writes
clause, or the consistency or ownership of \vcc{p} was updated after the current
function started executing.
Thus, after you unwrap an object, you modify consistency of all its fields,
which provides the write access to them.
Also, you modify ownership of all the objects that it used to own, providing
write access to unwrap these objects.
In case a write clause is specified on a loop, think of an implicit function
definition around the loop.
\end{note}

%This effectively tells the call site that it no longer has ownership of \vcc{s}.
%Additionally, when we look at the invariant of \vcc{c}, we can even figure out
%that \vcc{s} is indeed no longer wrapped.
%
%VCC does know that the invariant of \vcc{s} holds (because the object is consistent),
%but we need to explicitly assert it to bring it into theorem prover scope.
%Normally, this is done by \vcc{unwrap}, or \vcc{requires \wrapped(...)}.
%
%Note the distinction between not being able to prove \vcc{P} and 
%being able to prove \vcc{!P}.
%

\subsection{Ownership domains}

An \Def{ownership domain} of an object \vcc{p} is the set of objects
that it owns and their ownership domains, plus \vcc{p} itself.
In other words, it's the set of objects that are transitively
owned by a given object.

\begin{note}
\todo{this remark might be confusing, and possibly no longer true}
In general there can be cycles in the ownership graph,
and so the definition above should be understood in the least fix point sense.
However, every object has exactly one owner and threads own themselves,
and thus anything that
is owned by a thread will have a ownership domain that is a tree.
It is most useful to think about ownership as trees, and disregard
the degenerate cycle case.
\end{note}

Let's then take a look at an ownership graph:
we will have a bunch of threads as roots.
In particular,
the current thread will own a number of objects, some of them
mutable (inconsistent and thus not owning anything), but other wrapped (consistent and
so with possibly large ownership trees (domains) hanging off them).
The ownership domains of the wrapped objects are disjoint
(in the grand ownership tree of the thread they are all at the same level).

Mentioning a wrapped object in the writes clause gives the function
a right to unwrap it, and then unwrap everything it owns.
Thus, it effectively gives write access to its entire ownership domain.
\todo{make it a real example that does something useful}
Consider the following piece of code:

\begin{VCC}
void f(T *p) 
  _(writes p) { ... }
...
T *p, *q, *r;
_(assert \wrapped(p) && \wrapped(q) && p != q);
_(assert q \in \domain(q))
_(assert r \in \domain(q))
_(assert q->f == 1 && r->f == 2);
f(p);
_(assert q->f == 1 && r->f == 2);
\end{VCC}

\noindent
The function \vcc{\domain(o)} returns the ownership
domain of \vcc{o}.
We have three objects, two of them are wrapped.
We call a function that will update one of them.
We now want to know if the values of the other two are preserved.
Clearly, because \vcc{p != q} and both are wrapped, then
\vcc{q} is not in the ownership domain of \vcc{p},
so value of \vcc{q->f} should be preserved by the call.
The value of \vcc{r->f} will be preserved unless
\vcc{r \in \domain(p)},
because \vcc{f(p)} could have written everything
in \vcc{\domain(p)} (according to its writes clause).
Unfortunately, the underlying logic used by VCC
is not strong enough to show this directly
(technically: the transitive closure of a relation, ownership in this case, is not
expressible in first-order logic).
However, VCC knows that there is no way to change anything in an ownership
domain without writing its root.
This is because we always enforce that ownership domains are disjoint.
For example, the only way for \vcc{f(p)} to write something in \vcc{\domain(q)}
would be to list \vcc{q} in \vcc{f()}'s writes clause, which is not the case.
Thus, if VCC knows that \vcc{r \in \domain(q)}, and it knows that
\vcc{f(p)} couldn't have written \vcc{q}, then it also knows that \vcc{r->f} is unchanged.
Unfortunately, we need to explicitly tell VCC in
which ownership domain \vcc{r} is to make use of that.
This is what the assertion \vcc{r \in \domain(q)} is doing.
We currently also need to assert \vcc{q \in \domain(q)} to help VCC with reasoning. 
This is because it treats the fields of \vcc{q} similarly to objects owned by \vcc{q}.
We plan to fix that in future.

\subsection{Simple sequential admissibility}
\label{sect:admissibility0}

Until now we've been skimming on the issue of what you can actually
mention in an invariant. 
Intuitively the invariant of an object should talk about consistent states
of that very object, not some other objects.
However, for example the invariant of the \vcc{struct SafeString} 
talks about the values stored in its underlying array.
This also seems natural: one should be able to mention things
from the ownership domain of \vcc{p} in \vcc{p}'s invariant.

This is important, because VCC checks only invariants of objects
that you actually modify, and as we recall the most important
verification property we want to enforce (and which we rely on in our
verifications) is that all invariant of all (consistent) objects 
are always preserved (\secref{wrap-unwrap}).
For example, consider:

\begin{VCC}
struct A { int x; };
struct A *a; // global variable
struct B {
  int y;
  _(invariant a->x == y)
};
void foo()
  _(requires \wrapped(a))
  _(writes a)
{
  _(unwrapping a) { a->x = 7; }
}
\end{VCC}

\noindent
In \vcc{foo()}, when wrapping \vcc{a} we would only check invariant
of \vcc{a}, not all \vcc{struct B}s that could possibly depend on it.
Thus, an action which preserves invariant of modified object breaks invariant of another object.
For this reason VCC makes the invariant of \vcc{struct B} inadmissible.
In fact, for all invariants VCC will check that they are admissible.
Admissibility of type \vcc{T} is checked by verifying VCC-generated
function called \vcc{T#adm}.
You can see messages about them when you verify files with type
invariants.
%In the VS interface you can check admissibility by right-clicking
%on type definitions, while in the command line you can provide
%\lstinline|/f:T#adm| parameter.

We shall refrain now from giving a full definition of admissibility, as 
it only makes full sense after we learn about two-state object invariants
(see \secref{inv2}), but for sequential programs the useful approximation
is that invariants that only talk about their ownership domains are admissible.


\subsection{Type safety}
\label{sect:type-safety}

Throughout this section
we have been talking about typed memory ``objects'' as if this were a
meaningful concept.  This typed view of memory is supported by most
modern programming languages, like Java, C\#, and ML, where memory
consists of a collection of typed objects. Programs in these languages
don't allocate memory (on the stack or on the heap), they allocate
objects, and the type of an object remains fixed until the object is
destroyed. Moreover, a non-null reference to an object is guaranteed to
point to a ``valid'' object. But in C, a type simply provides a way to
interpret a sequence of bytes; nothing prevents a program from having
multiple pointers of different types pointing into the same memory, or
even having two instances of the same \vcc{struct} type partially
overlapping. Moreover, a non-null pointer might still point into an
invalid region of memory.

That said, most C functions really do access memory using a strict type
discipline and tacitly assume that their callers do
so also. For example, if the parameters of a function are a pointer to
an \vcc{int} and a pointer to a \vcc{char}, we shouldn't have to worry
about crazy possibilities like the \vcc{char} aliasing with the second
half of the \vcc{int}. (Without such assumptions, we would have to
provide explicit preconditions to this effect.)  On the other hand, if
the second parameter is a pointer to an \vcc{int}, we do consider the
possibility of aliasing (as we would in a strongly typed language).
Moreover, since in C objects of structured types literally contain
objects of other types, if the second argument were a struct that had
a member of type \vcc{int}, we would have to consider the possibility
of the first parameter aliasing that member. 

To support this, VCC essentially maintains a typed view of memory; in
any state, \vcc{p->\valid} means that \vcc{p} points to memory that is
currently ``has'' type \vcc{p}. The rules governing validity guarantee
that in any state, the valid pointers constitute a typesafe view of
memory.  In particular, if two valid pointers point to overlapping
portions of memory, one of them is properly contained in the other; if
a \vcc{struct} is typed, then each of its members is typed; and if a
\vcc{union} is typed, then exactly one of its members is typed.  The
definition of validity is folded into the definitions of
\vcc{\thread_local} and \vcc{\mutable}; these definitions check not
only that the memory pointed to exists, but that the pointer to it is
valid.

There are rare situations where a program needs to change type
assignment of a pointer.  The most common is in the memory allocator,
which needs to create and destroy objects of arbitrary types from
arrays of bytes in its memory pool. Therefore, VCC includes
annotations (explained in \secref{reint}) that explicitly change the
typestate.  Thus, while your program can access memory using pretty
much arbitrary types and typecasting, doing so will require additional
annotations. But for most programs, checking type safety is completely
transparent, so you don't have to worry about it.

