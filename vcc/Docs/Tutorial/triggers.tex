\section{Triggers}

\itodo{The current VCC version needs to be told to infer triggers, see \secref{todo-invoke}}

The triggers are likely the most difficult part of this tutorial.
As of July 2010, the trigger inference algorithm in VCC
(as opposed to the SMT solver)
has been implemented, so triggers need to be used less often.
Thus, we didn't need any trigger annotations for the examples
in the tutorial.
Still, you'll need them to deal with more complex VCC verification
tasks.

This appendix gives some background on the usage of triggers in the 
SMT solvers, the underlying VCC theorem proving technology.

SMT solvers
prove that the program is correct by looking for possible counterexamples,
or \Def{models}, where your program goes wrong (\eg by violating an assertion).
Once the solver goes through \emph{all} possible counterexamples, and finds them
all to be inconsistent (\ie impossible),
it considers the program to be correct.
Normally, it would take virtually forever, for there is very large number of
possible counterexamples, one per every input to the function (values stored in
the heap also count as input).
To workaround this problem, the SMT solver considers
\Def{partial models}, \ie 
sets of statements about the state of the program.
For example, the model description may say \vcc{x == 7}, \vcc{y > x}
and \vcc{*p == 12}, which describes all the concrete models, where
these statements hold. There is great many such models,
for example one for each different value of \vcc{y} and other program variables,
not even mentioned in the model.

It is thus useful to think of the SMT solver as sitting there with
a possible model, and trying to find out whether the model is consistent or not.
For example, if the description of the model says that \vcc{x > 7} and
\vcc{x < 3}, then the solver can apply rules of arithmetic, conclude this is
impossible, and move on to a next model.
The SMT solvers are usually very good in finding inconsistencies in models
where the statements describing them do not involve universal quantifiers.
With quantifiers things tend to get a bit tricky.

For example, let's say the model description states that the two
following facts are true:
\begin{VCC}
\forall unsigned i; i < 10 ==> a[i] > 7
a[4] == 3
\end{VCC}
The meaning of the universal quantifier is that it should hold
not matter what we substitute for \vcc{i}, for example
the universal quantifier above implies the following facts (which
are called \Def{instances} of the quantifier):
\begin{VCC}
 4 < 10 ==>  a[4] > 7  // for i == 4
\end{VCC}
which happens to be the one needed to refute our model,
\begin{VCC}
11 < 10 ==> a[11] > 7  // for i == 11
\end{VCC}
which is trivially true, because false implies everything, and
\begin{VCC}
 k < 10 ==>  a[k] > 7  // for i == k
\end{VCC}
where \vcc{k} is some program variable of type \vcc{unsigned}.

However, there is potentially infinitely many such instances, and certainly too many
to enumerate them all.
Still, to prove that our model candidate is indeed contradictory
we only need the first one, not the other two.
Once the solver adds it to the model description,
it will simplify \vcc{4 < 10} to true,
and then see that \vcc{a[4] > 7} and \vcc{a[4] == 3} cannot hold
at the same time.

The question remains: how does the SMT solver decide that the first
instance is useful, and the other two are not?
This is done through so called \Def{triggers}.
Triggers are either specified by the user or inferred automatically
by the SMT solver or the verification tool.
In all the examples before we relied on the automatic trigger
inference, but as we go to more complex examples, we'll need to consider
explicit trigger specification.

A trigger for a quantified formula is usually some subexpression 
of that formula, which contains all the variables that the formula
quantifies over.
For example, in the following formula:
\begin{VCC}
\forall int i; int p[int]; is_pos(p, i) ==> f(i, p[i]) && g(i)
\end{VCC}
possible triggers include the following expressions
\vcc|is_pos(p, i)|, \vcc|p[i]|, and also \vcc|f(i, p[i])|,
whereas \vcc|g(i)| would not be a valid trigger, because
it does not contain \vcc|p|.

Let's assume that \vcc|is_pos(p, i)| is the trigger. 
The basic idea is that when the SMT solvers considers a model,
which mentions \vcc|is_pos(q, 7)| (where \vcc|q| is, \eg a local variable), then the formula
should be instantiated with \vcc|q| and \vcc|7| substituted
for \vcc|p| and \vcc|i| respectively.

Note that the trigger \vcc|f(i, p[i])| is \Def{more restrictive} than
\vcc|p[i]|: if the model contains \vcc|f(k, q[k])| it also contains \vcc|q[k]|.
Thus, a ``bigger'' trigger will cause the formula to be instantiated less often,
generally leading to better proof performance (because the solver has
less formulas to work on), but also possibly preventing
the proof altogether (when the solver does not get the instantiation needed for the proof).

Triggers cannot contain boolean operators or the equality operator.
As of the current release, arithmetic operators are allowed, but cause warnings
and work unreliably, so you should avoid them.

A formula can have more than one trigger.
It is enough for one trigger to match in order for the formula
to be instantiated.

\begin{note}
\textbf{Multi-triggers}:
Consider the following formula:
\begin{VCC}
\forall int a, b, c; P(a, b) && Q(b, c) ==> R(a, c)
\end{VCC}
There is no subexpression here, which would contain all the variables
and not contain boolean operators.
In such case we need to use a \Def{multi-trigger}, which
is a set of expressions which together cover all variables.
An example trigger here would be \vcc|{P(a, b), Q(b, c)}|.
It means that for any model, which has both \vcc|P(a, b)|
and \vcc|Q(b, c)| (for the same \vcc|b|!), the quantifier
will be instantiated.
In case a formula has multiple multi-triggers, \emph{all}
expressions in at least \emph{one} of multi-triggers
must match for the formula to be instantiated.

If it is impossible to select any single-triggers in the formula,
and none are specified explicitly, Z3 will select \emph{some}
multi-trigger, which is usually not something that you want.
\end{note}

\subsection{Matching loops}

Consider a model description
\begin{VCC}
\forall struct Node *n; {\mine(n)} \mine(n) ==> \mine(n->next)
\mine(a)
\end{VCC}
Let's assume the SMT solver will instantiate the quantifier with \vcc{a}, yielding:
\begin{VCC}
\mine(a) ==> \mine(a->next)
\end{VCC}
It will now add \vcc{\mine(a->next)} to the set of facts
describing the model.
This however will lead to instantiating the quantifier again,
this time with \vcc{a->next}, and in turn again with
\vcc{a->next->next} and so forth.
Such situation is called a \Def{matching loop}.
The SMT solver would usually cut such loop at a certain depth,
but it might make the solver run out of time, memory, or both.

Matching loops can involve more than one quantified formula. 
For example consider the following, where \vcc{f} is a user-defined function.
\begin{VCC}
\forall struct Node *n; {\mine(n)} \mine(n) ==> f(n)
\forall struct Node *n; {f(n)} f(n) ==> \mine(n->next)
\mine(a)
\end{VCC}


\subsection{Trigger selection}
\label{sect:trigger-inference}

The explicit triggers are listed in \vcc|{...}|, after the quantified variables.
They don't have to be subexpressions of the formula. 
We'll see some examples of that later.
When there are no triggers specified explicitly, VCC selects the triggers for you.
These are always subexpressions of the quantified formula body.
To select default triggers VCC first considers all subexpressions which contain all the quantified variables,
and then it splits them into four categories:
\begin{itemize}
\item level 0 triggers, which are mostly ownership-related. 
These are \vcc{\mine(E)}, \vcc{E1 \in \owns(E2)}, and also \vcc{E1 \in0 E2} (which, except for triggering, is the same as \vcc{E1 \in E2}).
\item level 1 triggers: set membership and maps, that is expressions of 
the form \vcc{E1 \in E2} and \vcc{E1[E2]}.
\item level 2 triggers: default, \ie everything not mentioned elsewhere. 
It is mostly heap dereferences, like \vcc{*p}, \vcc{&a[i]} or \vcc{a[i]}, as well as bitwise arithmetic operators.
\item level 3 triggers: certain ``bad triggers'', which use internal VCC encoding functions.
\item level 4 triggers: which use interpreted arithmetic operations (\vcc{+}, \vcc{-}, and \vcc{*} on integers).
\end{itemize}

Expressions, which contain \vcc{<=}, \vcc{>=}, \vcc{<}, \vcc{>}, \vcc{==},
\vcc{!=}, \vcc{||}, \vcc{&&}, \vcc{==>}, \vcc{<==>}, and \vcc{!} are not
allowed in triggers.

Each of these expressions is then tested for immediate matching loop,
that is VCC checks if instantiating the formula with that trigger
will create a bigger instance of that very trigger.
Such looping triggers are removed from their respective categories.
This protects against matching loops consisting of a single
quantified formula, but matching loops with multiple formulas
are still possible.

To select the triggers, VCC iterates over levels, starting with 0.
If there are some triggers at the current level, these triggers are selected
and iteration stops.
This means that, \eg if there are set-membership triggers then heap dereference
triggers will not be selected.

If there are no triggers in levels lower than 4, VCC tries to select a multi-trigger.
It will only select one, with possibly low level, and some overlap between variables
of the subexpressions of the trigger.
Only if no multi-trigger can be found, VCC will try to use level 4 trigger.
Finally, if no triggers can be inferred VCC will print a warning.

As a post-processing step, VCC looks at the set of selected triggers, and if
any there are two triggers X and Y, such that X is a subexpression of Y, then Y
is removed, as it would be completely redundant.

You can place a \vcc@{:level N}@ annotation in place of a trigger.
It causes VCC to use all triggers from levels 0 to N inclusive.
If this results in empty trigger set, the annotation is silently ignored.

The flag \texttt{/dumptriggers:K} (or \texttt{/dt:K}) can be used to display inferred
triggers.
\texttt{/dt:1} prints the inferred triggers,
\texttt{/dt:2} prints what triggers would be inferred if \vcc@{:level ...}@ annotation
was supplied.
\texttt{/dt:3} prints the inferred triggers even when there are explicit triggers
specified. 
It does not override the explicit triggers, it just print what would happen if you
removed the explicit trigger.

Let's consider an example:
\begin{VCC}
int *buf;
unsigned perm[unsigned];
\forall unsigned i; i < len ==> perm[i] == i ==> buf[i] < 0
\end{VCC}
The default algorithm will infer \vcc@{perm[i]}@, and with \vcc@{:level 1}@
it will additionally select \vcc@{&buf[i]}@.
Note the ampersand.
This is because in C \vcc{buf[i]} is equivalent to \vcc{*(&buf[i])}, and thus
the one with ampersand is simpler.
You can also equivalently write it as \vcc@{buf + i}@.
Note that the plus is not integer arithmetic addition, and can thus be safely used
in triggers.

Another example would be:
\begin{VCC}
\forall struct Node *n; n \in q->\owns ==> perm[n->idx] == 0
\end{VCC}
By default we get level 0 \vcc@{n \in q->\owns}@, with level 1 we also get
\vcc@{perm[n->idx]}@ and with level 2 additionally \vcc@{&n->idx}@.

\subsection{Hints}
\label{sect:trigger-hints}

Consider a quantified formula \vcc|\forall T x; {:hint H} E|.
Intuitively the hint annotation states that the expression \vcc{H} (which can
refer to \vcc{x}) might have something to do with proving \vcc{E}.
A typical example, where you might need it is the following:
\begin{VCC}
\forall struct Node *n; \mine(n) ==> \mine(n->next) && n->next->prev == n
\end{VCC}
The default trigger selection will pick \vcc@{\mine(n->next)}@, which is also
the ``proper'' trigger here. 
However, when proving admissibility, to know that \vcc{n->next->prev} did not
change in the legal action, we need to know \vcc{\mine(n->next)}.
This is all good, it's stated just before, but the SMT solver
might try to prove \vcc{n->next->prev == n} first,
and thus miss the fact that \vcc{\mine(n->next)}.
Therefore, we will need to add \vcc@{:hint \mine(n->next)}@.
For inferred level 0 triggers, these are added automatically.


%\subsection{Typical triggers}
%
%\itodo{maybe we want a section like that?}
%
%\begin{VCC}
%\forall unsigned i; {a[i]} i < 100 ==> a[i] > 0
%\forall unsigned i, j; {f(i, j)} f(i, j) == i + j * 2
%\end{VCC}
%



