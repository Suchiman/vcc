\section{Assert and assume}
\label{sect:assert-assume}

%\itodo{Penny: This section dives right into the meat. First, you need two things: 
%1. A general guide to annotation syntax
%2. A section that describes how to think in VCC terms is, you have to describe
%not only exactly what your code does at each step of the way, but what your
%program *doesnt* do. This latter point will nail people on loop verification
%every time. 
%After you do the how to think part, I suggest some simple exercises, if this is in fact to be a tutorial. The whole book can come later. 
%}
%\itodo{Michal: I'm not sure about 1., 2. might be good idea.}

Let's begin with a simple example:

\vccInputSC{c/01_minInline2.c}
This program (internally) sets \vcc{z} to the minimum of \vcc{x} and
\vcc{y}. 
An annotation of the form \vcc{_( assert E )}, called an \emph{assertion}, asks VCC to prove that
\vcc{E} is guaranteed to hold\footnote{
  In C parlance, this means evaluating to a value other than \vcc{0}.
} whenever control reaches this assertion \footnote{
  This interpretation changes slightly if \vcc{E} refers to
  memory locations that might be concurrently written to by other
  threads; see section \secref{concurrency}}.  
Thus, the line \lstinline|_( assert z <= x && z <= y )| says
that when control reaches the assertion, \vcc{z} is no larger than \vcc{x} or
\vcc{y}. If VCC verifies a program, it promises that all such assertions
hold in \emph{every} possible execution of the program, considering all possible inputs,
all possible ways of scheduling concurrent threads, etc. 
% The following doesn't really clarify much
%% An \Def{execution} is a complete recording of all actions made by the program:
%% which branches were taken, what values were read and stored in memory
%% and so forth.
%% For a deterministic program, once you fix input data (\eg by saying in this and that
%% microsecond the user is going to click at that pixel and so forth),
%% you fix the execution.
%% An assertion causes VCC to try to \emph{prove} that \vcc{E} holds for \emph{all}
%% executions.

\begin{note}
It is instructive to compare \vcc{_( assert E )} with the macro
\lstinline|assert(E)| (defined in \lstinline|<assert.h>|), which
evaluates \vcc{E}  at runtime and aborts execution if \vcc{E}
doesn't hold. First, \lstinline|assert(E)| requires runtime overhead (at least
in builds where the check is made), whereas \vcc{_( assert E )} does
not. Second, \lstinline|assert(E)| will catch failure of the 
assertion only when it actually fails in an execution, whereas 
\vcc{_( assert E )} is guaranteed to catch the failure if it is
possible in \emph{any} execution. Third, because VCC doesn't have to
actually execute \vcc{_( assert E )}, it allows \vcc{E} to include
mathematical operations that cannot be executed, such as
quantification over infinite domains.
\end{note}

To verify this function using VCC from the command line, save the source in a file called \lstinline|minimum.c|
and run VCC as follows:

\begin{lstlisting}
C:\Somewhere\VCC Tutorial> vcc.exe minimum.c
Verification of main succeeded.
C:\Somewhere\VCC Tutorial>
\end{lstlisting}

If instead you wish to use VCC Visual Studio plugin, load the solution \lstinline|VCCTutorial.sln|
(distributed with VCC), 
\itodo{
This isn't yet true. The samples are available only with VCC sources, available from
\url{http://vcc.codeplex.com/SourceControl/list/changesets}: click Download on the right,
get the zip file and navigate to \lstinline|vcc/Docs/Tutorial/c|.
}
\todo{we need to package the examples, but only after it is clear which ones are included in the tutorial}
locate the file with the example, and right-click on the program text.
You should get options to verify the file or just this function (either will work).

\begin{note}
If you right-click within a C source file,
several VCC commands are made available, depending on what kind of
construction IntelliSense thinks you are in the middle of. The choice
of verifying the entire file is always available. If you click within
the definition of a struct type, VCC will offer you the choice of
checking admissibility for that type (a concept explained in \secref{admissibility0}).
If you click within the body of a function, VCC should offer
you the opportunity to verify just that function. However,
IntelliSense often gets confused about the syntactic structure of
VCC code, so it might not always present these context-dependent
options. However, if you select the name of a function and then right
click, it will allow you to verify just that function.
\end{note}

VCC verifies this function successfully, which means that its assertions
are indeed guaranteed to hold (and that, in addition, the program cannot
crash). If VCC is unable to prove an assertion, it reports an
error.  Try changing the assertion in this program to something that
isn't true and see what happens. (You might also want to try coming up
with some complex assertion that is true, just to see whether VCC is
smart enough to prove it.)

Assertions, like all VCC annotations, 
are surrounded by \vcc{_( ... )} to indicate that they are
only for VCC, and are not part of the program being verified
For the regular C compiler the \vcc{<vcc.h>} header file defines
\begin{VCC}
#define _(...) /* nothing */
\end{VCC}
VCC does not use this definition, and instead parses the inside of \vcc{_( ... )}
annotations.%
\footnote{
  One can prevent definition of \vcc{_}, and instead surround annotations with
  \vcc{__vcc_spec(...)}.
  This is controlled with \vcc{_VCC_DONT_USE_UNDERSCORE} macro-definition.}

To understand how VCC works, and to use it successfully, it is useful to
think in terms of what VCC ``knows'' at various program points. In
this example, VCC initially knows nothing about the variables (they
can initially hold any value). Just before the first assignment, VCC knows that 
\vcc{x <= y} (because execution followed that branch of the conditional), and
after the assignment, VCC knows additionally that \vcc{z} is equal to
\vcc{x}, so it knows that \vcc{z <= x && z <= y}. Similar reasoning
(with \vcc{x} and \vcc{y} reversed) holds in \vcc{else} branch, so VCC
knows that the assertion holds when control reaches it. In general,
VCC doesn't lose any information when reasoning about assignments and
conditionals. We will see in subsequent sections, however, that VCC
abstracts over (looses) some information when reasoning about loops and function calls,
which means additional annotations are necessary.

When VCC surprises you by failing to verify something that you think
it should be able to verify, it is usually because it doesn't know
something you thought it should. An assertion provides one way to
check whether VCC knows what you think it should know.  

You can add to what VCC knows at a particular point by adding an
\emph{assumption}.
An assumption \vcc{_( assume E )} is a promise that you make to VCC
that for whatever reasons \vcc{E} will always hold.
Equivalently, VCC will just ignore the executions where \vcc{E} doesn't hold.
Operationally, the assumption simply adds \vcc{E} to what VCC
knows, for subsequent reasoning. For example:
\begin{VCC}
int x, y;
_( assume x != 0 )
y = 100 / x;
\end{VCC}
Without the assumption, VCC would complain about possible division by
zero (it checks for division by zero because it would cause the
program to crash).  With the assumption, this error cannot happen.
An assumption essentially tells VCC to ignore all of those executions
where the assumption fails to hold; thus, when VCC successfully
verifies a program, it promises that in every execution where no
assumption fails, no assertion will fail either.
Of course assumptions, like all annotations, are not included in the executable, so
if an assumption fails the program doesn't stop and the following assertion might get violated.
Thus, you normally want to verify your program without any assumptions.

If assumptions are to be avoided, why are assumptions useful at
all? They are useful in several ways:
\begin{itemize}
\item If a verification is not yet complete, assumptions can be used
  to mark the knowledge that VCC is missing, and guide further work.
  If you maintain the discipline of checking in your work only when
  everything verifies (but perhaps with assumptions), other people can
  read your annotated text and see how far away from complete your
  verification is, without having to look at the verification output
  itself.
\item When debugging a failed verification, you can use assumptions to
  narrow down the failed verification to a more specific failure
  scenario.
\item Sometimes you want to assume something even though VCC can
  verify it, just to stop VCC from spending time proving it. For
  example, assuming \vcc{0} (i.e., \vcc{\false}) allows VCC to prove
  all subsequent assertions, effectively focussing its attention on
  other parts of the program.
\item Sometimes you will want to make assumptions about the operating
  environment of a program. For example, you might want to assume that
  a 64-bit counter doesn't overflow, but don't want to justify it
  formally because it depends on extra-logical assumptions (like the
  speed of the hardware and the lifetime of the software). 
\item Finally, assumptions provide a useful tool in explaining how VCC
  reasons about your program. We'll see examples of this throughout
  this tutorial.
\end{itemize}

%%  What is however much more common is
%% use of assumption when debugging your specifications: when the program
%% does not verify you might try to temporarily add assumptions in a bet
%% ``if VCC knew that, would it verify my program?''.  Once you find what
%% is that thing VCC doesn't know, but what makes it verify the program,
%% you can figure out how to specify rest of your program, so that is
%% knows it.

% The following is a bad example, since we would normally specify
% hardware by an invariant on a volatile field.
%% One case is interaction with hardware: you might know the hardware is
%% going to leave specific I/O register positive, and an assumption is a
%% way to communicate that to VCC. 

An assertion can also change what VCC knows after the assertion, if
the assertion fails; even though VCC will report the failure as an error,
it will assume the asserted fact holds afterward. For example
\begin{VCC}
int x;
_( assert x == 0 ) // fails
_( assert x == 0 ) // succeeds
\end{VCC}

When we talk about what VCC knows, we mean what it knows in an ideal
sense, where if it knows \vcc{E}, it also knows any logical
consequence of \vcc{E}. In such a world, adding an assertion that
succeeds would have no effect on whether later assertions succeed.
VCC's ability to deduce consequences is indeed complete for many types
of expressions (e.g. expressions that use only equalities,
inequalities, addition, subtraction, multiplication by constants, and
logical operators), but not for all expressions, so VCC will
sometimes fail to prove an assertion, even though it ``knows'' enough
to prove it.  

Because VCC's reasoning is incomplete\footnote{Note that this
  incompleteness in no way makes VCC \emph{unsound}; it can cause
  false failure reports, but cannot cause VCC to verify a program that
  is not correct.}, an assertion that succeeds can sometimes cause
later assertions to succeed, by drawing VCC's attention to a crucial
fact it needs to prove the later assertion.  This is relatively rare,
and typically involves ``nonlinear arithmetic'' (e.g. where variables
are multiplied together), bit-vector reasoning, or quantifiers. We'll
say more about these in \secref{bv}.

\subsection{Exercises}
\begin{enumerate}
\item
In the following program fragment, which assertions will fail?
\begin{VCC}
int x,y; 
_( assert x > 5 ) 
_( assert x > 3 ) 
_( assert x < 2 ) 
_( assert y < 3 )
\end{VCC}
\item
Is there any difference between 
\begin{VCC}
_( assume p ) _( assume q  )
\end{VCC}
and 
\begin{VCC}
_( assume q ) _( assume p )
\end{VCC}
? What about if the assumptions are replaced by assertions?
\item
Is there any difference between
\begin{VCC}
_( assume p ) _( assert q )
\end{VCC}
and 
\begin{VCC}
_( assert p ==> q )
\end{VCC}
? 
\end{enumerate}


\section{Function specifications}
\label{sect:functions}

Next we turn to the specification of functions. We'll take the example
from the previous section, and pull the computation of the minimum of
two numbers out into a separate function:\todo{match with the previous programs}

\vccInput{c/01_min2.c}

(The listing above presents both the source code and the output
of VCC, typeset in a different font and color, and 
the actual file name of the example is replaced with \vcc{/*`testcase`*/}.)
VCC failed to prove our assertion, even though it's easy to see that
it always holds. This is because the verification is modular: VCC
doesn't look inside the body of a function (such as \vcc{min()}) 
when verifying a function that calls it (such as \vcc{main()})%
\footnote{We will see later how to selectively override this default
  and perform such inlining during verification.}.  
That is, VCC doesn't ``know'' anything about the effect of calling 
\vcc{min()}, unless you specify it explicitly%
\footnote{
  Actually, VCC does know one important thing from this specification
  of \vcc{min()}: it knows that a call do \vcc{min()} has no side
  effects that are visible to the caller. We'll see in \secref{writes} that
  any such side effects have to be explicitly specified.
}.
Since the correctness of \vcc{main()} clearly depends on what \vcc{min()}
does, we need to specify \vcc{min()} in order to verify \vcc{main()}.

\begin{note}
There are at least three good reasons to use modular
verification. First, it allows verification to more easily scale to
large programs. Second, it allows you to modify the implementation of
a function like \vcc{min()} without having to worry about breaking the
verification of functions that call it (as long as you don't change
the specification of \vcc{min()}). This is especially important
because these callers normally aren't in scope, and the person
maintaining \vcc{min()} might not even know about them (e.g., if
\vcc{min()} is in a library). Third, you can verify a function like
\vcc{main()} even if the implementation of \vcc{min()} is unavailable, or
perhaps hasn't even been written yet. 
\end{note}

The specification of a function is sometimes called a \emph{contract},
because it gives obligations on both the function and its callers:
\begin{itemize}
\item The requirements on the caller (sometimes called the
  \emph{precondition} of the function) take the form 
  \vcc{_( requires E )}, 
  where \vcc{E} is an expression%
  \footnote{
    \vcc{E} can mention only names that are in scope at the function entry, 
    so it can mention parameters of the function, global variables, and
    types.}.  
  In verifying the function, VCC implicitly assumes
  \vcc{E} on function entry, and for each call to the function, VCC
  implicitly asserts \vcc{E} (after the function arguments have been
  evaluated and bound to the parameter names).

\item The requirement on the function  (sometimes called the
  \emph{postcondition} of the function) takes the form
  \vcc{_( ensures E )}, where \vcc{E} is an expression; this says that
  the function promises that \vcc{E} holds when control is returned to
  the caller.  In the code of the caller, VCC implicitly assumes \vcc{E}
  just after the call returns (but
  before the result is used). In the code of the function, at each return
  point of the function (including, implicitly,
  at the end of the function body), VCC asserts that the postcondition
  of the function holds (with the VCC keyword \vcc{\result} bound to the
  return value, and the function parameters 
  interpreted according to the values they had on function entry).
\end{itemize}

For example, we can provide a suitable specification for \vcc{min()} as
follows:
\vccInput[linerange={min-endmin,out-}]{c/01_min3.c}
\noindent
The precondition \vcc{_( requires \true )} of \vcc{min()} doesn't
really say anything (since \vcc{\true} holds in every state), but is included
to emphasize that the function can be called from any state and
with arbitrary parameter values.
The postcondition states that the value returned from \vcc{min()} 
is no bigger than either of the inputs.
Note that \vcc{\true} and \vcc{\result} are spelled with a backslash
to avoid clash with C identifiers.
All VCC keywords start with a backslash.
The backslash can optionally be omitted on the first keyword after \vcc{_}, because it cannot
be confused with a C identifier (thus you are still free to have, \eg
a function called \texttt{requires} or \texttt{assert}).

As described above, VCC translates the program roughly as follows:
\vccInput{c/01_min_assert.c}

\begin{note}
The assumptions introduced implicitly by VCC at function entry and
after the function return are justified by the assertions
introduced before the function call and before the function return, respectively.
This is why they do not compromise soundness of the verifier;
\ie if the entire program verifies, then 
these assumptions will never fail at runtime.
\end{note}

As for call itself, because the \vcc{min()} function above has no
visible specification, VCC doesn't know anything about the value it
might possibly return.  Therefore, as far as VCC's understanding is
concerned, the line:
\begin{VCC}
position = min(newPos, LIMIT);
\end{VCC}
\noindent
could be replaced with:
\begin{VCC}
position = anything();
\end{VCC}
\noindent
that is assigning an unspecified value to the variable \vcc{position}.
The only knowledge about it comes from the following assumption.

\subsection{Exercises}
\begin{enumerate}
\item
Try replacing the \vcc{<} in the return statement of \vcc{min()} with
\vcc{>}. Before running the verifier, can you guess which parts of the
verification will fail?
%% One way to illustrate that the verification of \vcc{main()} 
%% depends only on the specification of \vcc{min()} is to 
%% replace \vcc{<} with \vcc{>} in the return statement of \vcc{min()}.
%% The output of verification shows that \vcc{main()} is still correct---only 
%% \vcc{min()} is broken:
%% \vccInput[linerange={out-}]{c/01_min4.c}

\item
What is the effect of giving a function the specification
\vcc{_( requires \false )} ? How does it effect verification of
the function itself? What about its callers? Can you think of a good 
reason to use such a specification?
%% When verifying function body it translates to an assumption.
%% As we have noted this will make the body of function verify,
%% regardless what it contains.
%% On the other hand, you will be unable to verify call to such a function
%% from outside.
%% Conversely, \vcc{_( ensures \false )} will prevent verification
%% of the body of the function, but will make the function that calls
%% it verify.

%% As we can see, in each case there is something that does not verify.
%% However, because of this you should be careful when interpreting
%% VCC answers: successful verification of a function is only meaningful
%% if everything it calls was verified.

\item
Can you see anything wrong with the above specification of\vcc{min()}?
Can you give a simpler implementation than the one presented? Is this 
specification strong enough to be useful? If not, how might it be
strengthened to make it more useful?

\item
Specify a function that returns the (int) square root of its (int)
argument. (You can try writing an implementation for the function, but
won't be able to verify it until you've learned about loop
invariants.)

\item
Can you think of useful functions in a real program that might
legitimately guarantee only the trivial postcondition \vcc{_( ensures \true )}?
\end{enumerate}
\subsection{Write clauses}
\label{sect:writes}

So far, we have considered only functions without visible side effects; for
example, we assumed that the call to \vcc{min()} didn't have the side
effect of changing the local variables of \vcc{main()}. The
translation above captured this requirement implicitly, by allowing a
\vcc{main()} to call \vcc{min()} without losing any of its prior
knowledge. VCC requires each function specification to declare any side
effects of the function that might destroy knowledge that
the caller might have had coming into the call. However, it need not
declare side effects to the state that the caller
could not possibly know about (e.g., modification to data allocated
after the beginning of the call). 
%(We will see in section \ref{} that
%VCC only allows the caller to know very specific things, which allows
%the function to do other things with global side effects without
%having to report them.)

In VCC, the destruction of knowledge of the caller is specified 
by means of \emph{writes clauses}. 
Intuitively, a clause \vcc{_( writes p, q )} is equivalent to saying
the function is going to modify \emph{only} the object pointed to by \vcc{p}
and the object pointed to by \vcc{q}
(of the memory objects ``local'' to the caller before the call).
In other words, it is the same as having an ensures clause
saying that the values of all other objects local to the caller prior
to the call remain unchanged.  VCC allows you
to specify pointers in separate writes clauses, and implicitly
combines them into a single set. If a function spec contains no writes clauses, 
it is equivalent to specifying a writes clause with empty set of
pointers.

Here is a simple example of a function that has visible side effects:
\vccInput[linerange={swap-partition}]{c/04_partition.c}

\noindent
The expression \vcc{\old(E)}, when used in a postcondition of a function,
returns the value of expression \vcc{E}
as it was at just before the function was called. 
Thus, the postcondition means that the value of \vcc{*p} after the call
is the same as the value of \vcc{*q} before the call, and \emph{vice versa}.

Now, let's have a look at a function calling \vcc{swap()}:

\vccInput[linerange={foo-}]{c/01_swap1.c}

Because global variables (like \vcc{x}) might be visible to callers
of \vcc{foo()}, \vcc{foo()} needs to report that they might change.
Note that the writes clause lists pointers to memory locations, not l-values
(\ie \vcc{_(writes &x, &y, &z)} and not \vcc{_(writes x, y, z)}).
Note also that because \vcc{&z} aliases neither \vcc{&x} nor \vcc{&y}, VCC can
deduce that \vcc{swap(&x, &y)} does not change \vcc{z}.

Had we left out the writes clause from the specification of
\vcc{swap()}, VCC would report an error:

\vccInput[linerange={out-}]{c/01_swap2.c}

\noindent
The complaints about thread-locality refer to the places where the function
reads \vcc{*p} and \vcc{*q}. 
Reading memory doesn't necessarily require a writes clause (since it can't
destroy knowledge of the caller), but it can still be unsafe. 
For example, the pointer might point to memory that has been freed, 
lies outside of the address space of the program, or is owned by another thread. 
For these (and other) reasons VCC implicitly
asserts \vcc{\thread_local(p)} just before any read access to \vcc{*p};
this predicate means that that the pointer points to an object that
``belongs'' to the current thread (much more about this later in the tutorial).
In particular, if something is writable it is also thread local.

Similarly, just before write access to \vcc{*p}, VCC inserts the
assertion \vcc{\writable(p)}, which means:
\begin{itemize}
\item \vcc{\thread_local(p)} (the written memory needs to belong to the current thread),
\item \vcc{*p} is in the ``phase of life'' where it can be modified (it is possible
to specifically wrap objects to prevent modifications, more in \secref{invariants}),
\item \vcc{p} is in the current ``writes set''. (You can think of
the writes set as being the set of objects named in the writes clause,
along with those objects that the thread has gotten control of since
entry of the function; we will see how this can happen in \secref{invariants}.)
\end{itemize}
For example, the assert/assume translation of \vcc{swap()} is%
\footnote{
  VCC does not generate the read and writes checks for the local variable
  \vcc{tmp}. Because \vcc{tmp} is a local that never has its address
  taken, it can be thought of as remaining in a register, where it is
  guaranteed to remain writable.
}:

\vccInput[linerange={swap-}]{c/01_swap3.c}

\noindent
As we can see, one effect of \vcc{writes p, q} is the implicit
precondition \vcc{requires \writable(p) && \writable(q)}. Such precondition
needs to be checked in \vcc{foo()}, at the place where it calls \vcc{swap()}.
In other words, the called function can write at most what the caller can write.
In particular if we forget to list \vcc{&x} in the writes
clause of \vcc{foo()} we would get an error when it tries to call a
function that possibly writes \vcc{&x}:

\vccInput[linerange={out-}]{c/01_swap4.c}


%% You might wonder why cannot we just have 
%% \vcc{requires \writable(p)} and instead have the specialized writes clause.
%% The reason is that the writes clause also specifies that nothing
%% outside of the writes clause will be changed.
%% This is why we can prove that \vcc{y} is still \vcc{42} after the call
%% to \vcc{boundedIncr(&x)}. 


%A predicate \vcc{\mutable(p)} states that the object pointed to by \vcc{p}
%is allocated, ``belongs'' to the current thread, and is in a ``phase of life''
%that allows for modification.
%We will get into details all of these later.
%For now we just need to know that in order to be able to write to \vcc{*p}
%one needs to know that \vcc{p} was listed in the writes clause \emph{and} \vcc{\mutable(p)}.
%For example, if we remove the \vcc{_( writes ... )} clause from the
%\vcc{boundedIncr()} we get the following output:


