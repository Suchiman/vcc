\section{Object invariants}

Object invariants are predicates describing ``consistent'' states of objects.
Take for example a safe string structure implemented with a
statically allocated array (we'll move to dynamic allocation later).

\vccInput[linerange={beg-index}]{c/05_safestring.c}

\noindent
The invariant of \vcc{struct SafeString} states that consistent instances
of that structure will shorter than \vcc{SSTR_MAXLEN} and \vcc{'\0'}-terminated.
While this seems simple and intuitive, we need a quick overlook of how VCC
models C memory, to get a full understanding.

\subsection{Memory model and the object type}

Talking about instances of structures in C is a tricky business. 
In plain C a structure type generally just gives some guidelines how to interpret
arrays of bytes.
Some programming languages, like Java or ML, have a much more disciplined
view of memory:
one allocates an object of given type, not merely N bytes of memory,
and later there is no way to change the type of that object.
Most C programs also follow this model, most of the time:
when a function gets a pointer to \vcc{struct Foo}, it usually doesn't
expect to find data corresponding to some \vcc{struct Bar} there.
Still, there are rare situations where the program needs to
change type assignment of a pointer.
The most common is in the memory allocator, which needs to create
and destroy objects of arbitrary types from arrays of bytes
in its memory pool.
Therefore, the general rule in VCC is that programs are forced to
follow the strict Java-like type discipline, except for places
where explicit annotations indicate reinterpretations of type assignment
(these annotations are explained in \secref{TODO}).
In particular, as long as there are no type reinterpretations,
the type of a pointer is fixed.

In most situations in C the type of a pointer is statically known:
while at the machine code level the pointer is passed around as a type-less
word, at the C level, in places where it is used, we know its type.
VCC memory model makes this explicit: pointers are understood as pairs
of their type and address (an word or integer representing location in memory
understood as an array of bytes).
For any state of program execution, VCC maintains the set of \Def{proper pointers}.
\todo{we might want a better name}
Only proper pointers can be accessed (read or written).
There are rules on changing the proper pointer set --- \eg one can remove
a pointer \vcc{(T*)a}, and add pointers \vcc{(char*)a}, \vcc{(char*)(a+1)},
..., \vcc{(char*)(a+sizeof(T))}, or \emph{vice versa}.
These rules make sure that at any given time, representations of two
unrelated proper pointers do not overlap, which greatly simplifies reasoning.
Note that given a \vcc{struct SafeString *p}, when \vcc{\proper(p)}
we will also expect \vcc{\proper(&p->len)}.
That is, when a structure is proper, and thus safe to access, so should
be all its fields.
This is what ``unrelated'' means in the sentence above:
the representations overlap if and only if they pointer refer to a struct
and fields of that struct.
It is OK that fields overlap with their containing struct, but that
structs overlap each other.

\todo{this is essentially the take-away, we should consider trimming this
section}
The net result is that VCC needs to check if every memory access is proper,
but in return it gets that representations of proper pointers to structs
are disjoint, and thus pointers to structs can be treated as type-safe
Java-like objects.

The type of type/address pairs, written \vcc{\object}, is exposed in the annotation
language.
For example, functions like \vcc{\thread_local()} discussed before,
or the function \vcc{\proper()} checking if a pointer is in the proper set,
take values of type \vcc{\object}.
They actually need that:
the fact that \vcc{(short*)0xdeadf00d} is proper
does not imply that \vcc{(int*)0xdeadf00d} is proper.
Note that this is different than a function taking \vcc{void*} --- in
VCC there would be no way to tell on value of what actual type the function was called.

\subsection{Wrapped}

Consider the function \vcc{sstr_index_of()} in the example below. 

\vccInput[linerange={index-999}]{c/05_safestring.c}

\noindent
To avoid indexing the \vcc{s->content[]} array out of bounds it needs to know
that \vcc{s->len <= SSTR_MAXLEN}. 
One could argue that all objects at all times should be consistent (\ie their
invariants hold),
so if a function takes an object as an argument, it should better be consistent.
Unfortunately, the reality is more complicated.
In particular, the invariants of objects generally do not hold before they are fully
initialized, or in a middle of a function operating on them.
Thus, in VCC one needs to explicitly state which objects are consistent.
This is what the \vcc{requires \wrapped(s)} precondition achieves:
it states that before the function is called the object should
better be consistent and owned by the current thread.
Now we need to explain ownership, and later we shall have a take
at explaining the naming (\ie why is it wrapped).

\subsection{Ownership}

In VCC each object has a single owner, denoted \vcc{\owner(o)}.
The owner itself is also an object, but VCC provides objects, of \vcc{\thread} type, to represent threads.
Thus, intuitively an object is owned by another object or a thread.
Threads own themselves.
For now we're going to ignore the possibility of non-thread objects owning things.

The current thread of execution is referred to as \vcc{\me}.
The concept of ``currently executing'' threads has only meaning when verifying
functions, so \vcc{\me} cannot be used in invariants.

\begin{VCC}
bool \wrapped(\object o) 
  { return \consistent(o) && \owner(o) == \me; }
bool \unwrapped(\object o) 
  { return !\consistent(o) && \owner(o) == \me; }
\end{VCC}


