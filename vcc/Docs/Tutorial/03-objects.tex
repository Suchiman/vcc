\section{Object invariants}

Verification in VCC, especially when it comes to concurrent programs,
it not so much about loop invariants or functions contracts,
as about object invariants.
Object invariants are predicates describing ``consistent'' states of objects.
Take for example a safe string structure implemented with a
statically allocated array (we'll move to dynamic allocation later).

\vccInput[linerange={obj-init}]{c/05_safestring.c}

\noindent
The invariant of \vcc{struct SafeString} states that consistent instances
of that structure will shorter than \vcc{SSTR_MAXLEN} and \vcc{'\0'}-terminated.

Talking about instances of structures in C is a tricky business. 
In plain C a structure type generally just gives some guidelines how to interpret
arrays of bytes.
Some programming languages, like Java or ML, have a much more disciplined
view of memory:
one allocates an object of given type, not merely $n$ bytes of memory,
and later there is no way to change the type assigned to that memory location.
Most C programs also follow this model, most of the time:
when a function gets a pointer to \vcc{struct Foo}, it usually doesn't
expect to find data corresponding to some \vcc{struct Bar} there.
Still, there are rare situations where the program needs to
change type assignment of a pointer.
The most common is in the memory allocator, which needs to create
and destroy objects of arbitrary types from arrays of bytes
in its memory pool.
Therefore, the general rule in VCC is that programs are forced to
follow the strict Java-like type discipline, except for places
where explicit annotations indicate reinterpretations of type assignment
(these annotations are explained in \secref{memmodel}).

In (sequential) Java programs it is usually enough to check that
\vcc{o} is non-null in order to deduce that accessing \vcc{o.f} is safe.
This is not the case is C, particularly not in concurrent C.
The pointer \vcc{o} might be non-null, but still point to an invalid
(for example unallocated) memory location.
VCC inserts an assertion in front of every memory access,
\vcc{*p}%
\footnote{
  Of course this also applies to \vcc{x->y}, which is understood
  as \vcc{*(&x->y)}, and other memory accesses.
}, in the program. 
This assertion will check that memory location pointed to by \vcc{p}
is currently allocated (which is something any sound C verifier
would need to do), but also that this memory location 
\emph{is currently assigned type} \vcc{T}, where \vcc{T} is the
statically known type of \vcc{*p}.
There are restriction on type assignment (to be explained in \secref{memmodel}), which guarantee
that C pointers behave like type-safe, Java-like objects.
It should be noted that one can still access memory almost arbitrarily,
but it requires additional annotations.

Let's then have a look at how one initializes an instance of the \vcc{SafeString} structure.

\vccInput[linerange={init-append}]{c/05_safestring.c}

\noindent
First, the function announces it will write the span of \vcc{s}. This is a shorthand
for:
\begin{VCC}
  _(writes s, &s->len, &s->content[0], &s->content[1], ..., 
           &s->content[SSTR_MAXLEN])
\end{VCC}
%pointers to all fields of \vcc{s}.%
%\footnote{
%  This is a bite more complicated when embedded structures are involved,
%  see \secref{TODO}.
%}
Second, it announces that when it's done, \vcc{s} will be \Def{wrapped}, and indeed
at the end it ``wraps'' \vcc{s}.
This means, that at the end of the function the invariant of \vcc{s} holds.
One could ask why wouldn't the invariant always hold, but already the \vcc{sstr_init()}
function provides a good explanation: its whole purpose is to establish the invariant,
for a freshly allocated object, for which the invariant (generally) doesn't yet hold
(because it contains garbage).
Similarly, sometimes one wants to perform a few updates on an object, and between
them the invariant doesn't have to hold.
Finally, most functions that only read an object, will require that it is consistent.
For example:

\vccInput[linerange={append-out}]{c/05_safestring.c}

\noindent
A pattern emerges: an object comes from the memory allocator unwrapped, 
we initialize and wrap it.
When we want to read it, it should be wrapped, and we know its invariant holds.
A function operating on an object unwraps it to temporarily suspend invariants, perform updates, and wraps it again.

Let's then get a bit more precise about this wrap/unwrap protocol. 

\subsection{Wrap/unwrap Protocol}

Because invariants do not always hold,
in VCC one needs to explicitly state which objects are consistent.
This is expressed with \vcc{\consistent(...)} function.
One should think of this function as returning a value of special ghost field,
let's call it the \Def{consistency flag}, defined on every object.
The invariants need to hold only when consistency flag is true.
When the flag is false they need not hold (but of course still can).
The flag is initially (on newly allocated objects) false,
and one needs to set it to false before disposing objects.

In addition to the consistency flag each object has an \Def{owner field}.
We also refer to it using a function, \vcc{\owner(...)}.
The owner itself is also an object, but VCC provides objects, of \vcc{\thread} type, to represent threads,
so that threads can also own objects.
The idea is that the owner of \vcc{o} should have some special rights to \vcc{o} that others do not.
In particular, the owner of \vcc{o} can transfer ownership of \vcc{o} to
another object (\eg the memory allocator, in order to dispose \vcc{o}).

When verifying a body of a function VCC assumes that it is being executed by some
particular thread.
The \vcc{\thread} object representing it is referred to as \vcc{\me}.

(Some of) the rules of ownership:
\begin{enumerate}
\item threads are always consistent and own themselves
\item only threads can own inconsistent objects
\item only the owning thread can modify inconsistent object
\end{enumerate}
These rules are usually ensured by using the wrap/unwrap protocol.
The first approximation is that the wrap operation asserts that the invariant
indeed holds and sets the consistent flag to true.
The unwrap operation assumes the invariant and sets the consistent
flag to false.
Both require that the object is owned by the current thread.
Let's then have a look at the definitions of \vcc{\wrapped(...)} and \vcc{\unwrapped(...)}
and at the assert/assume desugaring of \vcc{sstr_append_char()} function.

\begin{VCC}
predicate \wrapped(\object o) =
  \consistent(o) && \owner(o) == \me;
predicate \unwrapped(\object o) =
  !\consistent(o) && \owner(o) == \me;
\end{VCC}
\vccInput[linerange={assert-999}]{c/05_safestring_assert.c}

\noindent
The definitions of \vcc{\wrapped(...)} and \vcc{\unwrapped(...)}
use the \vcc{\object} type.
It is much like \vcc{void*}, in the sense that it is a wildcard for any pointer type.
However, unlike \vcc{void*}, it also carries the dynamic information about the type of the pointer.
It can be only used in specifications.
Note that \vcc{\unwrapped(s)} does not mean \vcc{!\wrapped(s)}.





