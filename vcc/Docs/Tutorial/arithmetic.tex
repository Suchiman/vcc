\todo{if this section remains after the functions section, move the exercises from there to here,
since they require quantification.}
\section{Arithmetic and Quantifiers}
\todo{move arithmetic stuff appendix in, as well as mathint and a mention of maps}
\todo{add appendix section gathering together all annotations and extensions to C}

VCC provides a number of C extensions that can be used within VCC
annotations (such as assertions):
\begin{itemize}
\item
The Boolean operator \vcc{==>} denotes logical implication; formally,
\vcc{P ==> Q} means \vcc{((!P) || (Q))}, and is usually 
read as ``\vcc{P} implies \vcc{Q}''. Because \vcc{==>} has lower
precedence than the C operators, it is typically not necessary to
parenthesize \vcc{P} or \vcc{Q}.

\item
The expression \vcc{\forall T v; E} evaluates to \vcc{1} if the
expression \vcc{E} evaluates to a nonzero value for every value 
\vcc{v} of type \vcc{T}. For example, the assertion
\begin{VCC}
_(assert x > 1 &&
  \forall int i; 1 < i && i < x ==> x % i != 0)
\end{VCC}
\noindent checks that (\vcc{int}) \vcc{x} is a prime number. If \vcc{b}
is an \vcc{int} array of size \vcc{N}, then
\begin{VCC}
_(assert \forall int i; \forall int j;
  0 <= i && i <= j && j < N ==> b[i] <= b[j])
\end{VCC}
checks that \vcc{b} is sorted.

\item
Similarly, the expression \vcc{\exists T v; E} evaluates to \vcc{1} if there
is some value of \vcc{v} of type \vcc{T} for which \vcc{E} evaluates
to a nonzero value. For example, if \vcc{b} is an \vcc{int} array of
size \vcc{N}, the assertion
\begin{VCC}
_(assert \exists int i; 0 <= i && i < N && b[i] == 0)
\end{VCC}
asserts that  \vcc{b} contains a zero element.
\vcc{\forall} and \vcc{\exists} are jointly referred to as
\Def{quantifiers}. 

\item
VCC also provides some mathematical types that cannot be used in
ordinary C code (because they are too big to fit in memory);
these include mathematical (unbounded) integers and (possibly infinite) maps. They are described in
\secref{mathTypes}.

\item
Expressions within VCC annotations are restricted in their use of 
functions: you can only use functions that are proved to be 
\Def{pure}, \ie free from side effects (\secref{pureFunctions}).
\end{itemize}
