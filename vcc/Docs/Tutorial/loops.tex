\section{Loop invariants}

For the most part, VCC computes what it knows at a control point from
what it knows at earlier control points. This works even if there are
\vcc{goto}s from earlier control points; VCC just takes the disjunction of 
what it knows for each of the possible places it came from. 
But when the control flow contains a loop, VCC faces a
chicken-egg problem, since what it knows at the top of the loop (i.e.,
at the beginning of each loop iteration) depends not only on what it
knew just before the loop, but also on what it knew just before it
jumped back to the top of the loop from the loop body.

Rather than trying to guess what it should know at the top of a loop,
VCC lets you tell it what it should know, by providing \Def{loop
  invariants}. To make sure that loop invariants indeed hold 
whenever control reaches the top
of the loop, VCC asserts that the invariants hold wherever control
jumps to the top of the loop -- namely, on loop entry and at the end of
the loop body.

Let's look at an example:
\vccInput[linerange={begin-}]{c/5.1.div.c}
\noindent

The \vcc{divide()} function computes the quotient and remainder of
integer division of \vcc{x} by \vcc{d} using the classic division
algorithm.  The loop invariant says that we have a suitable answer,
except with a remainder that is possibly too big. VCC translates this
example roughly as follows:
\vccInput[linerange={begin-end}]{c/5.2.div_assert.c}

\noindent
Note that this translation has removed all cycles from the control
flow graph of the function (even though it has gotos); this means that
VCC can use the rules of the previous sections to reason about the
program. In VCC, all program reasoning is reduced to reasoning about
acyclic chunks of code in this way.

Note that the invariant is asserted wherever control moves to the top of the
loop (here, on entry to the loop and at the end of the loop body). On
loop entry, VCC forgets the value of each variable modified in the
loop (in this case just the local variables \vcc{lr} and \vcc{ld}),
%% \footnote{ Because of aliasing, it is not always obvious to VCC that a
%%   variable is not modified in the body of the loop. However, VCC can
%%   check it syntactically for a local variable if you never take the
%%   address of that variable.}
and assumes the invariant (which places some constraints on these
variables).  VCC doesn't have to consider the actual jump from the end
of the loop iteration back to the top of the loop (since it has
already checked the loop invariant), so further consideration of that
branch is cut off with \vcc{_(assume \false)}.  Each loop exit is
translated into a \vcc{goto} that jumps to just beyond the loop (to
\vcc{loopExit}). At this control point, we know the loop invariant
holds and that \vcc{lr < d}, which together imply that we have
computed the quotient and remainder.

For another, more typical example of a loop, consider 
the following function that uses linear search to determine if a value
occurs within an array:

\vccInput[linerange={begin-}]{c/5.3.lsearch_full.c}

\noindent
The postconditions say that the returned value is the minimal array
index at which \vcc{elt} occurs (or \vcc{UINT_MAX} if it does not occur).
The loop invariant says that \vcc{elt}  does not occur in \vcc{ar[0]}\dots
\vcc{ar[i - 1]}.

\subsection{Termination measures for loops}
\label{sect:loopTermination}
To prove that a loop terminates, it can be given a \vcc{_(decreases)}
clause, just as a function can. Before control returns from inside the
loop to the top of the loop, there is an implicit assertion that the
measure on the loop has gone down from its value at the beginning of
the iteration. (Note that if the loop body contains a function call,
its measure is checked against the measure assigned to the function,
not to the loop.)

For example, in the \vcc{divide} function, we could specify that the
loop terminates by adding the specification \vcc{_(decreases lr)} to
the loop specification. This would then allow us to add the
specification \vcc{_(decreases 0)} to the divide function itself.

If a function with a termination measure contains a \vcc{for} loop
without a termination measure, VCC tries to guess one from syntactic
form of the loop header. Thus, most boilderplate \vcc{for} loops do
not require explicit termination measures.

Here's an example, a function that sorts an array using
bubblesort. VCC infers a termination measure for the outer loop, but 
currently needs to be given one for the inner loop:

\vccInput[linerange={begin-out}]{c/5.4.sort.c}

The specification that we use is that the output of the sorting routine is sorted.
However, it doesn't say that the output is a permutation of the input.
We'll show how to do that in \secref{sorting-perm}.

\subsection{Writes clauses for loops}
\label{sect:sorting}

Loops are in many ways similar to recursive functions.
Invariants work as the combination of pre- and post-conditions.
Similarly to functions loops can also have writes clauses.
You can provide a writes clause using exactly the same syntax
as for functions.
When you do not write any heap location in the loop (which has been
the case in all examples so far), VCC will automatically infer
an empty writes clause.
Otherwise, it will take the writes clause that is specified on
the function.
So by default, the loop is allowed to write everything that the function
can.
Here is an example of such implicit writes clause,
a reinterpretation of \vcc{my_memcpy()} from \secref{arrays}.

\vccInput[linerange={begin-end}]{c/5.5.copy_array.c}
(Note that VCC also inferred an appropriate
termination measure for the \vcc{for} loop.)

If a loop does not write everything the function can write
you will often want to provide explicit write clauses.
Here's a variation of \vcc{memcpy()}, which clears (maybe for security reasons)
the source buffer after copying it.

\vccInput[linerange={begin-end}]{c/5.6.copy_and_clear_array.c}

\noindent
If the second loops did not provide a writes clause,
we couldn't prove the first postcondition---VCC 
would think that the second loop could have overwritten \vcc{dst}.

\subsection*{Exercises}
Specify and verify iterative implementations of the following functions:
\begin{enumerate}
\item
a function that takes two arrays and checks whether
the arrays are equal (\ie whether they contain the same sequence of
elements); 
\item
a function that checks whether two sorted arrays
contain a common element;
\item
a function that checks whether a sorted array contains a given value;
\item
a function that takes an array and checks whether it
contains any duplicate elements;
\item
a function that takes an array and reverses it.
\end{enumerate}

Solutions can be found in the file \vcc{5.7.solutions.c} in the tutorial directory.
