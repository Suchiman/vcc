\section{Loop invariants}

For the most part, what VCC knows at a control point can be
computed from what it knew at the immediately preceding control
points. But when the control flow contains a loop, VCC faces a
chicken-egg problem, since what it knows at the top of the loop (i.e.,
at the beginning of each loop iteration) depends not only on what it
knew just before the loop, but also on what it knew just before it
jumped back to the top of the loop from the loop body.

Rather than trying to guess what it should know at the top of a loop,
VCC lets you tell it what it should know, by providing a \Def{loop
  invariant}. To make sure that the loop invariant does indeed hold 
whenever control reaches the top
of the loop, VCC asserts that the invariant holds wherever control
jumps to the top of the loop -- namely, on loop entry and at the end of
the loop body.

% Repeated below. --MM
%In addition, VCC knows at the top of a loop
%that any variable that is not modified in the loop has 
%the same value it had on entry to the loop.
%We simulate this by having VCC forget
%(at loop entry) everything it knew about variables that are modified
%in the loop body.

Let's look at an example:
\vccInput[linerange={begin-}]{c/02_div.c}
\noindent

The \vcc{divide()} function computes the quotient and remainder of integer division
of \vcc{x} by \vcc{d} using the classic division algorithm.
%% (Later, we'll see similar examples with sets, which are built into
%% the specification language, but not into C.)
The loop invariant says that we have a suitable answer, except with a
remainder that is possibly too big. VCC translates this example roughly as follows:
\vccInput[linerange={begin-end}]{c/02_div_assert.c}

\noindent
Note that this translation has removed all cycles from the control
flow graph of the function (even though it has gotos); this means that
VCC can use the rules of the previous sections to reason about the
program. In VCC, all program reasoning is reduced to reasoning about
acyclic chunks of code in this way.

Note that the invariant is asserted wherever control moves to the top of the
loop (here, on entry to the loop and at the end of the loop body). On
loop entry, VCC forgets the value of each variable modified in the
loop (in this case just the local variables \vcc{lr} and \vcc{ld}),%
\footnote{ Because of aliasing, it is not always obvious to VCC that a
  variable is not modified in the body of the loop. However, VCC can
  check it syntactically for a local variable if you never take the
  address of that variable.}
and assumes the invariant (which places some constraints on these
variables).  VCC doesn't have to consider the actual jump from the end
of the loop iteration back to the top of the loop (since it has
already checked the loop invariant), so further consideration of that
branch is cut off with \vcc{_(assume \false)}.  Each loop exit is
translated into a \vcc{goto} that jumps to just beyond the loop (to
\vcc{loopExit}). At this control point, we know the loop invariant
holds and that \vcc{lr < d}, which together imply that we have
computed the quotient and remainder.

%% If the loop had been of the form
%% \begin{VCC}
%% while (res >= b) 
%%   _( invariant a % b == res % b )
%% {
%%   res -= b;
%% }
%% \end{VCC}
%% the assert/assume translation would be exactly the same: the invariant
%% would be assumed before checking the guard, despite the fact that
%% guard syntactically precedes the invariant.

For another, more typical example of a loop, consider 
the following function that uses linear search to determine if a value
occurs within an array:

%% \noindent
%% The first postcondition guarantees that when the returned value
%% is not \vcc{UINT_MAX}, then \vcc{elt} is indeed found at the returned
%% index.
%% \itodo{It is required if we promise to return the first one; make this an exercise?}
%% This doesn't require an invariant, because the non-\vcc{UINT_MAX} index
%% is returned only under that condition.
%% However this specification is not full: what about the case when the result is not found?
%% The second postcondition says that 
%% To express that an element is not in an array we will use a universal quantifier:

\vccInput[linerange={begin-}]{c/03_lsearch_full.c}

%% Moved the discussion of quantification to the section on
%% assertions; this just isn't an appropriate place to talk about
%% quantifiers, because they are not specific to loop invariants. -E
\noindent
The postconditions say that the returned value is the minimal array
index at which \vcc{elt} occurs (or \vcc{UINT_MAX} if it does not occur).
The loop invariant says that \vcc{elt}  does not occur in \vcc{ar[0]}\dots
\vcc{ar[i - 1]}.


\subsection{Writes clauses for loops}
\label{sect:sorting}

Loops are in many ways similar to recursive functions.
Invariants work as the combination of pre- and post-conditions.
Similarly to functions loops can also have writes clauses.
You can provide a writes clause using exactly the same syntax
as for functions.
When you do not write any heap location in the loop (which has been
the case in all examples so far), VCC will automatically infer
an empty writes clause.
Otherwise, it will take the writes clause that is specified on
the function.
So by default, the loop is allowed to write everything that the function
can.
Let's see an example of such implicit writes clause,
a reinterpretation of \vcc{my_memcpy()} from \secref{TODO}.

\vccInput[linerange={begin-end}]{c/04_copy_array.c}

If the loop does not write everything the function can write
you will often want to provide explicit write clauses.
Here's a variation of \vcc{memcpy()}, which clears (maybe for security reasons)
the source buffer after copying it.

\vccInput[linerange={begin-end}]{c/04_copy_and_clear_array.c}

\noindent
If the second loops did not provide a writes clause,
we couldn't prove the first postcondition---VCC 
would think that the second loop could have overwritten \vcc{dst}.

\itodo{one sorting example is needed if we want people to do other,
maybe we should use bubble or insertion sort though}
Equipped with that knowledge we can proceed to not only checking
if an array is sorted, as we did in \secref{TODO}, but to actually
sorting it.
The function below implements the bozo-sort algorithm.
The algorithm works by swapping two random elements in an array, checking if the resulting array
is sorted, and repeating otherwise.
We do not recommend using it in production code:
it's not stable, and moreover has a fairly bad time complexity.

\vccInput[linerange={begin-out}]{c/04_bozosort.c}

The specification that we use is that the output of the sorting routine is sorted.
Unfortunately, we do not say that it's actually a permutation of the input.
We'll show how to do that in \secref{sorting-perm}.


\subsection*{Exercises}
Return to the verification exercises of section \secref{recursiveExercises} and repeat them with iterative 
implementations.


%\Def{Ghost data} contains auxiliary information needed to convince VCC about correctness of a program.
%You can think of it as data that the program maintains for the purpose of debugging.
%An example might be a program which only keeps track of count of foobars, whereas
%the specifications of that program also need to use the set of these foobars.
%\Def{Ghost code} is code which manipulates such data (\ie when you increment the count of foobars
%you need to add the specific foobar to the set). Ghost variables are just pieces of local ghost
%data, and ghost functions are functions, which can be only used in specifications and ghost code.
%
%The regular C compiler doesn't see the ghost code.
%Therefore it has no runtime effect, it's only there to help VCC understand why the program works.

%We shall start with ghost functions, which in this case is just a
%macro for another formula.

% Let's leave this out for now -E
%\subsection{Review}
%\todo{Put in some way to get to the relevant info about quantifiers
%  and ghost data?}  
%
%\itodo{I'm not sure how useful this review is. In particular the parts
%where we explain the semantics of if statements and assignments
%in terms of what VCC know just seem confusing. Programmers already
%know what if statement or assignment does. 
%I think it would be useful just to emphasize what VCC doesn't know,
%for loops and function calls.
%--M }

%% You have now learned enough to verify some nontrivial sequential
%% programs that use only base types and arrays.  This is already a very
%% rich domain for programming, and you should take some time using VCC
%% to verify some of the ``toy'' algorithms you learned in school. It's
%% also a good opportunity to review what we've learned so far.

%% At each control point within a function, VCC ``knows'' certain things
%% about the state of the program. Included in this knowledge is what
%% memory locations it can safely read or write. The computation of this
%% knowledge can be summarized as follows:
%% \begin{itemize}
%% \item
%% On entry to a function, it knows the preconditions of the function.
%% \item
%% A memory object is writable if it is mutable and is either listed in the 
%% writes clause of the function or was mutable after the function was entered.
%% \item
%% After \vcc{_(assume E)}, it knows what it knew before the assumption,
%% and in addition knows \vcc{E != 0}.
%% \item
%% \vcc{_(assert E)} asks VCC to prove that what it knows before the
%% assertion implies \vcc{E} (and report an error if it can't). 
%% It also assumes \vcc{E} afterward.
%% %\item
%% %\todo{Break this up into variable assignment and memory assignment?}
%% %An assignment statement \vcc{v = E}, where \vcc{E} doesn't have a 
%% %function call and doesn't mention \vcc{v}%
%% %\footnote{
%% %  If \vcc{E} mentions \vcc{v}, we can imagine the value of \vcc{v} being
%% %  first copied into a fresh temporary variable, which is used in place
%% %  of \vcc{v} within \vcc{E}.
%% %}, asserts that the data needed to
%% %compute \vcc{E} is readable, and that \vcc{v} is writable. After the
%% %assignment, it knows everything it knew before the assignment (except
%% %for what it knew about the value of \vcc{v}), and additionally knows
%% %that \vcc{v == E}. 
%% \item 
%% A function call \vcc{f(args)}, 
%% %where \vcc{args} is a list of variables
%% %\footnote{
%% %  If the arguments to the function call are expressions, we can think
%% %  of these expressions being evaluated and assigned to temporary
%% %  variables before the function call.
%% %}, 
%% asserts that the \vcc{args} are readable, asserts that the objects
%% mentioned in the writes clauses of \vcc{f} are writable,
%% asserts the preconditions of \vcc{f} (with the actual parameters
%% substituted for the formal parameters), forgets what it knew about
%% \vcc{v} and any objects mentioned in the writes clause of \vcc{f}, and
%% finally assumes the postconditions of \vcc{f}.
%% If these postconditions use \vcc{\\old} to refer to parts of
%% the state before the call, we can think of these parts of the state
%% as copied to temporary variables prior to the call. The result of
%% the function can be viewed as being put into a temporary variable of
%% the caller.
%% %\item
%% %For a conditional \vcc{if (p) S1 else S2}, it first asserts that
%% %\vcc{p} is readable. At the beginning of \vcc{S1} (resp. \vcc{S2}), it
%% %knows what it knew before the conditional, and in addition knows
%% %\vcc{p != 0} (resp. \vcc{p == 0}). After the conditional, it knows the 
%% %disjunction (``or'') of what it knew at the end of \vcc{S1} and what
%% %we know at the end of \vcc{S2}.
%% \item
%% For a loop, it knows at the beginning of the loop body just what it
%% knew just before the loop (except that it forgets what it knew about
%% variables modified in the loop), and also knows the loop
%% invariant. Just before the loop, and at any point in the loop body where
%% control jumps back to the top of the loop (including the end of the
%% loop body), it asserts the loop invariant.
%% \end{itemize}

