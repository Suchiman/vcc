\section{Simple Programs}
We begin by describing how VCC verifies ``simple programs'' ---
sequential programs without loops, pointers, or function calls. This
might seem to be a trivial subset of C, but in fact VCC reasons about
more complex programs (with pointers, function calls, loops, and
concurrency) by reducing them to reasoning about simple programs.

\subsection{Assertions}
\label{sect:assert-assume}

%\itodo{Penny: This section dives right into the meat. First, you need two things: 
%1. A general guide to annotation syntax
%2. A section that describes how to think in VCC terms is, you have to describe
%not only exactly what your code does at each step of the way, but what your
%program *doesnt* do. This latter point will nail people on loop verification
%every time. 
%After you do the how to think part, I suggest some simple exercises, if this is in fact to be a tutorial. The whole book can come later. 
%}
%\itodo{Michal: I'm not sure about 1., 2. might be good idea.}

Let's begin with a simple example:

\vccInputSC[linerange={begin-}]{c/01_minInline2.c}
This program (internally) sets \vcc{z} to the minimum of \vcc{x} and
\vcc{y}. In addition to the ordinary C code, this program includes an
\Def{annotation}, starting with \vcc{_(}, terminating with a closing
parenthesis, with balanced parentheses inside. The first identifier
after the opening parenthesis (in the program above it's \vcc{assert})
is called an \Def{annotation tag} and
identifies the type of annotation provided (and hence its function).
(The tag plays this role only 
at the beginning of an annotation; elsewhere, it would be treated as
an ordinary program identifier.)

An annotation of the form \vcc{_(assert E)}, called an \Def{assertion}, asks VCC to prove that
\vcc{E} is guaranteed to hold (\ie, evaluate to a value other than \vcc{0})
whenever control reaches the assertion%
\footnote{
  This interpretation changes slightly if \vcc{E} refers to
  memory locations that might be concurrently modified by other
  threads; see \secref{concurrency}}.  
Thus, the line \vcc|_(assert z <= x)| says
that when control reaches the assertion, \vcc{z} is no larger than \vcc{x}.
When VCC successfully verifies a program, it promises that all assertions
hold in \emph{every} possible execution of the program, considering all possible inputs,
all possible ways of scheduling concurrent threads, etc. 
% The following doesn't really clarify much
%% An \Def{execution} is a complete recording of all actions made by the program:
%% which branches were taken, what values were read and stored in memory
%% and so forth.
%% For a deterministic program, once you fix input data (\eg by saying in this and that
%% microsecond the user is going to click at that pixel and so forth),
%% you fix the execution.
%% An assertion causes VCC to try to \emph{prove} that \vcc{E} holds for \emph{all}
%% executions.

% There are only a few of these notes, so let's avoid them for now -E
% Also, this note is something we probably want people to read.
%\todo{should these notes be less prominent?}
%\begin{note}
It is instructive to compare \vcc{_(assert E)} with the macro
\lstinline|assert(E)| (defined in \lstinline|<assert.h>|), which
evaluates \vcc{E}  at runtime and aborts execution if \vcc{E}
doesn't hold. First, \lstinline|assert(E)| requires runtime overhead (at least
in builds where the check is made), whereas \vcc{_(assert E)} does
not. Second, \lstinline|assert(E)| will catch failure of the 
assertion only when it actually fails in an execution, whereas 
\vcc{_(assert E)} is guaranteed to catch the failure if it is
possible in \emph{any} execution. Third, because VCC doesn't have to
actually execute \vcc{_(assert E)}, it allows \vcc{E} to include
mathematical operations that cannot be executed, such as
quantification over infinite domains.
%\end{note}

To verify the function using VCC from the command line, save the source in a file called \lstinline|minimum.c|
and run VCC as follows:

\begin{VCC}
/*`C:\Somewhere\VCC Tutorial> vcc.exe minimum.c
Verification of main succeeded.
C:\Somewhere\VCC Tutorial>`*/
\end{VCC}


If instead you wish to use VCC Visual Studio plugin, load the solution \lstinline|VCCTutorial.sln|
% moved the information on the samples to the preface - E
%% (distributed with VCC), 
%% \itodo{
%% This isn't yet true. The samples are available only with VCC sources, available from
%% \url{http://vcc.codeplex.com/SourceControl/list/changesets}: click Download on the right,
%% get the zip file and navigate to \lstinline|vcc/Docs/Tutorial/c|.
%% }
%% \todo{we need to package the examples, but only after it is clear which ones are included in the tutorial}
locate the file with the example, and right-click on the program text.
You should get options to verify the file or just this function (either will work).

\begin{note}
If you right-click within a C source file,
several VCC commands are made available, depending on what kind of
construction IntelliSense thinks you are in the middle of. The choice
of verifying the entire file is always available. If you click within
the definition of a struct type, VCC will offer you the choice of
checking admissibility for that type (a concept explained in \secref{admissibility0}).
If you click within the body of a function, VCC should offer
you the opportunity to verify just that function. However,
IntelliSense often gets confused about the syntactic structure of
VCC code, so it might not always present these context-dependent
options. However, if you select the name of a function and then right
click, it will allow you to verify just that function.
\end{note}

VCC verifies this function successfully, which means that its
assertions are indeed guaranteed to hold and that the program cannot
crash%
\footnote{
  VCC currently doesn't check that your program doesn't run forever or
  run out of stack space, but future versions will, at least for sequential
  programs.  
}
If VCC is unable
to prove an assertion, it reports an error.  Try changing the
assertion in this program to something that isn't true and see what
happens. (You might also want to try coming up with some complex
assertion that is true, just to see whether VCC is smart enough to
prove it.)

Assertions, like all VCC annotations, 
are surrounded by \vcc{_( ... )} to indicate that they are
only for VCC, and are not part of the program being verified
For the regular C compiler the \vcc{<vcc.h>} header file defines:
\begin{VCC}
#define _(...) /* nothing */
\end{VCC}
VCC does not use this definition, and instead parses the inside of \vcc{_( ... )}
annotations.%
% there's not much point in putting into the tutorial an obscure
% feature that doesn't work -E
%% \footnote{
%%   You can prevent definition of \vcc{_}, and instead surround annotations with
%%   \vcc{__vcc_spec(...)}.
%%   This is controlled with \vcc{_VCC_DONT_USE_UNDERSCORE} macro-definition.
%%   \itodo{This doesn't work at the moment.} }



To understand how VCC works, and to use it successfully, it is useful to
think in terms of what VCC ``knows'' at various program points. In
this example, VCC initially knows nothing about the variables (since they
can initially hold any value). 
Just before the first assignment, VCC knows that 
\vcc{x <= y} (because execution followed that branch of the conditional), and
after the assignment, VCC also knows that \vcc{z == x}, 
so it knows that \vcc{z <= x}. Similarly, in the \vcc{else} branch,
VCC knows that \vcc{y < x} (because execution didn't follow the
\vcc{if} branch), and after the assignment to \vcc{z}, it also knows
that \vcc{z == y}, so it knows \vcc{z <= x}. Since 
\vcc{z <= x} is known to hold at the end of each branch of the
conditional, it is known to hold at the end of the conditional, so the
assertion is known to hold when control reaches it. In general,
VCC doesn't lose any information when reasoning about assignments and
conditionals. However, we will see that VCC
 some information when reasoning about loops and function calls,
which means additional annotations are necessary.

When VCC surprises you by failing to verify something that you think
it should be able to verify, it is usually because it doesn't know
something you think it should know. An assertion provides one way to
check whether VCC knows what you think it knows.

% Too much information, too few full examples in one place. --M
% We can't use a complete program to introduce every single thing,
% especially those that should be easy to understand. Spreading things
% out over the text just makes it harder to grasp what facilities are
% available.
%

VCC provides a number of C extensions that can be used within VCC
annotations (such as assertions):
\begin{itemize}
\item
The Boolean operator \vcc{==>} denotes logical implication; formally,
\vcc{P ==> Q} means \vcc{((!P) || (Q))}, and is usually 
read as ``\vcc{P} implies \vcc{Q}''. Because \vcc{==>} has lower
precedence than the C operators, it is typically not necessary to
parenthesize \vcc{P} or \vcc{Q}.

\item
The expression \vcc{\forall T v; E} evaluates to \vcc{1} if the
expression \vcc{E} evaluates to a nonzero value for every value 
\vcc{v} of type \vcc{E}. For example, the assertion
\begin{VCC}
_(assert x > 1 && \forall int i; 1 < i && i < x ==> x % i != 0)
\end{VCC}
\noindent says that (\vcc{int}) \vcc{x} is a prime number. If \vcc{b}
is an \vcc{int} array of size \vcc{N},
\begin{VCC}
_(assert \forall int i; \forall int j; 0 <= i && i <= j && j < N ==>
  b[i] <= b[j])
\end{VCC}
says that \vcc{b} is sorted.

\item
Similarly, the expression \vcc{\exists T v; E} evaluates to \vcc{1} if there
is some value of \vcc{v} of type \vcc{T} for which \vcc{E} evaluates
to a nonzero value. For example, if \vcc{b} is an \vcc{int} array of
size \vcc{N}, the assertion
\begin{VCC}
_(assert \exists int i; 0 <= i && i < N && b[i] == 0)
\end{VCC}
asserts that  \vcc{b} contains a zero element.
\vcc{\forall} and \vcc{\exists} are jointly referred to as
\Def{quantifiers}. 

\item
VCC also provides some mathematical types that cannot be used in
ordinary C code (because they are too big to fit in memory);
these include mathematical integers and maps. They are described in
\secref{mathTypes}.

\item
Expressions within VCC annotations are restricted in their use of 
functions: you can only use functions that are proved to be 
\Def{pure}, \ie free from side effects (\secref{pureFunctions}).
\end{itemize}

%% Within a VCC declaration, the type \vcc|T[E]| where \vcc{T} is
%% a type and \vcc{E} is an expression that is convertible to an
%% \vcc{int}, is the type of arrays of \vcc{E} elements of type
%% \vcc{T}. For example, the declaration
%% \begin{VCC}
%% void copyArray(int *from, int *to, int len)
%% _(requires \\thread_local((int[len]) from))
%% _(writes ((int[len]) to))
%% ...
%% \end{VCC}
%defines a function that can read the first \vcc{len} elements of the
%array \vcc{from}, and read or write the first \vcc{len} elements of
%the array \vcc{to}.
%
%The expression \vcc{\old(E)}, when used in a postcondition of a
%function, returns the value of expression \vcc{E} as it was at just
%before the function was called. For example, the postcondition 
%\vcc{*x == \old(*x) + 1} says that the function under consideration
%increments \vcc{*x}.

\subsection*{Exercises}
\begin{enumerate}
\item Write an assertion that says that the \vcc{int} \vcc{x} is the average of
  the \vcc{int}s \vcc{y} and \vcc{z}.
\item Write an assertion that says that the \vcc{int} \vcc{x} is a
  perfect square.
\item Write an assertion that says that the \vcc{int} \vcc{x} occurs
  in the \vcc{int} array \vcc{b[10]}.
\item Write an assertion that says that the \vcc{int} array \vcc{b},
  of length \vcc{N}, contains no duplicates.
\item Write an assertion that says that all pairs of adjacent elements
  of the \vcc{int} array \vcc{b} of length \vcc{N} differ by at most
  \vcc{1}.
\item Write an assertion that says that an array \vcc{b} of length
  \vcc{N} contains only perfect squares.
\end{enumerate}

\subsection{Assumptions}

You can add to what VCC knows at a particular point with a second
type of annotation, called an \Def{assumption}.
The assumption \vcc{_(assume E)} tells VCC to ignore
the rest of an execution if \vcc{E} fails to hold (\ie if
\vcc{E} evaluates to \vcc{0}). 
Reasoning-wise, assumption simply adds \vcc{E} to what VCC
knows for subsequent reasoning. For example:
\begin{VCC}
int x, y;
_(assume x != 0)
y = 100 / x;
\end{VCC}
Without the assumption, VCC would complain about possible division by
zero (it checks for division by zero because it would cause the
program to crash).  Assuming the assumption, this error cannot happen.  
Since assumptions (like all annotations) are not seen by the compiler,
assumption failure won't cause the program to stop, and subsequent assertions
might be violated. To put it another way, if VCC verifies a program,
it guarantees that in any prefix of an execution
where all (user-provided) assumptions hold, all assertions will also
hold. Thus, your verification goal should be to eliminate as many assumptions as
possible (preferably all of them).

Although assumptions are generally to be avoided, they are nevertheless
sometimes useful:
\begin{itemize}
\item In an incomplete verification, assumptions can be used to mark
  the knowledge that VCC is missing, and to coordinate further
  verification work (possibly performed by other people). If you
  follow a discipline of keeping your code in a state where the whole 
  program verifies, the verification state can be judged by browsing
  the code (without having to run the verifier).

\item When debugging a failed verification, you can use assumptions to
  narrow down the failed verification to a more specific failure
  scenario, perhaps even to a complete counterexample. 
% We should have chapter about it, not a cryptic mention here. --M
%  You can then
%  add additional (non-assumption) annotations to eliminate the
%  counterexample, weaken assumptions until you either hit another
%  counterexample (or finish the verification), and repeat the process.

\item Sometimes you want to assume something even though VCC can
  verify it, just to stop VCC from spending time proving it. For
  example, assuming \vcc{\false} allows VCC to 
  easily prove subsequent assertions, thereby focussing its
  attention on other parts of the code. Temporarily adding assumptions
  is a common tactic when developing annotations for complex functions.

\item Sometimes you want to make assumptions about the operating
  environment of a program. For example, you might want to assume that
  a 64-bit counter doesn't overflow, but don't want to justify it
  formally because it depends on extra-logical assumptions (like the
  speed of the hardware or the lifetime of the software). 

\item Assumptions provide a useful tool in explaining how VCC
  reasons about your program. We'll see examples of this throughout
  this tutorial.
\end{itemize}

%%  What is however much more common is
%% use of assumption when debugging your specifications: when the program
%% does not verify you might try to temporarily add assumptions in a bet
%% ``if VCC knew that, would it verify my program?''.  Once you find what
%% is that thing VCC doesn't know, but what makes it verify the program,
%% you can figure out how to specify rest of your program, so that is
%% knows it.

% The following is a bad example, since we would normally specify
% hardware by an invariant on a volatile field.
%% One case is interaction with hardware: you might know the hardware is
%% going to leave specific I/O register positive, and an assumption is a
%% way to communicate that to VCC. 

An assertion can also change what VCC knows after the assertion, if
the assertion fails to verify: although VCC will report the failure as an error,
it will assume the asserted fact holds afterward. For example, in the following
VCC will only report an error for the first assumption and not the second:
\begin{VCC}
int x;
_(assert x == 0)
_(assert x == 0)
\end{VCC}

When we talk about what VCC knows, we mean what it knows in an ideal
sense, where if it knows \vcc{E}, it also knows any logical
consequence of \vcc{E}. In such a world, adding an assertion that
succeeds would have no effect on whether later assertions succeed.
VCC's ability to deduce consequences is indeed complete for many types
of formulas (e.g. formulas that use only equalities,
inequalities, addition, subtraction, multiplication by constants, and
boolean connectives), but not for all formulas, so VCC will
sometimes fail to prove an assertion, even though it ``knows'' enough
to prove it.  Conversely, an assertion that succeeds can sometimes cause later assertions that
would otherwise fail to succeed, by drawing VCC's attention to a
crucial fact it needs to prove the later assertion.  This is
relatively rare, and typically involves ``nonlinear arithmetic''
(e.g. where variables are multiplied together), bitvector reasoning
(\secref{bv}) or quantifiers.

\subsection*{Exercises}
\begin{enumerate}
\item
In the following program fragment, which assertions will fail?
\begin{VCC}
int x,y; 
_(assert x > 5) 
_(assert x > 3) 
_(assert x < 2) 
_(assert y < 3)
\end{VCC}
\item
Is there any difference between 
\begin{VCC}
_(assume p)
_(assume q)
\end{VCC}
and 
\begin{VCC}
_(assume q) 
_(assume p)
\end{VCC}
? What if the assumptions are replaced by assertions?
\item
Is there any difference between
\begin{VCC}
_(assume p)
_(assert q)
\end{VCC}
and 
\begin{VCC}
_(assert (!p) || (q))
\end{VCC}
? 

% we need to have a place where we collect all of the new operators -E
%% \vcc{p ==> q} means \vcc{p} implies \vcc{q}, and formally \vcc{!p || q}.
%\item
%\itodo{I'm not sure if this and the following exercise give
%any insight. --M}
% the objective is to let readers think about strongest postconditions
% and weakest preconditions -E
%Given the assumptions preceding each of the assignments below to int
%variables, what are the strongest assertions that would be valid?
%\begin{VCC}
%_(assume x < 5) x += 5; _(assert ??)
%_(assume u < v) u = v;  _(assert ??)
%_(assume a < b && a < c) a = b;  _(assert ??)
%\end{VCC}
%\item
%Given the assertions following each of the assignments below, what is 
%the weakest preceding assumption that would allow the assertions to be
%valid?
%\begin{VCC}
%_(assume ??) x = y _(assert x < 10)
%_(assume ??) u += 3 _(assert 2*u == 3*v)
%\end{VCC}
\end{enumerate}

\subsection{Havoc}
%% \itodo{Add havoc to VCC}

An assignment of the form \vcc{x = E}, where \vcc{E} is an expression
that doesn't mention \vcc{x}, is equivalent to setting \vcc{x} to an
arbitrary value of its type (an operation we write as \vcc{havoc(x)}%
), followed by the assumption that \vcc{x==y}. This can be naturally
generalized to havocing a collection of variables and then assuming
something about the resulting state (to constrain the chosen values of
these variables).

This idiom is central to how VCC reasons about more interesting kinds
of programs. Whenever something complex happens in the body of a
function (such as a loop, a function call, or interference from other
threads), VCC replaces it with an abstraction of this form, to reduce
reasoning about the original program to reasoning about a simple
program.  The annotations you put on the program determine the
abstraction that is used; successful verification guarantees that these
abstractions are \Def{sound}, \ie that each complex behavior
implements the abstraction used to replace it.




