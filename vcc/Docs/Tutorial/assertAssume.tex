\section{Verifying Simple Programs}
We begin by describing how VCC verifies ``simple programs'' ---
sequential programs without loops,  function calls, or concurrency. This
might seem to be a trivial subset of C, but in fact VCC reasons about
more complex programs by reducing them to reasoning about simple programs.

\subsection{Assertions}
\label{sect:assert-assume}

Let's begin with a simple example:

\vccInputSC[linerange={begin-}]{c/01_minInline2.c}
This program sets \vcc{z} to the minimum of \vcc{x} and
\vcc{y}. In addition to the ordinary C code, this program includes an
\Def{annotation}, starting with \vcc{_(}, terminating with a closing
parenthesis, with balanced parentheses inside. The first identifier
after the opening parenthesis (in the program above it's \vcc{assert})
is called an \Def{annotation tag} and
identifies the type of annotation provided (and hence its function).
(The tag plays this role only 
at the beginning of an annotation; elsewhere, it is treated as
an ordinary program identifier.)
Annotations are used only by VCC, and are not seen by the C compiler.
When using the regular C compiler the \vcc{<vcc.h>} header file defines:
\begin{VCC}
#define _(...) /* nothing */
\end{VCC}
VCC does not use this definition, and instead parses the inside of \vcc{_( ... )}
annotations.


An annotation of the form \vcc{_(assert E)}, called an \Def{assertion}, asks VCC to prove that
\vcc{E} is guaranteed to hold (\ie evaluate to a value other than \vcc{0})
whenever control reaches the assertion.%
\footnote{
  This interpretation changes slightly if \vcc{E} refers to
  memory locations that might be concurrently modified by other
  threads; see \secref{concurrency}}  
Thus, the line \vcc|_(assert z <= x)| says
that when control reaches the assertion, \vcc{z} is no larger than \vcc{x}.
If VCC successfully verifies a program, it promises that this will hold throughout 
every possible execution of the program, regardless of inputs, how concurrent
threads are scheduled, etc. More generally, VCC might verify some, but not all of your assertions; 
in this case, VCC promises that the first assertion to be violated will not be one of the verified ones.

\begin{note}
It is instructive to compare \vcc{_(assert E)} with the macro
\lstinline|assert(E)| (defined in \lstinline|<assert.h>|), which
evaluates \vcc{E}  at runtime and aborts execution if \vcc{E}
doesn't hold. First, \lstinline|assert(E)| requires runtime overhead (at least
in builds where the check is made), whereas \vcc{_(assert E)} does
not. Second, \lstinline|assert(E)| will catch failure of the 
assertion only when it actually fails in an execution, whereas 
\vcc{_(assert E)} is guaranteed to catch the failure if it is
possible in \emph{any} execution. Third, because
\vcc{_(assert E)} is not actually executed, \vcc{E} can include
unimplementable mathematical operations, such as
quantification over infinite domains.
\end{note}

To verify the function using VCC from the command line, save the source in a file called \lstinline|minimum.c|
and run VCC as follows:

\begin{VCC}
/*`C:\Somewhere\VCC Tutorial> vcc.exe minimum.c
Verification of main succeeded.
C:\Somewhere\VCC Tutorial>`*/
\end{VCC}


If instead you wish to use VCC Visual Studio plugin, load the solution \lstinline|VCCTutorial.sln|
\todo{make sure that VCC can verify code not in the project}
locate the file with the example, and right-click on the program text.
You should get options to verify the file or just this function (either will work).

If you right-click within a C source file,
several VCC commands are made available, depending on what kind of
construction IntelliSense thinks you are in the middle of. The choice
of verifying the entire file is always available. If you click within
the definition of a struct type, VCC will offer you the choice of
checking admissibility for that type (a concept explained in \secref{admissibility0}).
If you click within the body of a function, VCC should offer
you the opportunity to verify just that function. However,
IntelliSense often gets confused about the syntactic structure of
VCC code, so it might not always present these context-dependent
options. However, if you select the name of a function and then right
click, it will allow you to verify just that function.

VCC verifies this function successfully, which means that its
assertions are indeed guaranteed to hold and that the program cannot
crash.%
\footnote{
  VCC currently doesn't check that your program doesn't run forever or
  run out of stack space, but future versions will, at least for sequential
  programs.  
}
If VCC is unable
to prove an assertion, it reports an error.  Try changing the
assertion in this program to something that isn't true and see what
happens. (You might also want to try coming up with some complex
assertion that is true, just to see whether VCC is smart enough to
prove it.)

To understand how VCC works, and to use it successfully, it is useful to
think in terms of what VCC ``knows'' at each control point
of your program. In the current example, just before the first conditional,
VCC  knows nothing about the local variables,
since they can initially hold any value. 
Just before the first assignment, VCC knows that 
\vcc{x <= y} (because execution followed that branch of the conditional), and
after the assignment, VCC also knows that \vcc{z == x}, 
so it knows that \vcc{z <= x}. Similarly, in the \vcc{else} branch,
VCC knows that \vcc{y < x} (because execution didn't follow the
\vcc{if} branch), and after the assignment to \vcc{z}, it also knows
that \vcc{z == y}, so it also knows \vcc{z <= x}. Since 
\vcc{z <= x} is known to hold at the end of each branch of the
conditional, it is known to hold at the end of the conditional, so the
assertion is known to hold when control reaches it.
In general,
VCC doesn't lose any information when reasoning about assignments and
conditionals. However, we will see that VCC may lose
loses some information when reasoning about loops,
necessitating additional annotations.

When we talk about what VCC knows, we mean what it knows in an ideal
sense, where if it knows \vcc{E}, it also knows any logical
consequence of \vcc{E}. In such a world, adding an assertion that
succeeds would have no effect on whether later assertions succeed.
VCC's ability to deduce consequences is indeed complete for many types
of formulas (e.g. formulas that use only equalities,
inequalities, addition, subtraction, multiplication by constants, and
boolean connectives), but not for all formulas, so VCC will
sometimes fail to prove an assertion, even though it ``knows'' enough
to prove it.  Conversely, an assertion that succeeds can sometimes cause later assertions that
would otherwise fail to succeed, by drawing VCC's attention to a
crucial fact it needs to prove the later assertion.  This is
relatively rare, and typically involves ``nonlinear arithmetic''
(e.g. where variables are multiplied together), bitvector reasoning
(\secref{bv}) or quantifiers.

When VCC surprises you by failing to verify something that you think
it should be able to verify, it is usually because it doesn't know
something you think it should know. An assertion provides one way to
check whether VCC knows what you think it knows.


\subsection*{Exercises}
\begin{enumerate}
\item Can the assertion at the end of the example function be made
  stronger? What is the strongest valid assertion one could write? Try 
  verifying the program with your stronger assertion.
\item Write an assertion that says that the \vcc{int} \vcc{x} is the average of
  the \vcc{int}s \vcc{y} and \vcc{z}.
\item Modify the example program so that it sets \vcc{x} and \vcc{y} to values
that differ by at most \vcc{1} and sum to \vcc{z}. Prove that your program 
does this correctly.
\item Write an assertion that says that the \vcc{int} \vcc{x} is a
  perfect square (\ie a number being a square of an integer).
\item Write an assertion that says that the \vcc{int} \vcc{x} occurs
  in the \vcc{int} array \vcc{b[10]}.
\item Write an assertion that says that the \vcc{int} array \vcc{b},
  of length \vcc{N}, contains no duplicates.
\item Write an assertion that says that all pairs of adjacent elements
  of the \vcc{int} array \vcc{b} of length \vcc{N} differ by at most
  \vcc{1}.
\item Write an assertion that says that an array \vcc{b} of length
  \vcc{N} contains only perfect squares.
\end{enumerate}

\subsection{Assumptions}

You can add to what VCC knows at a particular point with a second
type of annotation, called an \Def{assumption}.
The assumption \vcc{_(assume E)} tells VCC to ignore
the rest of an execution if \vcc{E} fails to hold (\ie if
\vcc{E} evaluates to \vcc{0}). 
Reasoning-wise, the assumption simply adds \vcc{E} to what VCC
knows for subsequent reasoning. For example:
\begin{VCC}
int x, y;
_(assume x != 0)
y = 100 / x;
\end{VCC}
Without the assumption, VCC would complain about possible division by
zero. (VCC checks for division by zero because it would cause the
program to crash.)  Assuming the assumption, this error cannot happen.  
Since assumptions (like all annotations) are not seen by the compiler,
assumption failure won't cause the program to stop, and subsequent assertions
might be violated. To put it another way, if VCC verifies a program,
it guarantees that in any prefix of an execution
where all (user-provided) assumptions hold, all assertions will also
hold. Thus, your verification goal should be to eliminate as many assumptions as
possible (preferably all of them).

\begin{note}
Although assumptions are generally to be avoided, they are nevertheless
sometimes useful:
\begin{inparaenum}
\item In an incomplete verification, assumptions can be used to mark
  the knowledge that VCC is missing, and to coordinate further
  verification work (possibly performed by other people). If you
  follow a discipline of keeping your code in a state where the whole 
  program verifies, the verification state can be judged by browsing
  the code (without having to run the verifier).

\item When debugging a failed verification, you can use assumptions to
  narrow down the failed verification to a more specific failure
  scenario, perhaps even to a complete counterexample. 

\item Sometimes you want to assume something even though VCC can
  verify it, just to stop VCC from spending time proving it. For
  example, assuming \vcc{\false} allows VCC to 
  easily prove subsequent assertions, thereby focussing its
  attention on other parts of the code. Temporarily adding assumptions
  is a common tactic when developing annotations for complex functions.

\item Sometimes you want to make assumptions about the operating
  environment of a program. For example, you might want to assume that
  a 64-bit counter doesn't overflow, but don't want to justify it
  formally because it depends on extra-logical assumptions (like the
  speed of the hardware or the lifetime of the software). 

\item Assumptions provide a useful tool in explaining how VCC
  reasons about your program. We'll see examples of this throughout
  this tutorial.
\end{inparaenum}
\end{note}

An assertion can also change what VCC knows after the assertion, if
the assertion fails to verify: although VCC will report the failure as an error,
it will assume the asserted fact holds afterward. For example, in the following
VCC will only report an error for the first assumption and not the second:
\begin{VCC}
int x;
_(assert x == 1)
_(assert x > 0)
\end{VCC}

\subsection*{Exercises}
\begin{enumerate}
\item
In the following program fragment, which assertions will fail?
\begin{VCC}
int x,y; 
_(assert x > 5) 
_(assert x > 3) 
_(assert x < 2) 
_(assert y < 3)
\end{VCC}
\item
Is there any difference between 
\begin{VCC}
_(assume p)
_(assume q)
\end{VCC}
and 
\begin{VCC}
_(assume q) 
_(assume p)
\end{VCC}
? What if the assumptions are replaced by assertions?
\item
Is there any difference between
\begin{VCC}
_(assume p)
_(assert q)
\end{VCC}
and 
\begin{VCC}
_(assert (!p) || (q))
\end{VCC}
? 

\end{enumerate}
