\section{Function Contracts}
\label{sect:functions}

Next we turn to the specification of functions. We'll take the example
from the previous section, and pull the computation of the minimum of
two numbers out into a separate function:

\vccInput[linerange={begin-}]{c/01_min2.c}

(The listing above presents both the source code and the output
of VCC, typeset in a different fonts, and 
the actual file name of the example is replaced with \vcc{/*`testcase`*/}.)
VCC failed to prove our assertion, even though it's easy to see that
it always holds. This is because verification in VCC is \Def{modular}: 
VCC doesn't look inside the body of a function (such as the definition of \vcc{min()}) 
when understanding the effect of a call to the function (such as 
the call from \vcc{main()})%
\footnote{We will see later how to selectively override this default
  and perform such inlining during verification.}; 
all it knows about the effect of calling \vcc{min()} is that the call 
satisfies the specification of \vcc{min()}. 
%This is just because we have chosen the default to be so. It thus doesn't
% contradict that specification needs to say everything --M
%\footnote{
%  Actually, VCC does know one important thing from this specification
%  of \vcc{min()}: it knows that a call do \vcc{min()} has no side
%  effects that are visible to the caller. We'll see in \secref{writes} that
%  any such side effects have to be explicitly specified.
%}.
Since the correctness of \vcc{main()} clearly depends on what \vcc{min()}
does, we need to specify \vcc{min()} in order to verify \vcc{main()}.

The specification of a function is sometimes called its \Def{contract},
because it gives obligations on both the function and its callers. It
is provided by three types of annotations:
\begin{itemize}
\item A requirement on the caller (sometimes called the
  \Def{precondition} of the function) takes the form \vcc{_(requires E)}, 
  where \vcc{E} is an expression. It says that callers of the
  function promise that \vcc{E} will hold on function entry. 

\item A requirement on the function (sometimes called a
  \Def{postcondition} of the function) takes the form \vcc{_(ensures E)}, 
  where \vcc{E} is an expression. It says that the function
  promises that \vcc{E} holds just before control exits the
  function. 

\item The third type of contract annotation, a \Def{writes clause}, is described
  in the next section.
\end{itemize}

For example, we can provide a suitable specification for \vcc{min()} as
follows:
\vccInput[linerange={min-endmin,out-}]{c/01_min3.c}
\noindent
(Note that the specification of the function comes after the header and
before the function body; you can also put specifications on function
declarations (\eg in header files).)
The precondition \vcc{_(requires \true)} of \vcc{min()} doesn't
really say anything (since \vcc{\true} holds in every state), but is included
to emphasize that the function can be called from any state and
with arbitrary parameter values.
The postcondition states that the value returned from \vcc{min()} 
is no bigger than either of the inputs.
Note that \vcc{\true} and \vcc{\result} are spelled with a backslash
to avoid name clashes with C identifiers.%
\footnote{
  All VCC keywords start with a backslash; 
  this contrasts with annotation tags (like \vcc{requires}),
  which are only used at the beginning of annotation
  and therefore cannot be confused with C identifiers
  (and thus you are still free to have, \eg
  a function called \lstinline{requires} or \lstinline{assert}).}

VCC uses function specifications as follows. When verifying the body of a
function, VCC implicitly assumes each precondition of the function on
function entry, and implicitly asserts each postcondition of the
function (with \vcc{\result} bound to the return value and each
parameter bound to its value on function entry) just before the
function returns. For every call to the function, VCC replaces the
call with an assertion of the preconditions of the function, sets the
return value to an arbitrary value, and finally assumes each
postcondition of the function. 
For example, VCC translates the program above roughly as follows:
\vccInput[linerange={begin-end}]{c/01_min_assert.c}

Note that the assumptions above are ``harmless'', that is in a fully
verified program they will be never violated, as they follow
from assertion that proceed them.
For example, the precondition assumption would only fail if the precondition
assertion before it fails.

\begin{note}
\notehd{Why modular?}
Modular verification brings several benefits. 
First, it allows verification to more easily scale to
large programs. Second, by providing a precise interface between
caller and callee, it allows you to modify the implementation of
a function like \vcc{min()} without having to worry about breaking the
verifications of functions that call it (as long as you don't change
the specification of \vcc{min()}). This is especially important
because these callers normally aren't in scope, and the person
maintaining \vcc{min()} might not even know about them (e.g., if
\vcc{min()} is in a library). Third, you can verify a function like
\vcc{main()} even if the implementation of \vcc{min()} is unavailable
(\eg if it hasn't been written yet). 
\end{note}

\subsection*{Exercises}
\begin{enumerate}
\item
Try replacing the \vcc{<} in the return statement of \vcc{min()} with
\vcc{>}. Before running VCC, can you guess which parts of the
verification will fail?
%% One way to illustrate that the verification of \vcc{main()} 
%% depends only on the specification of \vcc{min()} is to 
%% replace \vcc{<} with \vcc{>} in the return statement of \vcc{min()}.
%% The output of verification shows that \vcc{main()} is still correct---only 
%% \vcc{min()} is broken:
%% \vccInput[linerange={out-}]{c/01_min4.c}

\item
What is the effect of giving a function the specification
\vcc{_(requires \false)} ? How does it effect verification of
the function itself? What about its callers? Can you think of a good 
reason to use such a specification?
%% When verifying function body it translates to an assumption.
%% As we have noted this will make the body of function verify,
%% regardless what it contains.
%% On the other hand, you will be unable to verify call to such a function
%% from outside.
%% Conversely, \vcc{_(ensures \false)} will prevent verification
%% of the body of the function, but will make the function that calls
%% it verify.

%% As we can see, in each case there is something that does not verify.
%% However, because of this you should be careful when interpreting
%% VCC answers: successful verification of a function is only meaningful
%% if everything it calls was verified.

\item
Can you see anything wrong with the above specification of \vcc{min()}?
Can you give a simpler implementation than the one presented? Is this 
specification strong enough to be useful? If not, how might it be
strengthened to make it more useful?

\item
Specify a function that returns the (int) square root of its (int)
argument. (You can try writing an implementation for the function, but
won't be able to verify it until you've learned about loop
invariants.)

\item
Can you think of useful functions in a real program that might
legitimately guarantee only the trivial postcondition \vcc{_(ensures \true)}?
\end{enumerate}

\subsection{Reading and writing}
\label{sect:writes}
When programming in C, it is important to distinguish two kinds of
memory access. \Def{Sequential} access, which is the default, is appropriate when
interference from other threads (or the outside world) is not an
issue, \eg when accessing unshared memory. Sequential accesses can be
safely optimized by the compiler by breaking it up into multiple
operations, caching reads, reordering operations, etc. 
\Def{Atomic} access is required when the access might race with other
threads, \eg a write to memory that is concurrently read or written,
or a read to memory that is concurrently written. In this section, we
consider only sequential access to memory, atomic access is considered
in \secref{concurrency}. 

For each memory access within a program, VCC checks that the access is accessing a 
\Def{valid} memory object. Validity implies that the object address
points to memory that is actually in the address space of the program
(i.e., it has been allocated, either on the stack or on the heap, and
has not been freed). (Validity in VCC additionally requires that the
access is appropriately typed; this aspect is 
described in more detail in \secref{type-safety}). 

To safely read memory, we additionally need to know that the memory
is not concurrently modified by other threads; such an object is said
to be \Def{readable}. In general, readability can depend on complex
program reasoning (and perhaps hints from the user), so by default,
VCC checks a stronger condition, called \Def{thread
  locality}. The precise definition of thread locality is given in 
\secref{ownership}, but intuitively, an object is
considered thread-local if it is controlled by the thread under
consideration. For sequential programs, readability essentially
reduces to simple validity; we point out thread-locality only because
if VCC is unable to check that a read is safe, it will report that it
was unable to show that the object read was thread-local.

You can specify that an object (pointed to by) a pointer \vcc{p} is
thread-local by the expression \vcc{\thread_local(p)}. 
The expression \vcc{\thread_local_array(ar, sz)} is
syntactic sugar for \vcc{\forall unsigned i; i < sz ==> \thread_local(&ar[i])}.

While thread locality is what is checked on a read, in practice it is
not often used in the specifications of functions with side effects.%
\footnote{This is because the thread-locality of an object could be
  falsified by the unwrapping of other objects; see
  \secref{invariants}).} 
More usual is to provide the stronger
precondition of the object being \vcc{\mutable} (also described in
\secref{ownership}), which essentially says that object is
private to the current thread; you should generally use this stronger
condition in specifications.%
\footnote{The other useful idiom is to require that the read object
  lies in the sequential ownership domain of some object that is
  wrapped, but not written, in the function, which guarantees that it
  remains thread-local throughout the function; see
  \secref{ownership}.} Similar to \vcc{\thread_local_array},
\vcc{\mutable_array(ar,sz)} is defined as
\vcc{\forall unsigned i; i < sz ==> \mutable(&ar[i])}.


When a program writes to an object, that object must be
\Def{writable}. Writability requires the object to be \vcc{\mutable}
(because a sequential write can in general cause unpredictable results
for a concurrent read to the same memory in another thread). 
However, a stronger check is needed because, by default, VCC assumes that a
function call has no side effects on its caller; for example, in
verifying \vcc{main()} above, VCC assumed that the call of
\vcc{min()} didn't change \vcc{newPos}. 
In VCC, possible side effects are specified using 
another form of function specification, called \Def{writes clauses}.
Intuitively, the clause \vcc{_(writes p, q)} says that, 
of the memory objects that are thread-local to the caller before the call,
the function is going to modify only the object pointed to by \vcc{p}
and the object pointed to by \vcc{q}.%
\footnote{More precisely, the function can modify objects that were in
  the sequential ownership domains of \vcc{p} and \vcc{q}; see 
  \secref{ownership}.}
In other words, it is roughly equivalent to a postcondition that ensures
that all other objects local to the caller prior
to the call remain unchanged.%
\footnote{It's not quite the same when considering objects that are
  wrapped; see \secref{invariants}.}  VCC allows you
to write multiple writes clauses, and implicitly
combines them into a single set. If a function spec contains no writes clauses, 
it is equivalent to specifying a writes clause with empty set of
pointers.

More precisely, an object is \vcc{\writable} if it is \vcc{\mutable} and 
it is either listed in a writes clause of the function,
or it became \vcc{\mutable} sometime after the function was entered; the 
latter condition guarantees that either \vcc{p} was listed in the
writes clause or was not thread-local in the caller when the call to
the function was made. 
In particular, formal function parameters and local variables are
writable as long as they are in scope and have not been explicitly
wrapped (\secref{invariants}) or reinterpreted to a
different type (\secref{reint}).  VCC asserts
\vcc{\writable(p)} on each attempt to write to \vcc{*p}, as well as on
each call to a function that lists \vcc{p} in a writes clause.

Here is a simple example of a function that visibly reads and writes
memory; it simply copies data from one location to another.
\vccInput[linerange={begin-}]{c/01_copy1.c}
In the postcondition the expression \vcc{\old(E)} returns the value
the expression \vcc{E} had on function entry.
Thus, our postcondition states that the new value of \vcc{*to}
equals the value \vcc{*from} had on call entry. 
VCC translates the function call \vcc{copy(&x,&y)} approximately as
follows:
\begin{VCC}
_(assert \thread_local(&x))
_(assert \mutable(&y))
// record the value of x
int _old_x = x;
// havoc the written variables
havoc(y);
// assume the postcondition
_(assume y == _old_x)
\end{VCC}

% Shouldn't \array_range be allowed for arrays of non-primitives also?
Programs that manipulate arrays often write to multiple array
locations. Writes clauses actually allow sets of pointers, rather than
individual pointers. We'll introduce sets in full generality later, but
note one special case: the expression \vcc{\array_range(ar, len)}
denotes the set of pointers \vcc@{&ar[0], &ar[1], ..., &ar[len-1]}@. 
Thus, a writes clause of the form
\vcc{_(writes \array_range(ar,len))}
allows writing to all elements of the array.

\subsection{Exercises}
\begin{enumerate}
\item
Could the postcondition of \vcc{copy} have been written equivalently
as the simpler \vcc{*to == *from}? If not, why not?
\item
Specify a program that takes two arrays and checks whether
the arrays are equal (\ie whether they contain the same sequence of
elements).
\item
Specify a program that takes an array and checks whether it
contains any duplicate elements.
\item
Specify a program that checks whether a given array
is a palindrome.
\item
Specify a program that checks whether two arrays contain a
common element.
\item
Specify, write, and verify a function that swaps the values pointed to
by two \vcc{int} pointers. Hint: use \vcc{\old(...)} in the
postcondition.
\item
Specify a function that copies one array to another array of the same
size. (Don't worry about array aliasing.)
\item 
Specify a function that takes a text (stored in an array) and a string
(stored in an array) and checks whether the string occurs within the
text.
\end{enumerate}

\subsection{Pure Functions and Logic Functions}
\label{sect:pureFunctions}
A \Def{pure function} is one that has no side effects on the program
state. In VCC, pure functions are not allowed to allocate memory, and can
write only to local variables. Only pure functions can be called within VCC
annotations. The function \vcc{min()} from \secref{functions} is an
example of a function that can be declared to be pure; this is done by
adding the modifier \vcc{_(vcc ispure)} to the beginning of the function specification,
\eg
\begin{VCC}
_(vcc ispure) min(int x, int y) ...
\end{VCC}

Being pure is a stronger condition that simply having an empty writes
clause. This is because a writes clause has only to mention those side
effects that might cause the caller to lose information (\ie
knowledge) about the state, and as we have seen, VCC takes advantage
of the kind of information callers maintain to limit the kinds of side
effects that have to be reported.

Sometimes you want a spec function that performs a logical operation
that cannot be performed easily with C, such as a quantification. (Of
course such functions can only be used in annotations, not in actual
code.) VCC lets you define such \Def{logic functions} using the syntax

\begin{VCC}
_(logic F(A) = E)
\end{VCC}
which is roughly equivalent to 
\begin{VCC}
#define F(A) (E)
\end{VCC}
except that it prevents
name capture and gives better error messages (\ie ones which mention
both \vcc{E} and \vcc{F(A)}). 
 
For example, you can specify a sorting function as follows:
\begin{VCC}
_(logic bool sorted(int *buf, unsigned len) =
  \forall unsigned i, j; i <= j && j < len ==> buf[i] <= buf[j])

void sort(int *buf, unsigned len)
  _(writes \array_range(buf, len))
  _(ensures sorted(buf, len));
\end{VCC}
\

\noindent Note that this specification says that the buffer is
sorted in the poststate, but does not say that it is a permutation of
the elements that it had in the prestate; we address this in
\secref{ghosts}.

%% \vccInput[linerange={swap-partition}]{c/04_partition.c}
%% \vccInput[linerange={foo-}]{c/01_swap1.c}

%% Because global variables (like \vcc{z}) might be visible to callers
%% of \vcc{copy()}, \vcc{copy()} needs to report that they might change.
%% Note that the writes clause lists pointers to memory locations, not lvalues
%% (\ie \vcc{_(writes &x, &y, &z)} and not \vcc{_(writes x, y, z)}).
%% Note also that because \vcc{&z} aliases neither \vcc{&x} nor \vcc{&y}, VCC can
%% deduce that \vcc{swap(&x, &y)} does not change \vcc{z}.

%% Had we left out the writes clause from the specification of
%% \vcc{swap()}, VCC would report several errors:
%% \vccInput[linerange={out-}]{c/01_swap2.c}
%% \noindent 

%% Whenever a memory object is read, VCC asserts that it is
%% thread-local; whenever a memory object is written or is
%% mentioned in the writes clause of a function call, VCC asserts that it
%% is writable. Try removing each of the function annotations in
%% this example (one at a time) to see what error messages you get.

%% For example, the assert/assume translation of \vcc{swap()} is%
%% \footnote{
%%   VCC does not generate the read and writes checks for the local variable
%%   \vcc{tmp}. Because \vcc{tmp} is a local that never has its address
%%   taken, it can be thought of as remaining in a register, where it is
%%   guaranteed to remain writable.
%% }:

%% \vccInput[linerange={swap-}]{c/01_swap3.c}

%% \noindent
%% As we can see, one effect of \vcc{writes p, q} is the implicit
%% precondition \vcc{requires \writable(p) && \writable(q)}. Such precondition
%% needs to be checked in \vcc{foo()}, at the place where it calls \vcc{swap()}.
%% In other words, the called function can write at most what the caller can write.
%% In particular if we forget to list \vcc{&x} in the writes
%% clause of \vcc{foo()} we would get an error when it tries to call a
%% function that possibly writes \vcc{&x}:

%% \vccInput[linerange={out-}]{c/01_swap4.c}


%% You might wonder why cannot we just have 
%% \vcc{requires \writable(p)} and instead have the specialized writes clause.
%% The reason is that the writes clause also specifies that nothing
%% outside of the writes clause will be changed.
%% This is why we can prove that \vcc{y} is still \vcc{42} after the call
%% to \vcc{boundedIncr(&x)}. 


%A predicate \vcc{\mutable(p)} states that the object pointed to by \vcc{p}
%is allocated, ``belongs'' to the current thread, and is in a ``phase of life''
%that allows for modification.
%We will get into details all of these later.
%For now we just need to know that in order to be able to write to \vcc{*p}
%one needs to know that \vcc{p} was listed in the writes clause \emph{and} \vcc{\mutable(p)}.
%For example, if we remove the \vcc{_(writes ...)} clause from the
%\vcc{boundedIncr()} we get the following output:

\subsection*{Exercises}
\begin{enumerate}
\item
Extend the specification of sorting to guarantee that the sort
is stable (i.e., it shouldn't change the array if it is already
sorted).  
\item
Extend the specification of sorting to guarantee that the sorting
doesn't change the set of numbers occurring in the array (though it
might change their multiplicities). 
\item
Write, without using the \vcc{%} operator, logic functions that
specify (1)~that one number (\vcc{unsigned int}) divides another
evenly, (2)~that a number is prime, (3)~that two numbers are
relatively prime, (4)~the greatest common divisor of two numbers.
\end{enumerate}
