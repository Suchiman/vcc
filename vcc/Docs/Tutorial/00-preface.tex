\section{Preface}

This document serves as an introduction to verifying programs with VCC.
VCC is a \Def{sound deductive verifier} for C programs.

VCC is a \emph{verifier} because it checks that the program is correct, that is it behaves according to
some specification.
In particular it will check that the program never crashes (because of division
by zero or segmentation violation).

VCC is \emph{deductive} because it creates a mathematical formula which
states ``the program is correct'', and then uses deduction system
to prove that this formula is actually a theorem, something that is always true.
This formula is called \emph{verification condition}.
The proof is performed using formal methods of mathematics,
and the deductive system that VCC most often uses, called Z3~\cite{z3},
is fully automatic.

VCC is \emph{sound}, because the verification condition models the program behavior conservatively.
This means that if the verification condition is a theorem, then the program will never
violate its specification.
In practice, if VCC tells you that everything is OK with your program,
then it doesn't mean it couldn't find any more bugs, but it means that there are none.%
\footnote{
  At least theoretically.
  There could be bugs in VCC itself, there could be bugs in the compiler, operating system, or the hardware.
  However it is all about confidence, and the levels of confidence you get from verifying your program with
  VCC are somewhere around (or above) what is currently required for airplane or space shuttle software.
}

This tutorial covers basics of VCC annotation language and some tricks you
need to know to get your program through it. 
It doesn't talk about the theoretical background, including soundness
arguments.
These topics are covered separately~\cite{lci}.
A high-level overview
of the VCC tool chain is also available separately~\cite{Cohen:TPHOLs2009-23}.

You will first need to install VCC, you can find instructions on doing so at \url{http://vcc.codeplex.com/}.
After you're done you have two choices: one is to edit the tutorial examples
using Visual Studio and verify them using VCC Visual Studio addin.
\todo{explain how to create VS project for VCC and stuff}
The other option is to invoke the VCC verifier directly from command line
(which is also what the addin does under the hood),
and use your favorite text editor to edit the examples.
In any event, the examples found in this tutorial come with VCC, you
can find them in the \texttt{.../VCC/Tutorial} directory.
\todo{set the location properly, maybe have a VS project there}
