\ifdense
\documentclass[preprint,nocopyrightspace]{sigplanconf}
\else
\documentclass{article}
\usepackage{fancyhdr}
%\pagestyle{fancy}
\fi
\bibliographystyle{plain}

\usepackage{amsmath}
\usepackage{stmaryrd}
\usepackage{color}
\usepackage{graphicx}
%\usepackage{marginnote}
\usepackage{hyperref}
\usepackage{amsmath, amssymb}
\usepackage{url, listings}
%\usepackage{msrtr}
\usepackage{vcc}
\usepackage{boogie}
\usepackage{xspace}
\definecolor{bgcode}{rgb}{0.92,0.92,0.92}

\definecolor{todocolor}{rgb}{0.9,0.0,0.0}
\definecolor{noteframe}{rgb}{0.5,0.5,0.5}
%\ifdense
\newcommand{\todo}[1]{[\textcolor{red}{\textbf{TODO:} {#1}}]}
%\else
%\newcommand{\todo}[1]{\marginnote{\scriptsize {[\textcolor{red}{\textbf{TODO:} {#1}}]}}}
%\fi
\newcommand{\itodo}[1]{{[\textcolor{red}{\textbf{TODO:} {#1}}]}}
%\renewcommand{\todo}[1]{}
%\renewcommand{\itodo}[1]{}
\newcommand{\lines}[1]{\begin{array}{l}#1\end{array}}
\newcommand{\linesi}[1]{\;\;\;\begin{array}{l}#1\end{array}}
\newcommand{\tuple}[1]{{\langle}{#1}\rangle}

\ifdense
\usepackage[T1]{fontenc}
\usepackage{epigrafica}
%\usepackage{iwona}
%\usepackage[scaled]{berasans}
%\usepackage{cmbright}
%\usepackage{lxfonts}
%\usepackage{kurier}
\usepackage{times}
\else
\usepackage[OT1]{fontenc}
\usepackage{courier}
%\usepackage[T1]{fontenc}
%\usepackage[scaled=0.81]{luximono}
\fi

\newcommand{\secref}[1]{\hyperref[sect:#1]{\textsection~\ref{sect:#1}}}
\newcommand{\figref}[1]{Figure~\ref{fig:#1}}
\newcommand{\Secref}[1]{Section~\ref{sect:#1}}
\newcommand{\Figref}[1]{Figure~\ref{fig:#1}}

\newcommand{\lemmaref}[1]{Lemma~\ref{lemma:#1}}
\newcommand{\thmref}[1]{Theorem~\ref{thm:#1}}
\newcommand{\lineref}[1]{line~\ref{line:#1}}

\newcommand{\ie}[0]{i.e.,{ }}
\newcommand{\eg}[0]{e.g.,{ }}
\newcommand{\cf}[0]{cf.{ }}


\ifdense
\newenvironment{note}{%
\vspace{-3mm}
\begin{list}{}%
    {\setlength{\leftmargin}{0.03\textwidth}}%
    \item[]%
  \textcolor{noteframe}{\rule{0.444\textwidth}{1pt}} \\
  %\small
  %\sffamily
  \selectfont
  %\textbf{Note:}
}{%
  \vspace{-2mm} \\ 
  \textcolor{noteframe}{\rule{0.444\textwidth}{1pt}} 
  \end{list}
\vspace{-2mm}
}
\else
\newenvironment{note}{%
\vspace{-3mm}
\begin{list}{}%
    {\setlength{\leftmargin}{0.1\textwidth}}%
    \item[]%
  \textcolor{noteframe}{\rule{0.9\textwidth}{1pt}} \\
  \small
  %\textbf{Note:}
}{%
  \vspace{-2mm} \\ 
  \textcolor{noteframe}{\rule{0.9\textwidth}{1pt}} 
  \end{list}
\vspace{-2mm}
}
\fi
\newcommand{\notehd}[1]{\textbf{#1} \\}

\newcommand{\eqspc}[1]{\;\;{#1}\;\;}
\setlength{\arraycolsep}{0mm}

%%% the model

%%% general
\newcommand{\MathOp}[2]{{}\mathbin{\hbox{$\mkern#2mu#1\mkern#2mu$}}{}}
\newcommand{\Iff}{\MathOp{\Leftrightarrow}{6}}
\newcommand{\Equal}{\MathOp{=}{6}}
\newcommand{\Xor}{\MathOp{\not\equiv}{6}}
\newcommand{\Implies}{\MathOp{\Rightarrow}{4}}
\renewcommand{\And}{\MathOp{\wedge}{2}}
\newcommand{\Or}{\MathOp{\vee}{2}}
\newcommand{\Neg}{\neg}
\newcommand{\dor}[0]{\MathOp{|}{2}}

\newcommand{\ON}[1]{\operatorname{#1}}
\newcommand{\Version}[0]{0.2}

% -----------------------------------------------------

\newcommand{\Def}[1]{\textit{\textbf{#1}}}

\begin{document}

\title{Verifying C Programs: \\ A VCC Tutorial \\
\vspace{2mm}
\Large  Working draft, version \Version, \today}
%\title{Developing VCC Specifications \\ for Concurrent C Programs}

\ifdense
\authorinfo{Ernie Cohen, Mark A. Hillebrand, \\ Stephan Tobies}{European Microsoft Innovation Center}{\{ecohen,mahilleb,stobies\}@microsoft.com}
\authorinfo{Micha{\l} Moskal, Wolfram Schulte}{Microsoft Research Redmond}{\{micmo,schulte\}@microsoft.com}
\preprintfooter{VCC Tutorial (working draft, ver. \Version)}
\else
\author{
Micha{\l} Moskal, Wolfram Schulte \\
\normalsize Microsoft Research Redmond \\
\and Ernie Cohen, Mark A. Hillebrand, Stephan Tobies \\
\normalsize European Microsoft Innovation Center}
\date{}
\pagestyle{fancy}
\fancyhf{}
\renewcommand{\headrulewidth}{0pt}
\renewcommand{\footrulewidth}{0.3pt}
\fancyfoot[LO,RE]{\footnotesize VCC Tutorial (working draft, ver. \Version, \today)}
\fancyfoot[RO,LE]{\thepage}
\fi


%\msrtrno{MSR-TR-2010-9}
%{\def\@titletext{foo}}
%\msrtrmaketitle

%\pagebreak
%\begin{figure*}
%\vspace{3in}
%\begin{center}
%This page intentionally left blank.

%\end{center}
%\end{figure*}

%\setcounter{page}{0}

\maketitle


\begin{abstract}
VCC is a verification environment for software written in C.
VCC takes a program (annotated with function
contracts, state assertions, and type invariants) and attempts to prove that these 
annotations are corrrect, i.e. that they hold for every possible program execution.
The environment includes tools for monitoring proof attempts and constructing
partial counterexample executions for failed proofs.
VCC handles fine-grained concurrency and low-level C features, and
has been used to verify the functional correctness of tens of thousands of 
lines of commercial concurrent system code. 

This tutorial describes how to use VCC to verify C code. It covers the
annotation language, the verification methodology, and the use of VCC itself.
\end{abstract}


\ifdense
\lstset{
  basicstyle=\small\sffamily,
  columns=fullflexible,
}
\else
\lstset{
  basicstyle=\small\ttfamily,
}
\fi

\definecolor{kwColor}{rgb}{0.2,0.2,0.8}

\lstset{
  keywordstyle=\bfseries, %\textcolor{kwColor},
  breaklines=true,
  breakatwhitespace=true,
  numberstyle=\tiny\sf,
  escapeinside={/*-}{*/},
  numbers=none,
  emptylines=1,
  rangeprefix=\/\*\{,
  rangesuffix=\}\*\/,
  includerangemarker=false,  
%  aboveskip=2mm,
%  belowskip=2mm,
%  xleftmargin=2mm,
%  xrightmargin=2mm,
}

\ifdense
\renewcommand{\labelitemi}{{\footnotesize \centeroncapheight{$\bullet$}}}
\fi

\section{Introduction}
This tutorial is an introduction to verifying C code with VCC. Our
primary audience is programmers who want to write correct code.
The only prerequisite is a working knowledge of C.

To use VCC, you first \emph{annotate} your code to specify 
what your code does (e.g., that your sorting function sorts its input),
and why it works (e.g., suitable invariants for your loops and 
data structures).
VCC then tries to \emph{prove} (mathematically) that
your program meets these specifications.  Unlike most program
analyzers, VCC doesn't look for bugs, or analyze an abstraction of
your program; if VCC certifies that your program is correct, then your
program really is correct%
\footnote{
  In reality, this isn'tnecessarily true, for two reasons. 
  First, VCC itself might have bugs; in practice, these are unlikely to cause you to accidentally verify an
  incorrect program, unless you find and intentionally exploit such a
  bug. Second, there are a few checks needed for soundness that haven't yet
  been added to VCC, such as checking that ghost code terminates. 
  \todo{Should we have a caveat list as an appendix to the tutorial?}
  % This is way too arcane to be mentioned in the intro. --M
  %In addition, VCC currently doesn't do the checks needed to
  %ignore memory system optimizations on multiprocessor machines, e.g.,
  %processor store buffering on x64 machines; this will be remedied in the near
  %future.
  }. 

To check your program, VCC uses the \emph{deductive verification} paradigm:
it generates a number of mathematical
statements (called \Def{verification conditions}), the validity of
which suffice to guarantee the program's correctness, and tries to
prove these statements using an automatic theorem prover. If any of
these proofs fail, VCC reflects these failures back to you in terms of
the program itself (as opposed to the formulas seen by the theorem
prover). Thus, you normally interact with VCC entirely at the level of
code and program states; you can usually ignore the mathematical
reasoning going on ``under the hood''.
For example, if your program uses division
somewhere, and VCC is unable to prove that the divisor is nonzero, it
will report this to you as a (potential) program error at that point in the
program. This doesn't mean that your program is necessarily incorrect;
most of the time, especially when verifying code that is already well-tested,
it is because you haven't provided enough information to allow VCC
to deduce that the suspected error doesn't occur.
(For example, you might have failed to specify that some function
parameter is required to be nonzero.)
Typically, you fix this ``error'' by strengthening your
annotations. This might lead to other error reports, forcing you to 
add additional annotations, so verification is in
practice an iterative process.  
Sometimes, this process will reveal a genuine programming error.
But even if it doesn't, you will have not only proved your code free from 
such errors, but you will have produced the precise specifications for your
code -- a very useful form of documentation.

This tutorial covers basics of VCC annotation language. By the time
you have finished working through it, you should be able to use VCC to
verify some nontrivial programs. It doesn't cover the theoretical
background of VCC \cite{lci}, implementation details \cite{vcc}
or advanced topics;
information on these can be found on the VCC
homepage\footnote{\url{http://vcc.codeplex.com/}}.
%
%%  These topics are covered
%% separately~\cite{lci}.  A high-level overview of the VCC tool chain is
%% also available separately~\cite{Cohen:TPHOLs2009-23}.
%
The examples in this tutorial are currently distributed with the VCC sources.%
\footnote{
Available from \url{http://vcc.codeplex.com/SourceControl/list/changesets}: click Download on the right,
get the zip file and navigate to \lstinline|vcc/Docs/Tutorial/c|. (In
future, these will form part of the distribution.)}

You can use VCC either from the command line or from Visual Studio
2008/2010 (VS); the VS interface offers easy access to different components of
VCC tool chain and is thus generally recommended.
VCC can be downloaded from the VCC homepage; 
be sure to read the installation instructions\footnote{\url{http://vcc.codeplex.com/wikipage?title=Install}},
which provide important information about installation prerequisites 
and how to set up tool paths.

%\subsection{Notational conventions}

% How about keeping these things in footnotes? -E
%% \begin{note}
%%   Throughout the tutorial, we'll use notes like this one to discuss
%%   topics, which can be skipped on the first reading, either because
%%   they are somewhat more advanced, arcane, or not so important.
%% \end{note}

Because this is a tutorial, we will occasionally provide simplified (and
therefore not strictly correct) explanations, providing
additional clarification in the footnotes\footnote{Like this one.},
which can be skipped on first reading. 
\begin{note}
Notes like this one are used for explanatory and additional material,
that is not necessary to understand the main flow of the text.
\end{note}
% Don't need this -E
%Definitions  be introduced with \Def{italic boldface}.

\subsection{A Note on VCC Invocation}
\todo{eliminate this section and move it into the instructions for running VCC.}
\label{sect:todo-invoke}
This tutorial assumes you're using VCC version 3. 
Currently, it is necessary to supply the flag \texttt{/3}
to VCC to enable this.
This will soon be the default.
% -3 now implies -2 and -it. --MM

\subsection{Revision history}
\todo{eliminate this section?}
%% \subsection{Top-level To-Do}

\noindent \textbf{Version 0.2, April ?? 2011.}
Improved thread-locality explanations in \secref{writes}.
Added description of pure functions.

\noindent \textbf{Version 0.1, July 21 2010}
Added \secref{overflows} about overflows, and explanations about arrays writes in \secref{sorting}.
Added \secref{trigger-inference} and \secref{trigger-hints} about triggering.

\noindent \textbf{Version 0.0, July 17 2010}
Initial.

%\subsection{Running VCC from the command line}
%\todo{More options?}
%The easiest way to call VCC on a set of files is
%\begin{verbatim}
%vcc [/f:funs] [/inspector] [/modelviewer] files
%\end{verbatim}
%Via the \verb!/f! switch a comma-separated list of functions to
%verify may be provided. The \verb!/inspector! (or \verb|/i|) switch causes the Z3 Inspector to be
%displayed while the verification is running, to monitor what VCC is
%trying to do. The \verb!/modelviewer! (or \verb|/mv|) switch causes VCC to display models for
%the errors that it finds.
%If there is more than one error, the active model can be switched in the Model menu.
% 
%\subsection{Running VCC from VS}
%\itodo{this should be introduced along with the examples}
%If you right-click within a C source file,
%several VCC commands are made available, depending on what kind of
%construction IntelliSense thinks you are in the middle of. The choice
%of verifying the entire file is always available. If you click within
%the definition of a struct type, VCC will offer you the choice of
%checking admissibility for that type (a concept explained in
%section \ref{}). If you click within the body of a function, VCC should offer
%you the opportunity to verify just that function. However,
%IntelliSense often gets confused about the syntactic structure of
%VCC code, so it may not give these context-dependent
%options. However, if you select the name of a function and then right
%click, it will allow you to verify just that function.
%
%If you want to run the VCC inspector during verification, this
%option can be selected from the Verify$\rightarrow$Settings menu. If you want
%to look at the error model for a particular error, right-click on
%error (in the source), and choose ``Show VCC error model''\todo{This
%  should really say ``VCC error model'', to not expose Z3.}

\section{Verifying Simple Programs}
We begin by describing how VCC verifies ``simple programs'' ---
sequential programs without loops,  function calls, or concurrency. This
might seem to be a trivial subset of C, but in fact VCC reasons about
more complex programs by reducing them to reasoning about simple programs.

\subsection{Assertions}
\label{sect:assert-assume}

Let's begin with a simple example:

\vccInputSC[linerange={begin-}]{c/01_minInline2.c}
This program sets \vcc{z} to the minimum of \vcc{x} and
\vcc{y}. In addition to the ordinary C code, this program includes an
\Def{annotation}, starting with \vcc{_(}, terminating with a closing
parenthesis, with balanced parentheses inside. The first identifier
after the opening parenthesis (in the program above it's \vcc{assert})
is called an \Def{annotation tag} and
identifies the type of annotation provided (and hence its function).
(The tag plays this role only 
at the beginning of an annotation; elsewhere, it is treated as
an ordinary program identifier.)
Annotations are used only by VCC, and are not seen by the C compiler.
When using the regular C compiler the \vcc{<vcc.h>} header file defines:
\begin{VCC}
#define _(...) /* nothing */
\end{VCC}
VCC does not use this definition, and instead parses the inside of \vcc{_( ... )}
annotations.


An annotation of the form \vcc{_(assert E)}, called an \Def{assertion}, asks VCC to prove that
\vcc{E} is guaranteed to hold (\ie evaluate to a value other than \vcc{0})
whenever control reaches the assertion.%
\footnote{
  This interpretation changes slightly if \vcc{E} refers to
  memory locations that might be concurrently modified by other
  threads; see \secref{concurrency}}  
Thus, the line \vcc|_(assert z <= x)| says
that when control reaches the assertion, \vcc{z} is no larger than \vcc{x}.
If VCC successfully verifies a program, it promises that this will hold throughout 
every possible execution of the program, regardless of inputs, how concurrent
threads are scheduled, etc. More generally, VCC might verify some, but not all of your assertions; 
in this case, VCC promises that the first assertion to be violated will not be one of the verified ones.

\begin{note}
It is instructive to compare \vcc{_(assert E)} with the macro
\lstinline|assert(E)| (defined in \lstinline|<assert.h>|), which
evaluates \vcc{E}  at runtime and aborts execution if \vcc{E}
doesn't hold. First, \lstinline|assert(E)| requires runtime overhead (at least
in builds where the check is made), whereas \vcc{_(assert E)} does
not. Second, \lstinline|assert(E)| will catch failure of the 
assertion only when it actually fails in an execution, whereas 
\vcc{_(assert E)} is guaranteed to catch the failure if it is
possible in \emph{any} execution. Third, because
\vcc{_(assert E)} is not actually executed, \vcc{E} can include
unimplementable mathematical operations, such as
quantification over infinite domains.
\end{note}

To verify the function using VCC from the command line, save the source in a file called \lstinline|minimum.c|
and run VCC as follows (as noted before, currently you need to specify the \lstinline|-3| flag to enable
VCC version 3):

%\todo{add something about /3}
\begin{VCC}
/*`C:\Somewhere\VCC Tutorial> vcc.exe -3 minimum.c
Verification of main succeeded.
C:\Somewhere\VCC Tutorial>`*/
\end{VCC}


If instead you wish to use VCC Visual Studio plugin, load the solution \lstinline|VCCTutorial.sln|
\todo{make sure that VCC can verify code not in the project}
locate the file with the example, and right-click on the program text.
You should get options to verify the file or just this function (either will work).

If you right-click within a C source file,
several VCC commands are made available, depending on what kind of
construction IntelliSense thinks you are in the middle of. The choice
of verifying the entire file is always available. If you click within
the definition of a struct type, VCC will offer you the choice of
checking admissibility for that type (a concept explained in \secref{admissibility0}).
If you click within the body of a function, VCC should offer
you the opportunity to verify just that function. However,
IntelliSense often gets confused about the syntactic structure of
VCC code, so it might not always present these context-dependent
options. However, if you select the name of a function and then right
click, it will allow you to verify just that function.

VCC verifies this function successfully, which means that its
assertions are indeed guaranteed to hold and that the program cannot
crash.%
\footnote{
  VCC currently doesn't check that your program doesn't run forever or
  run out of stack space, but future versions will, at least for sequential
  programs.  
}
If VCC is unable
to prove an assertion, it reports an error.  Try changing the
assertion in this program to something that isn't true and see what
happens. (You might also want to try coming up with some complex
assertion that is true, just to see whether VCC is smart enough to
prove it.)

To understand how VCC works, and to use it successfully, it is useful to
think in terms of what VCC ``knows'' at each control point
of your program. In the current example, just before the first conditional,
VCC  knows nothing about the local variables,
since they can initially hold any value. 
Just before the first assignment, VCC knows that 
\vcc{x <= y} (because execution followed that branch of the conditional), and
after the assignment, VCC also knows that \vcc{z == x}, 
so it knows that \vcc{z <= x}. Similarly, in the \vcc{else} branch,
VCC knows that \vcc{y < x} (because execution didn't follow the
\vcc{if} branch), and after the assignment to \vcc{z}, it also knows
that \vcc{z == y}, so it also knows \vcc{z <= x}. Since 
\vcc{z <= x} is known to hold at the end of each branch of the
conditional, it is known to hold at the end of the conditional, so the
assertion is known to hold when control reaches it.
In general,
VCC doesn't lose any information when reasoning about assignments and
conditionals. However, we will see that VCC may lose
loses some information when reasoning about loops,
necessitating additional annotations.

When we talk about what VCC knows, we mean what it knows in an ideal
sense, where if it knows \vcc{E}, it also knows any logical
consequence of \vcc{E}. In such a world, adding an assertion that
succeeds would have no effect on whether later assertions succeed.
VCC's ability to deduce consequences is indeed complete for many types
of formulas (e.g. formulas that use only equalities,
inequalities, addition, subtraction, multiplication by constants, and
boolean connectives), but not for all formulas, so VCC will
sometimes fail to prove an assertion, even though it ``knows'' enough
to prove it.  Conversely, an assertion that succeeds can sometimes cause later assertions that
would otherwise fail to succeed, by drawing VCC's attention to a
crucial fact it needs to prove the later assertion.  This is
relatively rare, and typically involves ``nonlinear arithmetic''
(e.g. where variables are multiplied together), bitvector reasoning
(\secref{bv}) or quantifiers.

When VCC surprises you by failing to verify something that you think
it should be able to verify, it is usually because it doesn't know
something you think it should know. An assertion provides one way to
check whether VCC knows what you think it knows.


\subsection*{Exercises}
\begin{enumerate}
\item Can the assertion at the end of the example function be made
  stronger? What is the strongest valid assertion one could write? Try 
  verifying the program with your stronger assertion.
\item Write an assertion that says that the \vcc{int} \vcc{x} is the average of
  the \vcc{int}s \vcc{y} and \vcc{z}.
\item Modify the example program so that it sets \vcc{x} and \vcc{y} to values
that differ by at most \vcc{1} and sum to \vcc{z}. Prove that your program 
does this correctly.
\item Write an assertion that says that the \vcc{int} \vcc{x} is a
  perfect square (\ie a number being a square of an integer).
\item Write an assertion that says that the \vcc{int} \vcc{x} occurs
  in the \vcc{int} array \vcc{b[10]}.
\item Write an assertion that says that the \vcc{int} array \vcc{b},
  of length \vcc{N}, contains no duplicates.
\item Write an assertion that says that all pairs of adjacent elements
  of the \vcc{int} array \vcc{b} of length \vcc{N} differ by at most
  \vcc{1}.
\item Write an assertion that says that an array \vcc{b} of length
  \vcc{N} contains only perfect squares.
\end{enumerate}

\subsection{Assumptions}

You can add to what VCC knows at a particular point with a second
type of annotation, called an \Def{assumption}.
The assumption \vcc{_(assume E)} tells VCC to ignore
the rest of an execution if \vcc{E} fails to hold (\ie if
\vcc{E} evaluates to \vcc{0}). 
Reasoning-wise, the assumption simply adds \vcc{E} to what VCC
knows for subsequent reasoning. For example:
\begin{VCC}
int x, y;
_(assume x != 0)
y = 100 / x;
\end{VCC}
Without the assumption, VCC would complain about possible division by
zero. (VCC checks for division by zero because it would cause the
program to crash.)  Assuming the assumption, this error cannot happen.  
Since assumptions (like all annotations) are not seen by the compiler,
assumption failure won't cause the program to stop, and subsequent assertions
might be violated. To put it another way, if VCC verifies a program,
it guarantees that in any prefix of an execution
where all (user-provided) assumptions hold, all assertions will also
hold. Thus, your verification goal should be to eliminate as many assumptions as
possible (preferably all of them).

\begin{note}
Although assumptions are generally to be avoided, they are nevertheless
sometimes useful:
\begin{inparaenum}
\item In an incomplete verification, assumptions can be used to mark
  the knowledge that VCC is missing, and to coordinate further
  verification work (possibly performed by other people). If you
  follow a discipline of keeping your code in a state where the whole 
  program verifies, the verification state can be judged by browsing
  the code (without having to run the verifier).

\item When debugging a failed verification, you can use assumptions to
  narrow down the failed verification to a more specific failure
  scenario, perhaps even to a complete counterexample. 

\item Sometimes you want to assume something even though VCC can
  verify it, just to stop VCC from spending time proving it. For
  example, assuming \vcc{\false} allows VCC to 
  easily prove subsequent assertions, thereby focussing its
  attention on other parts of the code. Temporarily adding assumptions
  is a common tactic when developing annotations for complex functions.

\item Sometimes you want to make assumptions about the operating
  environment of a program. For example, you might want to assume that
  a 64-bit counter doesn't overflow, but don't want to justify it
  formally because it depends on extra-logical assumptions (like the
  speed of the hardware or the lifetime of the software). 

\item Assumptions provide a useful tool in explaining how VCC
  reasons about your program. We'll see examples of this throughout
  this tutorial.
\end{inparaenum}
\end{note}

An assertion can also change what VCC knows after the assertion, if
the assertion fails to verify: although VCC will report the failure as an error,
it will assume the asserted fact holds afterward. For example, in the following
VCC will only report an error for the first assumption and not the second:
\begin{VCC}
int x;
_(assert x == 1)
_(assert x > 0)
\end{VCC}

\subsection*{Exercises}
\begin{enumerate}
\item
In the following program fragment, which assertions will fail?
\begin{VCC}
int x,y; 
_(assert x > 5) 
_(assert x > 3) 
_(assert x < 2) 
_(assert y < 3)
\end{VCC}
\item
Is there any difference between 
\begin{VCC}
_(assume p)
_(assume q)
\end{VCC}
and 
\begin{VCC}
_(assume q) 
_(assume p)
\end{VCC}
? What if the assumptions are replaced by assertions?
\item
Is there any difference between
\begin{VCC}
_(assume p)
_(assert q)
\end{VCC}
and 
\begin{VCC}
_(assert (!p) || (q))
\end{VCC}
? 

\end{enumerate}

\section{Function Contracts}
\label{sect:functions}

Next we turn to the specification of functions. We'll take the example
from the previous section, and pull the computation of the minimum of
two numbers out into a separate function:

\vccInput[linerange={begin-}]{c/01_min2.c}

(The listing above presents both the source code and the output
of VCC, typeset in a different fonts, and 
the actual file name of the example is replaced with \vcc{/*`testcase`*/}.)
VCC failed to prove our assertion, even though it's easy to see that
it always holds. This is because verification in VCC is \Def{modular}: 
VCC doesn't look inside the body of a function (such as the definition of \vcc{min()}) 
when understanding the effect of a call to the function (such as 
the call from \vcc{main()});
% We actually don't have that (except for atomics) --MM
%\footnote{We will see later how to selectively override this default
%  and perform such inlining during verification.}; 
all it knows about the effect of calling \vcc{min()} is that the call 
satisfies the specification of \vcc{min()}. 
%This is just because we have chosen the default to be so. It thus doesn't
% contradict that specification needs to say everything --M
%\footnote{
%  Actually, VCC does know one important thing from this specification
%  of \vcc{min()}: it knows that a call do \vcc{min()} has no side
%  effects that are visible to the caller. We'll see in \secref{writes} that
%  any such side effects have to be explicitly specified.
%}.
Since the correctness of \vcc{main()} clearly depends on what \vcc{min()}
does, we need to specify \vcc{min()} in order to verify \vcc{main()}.

The specification of a function is sometimes called its \Def{contract},
because it gives obligations on both the function and its callers. It
is provided by three types of annotations:
\begin{itemize}
\item A requirement on the caller (sometimes called the
  \Def{precondition} of the function) takes the form \vcc{_(requires E)}, 
  where \vcc{E} is an expression. It says that callers of the
  function promise that \vcc{E} will hold on function entry. 

\item A requirement on the function (sometimes called a
  \Def{postcondition} of the function) takes the form \vcc{_(ensures E)}, 
  where \vcc{E} is an expression. It says that the function
  promises that \vcc{E} holds just before control exits the
  function. 

\item The third type of contract annotation, a \Def{writes clause}, is described
  in the next section.
\end{itemize}

For example, we can provide a suitable specification for \vcc{min()} as
follows:
\vccInput[linerange={min-endmin,out-}]{c/01_min3.c}
\noindent
(Note that the specification of the function comes after the header and
before the function body; you can also put specifications on function
declarations (\eg in header files).)
The precondition \vcc{_(requires \true)} of \vcc{min()} doesn't
really say anything (since \vcc{\true} holds in every state), but is included
to emphasize that the function can be called from any state and
with arbitrary parameter values.
The postcondition states that the value returned from \vcc{min()} 
is no bigger than either of the inputs.
Note that \vcc{\true} and \vcc{\result} are spelled with a backslash
to avoid name clashes with C identifiers.%
\footnote{
  All VCC keywords start with a backslash; 
  this contrasts with annotation tags (like \vcc{requires}),
  which are only used at the beginning of annotation
  and therefore cannot be confused with C identifiers
  (and thus you are still free to have, \eg
  a function called \lstinline{requires} or \lstinline{assert}).}

VCC uses function specifications as follows. When verifying the body of a
function, VCC implicitly assumes each precondition of the function on
function entry, and implicitly asserts each postcondition of the
function (with \vcc{\result} bound to the return value and each
parameter bound to its value on function entry) just before the
function returns. For every call to the function, VCC replaces the
call with an assertion of the preconditions of the function, sets the
return value to an arbitrary value, and finally assumes each
postcondition of the function. 
For example, VCC translates the program above roughly as follows:
\vccInput[linerange={begin-end}]{c/01_min_assert.c}

Note that the assumptions above are ``harmless'', that is in a fully
verified program they will be never violated, as they follow
from assertion that proceed them.
For example, the precondition assumption would only fail if the precondition
assertion before it fails.

\begin{note}
\notehd{Why modular?}
Modular verification brings several benefits. 
First, it allows verification to more easily scale to
large programs. Second, by providing a precise interface between
caller and callee, it allows you to modify the implementation of
a function like \vcc{min()} without having to worry about breaking the
verifications of functions that call it (as long as you don't change
the specification of \vcc{min()}). This is especially important
because these callers normally aren't in scope, and the person
maintaining \vcc{min()} might not even know about them (e.g., if
\vcc{min()} is in a library). Third, you can verify a function like
\vcc{main()} even if the implementation of \vcc{min()} is unavailable
(\eg if it hasn't been written yet). 
\end{note}

\subsection*{Exercises}
\begin{enumerate}
\item
Try replacing the \vcc{<} in the return statement of \vcc{min()} with
\vcc{>}. Before running VCC, can you guess which parts of the
verification will fail?
%% One way to illustrate that the verification of \vcc{main()} 
%% depends only on the specification of \vcc{min()} is to 
%% replace \vcc{<} with \vcc{>} in the return statement of \vcc{min()}.
%% The output of verification shows that \vcc{main()} is still correct---only 
%% \vcc{min()} is broken:
%% \vccInput[linerange={out-}]{c/01_min4.c}

\item
What is the effect of giving a function the specification
\vcc{_(requires \false)}? How does it effect verification of
the function itself? What about its callers? Can you think of a good 
reason to use such a specification?
%% When verifying function body it translates to an assumption.
%% As we have noted this will make the body of function verify,
%% regardless what it contains.
%% On the other hand, you will be unable to verify call to such a function
%% from outside.
%% Conversely, \vcc{_(ensures \false)} will prevent verification
%% of the body of the function, but will make the function that calls
%% it verify.

%% As we can see, in each case there is something that does not verify.
%% However, because of this you should be careful when interpreting
%% VCC answers: successful verification of a function is only meaningful
%% if everything it calls was verified.

\item
Can you see anything wrong with the above specification of \vcc{min()}?
Can you give a simpler implementation than the one presented? Is this 
specification strong enough to be useful? If not, how might it be
strengthened to make it more useful?

\item
Specify a function that returns the (\vcc{int}) square root of its (\vcc{int})
argument. (You can try writing an implementation for the function, but
won't be able to verify it until you've learned about loop
invariants.)

\item
Can you think of useful functions in a real program that might
legitimately guarantee only the trivial postcondition \vcc{_(ensures \true)}?
\end{enumerate}

\subsection{Reading and writing through pointers}
\label{sect:writes}

When programming in C, it is important to distinguish two kinds of
memory access. \Def{Sequential} access, which is the default, is appropriate when
interference from other threads (or the outside world) is not an
issue, \eg when accessing unshared memory. Sequential accesses can be
safely optimized by the compiler by breaking it up into multiple
operations, caching reads, reordering operations, etc. 
\Def{Atomic} access is required when the access might race with other
threads, \eg a write to memory that is concurrently read or written,
or a read to memory that is concurrently written.

We have already seen that VCC verifies each function separately,
but it also verifies each thread separately. 
There is always a notion of a single \Def{current thread}
that is executing the verified function.
Other threads might be concurrently executing in the same
function, or in other functions. 

By default VCC enforces all memory accesses to be sequential,
which means you have to prove that the memory location read
or written ``belongs'' to the current thread.
The three simple ways of specifying this are listed below:

\begin{itemize}
\item A pointer \vcc{p} is considered \Def{thread-local}
when you specify:
\begin{VCC}
_(requires \thread_local(p))
\end{VCC}
When reading \vcc{*p} (without additional annotations) you will need to prove \vcc{\thread_local(p)}.

\item A pointer \vcc{p} is considered \Def{mutable} when you specify:
\begin{VCC}
_(requires \mutable(p))
\end{VCC}
All mutable pointers are also thread local.
Writing via pointers different than \vcc{p} cannot make \vcc{p} non-mutable.

\item A pointer \vcc{p} is considered \Def{writable} when you specify:
\begin{VCC}
_(writes p)
\end{VCC}
Additionally, freshly allocated pointers are are also writable.
All writable pointers are also mutable.
To write through \vcc{p} you will need to prove that the pointer is writable.
Deallocating \vcc{p} requires that it is writable, and afterwards
the memory is not even thread-local anymore.
\end{itemize}

\begin{note}
If VCC doesn't know why an object is thread local, then it has
hard time proving that the object stays thread local after an operation
with side effects (\eg a function call).
Thus, in preconditions you will sometimes want to use
\vcc{\mutable(p)} instead of \vcc{\thread_local(p)}.
The precise definitions of mutability and thread locality
is given in \secref{ownership},
where we also describe another form of guaranteeing thread locality
through so called ownership domains.
\end{note}

The \vcc{NULL} pointer, pointers outside bounds of arrays,
the memory protected by the operating system, or outside
the address space are never thread local (and thus also never mutable
nor writable).
Access to thread local memory will never crash the program.
The memory that is not thread-local is covered in 
\secref{concurrency}.

Let's have a look at an example:
\vccInput[linerange={begin-}]{c/01_rw.c}
\noindent
The function \vcc{write_wrong} fails because \vcc{p} is only
mutable, and not writable.
In \vcc{read_wrong} VCC complains that it does not know
anything about \vcc{p} (maybe it's \vcc{NULL}, who knows),
in particular it doesn't know it's thread-local.
\vcc{read2} is fine because \vcc{\mutable} is stronger
than \vcc{\thread_local}.
Finally, in \vcc{test_them} the first three assertions succeed
because if something is not listed in the writes clause
of the called function it cannot change.
The last assertion fails, because \vcc{write()} listed \vcc{&a}
in its writes clause.

%\noindent Without the \vcc{\thread_local()} annotation you would
%get the following:
%\vccInput[linerange={begin-}]{c/01_read_wrong.c}

% the concept of validity (typed pointers) is gone from VCC3 --MM
%For each memory access within a program, VCC checks that the access is accessing a 
%\Def{valid} memory object. Validity implies that the object address
%points to memory that is actually in the address space of the program
%(i.e., it has been allocated, either on the stack or on the heap, and
%has not been freed). (Validity in VCC additionally requires that the
%access is appropriately typed; this aspect is 
%described in more detail in \secref{type-safety}). 

Intuitively, the clause \vcc{_(writes p, q)} says that, 
of the memory objects that are thread-local to the caller before the call,
the function is going to modify only the object pointed to by \vcc{p}
and the object pointed to by \vcc{q}.%
\footnote{
And their ``ownership domains'',
but until \secref{ownership} we consider these to be empty.
}
In other words, it is roughly equivalent to a postcondition that ensures
that all other objects local to the caller prior
to the call remain unchanged.
VCC allows you
to write multiple writes clauses, and implicitly
combines them into a single set. If a function spec contains no writes clauses, 
it is equivalent to specifying a writes clause with empty set of
pointers.

%More precisely, an object is \vcc{\writable} if it is \vcc{\mutable} and 
%it is either listed in a writes clause of the function,
%or it became \vcc{\mutable} sometime after the function was entered; the 
%latter condition guarantees that either \vcc{p} was listed in the
%writes clause or was not thread-local in the caller when the call to
%the function was made. 
%In particular, formal function parameters and local variables are
%writable as long as they are in scope and have not been explicitly
%wrapped (\secref{invariants}) or reinterpreted to a
%different type (\secref{reint}).  VCC asserts
%\vcc{\writable(p)} on each attempt to write to \vcc{*p}, as well as on
%each call to a function that lists \vcc{p} in a writes clause.

Here is a simple example of a function that visibly reads and writes
memory; it simply copies data from one location to another.
\vccInput[linerange={begin-}]{c/01_copy1.c}
In the postcondition the expression \vcc{\old(E)} returns the value
the expression \vcc{E} had on function entry.
Thus, our postcondition states that the new value of \vcc{*to}
equals the value \vcc{*from} had on call entry. 
VCC translates the function call \vcc{copy(&x,&y)} approximately as
follows:
\begin{VCC}
_(assert \thread_local(&x))
_(assert \mutable(&y))
// record the value of x
int _old_x = x;
// havoc the written variables
havoc(y);
// assume the postcondition
_(assume y == _old_x)
\end{VCC}

%% Shouldn't \array_range be allowed for arrays of non-primitives also?
% Similar to \vcc{\thread_local_array},
%\vcc{\mutable_array(ar,sz)} is defined as
%\vcc{\forall unsigned i; i < sz ==> \mutable(&ar[i])}.
%
%The expression \vcc{\thread_local_array(ar, sz)} is
%syntactic sugar for \vcc{\forall unsigned i; i < sz ==> \thread_local(&ar[i])}.
%
%Programs that manipulate arrays often write to multiple array
%locations. Writes clauses actually allow sets of pointers, rather than
%individual pointers. We'll introduce sets in full generality later, but
%note one special case: the expression \vcc{\array_range(ar, len)}
%denotes the set of pointers \vcc@{&ar[0], &ar[1], ..., &ar[len-1]}@. 
%Thus, a writes clause of the form
%\vcc{_(writes \array_range(ar,len))}
%allows writing to all elements of the array.

\subsection{Arrays}

Array accesses are a kind of pointer accesses.
Thus, before allowing you to read an element of an array VCC checks if it's thread-local.
Usually you want to specify that all elements of an array are thread-local,
which is done using the expression \vcc{\thread_local_array(ar, sz)}.
It is essentially a syntactic sugar for
\vcc{\forall unsigned i; i < sz ==> \thread_local(&ar[i])}.
The annotation \vcc{\mutable_array()} is analogous.
To specify that an array is writable use:
\begin{VCC}
_(writes \array_range(ar, sz))
\end{VCC}
which is roughly a syntactic sugar for:
\begin{VCC}
_(writes &ar[0], &ar[1], ..., &ar[sz-1])
\end{VCC}

For example, the function below is reimplementation of the 
C standard library \vcc{memcpy()} function:
\vccInput[linerange={begin-end}]{c/01_copy_array.c}
It requires that array \vcc{src} is thread-local,
\vcc{dst} is writable, and they do not overlap.
It ensures that at all indices \vcc{dst} has the old
value of \vcc{src}.
The next section will show how to implemented it with a loop.

\subsection{Logic functions}

Just like with programs, as the specifications get more complex,
it's good to structure them somewhat.
One mechanism provided by VCC for that is \Def{logic functions}.
They essentially work as macros for pieces of specification,
but prevent name capture and give better error messages.
For example:
\vccInput[linerange={beginsp-endsp}]{c/01_issorted.c}
\noindent
A partial spec for a sorting routine could look like the following:%
\footnote{We will take care about input being permutation of the output in \secref{ghosts}.}
\vccInput[linerange={beginso-endso}]{c/01_issorted.c}

Logic functions are not executable, but they can be implemented:
\vccInput[linerange={beginim-endim}]{c/01_issorted.c}

A logic function and its C implementation can be combined into one using
\vcc{_(pure)} annotation.  This is covered in \secref{pureFunctions}.

 
\subsection{Exercises}
\begin{enumerate}
\item
Could the postcondition of \vcc{copy} have been written equivalently
as the simpler \vcc{*to == *from}? If not, why not?
\item
Specify a program that takes two arrays and checks whether
the arrays are equal (\ie whether they contain the same sequence of
elements).
\item
Specify a program that takes an array and checks whether it
contains any duplicate elements.
\item
Specify a program that checks whether a given array
is a palindrome.
\item
Specify a program that checks whether two arrays contain a
common element.
\item
Specify, write, and verify a function that swaps the values pointed to
by two \vcc{int} pointers. Hint: use \vcc{\old(...)} in the
postcondition.
\item
Specify a function that copies one array to another array of the same
size. (Don't worry about array aliasing.)
\item 
Specify a function that takes a text (stored in an array) and a string
(stored in an array) and checks whether the string occurs within the
text.
%\end{enumerate}
%\begin{enumerate}
\item
Extend the specification of sorting to guarantee that the sort
is stable (i.e., it shouldn't change the array if it is already
sorted).  
\item
Extend the specification of sorting to guarantee that the sorting
doesn't change the set of numbers occurring in the array (though it
might change their multiplicities). 
\item
Write, without using the \vcc{%} operator, logic functions that
specify (1)~that one number (\vcc{unsigned int}) divides another
evenly, (2)~that a number is prime, (3)~that two numbers are
relatively prime, and (4)~the greatest common divisor of two numbers.
\end{enumerate}


%% \vccInput[linerange={swap-partition}]{c/04_partition.c}
%% \vccInput[linerange={foo-}]{c/01_swap1.c}

%% Because global variables (like \vcc{z}) might be visible to callers
%% of \vcc{copy()}, \vcc{copy()} needs to report that they might change.
%% Note that the writes clause lists pointers to memory locations, not lvalues
%% (\ie \vcc{_(writes &x, &y, &z)} and not \vcc{_(writes x, y, z)}).
%% Note also that because \vcc{&z} aliases neither \vcc{&x} nor \vcc{&y}, VCC can
%% deduce that \vcc{swap(&x, &y)} does not change \vcc{z}.

%% Had we left out the writes clause from the specification of
%% \vcc{swap()}, VCC would report several errors:
%% \vccInput[linerange={out-}]{c/01_swap2.c}
%% \noindent 

%% Whenever a memory object is read, VCC asserts that it is
%% thread-local; whenever a memory object is written or is
%% mentioned in the writes clause of a function call, VCC asserts that it
%% is writable. Try removing each of the function annotations in
%% this example (one at a time) to see what error messages you get.

%% For example, the assert/assume translation of \vcc{swap()} is%
%% \footnote{
%%   VCC does not generate the read and writes checks for the local variable
%%   \vcc{tmp}. Because \vcc{tmp} is a local that never has its address
%%   taken, it can be thought of as remaining in a register, where it is
%%   guaranteed to remain writable.
%% }:

%% \vccInput[linerange={swap-}]{c/01_swap3.c}

%% \noindent
%% As we can see, one effect of \vcc{writes p, q} is the implicit
%% precondition \vcc{requires \writable(p) && \writable(q)}. Such precondition
%% needs to be checked in \vcc{foo()}, at the place where it calls \vcc{swap()}.
%% In other words, the called function can write at most what the caller can write.
%% In particular if we forget to list \vcc{&x} in the writes
%% clause of \vcc{foo()} we would get an error when it tries to call a
%% function that possibly writes \vcc{&x}:

%% \vccInput[linerange={out-}]{c/01_swap4.c}


%% You might wonder why cannot we just have 
%% \vcc{requires \writable(p)} and instead have the specialized writes clause.
%% The reason is that the writes clause also specifies that nothing
%% outside of the writes clause will be changed.
%% This is why we can prove that \vcc{y} is still \vcc{42} after the call
%% to \vcc{boundedIncr(&x)}. 


%A predicate \vcc{\mutable(p)} states that the object pointed to by \vcc{p}
%is allocated, ``belongs'' to the current thread, and is in a ``phase of life''
%that allows for modification.
%We will get into details all of these later.
%For now we just need to know that in order to be able to write to \vcc{*p}
%one needs to know that \vcc{p} was listed in the writes clause \emph{and} \vcc{\mutable(p)}.
%For example, if we remove the \vcc{_(writes ...)} clause from the
%\vcc{boundedIncr()} we get the following output:


\todo{if this section remains after the functions section, move the exercises from there to here,
since they require quantification.}
\section{Arithmetic and Quantifiers}
\todo{move arithmetic stuff appendix in, as well as mathint and a mention of maps}
\todo{add appendix section gathering together all annotations and extensions to C}

VCC provides a number of C extensions that can be used within VCC
annotations (such as assertions):
\begin{itemize}
\item
The Boolean operator \vcc{==>} denotes logical implication; formally,
\vcc{P ==> Q} means \vcc{((!P) || (Q))}, and is usually 
read as ``\vcc{P} implies \vcc{Q}''. Because \vcc{==>} has lower
precedence than the C operators, it is typically not necessary to
parenthesize \vcc{P} or \vcc{Q}.

\item
The expression \vcc{\forall T v; E} evaluates to \vcc{1} if the
expression \vcc{E} evaluates to a nonzero value for every value 
\vcc{v} of type \vcc{T}. For example, the assertion
\begin{VCC}
_(assert x > 1 &&
  \forall int i; 1 < i && i < x ==> x % i != 0)
\end{VCC}
\noindent checks that (\vcc{int}) \vcc{x} is a prime number. If \vcc{b}
is an \vcc{int} array of size \vcc{N}, then
\begin{VCC}
_(assert \forall int i; \forall int j;
  0 <= i && i <= j && j < N ==> b[i] <= b[j])
\end{VCC}
checks that \vcc{b} is sorted.

\item
Similarly, the expression \vcc{\exists T v; E} evaluates to \vcc{1} if there
is some value of \vcc{v} of type \vcc{T} for which \vcc{E} evaluates
to a nonzero value. For example, if \vcc{b} is an \vcc{int} array of
size \vcc{N}, the assertion
\begin{VCC}
_(assert \exists int i; 0 <= i && i < N && b[i] == 0)
\end{VCC}
asserts that  \vcc{b} contains a zero element.
\vcc{\forall} and \vcc{\exists} are jointly referred to as
\Def{quantifiers}. 

\item
VCC also provides some mathematical types that cannot be used in
ordinary C code (because they are too big to fit in memory);
these include mathematical (unbounded) integers and (possibly infinite) maps. They are described in
\secref{mathTypes}.

\item
Expressions within VCC annotations are restricted in their use of 
functions: you can only use functions that are proved to be 
\Def{pure}, \ie free from side effects (\secref{pureFunctions}).
\end{itemize}


\subsection{Overflows and unchecked arithmetic}
\label{sect:overflows}

Consider the C expression \vcc{a+b}, when \vcc{a} and \vcc{b} are,
say, \vcc{unsigned int}s. This might represent one of two programmer
intentions. Most of the time, it is intended to mean ordinary
arithmetic addition on numbers; program correctness is then likely to
depend on this addition not causing an overflow. However, sometimes
the program is designed to cope with overflow, so the programmer means
\vcc{(a + b) % UINT_MAX+1}. 
It is always sound to use this second interpretation, but VCC
nevertheless assumes the first by default, for several reasons:
\begin{itemize}
\item The first interpretation is much more common.
\item The second interpretation introduces an implicit \vcc{%}
  operator, turning linear arithmetic into nonlinear arithmetic and
  making subsequent reasoning much more difficult.
\item If the first interpretation is intended but the addition can in
  fact overflow, this potential error will only manifest later in the
  code, making the source of the error harder to track down.
\end{itemize}

Here is an example where the second interpretation is intended, but
VCC complains because it assumes the first:
\vccInput[linerange={begin-}]{c/4.1.hash_fail.c}

\noindent
VCC complains that the hash-computing operation might overflow.
To indicate that this possible overflow behavior is desired we use \vcc{_(unchecked)},
with syntax similar to a regular C type-cast.
This annotation applies to the following expression, and indicates that
you expect that there might be overflows in there.
Thus, replacing the body of the loop with the following
makes the program verify:

\vccInput[linerange={update-endupdate}]{c/4.2.hash.c}

Note that ``unchecked'' does not mean ``unsafe''.
The C standard mandates the second interpretation for unsigned overflows,
and signed overflows are usually implementation-defined to use two-complement.
It just means that VCC will loose information about the operation.
For example consider:
\begin{VCC}
int a, b;
// ...
a = b + 1;
_(assert a < b)
\end{VCC}
This will either complain about possible overflow of \vcc{b + 1} or succeed.
However, the following might complain about \vcc{a < b}, if VCC does not know
that \vcc{b + 1} doesn't overflow.
\begin{VCC}
int a, b;
// ...
a = _(unchecked)(b + 1);
_(assert a < b)
\end{VCC}
Think of \vcc{_(unchecked)E} as computing the expression using mathematical 
integers, which never overflow, and then casting the result to the desired range.
VCC knows that \vcc{_(unchecked)E == E} if \vcc{E} fits in the proper range,
and some other basic facts about \vcc{(unsigned)-1}.
If you need anything else, you will need to resort to bit-vector
reasoning (\secref{bv}).

\subsubsection{Bitvector Reasoning}
\label{sect:bv}

Every now and then, you need to prove some low-level fact that VCC
can't prove using ordinary logical reasoning. If the fact involves
can be expressed over a relatively small number of bits, you can ask
VCC to prove it using boolean reasoning at the level of bits, by 
putting \lstinline|{:bv}| after the \vcc{assert} tag. For example:

\vccInputSC[linerange={begin-}]{c/4.3.min5.c}

Assertions proved in this way cannot mention program variables, and
can use only variables of primitive C types.


\section{Loop invariants}

For the most part, VCC computes what it knows at a control point from
what it knows at earlier control points. This works even if there are
\vcc{goto}s from earlier control points; VCC just takes the disjunction of 
what it knows for each of the possible places it came from. 
But when the control flow contains a loop, VCC faces a
chicken-egg problem, since what it knows at the top of the loop (i.e.,
at the beginning of each loop iteration) depends not only on what it
knew just before the loop, but also on what it knew just before it
jumped back to the top of the loop from the loop body.

Rather than trying to guess what it should know at the top of a loop,
VCC lets you tell it what it should know, by providing \Def{loop
  invariants}. To make sure that loop invariants indeed hold 
whenever control reaches the top
of the loop, VCC asserts that the invariants hold wherever control
jumps to the top of the loop -- namely, on loop entry and at the end of
the loop body.

Let's look at an example:
\vccInput[linerange={begin-}]{c/5.1.div.c}
\noindent

The \vcc{divide()} function computes the quotient and remainder of
integer division of \vcc{x} by \vcc{d} using the classic division
algorithm.  The loop invariant says that we have a suitable answer,
except with a remainder that is possibly too big. VCC translates this
example roughly as follows:
\vccInput[linerange={begin-end}]{c/5.2.div_assert.c}

\noindent
Note that this translation has removed all cycles from the control
flow graph of the function (even though it has gotos); this means that
VCC can use the rules of the previous sections to reason about the
program. In VCC, all program reasoning is reduced to reasoning about
acyclic chunks of code in this way.

Note that the invariant is asserted wherever control moves to the top of the
loop (here, on entry to the loop and at the end of the loop body). On
loop entry, VCC forgets the value of each variable modified in the
loop (in this case just the local variables \vcc{lr} and \vcc{ld}),
%% \footnote{ Because of aliasing, it is not always obvious to VCC that a
%%   variable is not modified in the body of the loop. However, VCC can
%%   check it syntactically for a local variable if you never take the
%%   address of that variable.}
and assumes the invariant (which places some constraints on these
variables).  VCC doesn't have to consider the actual jump from the end
of the loop iteration back to the top of the loop (since it has
already checked the loop invariant), so further consideration of that
branch is cut off with \vcc{_(assume \false)}.  Each loop exit is
translated into a \vcc{goto} that jumps to just beyond the loop (to
\vcc{loopExit}). At this control point, we know the loop invariant
holds and that \vcc{lr < d}, which together imply that we have
computed the quotient and remainder.

For another, more typical example of a loop, consider 
the following function that uses linear search to determine if a value
occurs within an array:

\vccInput[linerange={begin-}]{c/5.3.lsearch_full.c}

\noindent
The postconditions say that the returned value is the minimal array
index at which \vcc{elt} occurs (or \vcc{UINT_MAX} if it does not occur).
The loop invariant says that \vcc{elt}  does not occur in \vcc{ar[0]}\dots
\vcc{ar[i - 1]}.

\subsection{Termination measures for loops}
\label{sect:loopTermination}
To prove that a loop terminates, it can be given a \vcc{_(decreases)}
clause, just as a function can. Before control returns from inside the
loop to the top of the loop, there is an implicit assertion that the
measure on the loop has gone down from its value at the beginning of
the iteration. (Note that if the loop body contains a function call,
its measure is checked against the measure assigned to the function,
not to the loop.)

For example, in the \vcc{divide} function, we could specify that the
loop terminates by adding the specification \vcc{_(decreases lr)} to
the loop specification. This would then allow us to add the
specification \vcc{_(decreases 0)} to the divide function itself.

If a function with a termination measure contains a \vcc{for} loop
without a termination measure, VCC tries to guess one from syntactic
form of the loop header. Thus, most boilderplate \vcc{for} loops do
not require explicit termination measures.

\subsection{Writes clauses for loops}
\label{sect:sorting}

Loops are in many ways similar to recursive functions.
Invariants work as the combination of pre- and post-conditions.
Similarly to functions loops can also have writes clauses.
You can provide a writes clause using exactly the same syntax
as for functions.
When you do not write any heap location in the loop (which has been
the case in all examples so far), VCC will automatically infer
an empty writes clause.
Otherwise, it will take the writes clause that is specified on
the function.
So by default, the loop is allowed to write everything that the function
can.
Here is an example of such implicit writes clause,
a reinterpretation of \vcc{my_memcpy()} from \secref{arrays}.

\vccInput[linerange={begin-end}]{c/5.4.copy_array.c}
(Note that VCC also inferred an appropriate
termination measure for the \vcc{for} loop.)

If a loop does not write everything the function can write
you will often want to provide explicit write clauses.
Here's a variation of \vcc{memcpy()}, which clears (maybe for security reasons)
the source buffer after copying it.

\vccInput[linerange={begin-end}]{c/5.5.copy_and_clear_array.c}

\noindent
If the second loops did not provide a writes clause,
we couldn't prove the first postcondition---VCC 
would think that the second loop could have overwritten \vcc{dst}.

Equipped with that knowledge we can proceed to not only checking
if an array is sorted, as we did in \secref{TODO}, but to actually
sorting it.
The function below implements the bozo-sort algorithm.
The algorithm works by swapping two random elements in an array, checking if the resulting array
is sorted, and repeating otherwise.
We do not recommend using it in production code:
it's not stable, and moreover has a fairly bad time complexity.

\vccInput[linerange={begin-out}]{c/5.6.bozosort.c}

The specification that we use is that the output of the sorting routine is sorted.
Unfortunately, we do not say that it's actually a permutation of the input.
We'll show how to do that in \secref{sorting-perm}.


\subsection*{Exercises}
Specify and verify iterative implementations of the following functions:
\begin{enumerate}
\item
a function that takes two arrays and checks whether
the arrays are equal (\ie whether they contain the same sequence of
elements); 
\item
a function that checks whether two sorted arrays
contain a common element;
\item
a function that checks whether a sorted array contains a given value;
\item
a function that takes an array and checks whether it
contains any duplicate elements;
\item
a function that takes an array and reverses it.
\end{enumerate}

Solutions can be found in the file \vcc{5.7.solutions.c} in the tutorial directory.

%\Def{Ghost data} contains auxiliary information needed to convince VCC about correctness of a program.
%You can think of it as data that the program maintains for the purpose of debugging.
%An example might be a program which only keeps track of count of foobars, whereas
%the specifications of that program also need to use the set of these foobars.
%\Def{Ghost code} is code which manipulates such data (\ie when you increment the count of foobars
%you need to add the specific foobar to the set). Ghost variables are just pieces of local ghost
%data, and ghost functions are functions, which can be only used in specifications and ghost code.
%
%The regular C compiler doesn't see the ghost code.
%Therefore it has no runtime effect, it's only there to help VCC understand why the program works.

%We shall start with ghost functions, which in this case is just a
%macro for another formula.

% Let's leave this out for now -E
%\subsection{Review}
%\todo{Put in some way to get to the relevant info about quantifiers
%  and ghost data?}  
%
%\itodo{I'm not sure how useful this review is. In particular the parts
%where we explain the semantics of if statements and assignments
%in terms of what VCC know just seem confusing. Programmers already
%know what if statement or assignment does. 
%I think it would be useful just to emphasize what VCC doesn't know,
%for loops and function calls.
%--M }

%% You have now learned enough to verify some nontrivial sequential
%% programs that use only base types and arrays.  This is already a very
%% rich domain for programming, and you should take some time using VCC
%% to verify some of the ``toy'' algorithms you learned in school. It's
%% also a good opportunity to review what we've learned so far.

%% At each control point within a function, VCC ``knows'' certain things
%% about the state of the program. Included in this knowledge is what
%% memory locations it can safely read or write. The computation of this
%% knowledge can be summarized as follows:
%% \begin{itemize}
%% \item
%% On entry to a function, it knows the preconditions of the function.
%% \item
%% A memory object is writable if it is mutable and is either listed in the 
%% writes clause of the function or was mutable after the function was entered.
%% \item
%% After \vcc{_(assume E)}, it knows what it knew before the assumption,
%% and in addition knows \vcc{E != 0}.
%% \item
%% \vcc{_(assert E)} asks VCC to prove that what it knows before the
%% assertion implies \vcc{E} (and report an error if it can't). 
%% It also assumes \vcc{E} afterward.
%% %\item
%% %\todo{Break this up into variable assignment and memory assignment?}
%% %An assignment statement \vcc{v = E}, where \vcc{E} doesn't have a 
%% %function call and doesn't mention \vcc{v}%
%% %\footnote{
%% %  If \vcc{E} mentions \vcc{v}, we can imagine the value of \vcc{v} being
%% %  first copied into a fresh temporary variable, which is used in place
%% %  of \vcc{v} within \vcc{E}.
%% %}, asserts that the data needed to
%% %compute \vcc{E} is readable, and that \vcc{v} is writable. After the
%% %assignment, it knows everything it knew before the assignment (except
%% %for what it knew about the value of \vcc{v}), and additionally knows
%% %that \vcc{v == E}. 
%% \item 
%% A function call \vcc{f(args)}, 
%% %where \vcc{args} is a list of variables
%% %\footnote{
%% %  If the arguments to the function call are expressions, we can think
%% %  of these expressions being evaluated and assigned to temporary
%% %  variables before the function call.
%% %}, 
%% asserts that the \vcc{args} are readable, asserts that the objects
%% mentioned in the writes clauses of \vcc{f} are writable,
%% asserts the preconditions of \vcc{f} (with the actual parameters
%% substituted for the formal parameters), forgets what it knew about
%% \vcc{v} and any objects mentioned in the writes clause of \vcc{f}, and
%% finally assumes the postconditions of \vcc{f}.
%% If these postconditions use \vcc{\\old} to refer to parts of
%% the state before the call, we can think of these parts of the state
%% as copied to temporary variables prior to the call. The result of
%% the function can be viewed as being put into a temporary variable of
%% the caller.
%% %\item
%% %For a conditional \vcc{if (p) S1 else S2}, it first asserts that
%% %\vcc{p} is readable. At the beginning of \vcc{S1} (resp. \vcc{S2}), it
%% %knows what it knew before the conditional, and in addition knows
%% %\vcc{p != 0} (resp. \vcc{p == 0}). After the conditional, it knows the 
%% %disjunction (``or'') of what it knew at the end of \vcc{S1} and what
%% %we know at the end of \vcc{S2}.
%% \item
%% For a loop, it knows at the beginning of the loop body just what it
%% knew just before the loop (except that it forgets what it knew about
%% variables modified in the loop), and also knows the loop
%% invariant. Just before the loop, and at any point in the loop body where
%% control jumps back to the top of the loop (including the end of the
%% loop body), it asserts the loop invariant.
%% \end{itemize}


\section{Object invariants}
\label{sect:invariants}

%% Verification in VCC, especially when it comes to concurrent programs,
%% it not so much about loop invariants or functions contracts,
%% as about object invariants. 

Pre- and postconditions allow for associating consistency conditions
with the code.
However, fairly often it is also possible to associate such consistency
conditions with the data and require all the code operating on such data
to obey the conditions.
As we will learn in \secref{concurrency} this is particularly important for
data accessed concurrently by multiple threads,
but even for sequential programs enforcing consistency conditions
on data reduces annotation clutter and allows for introduction of abstraction
boundaries.

% This is not a section about concurrency, thus we need a sequential
% rationale. --M
%When verifying a function, we are usually reasoning about only a very
%small part of the program state. This is particularly true for a
%concurrent program, where concurrency depends on minimizing the amount
%of state that a thread ``owns''. Since a thread only knows about data
%that it owns (and perhaps read-only data that it shares), another
%mechanism is needed to regulate data that is shared between
%threads. But these same mechanisms provide a way to minimize the
%amount of knowledge a thread needs at any program point.

In VCC, the mechanism for enforcing data consistency is \Def{object invariants}, which are conditions associated
with compound C types (\vcc{struct}s and \vcc{union}s).
The invariants of a type describe how ``proper'' objects of that type
behave. 
In this and the following section, we consider only the static aspects of this
behavior, namely what the ``consistent'' states of an object are. 
Dynamic aspects, \ie how objects can change, are covered in \secref{concurrency}.
For example, consider the following type definition of \vcc{'\0'}-terminated
safe strings implemented with statically allocated arrays (we'll see
dynamic allocation later).

\vccInput[linerange={obj-init}]{c/6.1.safestring.c}

\noindent
The invariant of \vcc{SafeString} states that consistent
\vcc{SafeString}s have length not more than \vcc{SSTR_MAXLEN} and are
\vcc{'\0'}-terminated.  Within a type invariant, \vcc{\this} refers to
(the address of) the current instance of the type (as in the first
invariant), but fields can also be referred to directly (as in the
second invariant). 

Because memory in C is allocated without initialization, no nontrivial
object invariant could be enforced to hold at all times
(they would not hold right after allocation).
\Def{Wrapped} objects are ones for which the invariant holds
and which the current thread directly owns (that is they are not
part of representation of some higher-level objects).
After allocating an object we would usually wrap it to make sure its invariant holds
and prepare it for later use:

\vccInput[linerange={init-append}]{c/6.1.safestring.c}

\noindent
For a pointer \vcc{p} of structured type, \vcc{\span(p)} returns the
set of pointers to members of \vcc{p}. Arrays of base types produce
one pointer for each base type component, so in this example,
\vcc{\span(s)} abbreviates the set
\begin{VCC}
  { s, &s->len, &s->content[0], &s->content[1], ..., &s->content[SSTR_MAXLEN] }
\end{VCC}
%pointers to all fields of \vcc{s}.%
%\footnote{
%  This is a bite more complicated when embedded structs are involved,
%  see \secref{TODO}.
%}
Thus, the writes clause says that the function 
%not only can wrap \vcc{s} but can also 
can write the fields of \vcc{s}. 
The postcondition says that the function returns with \vcc{s} wrapped,
which implies also that the invariant of \vcc{s} holds; this invariant
is checked when the object is wrapped. (You can see this check fail by
commenting any of the assignment statements.)

A function that modifies a wrapped object will first unwrap it, make
the necessary updates, and wrap the object again (which causes another
check of the object invariant). Unwrapping an object adds all of its
members to the writes set of a function, so such a function has to
report that it writes the object, but does not have to report writing
the fields of the object.

\vccInput[linerange={append-index}]{c/6.1.safestring.c}

\noindent
Finally, a function that only reads an object need not unwrap, and so
will not list it in its writes clause. For example:

\vccInput[linerange={index-out}]{c/6.1.safestring.c}

The following subsection explains this wrap/unwrap protocol in more details.

\subsection{Wrap/unwrap protocol}
\label{sect:wrap-unwrap}

Because invariants do not always hold,
in VCC one needs to explicitly state which objects are consistent,
using a field \vcc{\closed} which is defined on every object.
A \Def{closed object} is one for which the \vcc{\closed} field
is true, and an \Def{open object} is one where it is false.
The invariants have to (VCC enforces them) to hold only when for closed objects, but
can also hold for open objects.
Newly allocated objects are open, and you need to make them open before disposing them.

In addition to the \vcc{\closed} field each object has an \Def{owner field}.
The owner of \vcc{p} is \vcc{p->\owner}.
%While the consistency flag is always written to by the wrap
%and unwrap operations, the owner field can be under some
%conditions (\secref{dynamic-claims}) written to directly.
%Therefore instead of a function it is referred to with
%a field: the owner of \vcc{p} is \vcc{p->\owner}.
This field is of pointer (object) type, but
VCC provides objects, of \vcc{\thread}
type, to represent threads of execution, so that threads can also own objects.
The idea is that the owner of \vcc{p} should have some special rights to \vcc{p} that others do not.
In particular, the owner of \vcc{p} can transfer ownership of \vcc{p} to
another object (\eg a thread can transfer ownership of \vcc{p} from itself to the memory allocator, 
in order to dispose of \vcc{p}).

When verifying a body of a function VCC assumes that it is being executed by some
particular thread.
The \vcc{\thread} object representing it is referred to as \vcc{\me}.

(Some of) the rules of ownership and consistency are
\begin{enumerate}
\item on every atomic step of the program the invariants of all the closed objects have to hold,
\item only the owning thread can modify fields of an open object,
\item each thread owns itself, and
\item only threads can own open objects.
\end{enumerate}
Thus, by the first two rules, VCC allows updates of objects in the following two situations:
\begin{enumerate}
\item the updated object is closed, the update is atomic, and the update preserves the invariant of the object,
\item or the updated object is open and the update is performed by the owning thread.
\end{enumerate}
In the first case to ensure that an update is atomic, VCC requires that the
updated field has a \vcc{volatile} modifier.
There is a lot to be said about atomic updates in VCC, and we shall do
that in \secref{concurrency}, but for now we're only considering sequentially
accessed objects, with no \vcc{volatile} modifiers on fields.
For such objects we can assume that they \emph{do not change}
when they are closed, so the only way to change their fields is to
first make them open, \ie via method~2 above.

A thread needs to make the object open to update it.
Because making it open counts as an update, the thread needs
to own it first.
This is performed by the unwrap operation, which translates to the following steps:
\begin{enumerate}
\item assert that the object is in the writes set,
\item assert that the object is wrapped (closed and owned by the current thread), 
\item assume the invariant (as a consequence of rule~1, the invariant holds for every closed object),
\item set the \vcc{\closed} field to false, and
\item add the span of the object (\ie all its fields) to the writes set
\end{enumerate}
The wrap operation does just the reverse:
\begin{enumerate}
\item
assert that the object is mutable (open and owned by the current thread),
\item assert the invariant, and
\item set the \vcc{\closed} field to true (this implicitly prevents further writes to the fields of the object).
\end{enumerate}
Let's then have a look at the definitions of \vcc{\wrapped(...)} and \vcc{\mutable(...)}.

\begin{VCC}
logic bool \wrapped(\object o) =
  o->\closed && o->\owner == \me;
logic bool \mutable(\object o) =
  !o->\closed && o->\owner == \me;
\end{VCC}

The definitions of \vcc{\wrapped(...)} and \vcc{\mutable(...)}
use the \vcc{\object} type.
It is much like \vcc{void*}, in the sense that it is a wildcard for any pointer type.
However, unlike \vcc{void*}, it also carries the dynamic information about the type of the pointer.
It can be only used in specifications.

The assert\slash assume desugaring of the \vcc{sstr_append_char()} function looks as follows:

\vccInput[linerange={assert-out}]{c/6.2.safestring_assert.c}

\subsection{Ownership trees}
\label{sect:ownership}

Objects often stand for abstractions that are implemented with
more than just one physical object.
As a simple example, consider our \vcc{SafeString}, changed to have a dynamically
allocated buffer.
The logical string object consists of the control object holding the length
and the array of bytes holding the content.
In real programs such abstraction become hierarchical, \eg an address book might consists of a few hash tables, each
of which consists of a control object, an array of buckets,
and the attached linked lists.

\vccInput[linerange={obj-append}]{c/6.3.safestring_dynamic.c}

\noindent
In C the type \vcc{char[10]} denotes an array with exactly 10 elements.
VCC extends that location to allow
the type \vcc{char[capacity]} denoting an array with \vcc{capacity} elements
(where \vcc{capacity} is a variable).
Such types can be only used in casts. For example, \vcc{(char[capacity])content}
means to take the pointer \vcc{content} and interpret it as an array
of \vcc{capacity} elements of type \vcc{char}.
This notation is used so we can think of arrays as objects (of a special type).
The other way to think about it is that \vcc{content} represents just
one object of type \vcc{char}, whereas \vcc{(char[capacity])content}
is an object representing the array.

The invariant of \vcc{SafeString} specifies that it \Def{owns} the
array object.
The syntax \vcc{\mine(o1, ..., oN)} is roughly equivalent
(we'll get into details later) to:
\begin{VCC}
o1->\owner == \this && ... && oN->\owner == \this
\end{VCC}
Conceptually there isn't much difference between having the \vcc{char}
array embedded and owning a pointer to it.
In particular, the functions operating
on some \vcc{s} of type \vcc{SafeString}
should still list only \vcc{s} in their writes clauses,
and not also \vcc{(char[s->capacity])s->content},
or any other objects the string might comprise of.
To achieve that VCC performs \Def{ownership transfers},
\ie assignments to the \vcc{\owner} field.
Specifically, there is another step when unwrapping an object \vcc{p}:
\begin{enumerate}
\setcounter{enumi}{5}
\item
for each object \vcc{o} owned by \vcc{p},
set \vcc{o->\owner} to \vcc{\me} and add \vcc{o} to the writes set
\end{enumerate}
Similarly, when wrapping \vcc{p}, VCC additionally does:
\begin{enumerate}
\setcounter{enumi}{3}
\item
for each object \vcc{o} that needs to be owned by \vcc{p}
(which is determined by \vcc{p}'s invariant, as you'll see in the next section),
assert that \vcc{o} is wrapped and writable and set \vcc{o->\owner} to \vcc{p}.
\end{enumerate}
Let's have a look at an example:

\vccInput[linerange={append-alloc}]{c/6.3.safestring_dynamic.c}

\noindent
First, let's explain the syntax:
\begin{VCC}
_(unwrapping o) { ... }
\end{VCC}
is equivalent to:
\begin{VCC}
_(unwrap o) { ... } _(wrap o)
\end{VCC}
Let \vcc{cont = (char[s->capacity]) s->content}. 
At the beginning of the function, \vcc{s} is owned by the
current thread (\vcc{\me}) and closed (\ie \vcc{\wrapped}), whereas
(by the string invariant) \vcc{cont} is owned by \vcc{s} (and
therefore closed).  Unwrapping \vcc{s} transfers ownership of 
\vcc{cont} to \vcc{\me}, but \vcc{cont} remains closed.
Thus, unwrapping \vcc{s} makes the string mutable, and \vcc{cont}
wrapped.  Then we unwrap \vcc{cont} (which doesn't own anything, so
the thread gets no new wrapped objects), perform the changes, and wrap
\vcc{cont}.  Finally, we wrap \vcc{s}.  This transfers ownership
of \vcc{cont} from the current thread to \vcc{s}, so \vcc{cont}
is no longer wrapped (but still closed).  Here is
the assert\slash assume translation:

\vccInput[linerange={append-out}]{c/6.4.safestring_dynamic_assert.c}

\noindent
Here, \vcc{\inv(p)} means the (user-defined) invariant of object \vcc{p}.
There are two ownership transfers
of \vcc{cont} to and from \vcc{\me} because \vcc{s} owns \vcc{cont} beforehand,
as specified in its invariant.
However, suppose we had an invariant like the following:
\begin{VCC}
struct S {
  struct T *a, *b;
  _(invariant \mine(a) || \mine(b))
};
\end{VCC}
When wrapping an instance of \vcc{struct S}, VCC wouldn't know which object
to transfer ownership of to the instance. Therefore, 
VCC rejects such invariants, and only allow \vcc{\mine(...)}
as a top-level conjunct in an invariant, unless further annotation is given;
see \secref{dynamic-ownership}.

\subsection{Dynamic ownership}
\label{sect:dynamic-ownership}

When a struct is annotated with \vcc{_(dynamic_owns)} the ownership transfers
during wrapping need to performed explicitly, but \vcc{\mine(...)} can
be freely used in its invariant, including using it under a universal
quantifier.

\vccInput[linerange={obj-set}]{c/6.5.table.c}

\noindent
The invariant of \vcc{struct SafeContainer} states that it owns its underlying array,
as well as all elements pointed to from it.
It also states that there are no duplicates in that array.
Let's now say we want to change a pointer in that array,
from \vcc{x} to \vcc{y}.
After such an operation, the container should own whatever it used
to own minus \vcc{x} plus \vcc{y}.
To facilitate such transfers VCC introduces the \Def{owns set}.
It is essentially the inverse of the owner field.
It is defined on every object \vcc{p} and referred to as \vcc{p->\owns}.
VCC maintains that:
\begin{VCC}
\forall \object p, q; p->\closed ==> 
  (q \in p->\owns <==> q->\owner == p)
\end{VCC}
The operator \vcc{<==>} reads ``if and only if'', and is simply boolean
equality (or implication both ways), with a binding priority lower than implication.
That is, for closed \vcc{p}, the set \vcc{p->\owns} contains exactly
the objects that have \vcc{p} as their owner.
Additionally, the unwrap operation does not touch the owns set,
that is after unwrapping \vcc{p}, the \vcc{p->\owns} still contains
all that objects that \vcc{p} used to own.
Finally, the wrap operation will attempt to transfer ownership
of everything in the owns set to the object being wrapped.
This requires that the current thread has write access to these objects
and that they are wrapped.

Thus, the usual pattern is to unwrap the object, potentially modify the owns
set, and wrap the object.
Note that when no ownership transfers are needed, one can just unwrap
and wrap the object, without worrying about ownership.
Let's have a look at an example, which does perform an ownership transfer:

\vccInput[linerange={set-use}]{c/6.5.table.c}

\noindent
The \vcc{sc_set()} function transfers ownership of \vcc{s} to \vcc{c},
and additionally leaves object initially pointed to by \vcc{s->strings[idx]}
wrapped, \ie owned by the current thread.
\begin{note}
\todo{We should have entire section about BVD. --MM}
VCC needs a help in form of an assertion statement at the beginning:
\vcc{sc_set} gets a wrapped \vcc{c} and \vcc{s}, so it cannot be the
case that \vcc{c} owns \vcc{s}. This is what the assertion says.
Without spelling it out explicitly, VCC thinks that \vcc{s} might be 
somewhere in the \vcc{strings} array beforehand, and
thus after inserting it at \vcc{idx} the distinctness invariant could be violated.
If you look at the error model in that case, you can
see that VCC knows nothing about the truth value of \vcc{s \in c->\owns},
and thus adding an explicit assertion about it helps.
\end{note}
Moreover, it promises that this object is \Def{fresh}, \ie the thread did not own
it directly before.
This can be used at a call site:

\vccInput[linerange={use-out}]{c/6.5.table.c}

\noindent
In the contract of \vcc{sc_add} the string \vcc{s} is mentioned
in the writes clause, but in the postcondition we do not say it's wrapped.
Thus, asserting \vcc{\wrapped(s)} after the call fails.
On the other hand, asserting \vcc{\wrapped(o)} fails before the call,
but succeeds afterwards.
Additionally, \vcc{\wrapped(c)} holds before and after as expected.

\begin{note}
\textbf{How is the write set updated?} \\
Before allowing a write to \vcc{*p} VCC will assert \vcc{\mutable(p)}.
Additionally, it will assert that either \vcc{p} is in the writes
clause, or the consistency or ownership of \vcc{p} was updated after the current
function started executing.
Thus, after you unwrap an object, you modify consistency of all its fields,
which provides the write access to them.
Also, you modify ownership of all the objects that it used to own, providing
write access to unwrap these objects.
In case a write clause is specified on a loop, think of an implicit function
definition around the loop.
\end{note}

%This effectively tells the call site that it no longer has ownership of \vcc{s}.
%Additionally, when we look at the invariant of \vcc{c}, we can even figure out
%that \vcc{s} is indeed no longer wrapped.
%
%VCC does know that the invariant of \vcc{s} holds (because the object is closed),
%but we need to explicitly assert it to bring it into theorem prover scope.
%Normally, this is done by \vcc{unwrap}, or \vcc{requires \wrapped(...)}.
%
%Note the distinction between not being able to prove \vcc{P} and 
%being able to prove \vcc{!P}.
%

\subsection{Ownership domains}

The \Def{sequential ownership domain} of an object \vcc{o} (written
\vcc{\domain(o)}) consists of \vcc{o} along with\footnote{ The domains
  of the objects owned by \vcc{o} are included only if \vcc{o} is not
  declared as \vcc{_(volatile_owns)}; see \secref{concurrency}.} the
union of the ownership domains of all objects owned by \vcc{o}.  In
other words, it's the set of objects that are transitively owned by
\vcc{o}. For most purposes, it is useful to think of \vcc{o} as
``containing'' all of \vcc{\domain(o)}; indeed, if \vcc{o1 != o2} and
neither \vcc{o1} nor \vcc{o2} are in the other's \vcc{\domain}, their
\vcc{\domain}s are necessarily disjoint. In particular, if \vcc{o1}
and \vcc{o2} are owned by threads then (because threads own
themselves) \vcc{o1} and \vcc{o2} are necessarily disjoint.

Writability of \vcc{o} gives a thread potential access to all of
\vcc{\domain(o)}: writability allows the thread to unwrap \vcc{o},
which makes writable both the fields of \vcc{o} and any objects that were
owned by \vcc{o}. Conversely, a call to a function that does not list
a wrapped object \vcc{o} in its writes clause is guaranteed to leave 
all of \vcc{\domain(o)} unchanged\footnote{
  This applies to nonvolatile fields of objects in
  the domain; volatile fields might change silently (see section \secref{concurrency}).
}. However, VCC will only reason about the unchangedness of
\vcc{\domain(o)} if it is explicitly brought to its attention, as in
the following example:

\begin{VCC}
void f(T *p) 
  _(writes p) { ... }
...
T *p, *q, *r;
_(assert \wrapped(q) && q != p)
_(assert q \in \domain(q))
_(assert r \in \domain(q))
f(p);
\end{VCC}
\noindent
The second and third assertions bring to VCC's attention that as long
as \vcc{q} is not explicitly unwrapped or included in the writes
clause of a function call, \vcc{r} and its fields will not change.

\subsection{Simple sequential admissibility}
\label{sect:admissibility0}

Until now we've ignored the issue of constraints on the form of object invariants.
The basic rule is that a state change that preserves the invariants of
all updated objects should also preserve the invariants of any closed,
unupdated object \vcc{o}. If an invariant of \vcc{o} satisfies this
criterion, we say that the invariant is \Def{admissible}. VCC requires
all object invariants to be admissible; this admissibility check is
performed when verifying the declaration of \vcc{o}'s type.
The admissibility check allows VCC to check an update to the state by
just checking the invariants of objects that are actually updated;
admissibility of the remaining objects guarantees that all of their
invariants are preserved also.

Fortunately, the most common case is trivial: if an invariant of
\vcc{o} mentions only (nonvolatile) fields of objects in
\vcc{\domain(o)}, the invariant is necessarily admissible.
More sophisticated kinds of invariants are discussed in \secref{inv2}.

\subsection{Type safety}
\label{sect:type-safety}

In modern languages like Java, C\#, and ML, where memory consists of a
collection of typed objects. Programs in these languages allocate
objects (as opposed to memory), and the type of an object remains
fixed until the object is destroyed. Moreover, a non-null reference to
an object is guaranteed to point to a ``valid'' object. But in C, a
type simply provides a way to interpret a sequence of bytes; nothing
prevents a program from having multiple pointers of different types
pointing into the same physical memory, or having a non-null
pointer point into an invalid region of memory.

That said, most C programs really do access memory using a strict type
discipline and tacitly assume that their callers do
so also. For example, if the parameters of a function are a pointer to
an \vcc{int} and a pointer to a \vcc{char}, we shouldn't have to worry
about crazy possibilities like the \vcc{char} aliasing with the second
half of the \vcc{int}. (Without such assumptions, we would have to
provide explicit preconditions to this effect.)  On the other hand, if
the second parameter is a pointer to an \vcc{int}, we do consider the
possibility of aliasing (as we would in a strongly typed language).
Moreover, since in C objects of structured types literally contain
fields of other types, if the second argument were a struct that had
a member of type \vcc{int}, we would have to consider the possibility
of the first parameter aliasing that member. 

To support this kind of antialiasing by default, VCC essentially
maintains a typed view of memory; in any state, \vcc{p->\valid} means
that \vcc{p} points to memory that is currently ``has'' type
\vcc{p}. The rules governing validity guarantee that in any state, the
valid pointers constitute a typesafe view of memory.  In particular,
valid objects never alias, and valid fields are always fields of valid
objects. Only valid fields (and purely local variables) can be
accessed by a program.

There are rare situations where a program needs to change the type of
memory, i.e., make one object invalid while making valid an object
that aliases with it. The most common example is in the memory
allocator, which needs to create and destroy objects of arbitrary
types from arrays of bytes in its memory pool. Therefore, VCC includes
annotations (explained in \secref{reint}) that explicitly change
object validity (and are in fact the only means to do so).  Thus,
while your program can access memory using pretty much arbitrary types
and typecasting, doing so is likely to require additional
annotations. But for most programs, checking type safety is completely
transparent, so you don't have to worry about it.


\section{Ghosts}
\label{sect:ghosts}

VCC methodology makes heavy use of \Def{ghost} data and code - data
and code that are not seen by the C compiler (and therefore are not
part of the executable), but are included to aid reasoning about the
program. Part of the VCC philosophy is that programmers would rather
writing extra code than have to drive interactive theorem provers, so
ghosts are the preferred way to explain to VCC why your program works.

We have already seen some ghost fields of ordinary data structures
(e.g. \vcc{\closed}, \vcc{\owner}, \vcc{\owns}m \vcc{\valid}, etc.) as
well as built-in pieces of ghost code (e.g. \vcc{_(wrap ...)} and
\vcc{_(unwrap ...)}). This section is about how to write your own
ghost code. 

Our first tyical use of ghost data is to express data abstaction. If
you implement an abstract set as a list, it is good practice to expose
to the client only the set abstraction, while keeping the list
implementation private to the implementation of the data type. One way
to do this is to store the abstract value in a ghost field and update
it explicitly in ghost code when operating on the data
structure\footnote{ An alternative approach is to write a pure ghost
  function that takes a concrete data structure and returns its
  abstract value.  The disadvantage of this approach is that for
  recursive structures like lists, the abstraction function is
  likewise recursive, and so reasoning with it requires substantial
  guidance from the user.}. Functions operating on the data structure
are specified in terms of their effect on the abstract value only; the
connection between the abstract and concrete values is written as an
invariant of the data structure.

VCC's mathematical types are usually better suited to representing these
abstract values than C's built-in types. Here, we will use VCC maps.
Recal that a declaration \vcc{int m[T*]} defines a map \vcc{m} from
\vcc{T*} to \vcc{int}; for any pointer \vcc{p} of type \vcc{T*}
\vcc{m[p]} gives the \vcc{int} to which \vcc{m} maps \vcc{p}.
A map \vcc{bool s[int]} can be thought of as a set of \vcc{int}s: the operation
\vcc{s[k]} will return true if and only if the element \vcc{k} is in the set \vcc{s}.

For example, here is a simple example of a set of \vcc{int}s implemented with an array:
\vccInput[linerange={types-init}]{c/7.0.arraySet.c}

\noindent 
The map \vcc{mem} gives the abstract value of the set. The map
\vcc{idx} says where to find each abstract member of the set. We could
have eliminated \vcc{idx} and instead used existential quantification
to say that every member of the abstract set occurs somewhere in the
list. The disadvantage of using an explicit witness like \vcc{idx} is
that we have to update it appropriately. The advantage is that the
prover doesn't have to guess how to instantiate such existential
quantifiers, which makes the verification more robust.

Here is the initializer for these sets:
\vccInput[linerange={init-mem}]{c/7.0.arraySet.c} 
\noindent
The standard form of a constructor is to take some raw
memory\footnote{
In C, it is normal for constructors to take a pointer to raw memory,
to that they can be used to make objects that are embedded within data
structures. 
}, and
wrap it, while establishing some condition on its abstract value.
Values of maps are constructed using \Def{lambda expressions}.  The
expression \vcc{\lambda T x; E} returns a map, which for any \vcc{x}
returns the value of expression \vcc{E} (which can mention \vcc{x}).
If \vcc{S} is the type of \vcc{E}, then this map is of type
\vcc{S[T]}.

Here is the membership test:
\vccInput[linerange={mem-add}]{c/7.0.arraySet.c} 
\noindent
As usual, an accessor is marked as \vcc{_(pure)}, and reads only the
abstract value of the set. Finally, here is a function that adds a new
element to the set:
\vccInput[linerange={add-del}]{c/7.0.arraySet.c} 
\noindent
Note that in addition to updating the concrete representation, we also
update the abstract value and the witness. This example shows another
way to update a map, using array syntax; if \vcc{m} is a variable of
type \vcc{S[T]}, \vcc{e} is of type \vcc{T}, and \vcc{e2} is of
type \vcc{S},  then the statement \vcc{m[e1] = e2} abbreviates the
statement 
\vcc{m = \lambda T v; v == e1 ? e2 : m[v]}.

\subsection*{Exercises}
\begin{enumerate}
\item Extend \vcc{ArraySet} with a function that deletes a value from
  the set. 
\item Modify \vcc{ArraySet} to keep the set without duplication.
\item Modify \vcc{ArraySet} to keep the elements ordered. Use binary
  search to check for membership and for insertion of a new element.
\item Extend \vcc{ArraySet} with a function that adds the contents of
  one set into another. (Try to calculate the size of the combined list
  before modifying the target, so that you can fail gracefully.)
\end{enumerate}

\subsection{Linked Data Structures}
As an example of a more typical dynamic data structure, consider 
the following alternative implementation of \vcc{int} sets as lists:

\vccInput[linerange={types-init}]{c/7.1.list0.c}

\noindent
The invariant states that:
\begin{itemize}
\item the list owns the head node (if it's non-null)
\item if the list owns a node, it also owns the next node (provided it's non-null)
\item if the list owns a node \vcc{n} then \vcc{n->data} is in \vcc{val};
\item if \vcc{v} is in \vcc{val}, then it is the \vcc{data} for some
  node (\vcc{find[v]}) owned by the list.
\end{itemize}

Note that we have chosen to put all of the list
invariants in the \vcc{List} data structure, rather than in the nodes
themselves (which would also work). A disadvantage of putting all of
the invariants in the \vcc{List} type is that when you modify one of
the nodes, you have to check these invariants for all of the nodes
(although the invariants are easy to discharge for nodes that are not
immediate neighbors). Some advantages of this choice is that it is
easier to modify all of the nodes as a group, and that the same
\vcc{Node} type can be used for data structures with different
invariants (e.g., cyclic lists).

Here is the implementation of the \vcc{add} function:

\vccInput[linerange={endspec-member}]{c/7.1.list0.c}

\noindent
We allocate the node, unwrap the list, initialize the new node and
wrap it, and prepend the node at the beginning of the list.  Then we
update the owns set to include the new node, update the abstract value
\vcc{val} and the witness \vcc{find}, and finally wrap the list up
again (when exiting the \vcc{_(unwrapping)} block). 

\begin{note}
The rest of this section can be skipped on first reading.
\end{note}

The invariants of our list say that the abstract value contains
exactly the \vcc{data} fields of nodes owned by the list. It also said
that pointers from list nodes take you to list nodes. But it doesn't
say that every node of the list can be reached from the first node;
the invariants would hold if, in addition to those nodes, the list
also owned some unrelated cycle of nodes. As a result, the natural
algorithm for checking if a value is in the set (by walking down the
list) won't verify; if it finds a value, its data is guaranteed to be
in the set, but not vice-versa. Moreover, the invariants aren't strong
enough to guarantee that the list itself is acyclic.

Thus, we need to strengthen the invariant of the list to guarantee
that every node of the list is reachable from the head. This cannot be
expressed directly with first-order logic, but there are several ways
to express this using VCC using ghost data:
\begin{itemize}
\item you can keep track of the ``depth'' of each list node (i.e.,
  how far down the list it appears);
\item you can maintain the abstract sequence of list nodes (i.e., a
  map from \vcc{\natural} to nodes, along with a \vcc{\natural} giving
  the length of the sequence);
\item you can maintain the ``reachability'' relationship between
  ordered pairs of nodes.
\end{itemize}

For this example, we'll use the first approach. We add the following 
to the definition of the \vcc{List} type:

\vccInput[linerange={moreList-noMoreList}]{c/7.2.list.c}

The new invariants say that the head (if it exists) is at depth 0 (and
is the only node at depth 0), that depth increases by 1 when following
list pointers, and that you can never ``miss'' a node by following the
list. (These are similar to the invariants you would use if each node
had a key and you were maintaining an ordered list.)

The only change to the code we've previously seen is that when adding
a node to the list, we have to also update the node indices:
\begin{VCC}
l->idx = (\lambda Node *m; m == n ? 0 : l->idx[m] + 1);
\end{VCC}

We can now write and verify the membership test:
\vccInput[linerange={member-out}]{c/7.2.list.c}

\noindent Note that the second invariant of the loop is analogous to the
invariant we would use for linear search in an array.

The \vcc{_(assert)} is an example of a situation where VCC needs a
little bit to see why what you think is true really is true. The loop
invariant says that there are no nodes in the list with key \vcc{k},
but VCC on its own will fail to make the appropriate connection to
\vcc{l->val[k]} via \vcc{l->find[k]} without this hint (which is just
giving an instantiation for the last list invariant).

\subsection*{Exercises}
\begin{enumerate}
\item
Modify the list implementation so that on a successful membership
test, the node that is found is moved to the front of the list. (Note
that the resulting function is no longer pure.)
\item 
Implement sets using sorted lists.
\item
Implement sets using binary search trees.
\end{enumerate}


\subsection{Sorting revisited}
\label{sect:sorting-perm}

In \secref{sorting} we verified that bubblesort returns a sorted array.
But we didn't prove that it returned a permutation of the input
array\footnote{For arrays in which no value occurs more than once,
  this property can be expressed that every value in the output array
  is in the input array. But with multiple occurrances, this would
  require stating that the multiplicity of each value is the same, a
  property that isn't first-order.
}. To express this postcondition, we return a ghost map, which states the
exact permutation that the sorting algorithm produced:

\vccInput[linerange={begin-out}]{c/7.3.sort.c}

This sample introduces two new features.
The first is the output ghost parameter \vcc{_(out Perm p)}.
An \vcc{out} parameter is used to return data from the function to the
caller. (One could also do it with a pointer to ghost memory, but
using an \vcc{out} parameter is simpler and more efficient.)

To call \vcc{sort()} you need to supply a local variable to hold
the permutation when the function exits, as in:
\begin{VCC}
void f(int *buf, unsigned len)
  // ...
{
  _(ghost Perm myperm; )
  // ...
  sort(buf, len _(out myperm));
}
\end{VCC}
The effect is to copy the value of the local variable \vcc{p} of the
function to the variable \vcc{myperm} on exit from the function.

The second feature is the use of \Def{record} types. A record type is
introduced by putting \vcc{_(record)} before the definition of a 
like a \vcc{struct} type, but is a pure (ghost) value. Fields of
records do not have distinct addresses, and two records are equal iff
their fields are all equal (whereas C does not even allow \vcc{==} to be
applied to instances of \vcc{struct} types, because of
padding). Thus, VCC does not allow a \vcc{struct} type to be used as
the domian of a map, but does allow record types.

\subsection*{Exercises}
\begin{enumerate}

\item 
Verify your favorite sorting functions (quicksort, heapsort,
mergesort, etc.).

\end{enumerate}

\section{Atomics}
\label{sect:concurrency}

Writing concurrent programs is generally considered to be harder than writing
sequential programs.
Similar opinions are held about verification.
Surprisingly, in VCC the leap from verifying sequential programs to
verifying fancy lock-free code is not that big.
This is because the verification in VCC is inherently based on invariants:
conditions that are attached to data and need to hold \emph{no matter which thread}
accesses it.

But let us move from words to actions, and verify a canonical example
of a lock-free algorithm, which is the implementation of a spin-lock itself.
The spin-lock data-structure is really simple -- it contains just a single
boolean field, meant to indicate whether the spin-lock
is currently acquired.
However, in VCC we would like to attach some formal meaning to this boolean.
We do that through ownership -- the spin-lock will protect some object,
and will own it whenever it is not acquired.
Thus, the following invariant should come as no surprise:

\vccInput[linerange={lock-init}]{c/08_lockw.c}

\noindent
We use a ghost field to hold a reference to the object meant to be protected
by this lock.
If you wish to protect multiple objects with a single lock, you can make
the object referenced by \vcc{protected_obj} own them all.
The \vcc{locked} field is annotated with \vcc{volatile}.
It has the usual meaning for the regular C compiler (\ie it makes the compiler
assume that the environment might write to that field, outside the knowledge
of the compiler).
For VCC it means that the field can be written also when the object is
closed (that is after wrapping it).
The idea is that we will not unwrap the object, but write it atomically,
while preserving its invariant.
The attribute
\vcc{_(volatile_owns)} means that we want the \vcc{\owns} set
to be treated as a volatile field (\ie we want to be able to write
it while the object is closed; normally this is not possible).

First, let's have a look at lock initialization:

\vccInput[linerange={init-xchg}]{c/08_lockw.c}

\noindent
One new thing there is the use of \Def{ghost parameter}.
The regular lock initialization function prototype does not say which
object the lock is supposed to protect, but our lock invariant requires it.
Thus, we introduce additional parameter for the purpose of verification.
A call to the initialization will look like \vcc{InitializeLock(&l _(ghost o))}.

Second, we require that the object to be protected is wrapped (recall that wrapped means closed
and owned by the current thread).
We need it to be closed because we will want to make the lock own it, and
lock can only own closed objects.
We need the current thread to own it, because ownership transfer can
only happen between the current thread and an object,
and not for example some other thread and an object.
Third, we say we're going to write the protected object.
This allows for the transfer, and prevents the calling function from assuming
that the object stays wrapped after the call.
Note that this contract is much like the contract of the function
adding an object to a container data-structure, like
\vcc{sc_add()} from \secref{dynamic-ownership}.

Now we can see how we operate on volatile fields.
We shall start with the function releasing the lock, as it is simpler,
than the one acquiring it.

\vccInput[linerange={release-out}]{c/08_lockw.c}

\noindent
First, let's have a look at the contract.
\vcc{Release()} requires the lock to be wrapped.%
\footnote{ You might wonder how multiple threads can all own the lock (to have it
wrapped), we will fix that later. }
The preconditions on the protected object are very similar to the
preconditions on the \vcc{InitializeLock()}.
Note that the \vcc{Release()} does not mention the lock in its writes clause,
this is because the write it performs is volatile.
Intuitively, VCC needs to assume such writes can happen at any time, so one additional
write from this function doesn't make a difference.

The \vcc{atomic} block is similar in spirit to the \vcc{unwrapping} block ---
it allows for modifications of listed objects and checks if their invariants
are preserved.
The difference is that the entire update happens instantaneously from the point
of view of other threads.
We needed the unwrapping operation because we wanted to mark that we temporarily
break the object invariants.
Here, there is no point in time where other threads can observe that the invariants
are broken.
Invariants hold before the beginning of the atomic block (by our principal reasoning
rule, \secref{wrap-unwrap}), and we check the invariant at the end of the atomic block.

The question arises, what guarantees that other threads won't interfere with the atomic
action?
VCC allows only one physical memory operation inside of an atomic block,
which is indeed atomic from the point of view of the hardware.
Here, that operation is writing to the \vcc{l->locked}.
Other possibilities include reading from a volatile field, or a performing
a primitive operation supported by the hardware, like interlocked
compare-and-exchange.
However, inside our atomic block we can also see the update of the owns set.
This is fine, because the ghost code is not executed by the actual hardware.

% I'm not sure if we need this...
\begin{note}
The reason we can use ghost code is a simulation relation between two machines.
Machine A executes the program with ghost code, and machine B executes the program
without ghost code.
Because ghost code cannot write physical data or influence the control flow of physical code
in any way, the contents of physical memory of machines A and B is the same.
Therefore any property we prove about physical memory of A also holds for B.
Now, if we imagine that both machines are multi-threaded, and the machine A blocks
other threads when it's executing ghost code, the same simulation property will still hold.
\end{note}

It is not particularly difficult to see that this atomic operation preserves the
invariant of the lock.
But this isn't the only condition imposed by VCC here.
To transfer ownership of \vcc{l->protected_obj} to the lock, we also need
write permission to the object being transferred, and
we need to know it is closed.
For example, should we forget to mention \vcc{l->protected_obj}
in the writes clause VCC will complain about:

\vccInput[linerange={out-999}]{c/08_lockw_wrong.c}

\noindent
As another example, should we forget to perform the ownership transfer inside of \vcc{Release()}, VCC will complain
about the invariant of the lock:

\vccInput[linerange={out-999}]{c/08_lockw_wrong2.c}

Let's then move to \vcc{Acquire()}. 
The specification is not very surprising: it requires the lock to be wrapped,
and ensures that after the call the thread will own the protected object,
and moreover, that the thread didn't directly own it before.
This is much like the postcondition on \vcc{sc_add()} function
from \secref{dynamic-ownership}.

\vccInput[linerange={acquire-release}]{c/08_lockw.c}

\noindent
The \vcc{InterlockedCompareAndExchange()} function is a compiler built-in,
which on the x86/x64 hardware translates to the \vcc{cmpxchg} assembly instruction.
It takes a memory location and two values.
If the memory location contains the first value, then it is replaced with the second.
It returns the old value.
The entire operation is performed atomically (and is also a write barrier).

\begin{note}
VCC doesn't have all the primitives of all the C compilers predefined.
One can define them by suppling a body.
It is presented only to the VCC compiler (it is enclosed in
\vcc{_(atomic_inline ...)}) so that the normal compiler doesn't get confused
about it.

\vccInput[linerange={xchg-acquire}]{c/08_lockw.c}

\noindent
This is one of the places where one needs to be very careful,
as there is no way for VCC to know if the definition you provided matches
the semantics of your regular C compiler.
Make sure to check with the regular C compiler manual for exact semantics
of its built-in functions.

We plan to include a header file with VCC containing a handful of popular operations,
so you can just rename them to fit your compiler. 
\end{note}

\subsection{Using claims}
\label{sect:using-claims}

The contracts of functions operating on the lock require that the lock
is wrapped.
This is because one can only perform atomic operations on objects
that are closed. 
If an object is open, then the owning thread is in full control of it.
However, wrapped means not only closed, but also owned by the current thread,
which defeats the purpose of the lock --- it should be possible
for multiple threads to compete for the lock.
Let's then say, there is a thread which owns the lock.
Assume some other thread \vcc|t| got to know that the lock is closed.
How would \vcc|t| know that the owning thread won't unwrap (or worse yet, deallocate) the lock, just
before \vcc|t| tries an atomic operation on the lock?
The owning thread thus needs to somehow promise \vcc|t|
that lock will stay closed.
In VCC such a promise takes the form of a \Def{claim}.
Later we'll see that claims are more powerful, but for
now consider the following to be the definition of a claim:

\begin{VCC}
_(ghost 
typedef struct {
  \ptrset claimed;
  _(invariant \forall \object o; o \in claimed ==> o->\closed)
} \claim_struct, *\claim;
)
\end{VCC}

\noindent
Thus, a claim is an object, with an invariant stating that a number of other objects
(we call them \Def{claimed objects}) are closed.
As this is stated in the invariant of the claim, it only needs to be true
as long as the claim itself stays closed.

Recall that what can be written in invariants is subject to the admissibility
condition, which we have seen partially explained in \secref{admissibility0}.
There we said that an invariant should talk only about things the object owns.
But here the claim doesn't own the claimed objects,
so how should the claim know the object will stay closed?
In general, an admissible invariant can depend on other objects invariants always being
preserved (we'll see the precise rule in \secref{inv2}).
So VCC adds an implicit invariant to all types
marked with \vcc{_(claimable)} attribute.
This invariant states that the object cannot be unwrapped when
there are closed claims on it.
More precisely, each claimable object keeps track of the count of outstanding
claims.
The number of outstanding claims on an object is stored in
\vcc{\claim_count} field.

Now, getting back to our lock example, the trick is that there can be
multiple claims claiming the lock (note that this is orthogonal to
the fact that a single claim can claim multiple objects).
The thread that owns the lock will need to keep track of who's using
the lock.
The owner won't be able to destroy the lock (which requires unwrapping it),
before it makes sure there is no one using the lock.
Thus, we need to add \vcc{_(claimable)} attribute to our lock
definition, and change the contract on the functions operating
on the lock. As the changes are very similar we'll only
show \vcc{Release()}.

\vccInput[linerange={release-struct_data}]{c/08_lock_claimsobj.c}

\noindent
We pass a ghost parameter holding a claim.
The claim should be wrapped.
The function \vcc{\claims_obj(c, l)} is defined to be
\vcc{l \in c->claimed}, \ie that the claim claims the lock.
We also need to know that the claim is not the protected object,
otherwise we couldn't ensure that the claim is wrapped after the call.
This is the kind of weird corner case that VCC is very good catching
(even if it's bogus in this context).
Other than the contract, the only change is that we list the claim
as parameter to the atomic block.
Listing a normal object as parameter to the atomic makes VCC know you're
going to modify the object.
For claims, it is just a hint, that it should use this claim when trying
to prove that the object is closed.

Additionally, the \vcc{InitializeLock()} needs to ensure \vcc{l->\claim_count} \vcc{== 0}
(\ie no claims on freshly initialized lock).
VCC even provides a syntax to say something is wrapped and has no claims: \vcc{\wrapped0(l)}.

\subsection{Creating claims}
\label{sect:creating-claims}

When creating (or destroying) a claim one needs to list the claimed objects.
Let's have a look at an example.

\vccInput[linerange={create_claim-out}]{c/08_lock_claimsobj.c}

This function tests that we can actually create a lock, create a claim on it,
use the lock, and then destroy it.
The \vcc{InitializeLock()} leaves the lock wrapped and writable by the current thread.
This allows for the creation of an appropriate claim, which is then passed to \vcc{Acquire()} and \vcc{Release()}.
Finally, we destroy the claim, which allows for unwrapping of the lock, and subsequently deallocating
it when the function activation record is popped off the stack.

The \vcc{\make_claim(...)} function takes the set of objects to be claimed
and a property (an invariant of the claim, we'll get to that in the next section).
Let us give desugaring of \vcc{\make_claim(...)} for a single object
in terms of the \vcc{\claim_struct} defined in the previous section.

\begin{VCC}
// c = \make_claim({o}, \true) expands to
o->\claim_count += 1;
c = malloc(sizeof(\claim_struct));
c->claimed = {o};
_(wrap c);

// \destroy_claim(c, {o}) expands to
assert(o \in c->claimed);
o->\claim_count -= 1;
_(unwrap c);
free(c);
\end{VCC}


Because creating or destroying a claim on \vcc{c} assigns to
\vcc{c->\claim_count}, it requires write access to that memory location.
One way to obtain such access is getting sequential write access to \vcc{c} itself:
in our example the lock is created on the stack and thus sequentially writable.
We can thus create a claim and immediately use it.
A more realistic claim management scenario is described in \secref{dynamic-claims}.
%when a thread creates an object, constructs
%a number of claims on it, and stores the claims in some shared, possibly global, data-structures
%(\eg a work-queue) where other threads can access them.

The \vcc{\true} in \vcc{\make_claim(...)} is the claimed property (an invariant
of the claim), which will be explained in the next section.

\begin{note}
The destruction can possibly leak claim counts, \ie one could say:
\begin{VCC}
\destroy_claim(c, {});
\end{VCC}
\noindent
and it would verify just fine.
This avoids the need to have write access to \vcc{p}, but on the other hand prevents
\vcc{p} from unwrapping forever (which might be actually fine if \vcc{p} is a ghost object).
%It seems clear why the claimed objects need to be listed when creating a claim, but
%why do we need them for destruction?
\end{note}

\subsection{Two-state invariants}
\label{sect:inv2}

Sometimes it is not only important what are the valid states of objects,
but also what are the allowed \emph{changes} to objects.
For example, let's take a counter keeping track of certain operations
since the beginning of the program.

\vccInput[linerange={counter-reading}]{c/09_counter.c}

\noindent
Its first invariant is a plain single-state invariant -- for some reason
we decided to exclude zero as the valid count.
The second invariant says that for any atomic update of (closed)
counter, \vcc{v} can either stay unchanged or increment by exactly one.
The syntax \vcc{\old(v)} is used to refer to value of \vcc{v} before
an atomic update, and plain \vcc{v} is used for the value of \vcc{v}
after the update.
(Note that the argument to \vcc{\old(...)} can be an arbitrary expression.)
That is, when checking that an atomic update preserves the invariant
of a counter, we will take the state of the program right
before the update, the state right after the update, and check
that the invariant holds for that pair of states.

\begin{note}
In fact, it would be easy to prevent any changes to some field \vcc{f}, by
saying \vcc{_(invariant \old(f) == f)}.
This is roughly what happens under the hood when a field is
declared without the \vcc{volatile} modifier.
\end{note}

As we can see the single- and two-state invariants are both defined
using the \vcc{_(invariant ...)} syntax.
The single-state invariants are just two-state invariants, which do not use
\vcc{\old(...)}.
However, we often need an interpretation of an object invariant in a single state \vcc{S}.
For that we use the \Def{stuttering transition} from \vcc{S} to \vcc{S} itself.
VCC enforces that all invariants are \Def{reflexive} that is if they hold
over a transition \vcc{S0, S1}, then they should hold in just \vcc{S1}
(\ie over \vcc{S1, S1}).
In practice,
this means that \vcc{\old(...)} should be only used to describe
how objects change, and not what are their proper values.
In particular,
all invariants which do not use \vcc{\old(...)} are reflexive, and so
are all invariants of the form \vcc{\old(E) == (E) || (P)}, for any expression \vcc{E} and condition \vcc{P}.
On the other hand, the invariants \vcc{\old(f) < 7} and \vcc{x == \old(x) + 1} are not reflexive.

Let's now discuss where can you actually rely on invariants being preserved.

\begin{VCC}
void foo(struct Counter *n)
  _(requires \wrapped(n))
{
  int x, y;
  atomic(n) { x = n->v; }
  atomic(n) { y = n->v; }
}
\end{VCC}

\noindent
The question is what do we know about \vcc{x} and \vcc{y}
at the end of \vcc{foo()}.
If we knew that nobody is updating \vcc{n->v} while \vcc{foo()}
is running we would know \vcc{x == y}.
This would be the case if \vcc{n} was unwrapped, but it is wrapped.
In our case, because \vcc{n} is closed, other threads can update it,
while \vcc{foo()} is running, but they will need to
adhere to \vcc{n}'s invariant.
So we might guess that at end of \vcc{foo()} we know
\vcc{y == x || y == x + 1}.
But this is incorrect: \vcc{n->v} might get incremented
by more than one, in several steps.
The correct answer is thus \vcc{x <= y}.
Unfortunately, in general, such properties are very difficult to deduce
automatically, which is why we use plain object invariants and admissibility
check to express such properties in VCC.

\begin{note}
An invariant is \Def{transitive} if it holds over states \vcc{S0, S2},
provided that it holds over \vcc{S0, S1} and \vcc{S1, S2}.
Transitive invariants could be assumed over arbitrary
pairs of states, provided that the object stays closed
in between them. 
VCC does not require invariants to be transitive, though.

Some invariants are naturally transitive (\eg we could say
\vcc{_(invariant \old(x) <= x)} in \vcc{struct Counter},
and it would be almost as good our current invariant).
Some other invariants, especially the more complicated ones,
are more difficult to make transitive.
For example, an invariant on a reader-writer lock might say
\begin{VCC}
_(invariant writer_waiting ==> old(readers) >= readers)
\end{VCC}
\noindent
To make it transitive one needs to introduce version numbers.
Some invariants describing hardware (\eg a step of physical CPU)
are impossible to make transitive.
\end{note}

Consider the following structure definition:

\vccInput[linerange={reading-endreading}]{c/09_counter.c}

\noindent 
It is meant to represent a reading from a counter.
Let's consider its admissibility.
It has a pointer to the counter, and a owns a claim, which
claims the counter.
So far, so good.
It also states that the current value of the counter is no less than \vcc{r}.
Clearly, the \vcc{Reading} doesn't own the counter, so our previous rule
from \secref{admissibility0}, which states
that you can mention in your invariant everything that you own, doesn't apply.
It would be tempting to extend that rule to say ``everything that you own
or have a claim on'', but VCC actually uses a more general rule.
In a nutshell, the rule says that every invariant should be preserved
under changes to other objects, provided that these other objects change
according to their invariants.
When we look at our \vcc{struct Reading}, its invariant cannot be broken when
its counter increments, which is the only change allowed by counters invariant.
On the other hand, an invariant like \vcc{r == n->v} or \vcc{r >= n->v}
could be broken by such a change.
But let us proceed with somewhat more precise definitions.

First, assume that every object invariant holds when the object is not closed.
This might sound counter-intuitive, but remember that closedness is controlled
by a field.
When that field is set to false, we want to \emph{effectively} disable the invariant,
which is the same as just forcing it to be true in that case.
Alternatively, you might try to think of all objects as being closed for a while.

An atomic action, which updates state \vcc{S0} into \vcc{S1}, is \Def{legal} if and only if the invariants of
objects that have changed between \vcc{S0} and \vcc{S1} hold over \vcc{S0, S1}.
In other words, a legal action preservers invariants of updated objects.
This should not come as a surprise: this is exactly what VCC checks
for in atomic blocks.

An invariant is \Def{stable} if and only if it cannot be broken by legal updates.
More precisely, to prove that an invariant of \vcc{p} is stable,
VCC needs to ``simulate'' an arbitrary legal update:
\begin{itemize}
\item Take two arbitrary states \vcc{S0} and \vcc{S1}.
\item Assume that all invariants (including \vcc{p}'s) hold over \vcc{S0, S0}.
\item Assume that for all objects, some fields of which are not the same in \vcc{S0} and \vcc{S1},
their invariants hold over \vcc{S0, S1}.
\item Assume that all fields of \vcc{p} are the same in \vcc{S0} and \vcc{S1}.
\item Check that invariant of \vcc{p} holds over \vcc{S0, S1}.
\end{itemize}
The first assumption comes from the fact that all invariants are reflexive.
The second assumption is legality.
The third assumption follows from the second (if \vcc{p} did change, its invariant would
automatically hold).

An invariant is \Def{admissible} if and only if it is stable and reflexive.

Let's see how our previous notion of admissibility relates to this one.
If \vcc{p} owns \vcc{q}, then \vcc{q \in p->\owns}.
By the third admissibility assumption, after the simulated action \vcc{p} still owns \vcc{q}.
By the rules of ownership (\secref{wrap-unwrap}), only threads can own
open objects, so we know that \vcc{q} is closed in both \vcc{S0}
and \vcc{S1}.
Therefore non-volatile fields of \vcc{q} do not change between \vcc{S0} and \vcc{S1},
and thus the invariant of \vcc{p} can freely talk about their values:
whatever property of them was true in \vcc{S0}, will also be true in \vcc{S1}.
Additionally, if \vcc{q} owned \vcc{r} before the atomic action, and the \vcc{q->\owns} is non-volatile,
it will keep owning \vcc{r}, and thus non-volatile fields of \vcc{r}
will stay unchanged.
Thus our previous notion of admissibility is a special case of this one.

Getting back to our \vcc{foo()} example, to deduce that \vcc{x <= y}, after
the first read we could create a ghost \vcc{Reading} object, and
use its invariant in the second action.
While we need to say that \vcc{x <= y} is what's required,
using a full-fledged object might seem like an overkill.
Luckily, definitions of claims themselves can specify additional invariants.

\begin{note}
The admissibility condition above is semantic: it will be checked by the theorem
prover. 
This allows construction of the derived concepts like claims and ownership,
and also escaping their limitations if needed.
It is therefore the most central concept of VCC verification methodology,
even if it doesn't look like much at the first sight.
\end{note}

\subsection{Guaranteed properties in claims}
\label{sect:claim-props}

When constructing a claim, you can specify additional invariants to put on
the imaginary definition of the claim structure.
Let's have a look at annotated version of our previous \vcc{foo()} function.

\vccInput[linerange={readtwice-endreadtwice}]{c/09_counter.c}

\noindent
Let's give a high-level description of what's going on.
Just after reading \vcc{n->v} we create a claim \vcc{r}, which guarantees
that in every state, where \vcc{r} is closed,
the current value of \vcc{n->v} is no less than the value of \vcc{x}
at the time when \vcc{r} was created.
Then, after reading \vcc{n->v} for the second time, we tell VCC to
make use of \vcc{r}'s guaranteed property, by asserting that it is ``active''.
This makes VCC know \vcc{x <= n->v} in the current state, where also
\vcc{y == n->v}.
From these two facts VCC can conclude that \vcc{x <= y}.

The general syntax for constructing a claim is:

\begin{VCC}
_(ghost c = \make_claim(S, P))
\end{VCC}

\noindent
We already explained, that this requires that \vcc{s->\claim_count} is writable for \vcc{s \in S}.
As for the property \vcc{P}, we pretend it forms the invariant of the claim.
Because we're just constructing the claim, just like during regular object initialization,
the invariant has to hold initially (\ie at the moment when the claim is created,
that is wrapped).
Moreover, the invariant has to be admissible, under the condition
that all objects in \vcc{S} stay closed as long as the claim itself
stays closed.
The claimed property cannot use \vcc{\old(...)}, and therefore it's automatically
reflexive, thus it only needs to be stable to guarantee admissibility.

But what about locals?
Normally, object invariants are not allowed to reference locals.
The idea is that when the claim is constructed, all the locals that the
claim references are copied into imaginary fields of the claim.
The fields of the claim never change, once it is created.
Therefore an assignment \vcc{x = UINT_MAX;} in between the atomic
blocks would not invalidate the claim --- the claim would still
refer to the old value of \vcc{x}.
Of course, it would invalidate the final \vcc{x <= y} assert.

\begin{note}
For any expression \vcc{E} you can use \vcc{\at(\now(), E)} in \vcc{P}
in order to have the value of \vcc{E} be evaluated in the state
when the claim is created, and stored in the field of the claim.
\end{note}

This copying business doesn't affect initial checking of the \vcc{P},
\vcc{P} should just hold at the point when the claim is created.
It does however affect the admissibility check for \vcc{P}:
\begin{itemize}
\item Consider an arbitrary legal action, from \vcc{S0} to \vcc{S1}.
\item Assume that all invariants hold over \vcc{S0, S0}, including assuming \vcc{P} in \vcc{S0}.
\item Assume that fields of \vcc{c} didn't change between \vcc{S0} and \vcc{S1}
(in particular locals referenced by the claim are the same as at the moment of its creation).
\item Assume all objects in \vcc{S} are closed in both \vcc{S0} and \vcc{S1}.
\item Assume that for all objects, fields of which are not the same in \vcc{S0} and \vcc{S1},
their invariants hold over \vcc{S0, S1}.
\item Check that \vcc{P} holds in \vcc{S1}.
\end{itemize}

To prove \vcc{\active_claim(c)} one needs to prove \vcc{c->\closed} and that
the current state is a \Def{full-stop} state, \ie state where all invariants
are guaranteed to hold.
Any execution state outside of an atomic block is full-stop.
The state right at the beginning of an atomic block is also full-stop.
The states in the middle of it (\ie after some state updates) might not be.

\begin{note}
Such middle-of-the-atomic states are not observable by other threads, and therefore
the fact that the invariants don't hold there does not create soundness problems.
\end{note}

The fact that \vcc{P} follows from \vcc{c}'s invariant after the construction
is expressed using \vcc{\claims(c, P)}.
It is roughly equivalent to saying:
\begin{VCC}
\forall \state s {\at(s, \active_claim(c))};
  \at(s, \active_claim(c)) ==> \at(s, P)
\end{VCC}
Thus, after asserting \vcc{\active_claim(c)} in some state \vcc{s},
\vcc{\at(s, P)} will be assumed, which means VCC will
assume \vcc{P}, where all heap references are replaced by their values in
\vcc{s}, and all locals are replaced by the values at the point
when the claim was created.

\itodo{I think we need more examples about that at() business,
claim admissibility checks and so forth}

\subsection{Dynamic claim management}
\label{sect:dynamic-claims}

So far we have only considered the case of creating claims to wrapped objects.
In real systems some resources are managed dynamically:
threads ask for ``handles'' to resources, operate on them,
and give the handles back.
These handles are usually purely virtual --- asking for a handle amounts to incrementing
some counter.
Only after all handles are given back the resource can be disposed.
This is pretty much how claims work in VCC, and indeed they were modeled after this
real-world scenario. 
Below we have an example of prototypical reference counter.

\vccInput[linerange={refcnt-init}]{c/10_rundown.c}

\noindent
Thus, a \vcc{struct RefCnt} owns a resource, and makes sure that the number of outstanding
claims on the resource matches the physical counter stored in it.
\vcc{\claimable(p)} means that the type of object pointed to by \vcc{p} was marked
with \vcc{_(claimable)}.
The lowest bit is used to disable giving out of new references
(this is expressed in the last invariant).

\vccInput[linerange={init-incr}]{c/10_rundown.c}

\noindent
Initialization shouldn't be very surprising:
\vcc{\wrapped0(o)} means \vcc{\wrapped(o) && o->\claim_count == 0},
and thus on initialization we require a resource without any outstanding
claims.

\vccInput[linerange={incr-decr}]{c/10_rundown.c}

\noindent
First, let's have a look at the function contract.
The syntax \vcc{_(always c, P)} is equivalent to:
\begin{VCC}
  _(requires \wrapped(c) && \claims(c, P))
  _(ensures \wrapped(c))
\end{VCC}
Thus, instead of requiring \vcc{\claims_obj(c, r)}, we require that the claim
guarantees \vcc{r->\closed}.
One way of doing this is claiming \vcc{r}, but another is claiming the owner
of \vcc{r}, as we will see shortly.

As for the body, we assume our reference counter will never overflow.
This clearly depends on the running time of the system and usage patterns,
but in general it would be difficult to specify this, and thus we just
hand-wave it.

The new thing about the body is that we make a claim on the resource,
even though it's not wrapped.
There are two ways of obtaining write access to \vcc{p->\claim_count}:
either having \vcc{p} writable sequentially and wrapped,
or in case \vcc{p->\owner} is a non-thread object, checking
invariant of \vcc{p->\owner}.
Thus, inside an atomic update on \vcc{p->\owner} (which will check the invariant of \vcc{p->\owner}) one can create
claims on \vcc{p}.
The same rule applies to claim destruction:

\vccInput[linerange={decr-use}]{c/10_rundown.c}

\noindent
A little tricky thing here, is that we need to make use of the \vcc{handle} claim
right after reading \vcc{r->cnt}. 
Because this claim is valid, we know that the claim count on the resource
is positive and therefore (by reference counter invariant) \vcc{v >= 2}.
Without using the \vcc{handle} claim to deduce it we would get a complaint
about overflow in \vcc{v - 2} in the second atomic block.

Finally, let's have a look at a possible use scenario of our reference counter.

\vccInput[linerange={use-enduse}]{c/10_rundown.c}

\noindent
The \vcc{struct B} contains a \vcc{struct A} governed by a reference counter.
It owns the reference counter, but not \vcc{struct A} (which is owned by the reference
counter).
A claim guaranteeing that \vcc{struct B} is closed also guarantees
that its counter is closed, so we can pass it to \vcc{try_incr()},
which gives us a handle on \vcc{struct A}.

Of course a question arises where one does get a claim on \vcc{struct B} from?
In real systems the top-level claims come either from global objects that are
always closed, or from data passed when the thread is created.



\appendix
\section{Triggers}

The triggers are likely the most difficult part of this tutorial.
VCC tries to infer appropriate triggers automatically, so 
trigger annotations were not needed for the examples in the tutorial.
However, you may need them to deal with more complex VCC verification
tasks.

This appendix gives some background on the usage of triggers in the 
SMT solvers, the underlying VCC theorem proving technology.

SMT solvers
prove that the program is correct by looking for possible counterexamples,
or \Def{models}, where your program goes wrong (\eg by violating an assertion).
Once the solver goes through \emph{all} possible counterexamples, and finds them
all to be inconsistent (\ie impossible),
it considers the program to be correct.
Normally, it would take virtually forever, for there is very large number of
possible counterexamples, one per every input to the function (values stored in
the heap also count as input).
To workaround this problem, the SMT solver considers
\Def{partial models}, \ie 
sets of statements about the state of the program.
For example, the model description may say \vcc{x == 7}, \vcc{y > x}
and \vcc{*p == 12}, which describes all the concrete models, where
these statements hold. There is great many such models,
for example one for each different value of \vcc{y} and other program variables,
not even mentioned in the model.

It is thus useful to think of the SMT solver as sitting there with
a possible model, and trying to find out whether the model is consistent or not.
For example, if the description of the model says that \vcc{x > 7} and
\vcc{x < 3}, then the solver can apply rules of arithmetic, conclude this is
impossible, and move on to a next model.
The SMT solvers are usually very good in finding inconsistencies in models
where the statements describing them do not involve universal quantifiers.
With quantifiers things tend to get a bit tricky.

For example, let's say the model description states that the two
following facts are true:
\begin{VCC}
\forall unsigned i; i < 10 ==> a[i] > 7
a[4] == 3
\end{VCC}
The meaning of the universal quantifier is that it should hold
not matter what we substitute for \vcc{i}, for example
the universal quantifier above implies the following facts (which
are called \Def{instances} of the quantifier):
\begin{VCC}
 4 < 10 ==>  a[4] > 7  // for i == 4
\end{VCC}
which happens to be the one needed to refute our model,
\begin{VCC}
11 < 10 ==> a[11] > 7  // for i == 11
\end{VCC}
which is trivially true, because false implies everything, and
\begin{VCC}
 k < 10 ==>  a[k] > 7  // for i == k
\end{VCC}
where \vcc{k} is some program variable of type \vcc{unsigned}.

However, there is potentially infinitely many such instances, and certainly too many
to enumerate them all.
Still, to prove that our model candidate is indeed contradictory
we only need the first one, not the other two.
Once the solver adds it to the model description,
it will simplify \vcc{4 < 10} to true,
and then see that \vcc{a[4] > 7} and \vcc{a[4] == 3} cannot hold
at the same time.

The question remains: how does the SMT solver decide that the first
instance is useful, and the other two are not?
This is done through so called \Def{triggers}.
Triggers are either specified by the user or inferred automatically
by the SMT solver or the verification tool.
In all the examples before we relied on the automatic trigger
inference, but as we go to more complex examples, we'll need to consider
explicit trigger specification.

A trigger for a quantified formula is usually some subexpression 
of that formula, which contains all the variables that the formula
quantifies over.
For example, in the following formula:
\begin{VCC}
\forall int i; int p[int]; is_pos(p, i) ==> f(i, p[i]) && g(i)
\end{VCC}
possible triggers include the following expressions
\vcc|is_pos(p, i)|, \vcc|p[i]|, and also \vcc|f(i, p[i])|,
whereas \vcc|g(i)| would not be a valid trigger, because
it does not contain \vcc|p|.

Let's assume that \vcc|is_pos(p, i)| is the trigger. 
The basic idea is that when the SMT solvers considers a model,
which mentions \vcc|is_pos(q, 7)| (where \vcc|q| is, \eg a local variable), then the formula
should be instantiated with \vcc|q| and \vcc|7| substituted
for \vcc|p| and \vcc|i| respectively.

Note that the trigger \vcc|f(i, p[i])| is \Def{more restrictive} than
\vcc|p[i]|: if the model contains \vcc|f(k, q[k])| it also contains \vcc|q[k]|.
Thus, a ``bigger'' trigger will cause the formula to be instantiated less often,
generally leading to better proof performance (because the solver has
less formulas to work on), but also possibly preventing
the proof altogether (when the solver does not get the instantiation needed for the proof).

Triggers cannot contain boolean operators or the equality operator.
As of the current release, arithmetic operators are allowed, but cause warnings
and work unreliably, so you should avoid them.

A formula can have more than one trigger.
It is enough for one trigger to match in order for the formula
to be instantiated.

\begin{note}
\textbf{Multi-triggers}:
Consider the following formula:
\begin{VCC}
\forall int a, b, c; P(a, b) && Q(b, c) ==> R(a, c)
\end{VCC}
There is no subexpression here, which would contain all the variables
and not contain boolean operators.
In such case we need to use a \Def{multi-trigger}, which
is a set of expressions which together cover all variables.
An example trigger here would be \vcc|{P(a, b), Q(b, c)}|.
It means that for any model, which has both \vcc|P(a, b)|
and \vcc|Q(b, c)| (for the same \vcc|b|!), the quantifier
will be instantiated.
In case a formula has multiple multi-triggers, \emph{all}
expressions in at least \emph{one} of multi-triggers
must match for the formula to be instantiated.

If it is impossible to select any single-triggers in the formula,
and none are specified explicitly, Z3 will select \emph{some}
multi-trigger, which is usually not something that you want.
\end{note}

\subsection{Matching loops}

Consider a model description
\begin{VCC}
\forall struct Node *n; {\mine(n)} \mine(n) ==> \mine(n->next)
\mine(a)
\end{VCC}
Let's assume the SMT solver will instantiate the quantifier with \vcc{a}, yielding:
\begin{VCC}
\mine(a) ==> \mine(a->next)
\end{VCC}
It will now add \vcc{\mine(a->next)} to the set of facts
describing the model.
This however will lead to instantiating the quantifier again,
this time with \vcc{a->next}, and in turn again with
\vcc{a->next->next} and so forth.
Such situation is called a \Def{matching loop}.
The SMT solver would usually cut such loop at a certain depth,
but it might make the solver run out of time, memory, or both.

Matching loops can involve more than one quantified formula. 
For example consider the following, where \vcc{f} is a user-defined function.
\begin{VCC}
\forall struct Node *n; {\mine(n)} \mine(n) ==> f(n)
\forall struct Node *n; {f(n)} f(n) ==> \mine(n->next)
\mine(a)
\end{VCC}


\subsection{Trigger selection}
\label{sect:trigger-inference}

The explicit triggers are listed in \vcc|{...}|, after the quantified variables.
They don't have to be subexpressions of the formula. 
We'll see some examples of that later.
When there are no triggers specified explicitly, VCC selects the triggers for you.
These are always subexpressions of the quantified formula body.
To select default triggers VCC first considers all subexpressions which contain all the quantified variables,
and then it splits them into four categories:
\begin{itemize}
\item level 0 triggers, which are mostly ownership-related. 
These are \vcc{\mine(E)}, \vcc{E1 \in \owns(E2)}, and also \vcc{E1 \in0 E2} (which, except for triggering, is the same as \vcc{E1 \in E2}).
\item level 1 triggers: set membership and maps, that is expressions of 
the form \vcc{E1 \in E2} and \vcc{E1[E2]}.
\item level 2 triggers: default, \ie everything not mentioned elsewhere. 
It is mostly heap dereferences, like \vcc{*p}, \vcc{&a[i]} or \vcc{a[i]}, as well as bitwise arithmetic operators.
\item level 3 triggers: certain ``bad triggers'', which use internal VCC encoding functions.
\item level 4 triggers: which use interpreted arithmetic operations (\vcc{+}, \vcc{-}, and \vcc{*} on integers).
\end{itemize}

Expressions, which contain \vcc{<=}, \vcc{>=}, \vcc{<}, \vcc{>}, \vcc{==},
\vcc{!=}, \vcc{||}, \vcc{&&}, \vcc{==>}, \vcc{<==>}, and \vcc{!} are not
allowed in triggers.

Each of these expressions is then tested for immediate matching loop,
that is VCC checks if instantiating the formula with that trigger
will create a bigger instance of that very trigger.
Such looping triggers are removed from their respective categories.
This protects against matching loops consisting of a single
quantified formula, but matching loops with multiple formulas
are still possible.

To select the triggers, VCC iterates over levels, starting with 0.
If there are some triggers at the current level, these triggers are selected
and iteration stops.
This means that, \eg if there are set-membership triggers then heap dereference
triggers will not be selected.

If there are no triggers in levels lower than 4, VCC tries to select a multi-trigger.
It will only select one, with possibly low level, and some overlap between variables
of the subexpressions of the trigger.
Only if no multi-trigger can be found, VCC will try to use level 4 trigger.
Finally, if no triggers can be inferred VCC will print a warning.

As a post-processing step, VCC looks at the set of selected triggers, and if
any there are two triggers X and Y, such that X is a subexpression of Y, then Y
is removed, as it would be completely redundant.

You can place a \vcc@{:level N}@ annotation in place of a trigger.
It causes VCC to use all triggers from levels 0 to N inclusive.
If this results in empty trigger set, the annotation is silently ignored.

The flag \texttt{/dumptriggers:K} (or \texttt{/dt:K}) can be used to display inferred
triggers.
\texttt{/dt:1} prints the inferred triggers,
\texttt{/dt:2} prints what triggers would be inferred if \vcc@{:level ...}@ annotation
was supplied.
\texttt{/dt:3} prints the inferred triggers even when there are explicit triggers
specified. 
It does not override the explicit triggers, it just print what would happen if you
removed the explicit trigger.

Let's consider an example:
\begin{VCC}
int *buf;
unsigned perm[unsigned];
\forall unsigned i; i < len ==> perm[i] == i ==> buf[i] < 0
\end{VCC}
The default algorithm will infer \vcc@{perm[i]}@, and with \vcc@{:level 1}@
it will additionally select \vcc@{&buf[i]}@.
Note the ampersand.
This is because in C \vcc{buf[i]} is equivalent to \vcc{*(&buf[i])}, and thus
the one with ampersand is simpler.
You can also equivalently write it as \vcc@{buf + i}@.
Note that the plus is not integer arithmetic addition, and can thus be safely used
in triggers.

Another example would be:
\begin{VCC}
\forall struct Node *n; n \in q->\owns ==> perm[n->idx] == 0
\end{VCC}
By default we get level 0 \vcc@{n \in q->\owns}@, with level 1 we also get
\vcc@{perm[n->idx]}@ and with level 2 additionally \vcc@{&n->idx}@.

\subsection{Hints}
\label{sect:trigger-hints}

Consider a quantified formula \vcc|\forall T x; {:hint H} E|.
Intuitively the hint annotation states that the expression \vcc{H} (which can
refer to \vcc{x}) might have something to do with proving \vcc{E}.
A typical example, where you might need it is the following:
\begin{VCC}
\forall struct Node *n; \mine(n) ==> \mine(n->next) && n->next->prev == n
\end{VCC}
The default trigger selection will pick \vcc@{\mine(n->next)}@, which is also
the ``proper'' trigger here. 
However, when proving admissibility, to know that \vcc{n->next->prev} did not
change in the legal action, we need to know \vcc{\mine(n->next)}.
This is all good, it's stated just before, but the SMT solver
might try to prove \vcc{n->next->prev == n} first,
and thus miss the fact that \vcc{\mine(n->next)}.
Therefore, we will need to add \vcc@{:hint \mine(n->next)}@.
For inferred level 0 triggers, these are added automatically.


%\subsection{Typical triggers}
%
%\itodo{maybe we want a section like that?}
%
%\begin{VCC}
%\forall unsigned i; {a[i]} i < 100 ==> a[i] > 0
%\forall unsigned i, j; {f(i, j)} f(i, j) == i + j * 2
%\end{VCC}
%




\section{Memory model}
\label{sect:memmodel}

In most situations in C the type of a pointer is statically known:
while at the machine code level the pointer is passed around as a type-less
word, at the C level, in places where it is used, we know its type.
VCC memory model makes this explicit: pointers are understood as pairs
of their type and address (an word or integer representing location in memory
understood as an array of bytes).
For any state of program execution, VCC maintains the set of \Def{proper pointers}.
\todo{we might want a better name}
Only proper pointers can be accessed (read or written).
There are rules on changing the proper pointer set --- \eg one can remove
a pointer \vcc{(T*)a}, and add pointers \vcc{(char*)a}, \vcc{(char*)(a+1)},
\dots, \vcc{(char*)(a+sizeof(T))}, or \textit{vice versa}.
These rules make sure that at any given time, representations of two
unrelated proper pointers do not overlap, which greatly simplifies reasoning.
Note that given a \vcc{struct SafeString *p}, when \vcc{\proper(p)}
we will also expect \vcc{\proper(&p->len)}.
That is, when a structure is proper, and thus safe to access, so should
be all its fields.
This is what ``unrelated'' means in the sentence above:
the representations overlap if and only if they pointer refer to a struct
and fields of that struct.
It is OK that fields overlap with their containing struct, but that
structs overlap each other.

\subsection{Reinterpretation}
\label{sect:reint}


%% Can the following section move to the appendix? -E
\section{Overflows and unchecked arithmetic}
\label{sect:overflows}

Consider the C expression \vcc{a+b}, when \vcc{a} and \vcc{b} are,
say, \vcc{UINT}s. This might represent one of two programmer
intentions. Most of the time, it is intended to mean ordinary
arithmetic addition on numbers; program correctness is then likely to
depend on this addition not causing an overflow. However, sometimes
the program is designed to cope with overflow, so the programmer means
\vcc{(a + b) \% UINT_MAX+1}. It is always sound to use this
second interpretation, but VCC nevertheless assumes the first by
default, for several reasons:
\begin{itemize}
\item The first interpretation is much more common.
\item The second interpretation introduces an implicit \vcc{\%}
  operator, turning linear arithmetic into nonlinear arithmetic and
  making subsequent reasoning much more difficult.
\item If the first interpretation is intended but the addition can in
  fact overflow, this potential error will only manifest later in the
  code, making the source of the error harder to track down.
\end{itemize}

Here is an example where the second interpretation is intended, but
VCC complains because it assumes the first:
\vccInput{c/02_hash_fail.c}

\noindent
VCC complains that the hash-computing operation might overflow.
To indicate that this possible overflow behavior is desired we use \vcc{_(unchecked)},
with syntax similar to a regular C type-cast.
This annotation applies to the following expression, and indicates that
you expect that there might be overflows in there.
Thus, replacing the body of the loop with the following
makes the program verify:

\vccInput[linerange={update-endupdate}]{c/02_hash.c}

Note that ``unchecked'' does not mean ``unsafe''.
The C standard mandates the second interpretation for unsigned overflows,
and signed overflows are usually implementation-defined to use two-complement.
It just means that VCC will loose information about the operation.
For example consider:
\begin{VCC}
int a, b;
// ...
a = b + 1;
_(assert a < b)
\end{VCC}
This will either complain about possible overflow of \vcc{b + 1} or succeed.
However, the following might complain about \vcc{a < b}, if VCC does not know
that \vcc{b + 1} doesn't overflow.
\begin{VCC}
int a, b;
// ...
a = _(unchecked)(b + 1);
_(assert a < b)
\end{VCC}
Think of \vcc{_(unchecked)E} as computing the expression using mathematical 
integers, which never overflow, and then casting the result to the desired range.
VCC knows that \vcc{_(unchecked)E == E} if \vcc{E} fits in the proper range,
and some other basic facts about \vcc{(unsigned)-1}.
If you need anything else, you will need to resort to bit-vector
reasoning (\secref{bv}).

%% I don't think the following example is worth the trouble -E
%% Let's have a look at another example:
%% \vccInput{c/02_rand.c}
%% \noindent
%% The reason we needed to use unchecked cast here, is that the C library
%% \vcc{rand()} function is specified to return a signed integer.

%% \begin{note}
%% In fact, the C standard, mandates the following specification:
%% \begin{VCC}
%% int rand(void)
%%   _(ensures 0 <= \result && \result <= RAND_MAX);
%% \end{VCC}
%% Thus, in principle the \vcc{_(unchecked)} shouldn't be required in the
%% example above. 
%% However,
%% VCC currently does not come with specifications for C standard library functions.
%% We plan to setup a open-source project, where you'll be able to contribute such
%% specifications.
%% It is indeed unclear why does it return a signed integer, only to ensure that the return value
%% is never negative.
%% \end{note}

\section{Bitvector reasoning}
\label{sect:bv}

Remember our first \vcc{min()} example?  Surprisingly it can get more involved.
For example the one below does not use a branch.

\vccInput[linerange={begin-}]{c/01_min5.c}

\noindent
The syntax:
\begin{VCC}
_( assert {bv} \forall int x; (x & -1) == x )
\end{VCC}

\noindent
VCC to prove the assertion using the fixed-length bit vector theory, a.k.a. machine integers.
This is then used as a lemma to prove the postcondition.




\section{Other VCC Features}

This appendix provides short description of several other
features of the VCC annotation language and the verifier itself.

\subsection{Globals}
\subsection{Groups}
\subsection{Structure inlining}
\subsection{Structure equality}
\subsection{Out parameters}
\subsection{Skinny expose}
\subsection{Mathint}
\subsection{Allocating ghost objects}
\subsection{Smoke}
\subsection{Isabelle}

\section{Soundness}

  \todo{Should we have a caveat list as an appendix to the tutorial?}
  % This is way too arcane to be mentioned in the intro. --M
  %In addition, VCC currently doesn't do the checks needed to
  %ignore memory system optimizations on multiprocessor machines, e.g.,
  %processor store buffering on x64 machines; this will be remedied in the near
  %future.
\section{Solutions to Exercises} %\todo{should exercise sections be numbered?}
\bibliography{tutorial}


\end{document}

% vim: spell
