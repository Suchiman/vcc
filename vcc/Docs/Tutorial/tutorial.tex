\newif\ifdense
\densetrue

\ifdense
\documentclass[preprint,nocopyrightspace]{sigplanconf}
\else
\documentclass{article}
\fi
\bibliographystyle{plain}

\usepackage{amsmath}
\usepackage{stmaryrd}
\usepackage{color}
\usepackage{graphicx}
\usepackage{marginnote}
\usepackage{hyperref}
\usepackage{amsmath, amssymb}
\usepackage{url, listings}
%\usepackage{msrtr}
\usepackage{vcc}
\usepackage{boogie}
\usepackage{xspace}
\definecolor{bgcode}{rgb}{0.92,0.92,0.92}

\definecolor{todocolor}{rgb}{0.9,0.0,0.0}
\definecolor{noteframe}{rgb}{0.5,0.5,0.5}
\ifdense
\newcommand{\todo}[1]{[\textcolor{red}{\textbf{TODO:} {#1}}]}
\else
\newcommand{\todo}[1]{\marginnote{\scriptsize {[\textcolor{red}{\textbf{TODO:} {#1}}]}}}
\fi
\newcommand{\itodo}[1]{{[\textcolor{red}{\textbf{TODO:} {#1}}]}}
%\renewcommand{\todo}[1]{}
%\renewcommand{\itodo}[1]{}
\newcommand{\lines}[1]{\begin{array}{l}#1\end{array}}
\newcommand{\linesi}[1]{\;\;\;\begin{array}{l}#1\end{array}}
\newcommand{\tuple}[1]{{\langle}{#1}\rangle}

\ifdense
\usepackage[T1]{fontenc}
\usepackage{epigrafica}
%\usepackage{iwona}
%\usepackage[scaled]{berasans}
%\usepackage{cmbright}
%\usepackage{lxfonts}
%\usepackage{kurier}
\usepackage{times}
\else
\usepackage[OT1]{fontenc}
\usepackage{courier}
%\usepackage[T1]{fontenc}
%\usepackage[scaled=0.81]{luximono}
\fi

\newcommand{\secref}[1]{Section~\ref{sect:#1}}
\newcommand{\figref}[1]{Figure~\ref{fig:#1}}
\newcommand{\Secref}[1]{Section~\ref{sect:#1}}
\newcommand{\Figref}[1]{Figure~\ref{fig:#1}}

\newcommand{\lemmaref}[1]{Lemma~\ref{lemma:#1}}
\newcommand{\thmref}[1]{Theorem~\ref{thm:#1}}
\newcommand{\lineref}[1]{line~\ref{line:#1}}

\newcommand{\ie}[0]{i.e.,{ }}
\newcommand{\eg}[0]{e.g.,{ }}
\newcommand{\cf}[0]{cf.{ }}

\newenvironment{note}{%
  \noindent\hspace*{0.1\textwidth}%
  \begin{minipage}{0.9\textwidth}%
  \vspace{1.5mm}
  \textcolor{noteframe}{\rule{\textwidth}{1pt}} \\
  \small
  \textbf{Note:}
}{%
  \vspace{-2mm} \\ 
  \textcolor{noteframe}{\rule{\textwidth}{1pt}} 
  \end{minipage}
}

\ifdense
\renewenvironment{note}{%
\vspace{-3mm}
\begin{list}{}%
    {\setlength{\leftmargin}{0.03\textwidth}}%
    \item[]%
  \textcolor{noteframe}{\rule{0.444\textwidth}{1pt}} \\
  %\small
  %\sffamily
  \selectfont
  %\textbf{Note:}
}{%
  \vspace{-2mm} \\ 
  \textcolor{noteframe}{\rule{0.444\textwidth}{1pt}} 
  \end{list}
\vspace{-2mm}
}
\else
\renewenvironment{note}{%
\vspace{-3mm}
\begin{list}{}%
    {\setlength{\leftmargin}{0.1\textwidth}}%
    \item[]%
  \textcolor{noteframe}{\rule{0.9\textwidth}{1pt}} \\
  \small
  \sffamily
  \selectfont
  %\textbf{Note:}
}{%
  \vspace{-2mm} \\ 
  \textcolor{noteframe}{\rule{0.9\textwidth}{1pt}} 
  \end{list}
\vspace{-2mm}
}
\fi

\newcommand{\eqspc}[1]{\;\;{#1}\;\;}
\setlength{\arraycolsep}{0mm}

%%% the model

%%% general
\newcommand{\MathOp}[2]{{}\mathbin{\hbox{$\mkern#2mu#1\mkern#2mu$}}{}}
\newcommand{\Iff}{\MathOp{\Leftrightarrow}{6}}
\newcommand{\Equal}{\MathOp{=}{6}}
\newcommand{\Xor}{\MathOp{\not\equiv}{6}}
\newcommand{\Implies}{\MathOp{\Rightarrow}{4}}
\renewcommand{\And}{\MathOp{\wedge}{2}}
\newcommand{\Or}{\MathOp{\vee}{2}}
\newcommand{\Neg}{\neg}
\newcommand{\dor}[0]{\MathOp{|}{2}}

\newcommand{\ON}[1]{\operatorname{#1}}

% -----------------------------------------------------

\newcommand{\Def}[1]{\textit{\textbf{#1}}}

\begin{document}

\title{Verifying Concurrent C Programs with VCC}
%\title{Developing VCC Specifications \\ for Concurrent C Programs}

\ifdense
\authorinfo{Ernie Cohen, Stephan Tobies}{European Microsoft Innovation Center}{\{ecohen,stobies\}@microsoft.com}
\authorinfo{Micha{\l} Moskal, Wolfram Schulte}{Microsoft Research Redmond}{\{micmo,schulte\}@microsoft.com}
\preprintfooter{VCC Tutorial (working draft)}
\else
\author{ }
\pagestyle{plain} % turn on page numbers
\fi


%\msrtrno{MSR-TR-2010-9}
%{\def\@titletext{foo}}
%\msrtrmaketitle

%\pagebreak
%\begin{figure*}
%\vspace{3in}
%\begin{center}
%This page intentionally left blank.

%\end{center}
%\end{figure*}

%\setcounter{page}{0}

\maketitle


\begin{abstract}
This tutorial provides basic information about developing 
specifications and annotations for concurrent C programs,
so that they can be verified with VCC.
\itodo{add more}
\end{abstract}


\ifdense
\lstset{
  basicstyle=\small\sffamily,
  columns=fullflexible,
}
\else
\lstset{
  basicstyle=\small\ttfamily,
}
\fi

\definecolor{kwColor}{rgb}{0.2,0.2,0.8}

\lstset{
  keywordstyle=\bfseries, %\textcolor{kwColor},
  breaklines=true,
  breakatwhitespace=true,
  numberstyle=\tiny\sf,
  escapeinside={/*-}{*/},
  numbers=none,
  emptylines=1,
  rangeprefix=\/\*\{,
  rangesuffix=\}\*\/,
  includerangemarker=false,  
%  aboveskip=2mm,
%  belowskip=2mm,
%  xleftmargin=2mm,
%  xrightmargin=2mm,
}

\ifdense
\renewcommand{\labelitemi}{{\footnotesize \centeroncapheight{$\bullet$}}}
\fi

\section{Introduction}
This tutorial is an introduction to verifying C code with VCC. Our
primary audience is programmers who want to write correct code and
verification engineers who want to check code for correctness. The
only prerequisite is a working understanding of C. 

To use VCC, you must first \emph{annotate} your code to specify how
your functions and data structures are meant to be used, and what your
functions guarantee to their callers.
VCC then
takes your program and tries to \emph{prove} (mathematically) that
your program meets these specifications.  Unlike most program
analyzers, VCC doesn't look for bugs, or analyze an abstraction of
your program; if VCC certifies that your program is correct, then your
program really should be correct
\footnote{
  In reality, this isn't
  really a guarantee, because VCC itself might have bugs. But in
  practice, this is unlikely to cause you to accidentally verify an
  incorrect program, unless you find and intentionally exploit such a
  bug. 
  Another source of bugs that could slip into verified software would be the compiler,
  the operating system, and the hardware. 
  These are generally more likely to introduce bug than VCC.
  % This is way too arcane to be mentioned in the intro. --M
  %In addition, VCC currently doesn't do the checks needed to
  %ignore memory system optimizations on multiprocessor machines, e.g.,
  %processor store buffering on x64 machines; this will be remedied in the near
  %future.
  }. 

To check your program, VCC generates a number of mathematical
statements (called \Def{verification conditions}) whose proofs
suffice to guarantee the program's correctness, and tries to prove
these statements using an automatic theorem prover. If any of these
proofs fail, VCC reflects these failures back to you in terms of the
program itself (as opposed to the formulas used in the theorem prover).
For example, if your program uses division
somewhere, and VCC is unable to prove (from what it thinks holds at the
point at which the division is done) that the divisor is nonzero, it
will report this to you as a program error at that point in the
program. 
% Seems like a redundant information.
%(If you use VCC from within Visual Studio, it will show in
%your program text the place where the error is, just like a syntax
%error.) 
This doesn't mean that your program is necessarily incorrect;
most of the time, especially when verifying code that is already well-tested,
it is because your specifications aren't strong
enough to guarantee that the suspected error doesn't occur.
Typically, you fix this ``error'' by strengthening your
specifications. This might lead to other error reports, necessitating
the strengthening of other specifications, so verification is in
practice an iterative process.  
Quite often this process will reveal a genuine programming error.

\itodo{the following paragraph should be moved out of here}
Annotating your program sometimes requires doing some extra
programming.  For example, your annotations might need to talk about a
set of users of a data structure, where the implementation only
maintains a count of them.  You would then need to include \Def{ghost
  data} to store this set and write small bits of \Def{ghost code} to
update it.  Ghost code is seen by VCC but not by the C compiler, and
so introduces no runtime overhead. Part of the VCC philosophy is that
programmers would rather do extra programming than drive interactive
theorem provers, so ghost code is the preferred way to help VCC
understand why your program works.  Thus, you interact with VCC
entirely at the level of of code and program states; usually, you can
safely ignore the mathematical reasoning going on ``under the hood''.

This tutorial covers basics of VCC annotation language. By the time
you have finished working through it, you should be able to use VCC to
verify some nontrivial programs. It doesn't cover the theoretical
background of VCC, implementation details, or advanced topics;
information on these can be found on the VCC
homepage\footnote{\url{http://vcc.codeplex.com/}}.

%%  These topics are covered
%% separately~\cite{lci}.  A high-level overview of the VCC tool chain is
%% also available separately~\cite{Cohen:TPHOLs2009-23}.

You can use VCC either from the command line or from Visual Studio
2008 (VS).
The VS interface offers easy access to different components of
VCC tool chain and is thus generally recommended,
but invoking VCC from command line will be also covered.
VCC can be downloaded from its homepage.
Make sure to check out installation instructions\footnote{\url{http://vcc.codeplex.com/wikipage?title=Install}},
as they contain important information about installation prerequisites 
and setting include paths.

%\subsection{Notational conventions}

\begin{note}
  Throughout the tutorial, we'll use notes like this one to discuss
  topics, which can be skipped on the first reading, either because
  they are somewhat more advanced, arcane, or not so important.

  Because this is a tutorial, we will occasionally provide a simplified (and
  therefore not strictly correct) explanation of what is going on, providing
  some additional clarification in the footnotes,\footnote{Like this one.}
  which can be skipped on first reading.
\end{note}

%\subsection{Running VCC from the command line}
%\todo{More options?}
%The easiest way to call VCC on a set of files is
%\begin{verbatim}
%vcc [/f:funs] [/inspector] [/modelviewer] files
%\end{verbatim}
%Via the \verb!/f! switch a comma-separated list of functions to
%verify may be provided. The \verb!/inspector! (or \verb|/i|) switch causes the Z3 Inspector to be
%displayed while the verification is running, to monitor what VCC is
%trying to do. The \verb!/modelviewer! (or \verb|/mv|) switch causes VCC to display models for
%the errors that it finds.
%If there is more than one error, the active model can be switched in the Model menu.
% 
%\subsection{Running VCC from VS}
%\itodo{this should be introduced along with the examples}
%If you right-click within a C source file,
%several VCC commands are made available, depending on what kind of
%construction IntelliSense thinks you are in the middle of. The choice
%of verifying the entire file is always available. If you click within
%the definition of a struct type, VCC will offer you the choice of
%checking admissibility for that type (a concept explained in
%section \ref{}). If you click within the body of a function, VCC should offer
%you the opportunity to verify just that function. However,
%IntelliSense often gets confused about the syntactic structure of
%VCC code, so it may not give these context-dependent
%options. However, if you select the name of a function and then right
%click, it will allow you to verify just that function.
%
%If you want to run the VCC inspector during verification, this
%option can be selected from the Verify$\rightarrow$Settings menu. If you want
%to look at the error model for a particular error, right-click on
%error (in the source), and choose ``Show VCC error model''\todo{This
%  should really say ``VCC error model'', to not expose Z3.}

\section{Assertion and assumptions}
Let's begin with a simple example:

\vccInputSC{c/01_minInline2.c}
This program (internally) sets \vcc{z} to the minimum of \vcc{x} and
\vcc{y}. An annotation of the form \vcc{_( assert E )} says that whenever
control reaches the annotation, the asserted expression should hold (i.e.,
evaluate to something other than 0), so the line 
\vcc{_( assert z <= x && z <= y )} says that when control reaches the
assertion, \vcc{z} is no larger than \vcc{x} or \vcc{y}. Go ahead and
verify this function in VCC. 

VCC verifies this function succesfully which means that this assertion
is indeed correct (and that, in addition, the program cannot
crash). If VCC is unable to prove such an assertion, it reports an
error. Try changing the assertion in this program to something that
isn't true and see what happens.

Assertions, like all VCC annotations, 
are surrounded by \vcc{_( ... )} to indicate that they are
only for VCC, and are not part of the program being verified\footnote{
  The \vcc{<assert.h>} header file defines an \vcc{assert(E)} macro
  with similar meaning, but a different implementation: 
  \vcc{E} is evaluated at runtime (in checked builds) and
  the program terminated if \vcc{E} is false.  The two fundamental
  differences between these runtime assertions and VCC assertions are
  (1) VCC assertions allow expressions that use additional constructs
  (like quantification) that might not be executable, and (2) VCC
  assertions are checked for \emph{all} possible program executions.}.  
For the regular C compiler the \vcc{<vcc.h>} header file defines
\begin{VCC}
#define _(...) /* nothing */
\end{VCC}
VCC does not use this definition, and instead parses the inside of \vcc{_( ... )}
annotations.%
\footnote{
  One can prevent definition of \vcc{_}, and instead surround annotations with
  \vcc{__vcc_spec(...)}.
  This is controlled with \vcc{_VCC_DONT_USE_UNDERSCORE} macro-definition.}

To understand how VCC works, and to use if succesfully, it is useful to
think in terms of what VCC ``knows'' at various program points. In
this example, VCC initially knows nothing about the variables (they
can initially hold any value). Just before the first assignment, VCC knows that 
\vcc{x  <= y} (because execution followed that branch of the conditional), and
after the assignment, VCC knows additionally that \vcc{z} is equal to
\vcc{x}, so it knows that \vcc{z <= x && z <= y}. Similar reasoning
(with \vcc{x} and \vcc{y} reversed) hold in \vcc{else} branch, so VCC
knows that the assertion holds when control reaches it. In general,
VCC doesn't lose any information when reasoning about assignments and
conditionals. We will see in subsequent sections, however, that VCC
can lose information when reasoning about loops and function calls.

When VCC surprises you by failing to verify something that you think
it should be able to verify, it is usually because it doesn't know
something you thought it should. An assertion provides a simple way to
check whether VCC knows what you think it should know.

You can add to what VCC knows at a particular point by adding an
\emph{assumption}. An assumption \vcc{_( assume E )} effectively stops
the program from executing further if \vcc{E} doesn't
hold. Equivalently, you can think of it as telling VCC to just ignore
such executions. Operationally, it simply adds \vcc{E} to what VCC
knows, for subsequent reasoning. For example
\begin{VCC}
  int x,y;
  _( assume x != 0 )
  y = 1 / x;
\end{VCC}

Without the assumption, VCC would complain about possible division by
zero (it checks for division by zero because it would cause the
program to crash).  With the assumption, this error cannot happen.
Succesful verification with VCC only guarantees absence of errors for
those executions in which no assumption is is violated. Thus, you
normally want to verify your program without any assumptions.

So if assumptions are to be avoided, why are assumptions useful at
all? They are useful in several ways:
\begin{itemize}
\item If a verification is not yet complete, assumptions can be used
  to mark the knowledge that VCC is missing, and guide further work.
  If you maintain the discipline of checking in your work only when
  everything verifies (but perhaps with assumptions), other people can
  read your annotated text and see how far away from complete your
  verification is, without having to look at the verification output
  itself.
\item When debugging a failed verification, you can use assumptions to
  narrow down the failed verification to a more specific failure
  scenario.
\item Sometimes you want to assume something even though VCC can
  verify it, just to stop VCC from spending time proving it. For
  example, assuming \vcc{0} (i.e., \vcc{\false}) allows VCC to prove
  all subsequent assertions, effectively focussing its attention on
  other parts of the program.
\item Sometimes you will want to make assumptions about the operating
  environment of a program. For example, you might want to assume that
  a 64-bit counter doesn't overflow, but don't want to justify it
  formally because it depends on extra-logical assumptions (like the
  speed of the hardware and the lifetime of the software). 
\item Finally, assumptions provide a useful tool in explaining how VCC
  reasons about your program. We'll see examples of this throughout
  this tutorial.
\end{itemize}

%%  What is however much more common is
%% use of assumption when debugging your specifications: when the program
%% does not verify you might try to temporarily add assumptions in a bet
%% ``if VCC knew that, would it verify my program?''.  Once you find what
%% is that thing VCC doesn't know, but what makes it verify the program,
%% you can figure out how to specify rest of your program, so that is
%% knows it.

% The following is a bad example, since we would normally specify
% hardware by an invariant on a volatile field.
%% One case is interaction with hardware: you might know the hardware is
%% going to leave specific I/O register positive, and an assumption is a
%% way to communicate that to VCC. 

An assertion can also change what VCC knows after the assertion, if
the assertion fails; even though VCC will report the failure as an error,
it will assume the asserted fact holds afterward. For example
\begin{VCC}
int x;
_( assert x == 0 ) // fails
_( assert x == 0 ) // succeeds
\end{VCC}

\todo{exercises?}
\section{Function specifications}

Now we turn to the specification of functions. We'll take the example
from the previous section, and pull the computation of the miminum of
two numbers out into a separate function:\todo{match with the previous programs}

\vccInput{c/01_min2.c}

(The listing above presents both the source code and the output
of VCC, typeset in a different font and color, and 
the actual file name of the example is replaced with \vcc{/*`testcase`*/}.)
VCC failed to prove our assertion, even though it's easy to see that
it always holds. This is because the verification is modular: VCC
doesn't look inside the body of \vcc{min()} when verifying \vcc{main()}.%
\footnote{
  If VCC did look into \vcc{min()} when verifying \vcc{main()}
  then in general we would quickly run into scalability problems.
  We will see later how to perform such inlining during verification.
}
Or to put it differently, VCC doesn't know anything about what the
\vcc{min()} function does, unless you specify it explicitly\footnote{
  Actually, VCC does know one important thing from this specification
  of \vcc{min()}: it knows that a call do \vcc{min()} has no side
  effects that are visible to the caller. We'll see in section ?? that
  any such side effects have to be explicitly specified.
}.
Since the correctness of \vcc{main()} clearly depends on what \vcc{min()}
does, we need to specify \vcc{min()} in order to verify \vcc{main()}.

The specification of a function is sometimes called a \emph{contract},
because it specifies obligations on both the function and its caller:
\begin{itemize}
\item The requirement on the caller (sometimes called the
  \emph{precondition} of the function) take the form 
  \vcc{_( requires E )}, 
  where \vcc{E} is an expression\footnote{\vcc{E} can mention
  only names that are in scope at the function entry, so it can
  mention parameters of the function and global variables and
  types.}.  In verifying the function, VCC implicitly assumes
  \vcc{E} on function entry, and for each call to the function, VCC
  implicitly asserts \vcc{E} (after the function arguments have been
  evaluated and bound to the parameter names).

\item The requirement on the function (sometimes called the
  \emph{postcondition} of the function) takes the form
  \vcc{_( ensures E )}, where \vcc{E} is an expression; this says that
  the function promises that \vcc{E} holds when control is returned to
  the caller.  In the code of the caller, VCC implicitly assumes \vcc{E}
  just after the call returns (but
  before the result is used). In the code of the function, at each return
  point of the function (including, implicitly,
  at the end of the function body), VCC asserts that the postcondition
  of the function holds (with the vcc keyword \vcc{\result} bound to the
  return value, and the function parameters 
  interpreted according to the values they had on function entry).
\end{itemize}

For example, we can provide a suitable specification for \vcc{min} as
follows: 
\vccInput[linerange={min-endmin,out-}]{c/01_min3.c}
\noindent
The precondition \vcc{_( requires \true )} of \vcc{min()} really doesn't say
anything (since \vcc{\true} holds in every state), and is included
only to emphasize that the function can be called from any state and
with arbitrary input values.
The postcondition states that the value returned from \vcc{Min()} 
is no bigger than either of the inputs.
Note that \vcc{\true} and \vcc{\result} are spelled with a backslash.
All VCC keywords, which can be used in expressions, start with a backslash.
This does not apply to the first keyword after \vcc{_}, because it cannot
be confused with a C identifier (thus you are still free to have, \eg
a function called \texttt{requires} or \texttt{assert}).

As described above, VCC translates the program roughly as follows:
\vccInput{c/01_min_assert.c}

As for call itself, because the \vcc{min()} function above has no
visible specification, VCC doesn't know anything about the value it
might possibly return.  Therefore, as far as VCC's understanding is
concerned, the line:
\begin{VCC}
position = min(newPos, LIMIT);
\end{VCC}
\noindent
could be replaced with:
\begin{VCC}
position = anything();
\end{VCC}
\noindent
that is assigning an unspecified value to the variable \vcc{position}.
The only knowledge about it comes from the following assumption.

\todo{put this into an exercise}
One way to illustrate that the verification of \vcc{main()} 
depends only on the specification of \vcc{min()} is to 
replace \vcc{<} with \vcc{>} in the return statement of \vcc{min()}.
The output of verification shows that \vcc{main()} is still correct -
only \vcc{min()} is broken:
\vccInput[linerange={out-}]{c/01_min4.c}

\todo{put this into an exercise also}
Let's also consider the meaning of \vcc{_( requires \false )}.
When verifying function body it translates to an assumption.
As we have noted this will make the body of function verify,
regardless what it contains.
On the other hand, you will be unable to verify call to such a function
from outside.
Conversely, \vcc{_( ensures \false )} will prevent verification
of the body of the function, but will make the function that calls
it verify.

As we can see, in each case there is something that does not verify.
However, because of this one should be careful when interpreting
VCC answers: successful verification of a function is only meaningful
if everything it calls was verified.

\subsection{Fancy example}
\todo{put this somewhere later in a section about bit vectors}
Finally, 
to show something little less trivial, the implementation of \vcc{min()} can surprisingly be more involved.
For example the one below does not use a branch.

\vccInput[]{c/01_min5.c}

\noindent
The syntax:
\begin{VCC}
_( assert {bv} \forall int x; (x & -1) == x )
\end{VCC}

\noindent
VCC to prove the assertion using the fixed-length bit vector theory, a.k.a. machine integers.
This is then used as a lemma to prove the postcondition.


\subsection{Write Clauses and old(...)}
So far, we have considered only functions without side effects. We
used this implicitly in the translations above; for example, we assumed
that the call to \vcc{min()} didn't have the side effect of changing
the local variables of \vcc{main()}. VCC requires that any side
effects of a function that might be visible to the caller (in the
sense of destroying knowledge that the caller might have) must be explicitly
declared in the specification of the function. For example, because
the caller cannot have any knowledge about the local variables of the
function, changes to these don't have to be reported, whereas changes
to global variables typically do. (We will see later that
VCC only allows the caller to know very specific things, which allows
the function to do other things with global side effects without
having to report them.)

In VCC, you declare that a function might have a visible side effect
my means of a \emph{writes clause}. 
Intuitively, a clause \vcc{_( writes p, q )} is equivalent to saying
that the function is going to modify \emph{only} objects pointed to by \vcc{p}
and objects pointed to by \vcc{q}.
In other words, it is the same as having an ensures clause
saying that everything the caller might know about, other than the objects
pointed to by \vcc{*p} and \vcc{*q}, stays unchanged. VCC allows you
to specify pointers in separate writes clauses, and implicitly
combines them into a single set. If a function spec contains no writes clauses, 
it is equivalent to specifying a writes clause with empty set of pointers.

Here is a simple example of a function that has visible side effects:
\vccInput[linerange={swap-partition}]{c/04_partition.c}

\noindent
The expression \vcc{\old(*p)}, when used in a postcondition of a function,
returns the value of expression \vcc{*p}
as it was at just before the function was called. 

Now, let's have a look at a function calling \vcc{swap()}.

\vccInput[linerange={foo-}]{c/01_swap1.c}

There are two things to note here.
First, one lists pointers to memory locations, not l-values, in the writes
clause (\ie \vcc{_(writes &x, &y, &z)} and not \vcc{_(writes x, y, z)}).
Second, VCC has no problems showing that \vcc{swap(&x, &y)} does not change
\vcc{z}.

Should we forget the writes clause on the \vcc{swap()}, we would get an error:

\vccInput[linerange={out-}]{c/01_swap2.c}

\noindent
\todo{change thread-locality to something better}
The complaints about thread-locality refer to the places where we
read \vcc{*p} and \vcc{*q}.
It is generally unsafe to read something just because you happen to have a pointer to it.
For example, the pointer might be \vcc{NULL}, or point to an object
that is lock-protected, or might have also been \vcc{free()}'d.
For these (and other) reasons VCC implicitly asserts
\vcc{\thread_local(p)} just before any read access to \vcc{*p}.
This predicate implies that that the pointer points to an object that
``belongs'' to the current thread (much more about this later).
Similarly, for write access to \vcc{*p} it inserts the assertion
\vcc{\writable(p)}.
It implies \vcc{\thread_local(p)},
and additionally that \vcc{*p} is in the ``phase of life'' where
it can be modified, and moreover, that it was listed in the current
``writes set''. (You can think of the writes set as being the set
of objects named in the writes clause, along with those objects that
the thread has gotten control of since entry of the function; we will
see how this can happen in section ??.)
For example, the assert/assume translation of \vcc{swap()} is:

\vccInput[linerange={swap-}]{c/01_swap3.c}

\noindent
Because the clause \vcc{writes p, q} essentially gives rise to a precondition
\vcc{requires \writable(p) && \writable(q)}, at a call  \vcc{swap(p, q)},
VCC also inserts \vcc{assert \writable(p) && \writable(q)}.
In particular if we forget to list \vcc{&x} in the writes
clause of \vcc{foo()} we will get:

\vccInput[linerange={out-}]{c/01_swap4.c}

One might wonder why cannot we just have 
\vcc{requires \writable(p)} and instead have the specialized writes clause.
The reason is that the writes clause also specifies that nothing
outside of the writes clause will be changed.
This is why we can prove that \vcc{y} is still \vcc{42} after the call
to \vcc{boundedIncr(&x)}. 


%A predicate \vcc{\mutable(p)} states that the object pointed to by \vcc{p}
%is allocated, ``belongs'' to the current thread, and is in a ``phase of life''
%that allows for modification.
%We will get into details all of these later.
%For now we just need to know that in order to be able to write to \vcc{*p}
%one needs to know that \vcc{p} was listed in the writes clause \emph{and} \vcc{\mutable(p)}.
%For example, if we remove the \vcc{_( writes ... )} clause from the
%\vcc{boundedIncr()} we get the following output:



\section{Loop invariants}

For the most part, what VCC knows at a control point can be
computed from what it knew at the immediately preceding control
points. But when the control flow contains a loop, VCC faces a
chicken-egg problem, since what it knows at the top of the loop (i.e.,
at the beginning of each loop iteration) depends not only on what it
knew just before the loop, but also on what it knew just before it
jumped back to the top of the loop from the loop body.

Rather than trying to guess what it should know at the top of a loop,
VCC lets you tell it what it should know, by providing a \Def{loop
  invariant}. To make sure that the loop invariant does indeed hold 
whenever control reaches the top
of the loop, VCC asserts that the invariant holds wherever control
jumps to the top of the loop -- namely, on loop entry, at the end of
the loop body, and at \vcc{continue} statements within the loop body.
In addition, VCC knows at the top of a loop
that any variable that is not modified in the loop has 
the same value it had on entry to the loop%
\footnote{ Because of aliasing, it is not always obvious to VCC that a
  variable is not modified in the body of the loop. However, VCC can
  check it syntactically for a local variable if you never take the
  address of that variable.}. We simulate this by having VCC forget
(at loop entry) everything it knew about variables that are modified
in the loop body.

Let's look at an example:
\vccInput[]{c/02_div.c}
\noindent

The \vcc{divide()} function computes the quotient and reminder of integer division
of \vcc{x} by \vcc{d} using the classic division algorithm.
%% (Later, we'll see similar examples with sets, which are built into
%% the specification language, but not into C.)
The loop invariant says that we have a suitable answer, except with a
remainder that is possibly too big. VCC translates this example roughly as follows:
\vccInput[]{c/02_div_assert.c}

\noindent
Note that this translation has removed all cycles from the control
flow graph of the function (even though it has gotos); this means that
VCC can use the rules of the previous sections to reason about the
program. In VCC, all program reasoning is reduced to reasoning about
acyclic chunks of code in this way.

Note that the invariant is asserted wherever control moves to the top of the
loop (here, on entry to the loop and at the end of the loop body). On
loop entry, VCC forgets the value of each variable modified in the
loop (in this case just the local variables \vcc{lr} and \vcc{ld}),
and assumes the invariant (which places some constraints on these
variables).  VCC doesn't have to consider the actual jump from the end
of the loop iteration back to the top of the loop (since it has
already checked the loop invariant), so further consideration of that
branch is cut off with \vcc{_(assume \false)}.  Each loop exit is
translated into a \vcc{goto} that jumps to just beyond the loop (to
\vcc{loopExit}). At this control point, we know the loop invariant
holds and that \vcc{lr < d}, which together imply that we have
computed the quotient and remainder.

%% If the loop had been of the form
%% \begin{VCC}
%% while (res >= b) 
%%   _( invariant a % b == res % b )
%% {
%%   res -= b;
%% }
%% \end{VCC}
%% the assert/assume translation would be exactly the same: the invariant
%% would be assumed before checking the guard, despite the fact that
%% guard syntactically precedes the invariant.

% This is still about array 
% This example isn't really about the unnecessity of invariants
%% Sometimes invariants are not required, in particular when the loop
%% doesn't modify any interesting locations. 
For another, more typical example of a loop, consider 
the following function that uses linear search to determine if a value
occurs within an array:
\vccInput{c/03_lsearch.c}

\noindent
The syntax \vcc{\thread_local_array(ar, sz)} means that the
array pointed to by \vcc{ar}, with \vcc{sz} elements is thread-local.
This is required so that the function can read the array.
The postcondition guarantees that when the returned value
is not \vcc{UINT_MAX}, then \vcc{elt} is indeed found at the returned
index.
This doesn't require an invariant, because the non-\vcc{UINT_MAX} index
is returned only under that condition.
However this specification is not full: what about the case when the result is not found?
To express that an element is not in an array we will use a universal quantifier:

\vccInput{c/03_lsearch_full.c}

\noindent
The invariant \vcc{\forall unsigned j; j < i ==> ar[j] != elt} means that 
\vcc{ar[j]} is not \vcc{elt} for all \vcc{j} between \vcc{0} and \vcc{i - 1}.
In other words, take an arbitrary unsigned (and thus non-negative) integer \vcc{j}. 
If it happens to be less than \vcc{i} then the value in \vcc{ar[j]} is not \vcc{elt}.
Thus, the invariant says that the element was not found between \vcc{0}
and \vcc{i - 1}.
If the loop terminates with \vcc{i == sz}, then the postcondition follows.

\subsection{Overflows and unchecked arithmetic}
\label{sect:overflows}

This topic is not really related to loops, but you're very likely to run into it
while verifying loops operating on integers.
Let's have a look at an example:

\vccInput{c/02_hash_fail.c}

\noindent
VCC complains that the hash-computing operation might overflow.
Normally, this is not something to be expected (integer overflows
are a common source of security vulnerabilities nowadays),
which is why VCC warns about it by default.
However, we do expect this in a function computing a hash of a buffer.
To indicate that this overflow behavior is desired we use \vcc{_(unchecked)},
with syntax similar to a regular C type-cast.
This annotation applies to the following expression, and indicates that
you expect that there might be overflows in there.
Thus, replacing the body of the loop with the following
makes the program verify:

\vccInput[linerange={update-endupdate}]{c/02_hash.c}

Note that ``unchecked'' does not mean ``unsafe''.
The C standard mandates modulo interpretation for unsigned overflows,
and signed overflows are usually implementation-defined to use two-complement.
It just means that VCC will loose information about the operation.
For example consider:
\begin{VCC}
int a, b;
// ...
a = b + 1;
_(assert a < b)
\end{VCC}
This will either complain about possible overflow of \vcc{b + 1} or succeed.
However, the following might complain about \vcc{a < b}, if VCC does not know
that \vcc{b + 1} doesn't overflow.
\begin{VCC}
int a, b;
// ...
a = _(unchecked)(b + 1);
_(assert a < b)
\end{VCC}
Think of \vcc{_(unchecked)E} as computing the expression using mathematical 
integers, which never overflow, and then casting the result to the desired range.
VCC knows that \vcc{_(unchecked)E == E} if \vcc{E} fits in the proper range,
and some other basic facts about \vcc{(unsigned)-1}.
If you need anything else, you will need to resort to bit-vector
reasoning (\secref{bv}).

Let's have a look at another example:
\vccInput{c/02_rand.c}
\noindent
The reason we needed to use unchecked cast here, is that the C library
\vcc{rand()} function is specified to return a signed integer.

\begin{note}
In fact, the C standard, mandates the following specification:
\begin{VCC}
int rand(void)
  _(ensures 0 <= \result && \result <= RAND_MAX);
\end{VCC}
Thus, in principle the \vcc{_(unchecked)} shouldn't be required in the
example above. 
However,
VCC currently does not come with specifications for C standard library functions.
We plan to setup a open-source project, where you'll be able to contribute such
specifications.
It is indeed unclear why does it return a signed integer, only to ensure that the return value
is never negative.
\end{note}

\subsection{Sorting}
\label{sect:sorting}

%\Def{Ghost data} contains auxiliary information needed to convince VCC about correctness of a program.
%You can think of it as data that the program maintains for the purpose of debugging.
%An example might be a program which only keeps track of count of foobars, whereas
%the specifications of that program also need to use the set of these foobars.
%\Def{Ghost code} is code which manipulates such data (\ie when you increment the count of foobars
%you need to add the specific foobar to the set). Ghost variables are just pieces of local ghost
%data, and ghost functions are functions, which can be only used in specifications and ghost code.
%
%The regular C compiler doesn't see the ghost code.
%Therefore it has no runtime effect, it's only there to help VCC understand why the program works.

%We shall start with ghost functions, which in this case is just a macro for another formula.
The function below implements the bozo-sort algorithm.
The algorithm works by swapping two random elements in an array, checking if the resulting array
is sorted, and repeating otherwise.
We do not recommend using it in production code:
it's not stable, and moreover has a fairly bad time complexity.
Still, it will serve us to illustrate the use of logic functions, and later (\secref{sorting-perm}) ghost data.

\vccInput[linerange={begin-out}]{c/04_bozosort.c}

\noindent
The meaning of:
\begin{VCC}
_(logic F(A) = E)
\end{VCC}
is similar to:
\begin{VCC}
#define F(A) E
\end{VCC}
except that it prevents
name capture and gives better error messages (\ie ones which mention both \vcc{E} and \vcc{F(A)}).

The other novel thing about this function is the writes clause on the array.
Indeed, all previous examples were only reading arrays, and this one is actually writing
the array.
The notation \vcc{\array_range(arr, len)} refers to all pointers into array
of size \vcc{len} pointed to by \vcc{arr}.
When \vcc{arr} is a pointer to non-struct type (as in the example above),
it expands to \vcc@{ &arr[0], &arr[1], ..., &arr[len-1] }@.
In addition to specifying the writes clause on the function, we also specify it on
the loop.
This is common for loops, which write the heap.
Generally, a writes clause on the loop is the same as the writes clause you would need to
write on a function which contained \emph{only} that loop.
If there is no writes clauses on the loop, VCC takes the writes clause from the 
function, but interprets it with respect to the beginning of the function,
and not beginning of the loop.
Thus, in general, when the loop writes the heap, you will want to specify a writes clause on it.

The specification that we use is that the output of the sorting routine is sorted.
Unfortunately, we do not say that it's actually a permutation of the input.
We'll show how to do that in \secref{sorting-perm}.

\subsection{Review}
%\todo{Put in some way to get to the relevant info about quantifiers
%  and ghost data?}  
%
%\itodo{I'm not sure how useful this review is. In particular the parts
%where we explain the semantics of if statements and assignments
%in terms of what VCC know just seem confusing. Programmers already
%know what if statement or assignment does. 
%I think it would be useful just to emphasize what VCC doesn't know,
%for loops and function calls.
%--M }

You have now learned enough to verify some nontrivial sequential
programs that use only base types and arrays.  This is already a very
rich domain for programming, and you should take some time using VCC
to verify some of the ``toy'' algorithms you learned in school. It's
also a good opportunity to review what we've learned so far.

At each control point within a function, VCC ``knows'' certain things
about the state of the program. Included in this knowledge is what
memory locations it can safely read or write. The computation of this
knowledge can be summarized as follows:
\begin{itemize}
\item
On entry to a function, it knows the preconditions of the function.
\item
A memory object is writable if it is mutable and is either listed in the 
writes clause of the function or was mutable after the function was entered.
\item
After \vcc{_(assume E)}, it knows what it knew before the assumption,
and in addition knows \vcc{E != 0}.
\item
\vcc{_(assert E)} asks VCC to prove that what it knows before the
assertion implies \vcc{E} (and report an error if it can't). 
It also assumes \vcc{E} afterward.
%\item
%\todo{Break this up into variable assignment and memory assignment?}
%An assignment statement \vcc{v = E}, where \vcc{E} doesn't have a 
%function call and doesn't mention \vcc{v}%
%\footnote{
%  If \vcc{E} mentions \vcc{v}, we can imagine the value of \vcc{v} being
%  first copied into a fresh temporary variable, which is used in place
%  of \vcc{v} within \vcc{E}.
%}, asserts that the data needed to
%compute \vcc{E} is readable, and that \vcc{v} is writeable. After the
%assignment, it knows everything it knew before the assignment (except
%for what it knew about the value of \vcc{v}), and additionally knows
%that \vcc{v == E}. 
\item 
A function call \vcc{f(args)}, 
%where \vcc{args} is a list of variables
%\footnote{
%  If the arguments to the function call are expressions, we can think
%  of these expressions being evaluated and assigned to temporary
%  variables before the function call.
%}, 
asserts that the \vcc{args} are readable, asserts that the objects
mentioned in the writes clauses of \vcc{f} are writeable,
asserts the preconditions of \vcc{f} (with the actual parameters
substituted for the formal parameters), forgets what it knew about
\vcc{v} and any objects mentioned in the writes clause of \vcc{f}, and
finally assumes the postconditions of \vcc{f}.
If these postconditions use \vcc{\\old} to refer to parts of
the state before the call, we can think of these parts of the state
as copied to temporary variables prior to the call. The result of
the function can be viewed as being put into a temporary variable of
the caller.
%\item
%For a conditional \vcc{if (p) S1 else S2}, it first asserts that
%\vcc{p} is readable. At the beginning of \vcc{S1} (resp. \vcc{S2}), it
%knows what it knew before the conditional, and in addition knows
%\vcc{p != 0} (resp. \vcc{p == 0}). After the conditional, it knows the 
%disjunction (``or'') of what it knew at the end of \vcc{S1} and what
%we know at the end of \vcc{S2}.
\item
For a loop, it knows at the beginning of the loop body just what it
knew just before the loop (except that it forgets what it knew about
variables modified in the loop), and also knows the loop
invariant. Just before the loop, and at any point in the loop body where
control jumps back to the top of the loop (including the end of the
loop body), it asserts the loop invariant.
\end{itemize}



\subsection*{Exercises}

\begin{enumerate}
\item
Write and verify an iterative program that copies an array if ints from one
location to another.% Repeat with 2-dimensional arrays.
\item
Write and verify an iterative program for binary search (a program
that checks whether a sorted array contains a given value).
\item
Write and verify an iterative program that sorts an array of ints using
bubblesort. The specification should be the same as for bozo-sort above.
%\item 
%Write and verify a program that takes a 2-dimensional array of ints in
%which every row and column is sorted, and checks whether a given int
%occurs in the array.
\end{enumerate}


\section{Object invariants}

Object invariants are predicates describing ``consistent'' states of objects.
Take for example a safe string structure implemented with a
statically allocated array (we'll move to dynamic allocation later).

\vccInput[linerange={beg-index}]{c/05_safestring.c}

\noindent
The invariant of \vcc{struct SafeString} states that consistent instances
of that structure will shorter than \vcc{SSTR_MAXLEN} and \vcc{'\0'}-terminated.
While this seems simple and intuitive, we need a quick overlook of how VCC
models C memory, to get a full understanding.

\subsection{Memory model and the object type}

Talking about instances of structures in C is a tricky business. 
In plain C a structure type generally just gives some guidelines how to interpret
arrays of bytes.
Some programming languages, like Java or ML, have a much more disciplined
view of memory:
one allocates an object of given type, not merely N bytes of memory,
and later there is no way to change the type of that object.
Most C programs also follow this model, most of the time:
when a function gets a pointer to \vcc{struct Foo}, it usually doesn't
expect to find data corresponding to some \vcc{struct Bar} there.
Still, there are rare situations where the program needs to
change type assignment of a pointer.
The most common is in the memory allocator, which needs to create
and destroy objects of arbitrary types from arrays of bytes
in its memory pool.
Therefore, the general rule in VCC is that programs are forced to
follow the strict Java-like type discipline, except for places
where explicit annotations indicate reinterpretations of type assignment
(these annotations are explained in \secref{TODO}).
In particular, as long as there are no type reinterpretations,
the type of a pointer is fixed.

In most situations in C the type of a pointer is statically known:
while at the machine code level the pointer is passed around as a type-less
word, at the C level, in places where it is used, we know its type.
VCC memory model makes this explicit: pointers are understood as pairs
of their type and address (an word or integer representing location in memory
understood as an array of bytes).
For any state of program execution, VCC maintains the set of \Def{proper pointers}.
\todo{we might want a better name}
Only proper pointers can be accessed (read or written).
There are rules on changing the proper pointer set --- \eg one can remove
a pointer \vcc{(T*)a}, and add pointers \vcc{(char*)a}, \vcc{(char*)(a+1)},
..., \vcc{(char*)(a+sizeof(T))}, or \emph{vice versa}.
These rules make sure that at any given time, representations of two
unrelated proper pointers do not overlap, which greatly simplifies reasoning.
Note that given a \vcc{struct SafeString *p}, when \vcc{\proper(p)}
we will also expect \vcc{\proper(&p->len)}.
That is, when a structure is proper, and thus safe to access, so should
be all its fields.
This is what ``unrelated'' means in the sentence above:
the representations overlap if and only if they pointer refer to a struct
and fields of that struct.
It is OK that fields overlap with their containing struct, but that
structs overlap each other.

\todo{this is essentially the take-away, we should consider trimming this
section}
The net result is that VCC needs to check if every memory access is proper,
but in return it gets that representations of proper pointers to structs
are disjoint, and thus pointers to structs can be treated as type-safe
Java-like objects.

The type of type/address pairs, written \vcc{\object}, is exposed in the annotation
language.
For example, functions like \vcc{\thread_local()} discussed before,
or the function \vcc{\proper()} checking if a pointer is in the proper set,
take values of type \vcc{\object}.
They actually need that:
the fact that \vcc{(short*)0xdeadf00d} is proper
does not imply that \vcc{(int*)0xdeadf00d} is proper.
Note that this is different than a function taking \vcc{void*} --- in
VCC there would be no way to tell on value of what actual type the function was called.

\subsection{Wrapped}

Consider the function \vcc{sstr_index_of()} in the example below. 

\vccInput[linerange={index-999}]{c/05_safestring.c}

\noindent
To avoid indexing the \vcc{s->content[]} array out of bounds it needs to know
that \vcc{s->len <= SSTR_MAXLEN}. 
One could argue that all objects at all times should be consistent (\ie their
invariants hold),
so if a function takes an object as an argument, it should better be consistent.
Unfortunately, the reality is more complicated.
In particular, the invariants of objects generally do not hold before they are fully
initialized, or in a middle of a function operating on them.
Thus, in VCC one needs to explicitly state which objects are consistent.
This is what the \vcc{requires \wrapped(s)} precondition achieves:
it states that before the function is called the object should
better be consistent and owned by the current thread.
Now we need to explain ownership, and later we shall have a take
at explaining the naming (\ie why is it wrapped).

\subsection{Ownership}

In VCC each object has a single owner, denoted \vcc{\owner(o)}.
The owner itself is also an object, but VCC provides objects, of \vcc{\thread} type, to represent threads.
Thus, intuitively an object is owned by another object or a thread.
Threads own themselves.
For now we're going to ignore the possibility of non-thread objects owning things.

The current thread of execution is referred to as \vcc{\me}.
The concept of ``currently executing'' threads has only meaning when verifying
functions, so \vcc{\me} cannot be used in invariants.

\begin{VCC}
bool \wrapped(\object o) 
  { return \consistent(o) && \owner(o) == \me; }
bool \unwrapped(\object o) 
  { return !\consistent(o) && \owner(o) == \me; }
\end{VCC}



\section{Abstraction}

Usually there are many ways of implementing a given data structure.
For example, a set might be implemented as a linked list, an array, or a hash table.

When reasoning about a program which uses a data structure we don't want to be
concerned with implementation details of the data structure.
We should reason at somewhat higher, abstract level.
For example, when we use a linked list as a representation of a set, we should not be concerned
with how the list nodes are laid out in memory.

In VCC we abstract over such details using ghost data: data-structures contain ghost fields with mathematical
abstractions of what they represent.
It is only rarely the case that a simple C type, say \vcc{unsigned int}, would be suitable
to store such abstraction (how would one store an unbounded set in a primitive C type?).
To that end, VCC provides \Def{map types}.
The syntax is similar to syntax of array types, \vcc{int m[T*]} defines a map \vcc{m} from \vcc{T*}
to \vcc{int}.
That is the type of expression \vcc{m[p]} is \vcc{int}, provided that \vcc{p} is a pointer
of type \vcc{T*}.
A map \vcc{T* a[unsigned]} is similar to an array of pointers of length $2^{32}$.%
\footnote{
  Because a map can be used only in ghost code, 
  the issue of runtime memory consumption does not apply to it.}
A map \vcc{bool s[int]} can be thought of as a set of \vcc{int}s: the operation
\vcc{s[k]} will return true if and only if the element \vcc{k} is in the set \vcc{s}.

Let's then have a look at an example of a list abstracted as a set:

\vccInput[linerange={types-init}]{c/07_list0.c}

\noindent
The expressions inside \vcc|{...}| after the quantified variables are hints for
the theorem prover called triggers. Ignore them for a moment.
The invariant states that:
\begin{itemize}
\item the list owns the head node (if it's non-null)
\item if the list owns a node, it also owns the next node (provided it's non-null)
\item if the list owns a node, then its data is present in the set \vcc{val};
      this binds the values stored in the implementation to the abstract representation
\end{itemize}
You may note that the set \vcc{val} is under-specified: 
it might be that it has some elements not stored in the list.
We'll get back to this issue later.
Now let's have a look at the specification of a function adding a
node to the list:

\vccInput[linerange={add-endspec}]{c/07_list0.c}

\noindent
The writes-clause and contracts about the list being wrapped are similar to what
we've seen before.
Then, there are the contracts talking about the result value.
This function might fail because there is not enough memory to allocate list node,
in such case it will return a non-zero value (an error code perhaps),
and the contracts guarantee that the set represented by the list will not be changed.
However, if the function succeeds (and thus returns zero), the contract specifies
that if we take an arbitrary integer \vcc{p}, then it is a member of the new abstract
value if and only if it was already a member before or it is \vcc{k}.

In other words, the new value of \vcc{l->val} will be the union of the old
abstract value and the element \vcc{k}.
Ideally, this contract is all that the caller will need to know about that function:
what kind of effect does it have on the abstract state. 
It doesn't specify if the node will be appended at the beginning, or in the middle
of the list.
It doesn't talk about possible duplicates or memory allocation.
Everything about implementation is completely abstracted away.
Still, we need a concrete implementation, and here it goes:

\vccInput[linerange={endspec-member}]{c/07_list0.c}

\noindent
We allocate the node, unwrap the list, initialize the new node
and wrap it (we want to wrap the list, and thus everything it is going to own will
need to be wrapped beforehand; the list is wrapped at the end of the 
unwrapping block), and
prepend the node at the beginning of the list.
Then we update the owns set (we've also already seen that).
Finally, we update the abstract value using a \Def{lambda expression}.
%\subsection{Lambda expressions}
The expression \vcc{\lambda T x; E} returns a map, which for
any \vcc{x} returns the value of expression \vcc{E},
which can reference \vcc{x}.
If the type of \vcc{E} is \vcc{S}, then the type of map, returned by the
lambda expression, is \vcc{S[T]}.
An assignment \vcc{m = \lambda T x; E} has a similar effect to
the following assumption (note that \vcc{E} will most likely reference \vcc{x}):
\begin{VCC}
_(assume \forall T x; m[x] == E)
\end{VCC}
\noindent
Unlike assumptions, lambda expressions do not compromise 
soundness of the verifier.
Just like for assumptions,
the expression is always evaluated in the state as it was
when the lambda was first defined, for example:
\begin{VCC}
int x = 1;
int m[int] = \lambda int y; y + x;
_(assert m[0] == 1) // succeeds
x = 2;
_(assert m[0] == 1) // still succeeds
\end{VCC}
One can imagine, that when this lambda expression is defined,
VCC will iterate over all possible values of \vcc{x},
and store the value of \vcc{E} in \vcc{m[x]}.
Lambda expressions are much like function values in functional
languages or delegates in C\#.

\begin{note}
The body of our lambda expression shows similarity to the body
of the quantifier we have used in specification.
It doesn't, however, have to be the same:
\begin{VCC}
int[int] foo(int v)
  _(ensures \forall int x; x >= 7 ==> \result[x] >= v)
{
  return \lambda int y; (y&1) == 0 ? INTMAX : v;
}
\end{VCC}
Thus, the specifications for lambda expressions can hide information.
\end{note}


\subsection{More ghost state}

\begin{note}
The following subsection, till the beginning of \secref{concurrency} (which is
about concurrency), might somewhat difficult upon first reading of this
tutorial.
It deals with the concept of the reachable set bookkeeping.
It is not required to understand \secref{concurrency}.
\end{note}

At minimum the list should support adding elements and checking
for membership. For example, we would expect:

\vccInput[linerange={member-endspec}]{c/07_list.c}

\noindent
Our current list invariant is strong enough only to show
\vcc{\result != 0 ==> l->val[k]}, because it only says
that if the list owns something, then it's in the \vcc{val}.
It also says that if something can be reached by following
the \vcc{next} field from the \vcc{head}, then it is owned.
What we want to additionally say is that if something is in the \vcc{val}
set, then it can be reached from the \vcc{head}.
Unfortunately, such property is not directly expressible in first-order
logic (which is the underlying logic of VCC specifications).
To workaround this problem we associate with each node
the set of values stored in all the following nodes and
the current node.
Additionally we say that the set for \vcc|NULL| node is empty.
This way, as we walk down the \vcc|next| pointers we can keep
track of all the elements that can be still reached.
Once we reach the \vcc|NULL| pointer, we know that nothing
more can be reached.
The set of reachable nodes are stored as maps from
\vcc|int| to \vcc|bool|.
We need one such map per each node, so we just
put a ghost map from \vcc|struct Node*| to the sets.
Alternatively, we could store these sets as a field inside of each node,
but maps gives more flexibility in updating it using lambda expressions.

\vccInput[linerange={type-init}]{c/07_list.c}

\noindent
All these changes in the invariant do not affect
the contract of \vcc{add()} function, and the only change in the body
is that we need to replace the update of \vcc{l->val} with the following:

\vccInput[linerange={specupdate-updateend}]{c/07_list.c}

\noindent
That is adding a node at the head only affect the followers set of the new head,
and the followers sets of all the other nodes remain unchanged.
Now let us have a look at the \vcc{member()} function:

\vccInput[linerange={member-out}]{c/07_list.c}

\noindent
The invariants of the \vcc{for} loop state that we only iterate over
nodes owned by the list, and that at each iteration \vcc{k} is in the
set of values represented by the list if and only if it is in the followers
set of the current node.
Both are trivially true for the head of the list, for the first iteration
of the loop.
For each next iteration, the invariant of the list tells us that by following
the \vcc|next| pointer we stay in the owns set.
It also tells us, that the \vcc|followers[n->next]| differs
from \vcc|followers[n]| only by \vcc|n->data|.
Thus, if \vcc|n->data| is not \vcc|val|, then the element,
if it's in \vcc|followers[n]| must be also in \vcc|followers[n->next]|.

\subsection{Sorting revisited}
\label{sect:sorting-perm}

In \secref{sorting} we have considered the bozo-sort algorithm. 
We have verified that the array after it returns is sorted.
But we would also like to know that it's a permutation of the input array.
To do that we will return a ghost map, which states the exact permutation
that the sorting algorithm produced.

\vccInput{c/04_bozosort_perm.c}

This sample introduces two new features.
First is the output ghost parameter \vcc{_(out perm_t perm)}.
We use it when we need a function to return something in addition to what it normally returns.
To call \vcc{bozo_sort()} you need to supply a local variable to hold
the permutation when the function exits, as in:
\begin{VCC}
void f(int *buf, unsigned len)
  // ...
{
  _(ghost perm_t myperm; )
  // ...
  bozo_sort(buf, len _(out myperm));
}
\end{VCC}
The value is only copied on exit of \vcc{bozo_sort()}, thought its execution 
it has its own copy.
It is thus different than passing a pointer to the local.
It is also more efficient for the verifier.

The other, somewhat more advanced, feature is explicit state manipulation.
The function \vcc{\now()} returns the current state of the heap (\ie dynamically
allocated memory; in future it will also work for locals, but for now it only applied
to memory location, address of which was taken). 
The state is encapsulated in a value of type \vcc{\state}.
The expression \vcc{\at(s, E)} returns value of expression
\vcc{E} as evaluated in state \vcc{s}. 
You can see  \vcc{\old(...)} as a special case of this.

Thus, the algorithm maintains the map containing the current permutation of
the data, with respect to the initial data (we store the initial state in
\vcc{s0}).
The initial permutation is just identity, and whenever we swap elements of
the array, we also swap elements of the permutation.

`

\section{Atomics}
\label{sect:concurrency}

Writing concurrent programs is generally considered to be harder than writing
sequential programs.
Similar opinions are held about verification.
Surprisingly, in VCC the leap from verifying sequential programs to
verifying fancy lock-free stuff is not that big.
This is because the verification in VCC is inherently based on invariants:
conditions that attached to data and need hold \emph{no matter which thread}
accesses it.

But let us move from words to actions, and verify a canonical example
of a lock-free algorithm, which is the implementation of a spin-lock itself.
The spin-lock data-structure is really simple -- it contains just a single
field, meant to be interpreted as a boolean stating whether the spin-lock
is currently acquired.
However, in VCC we would like to attach some formal meaning to this boolean.
We do that through ownership -- the spin-lock will protect some object,
and will own it whenever it is not acquired.
Thus, the following invariant should come as no surprise:

\vccInput[linerange={lock-init}]{c/08_lockw.c}

\noindent
We use a ghost field to hold a reference to the object meant to be protected
by this lock.
If one wishes to protect multiple objects with a single lock, one can make
the object referenced by \vcc{protected_obj} own them all.
The \vcc{locked} filed is annotated with \vcc{volatile}.
It has the usual meaning for the regular C compiler (\ie it makes the compiler
assume that the environment might write to that field, outside the knowledge
of the compiler).
For VCC it means that the field can be written also when the object is
consistent.
The idea is that we will not unwrap the object, but write it atomically,
while preserving its invariant.
The attribute
\vcc{_(volatile_owns)} means that we want the \vcc{\owns} set
to be treated as a volatile field (\ie we want to be able to write
it while to object is consistent; normally this is not possible).

Now we can see how one operates on volatile fields.
We shall start with the function releasing the lock, as it is simpler.

\vccInput[linerange={release-out}]{c/08_lockw.c}

\noindent
First, let's have a look at the contract.
\vcc{Release()} requires the lock to be wrapped.
You might wonder how multiple threads can all own the lock (to have it
wrapped), we will fix that later.
Second, we require that the protected object is wrapped.
We need it to be consistent because we will want to make the lock own it, and
lock can only own consistent objects.
We need the current thread to own it, because ownership transfer can
only happen between the current thread and an object.
Third, we say we're going to write the protected object.
This allows for the transfer, and prevents the calling function from assuming
that the object stays wrapped after the call.
Note that this contract is much like the contract of the function
adding an object to a container data-structure (see TODO).

The \vcc{atomic} block is similar in spirit to the \vcc{unwrapping} block ---
it allows for modifications of listed objects and checks if their invariants
are preserved.
The difference is that the entire update happens instantaneously from the point
of view of other threads.
We needed the unwrapping operation because we wanted to mark that we temporarily
break the object invariants.
Here, there is no point in time, where other threads can observe that the invariants
are broken.
Invariants hold before the beginning of the atomic block (by our principal reasoning
rule, \secref{TODO}), and we check the invariant at the end of the atomic block.

The question arises, what guarantees that other threads won't interfere with the atomic
action?
VCC allows only one physical memory operation inside of an atomic block,
which is indeed atomic from the point of view of the hardware.
Here, that operation is writing to the \vcc{l->locked}.
Other possibilities include reading from a volatile field, or a performing
a primitive operation supported by the hardware, like interlocked
compare-and-exchange.
However, inside our atomic block we can also see the update of the owns set
(remember that \vcc{\union_with(\owns(A), B)} is a shorthand for \vcc{\set(\owns(A), \union(\owns(A), B))}).
\todo{need better explanation}
This is fine, because the ghost code is not executed by the actual hardware.
If we were to imagine a VCC machine, which would actually execute the 
ghost code, we could say that it blocks other threads when the ghost
code is executing.
Because the real machine doesn't execute any ghost code, it doesn't need to do
any such blocking.

It is not particularly difficult to see that this atomic operation preserves the
invariant of the lock.
But this isn't the only condition imposed by VCC here.
To transfer ownership of \vcc{l->protected_obj} to the lock, we also need
write permission to the object being transferred, and
we need to know its consistent.
For example, should we forget to mention \vcc{l->protected_obj}
in the writes clause VCC will complain about:

\vccInput[linerange={out-999}]{c/08_lockw_wrong.c}

\noindent
Should we forget to perform the ownership transfer inside of \vcc{Release()}, we'll get complain
about the invariant of lock.

\vccInput[linerange={out-999}]{c/08_lockw_wrong2.c}

Let's then move to \vcc{Acquire()}. 
The specification is not very surprising: it requires the lock to be wrapped,
and ensures that after the call the thread will own the protected object,
and moreover, that the thread didn't own it before.
\todo{is-fresh should be explained earlier}

\vccInput[linerange={acquire-release}]{c/08_lockw.c}

\noindent
The \vcc{InterlockedCompareAndExchange()} function is a compiler built-in,
which on the x86/x64 hardware translates to \vcc{cmpxchg} assembly instruction.
It takes a memory location and two values.
If the memory location contains the first value, then it is replaced with the second.
It returns the old value.
The entire operation is performed atomically (and is also a write barrier).

\subsection{Atomic inline}
\label{sect:atomic-inline}

VCC doesn't have all the primitives of all the C compilers predefined.
One can define them by suppling a body.
It is presented only to the VCC compiler (it is enclosed in
\vcc{_(atomic_inline ...)}) so that the normal compiler doesn't get confused
about it.

\vccInput[linerange={xchg-acquire}]{c/08_lockw.c}

\noindent
This is one of the places where one needs to be very careful,
as there is no way for VCC to know if the definition you provided matches
the semantics of your regular C compiler.
Make sure to check with the regular C compiler manual for exact semantics
of its built-in functions.

The header files coming with VCC provide a handful of popular operations,
you can just rename them to fit your compiler.

\subsection{Using claims}
\label{sect:using-claims}

The contracts of functions operating on the lock require that the lock
is wrapped.
This is because one can only perform atomic operations on objects
that are consistent. 
If object is inconsistent, then the owning thread is in full control of it.
However, wrapped means not only consistent, but also owned by the current thread,
which defeats the purpose of the lock --- it should be possible
for multiple threads to compete for the lock.
Let's then say, there is a thread which owns the lock.
Assume some other thread got to know that the lock is consistent.
How would it know that the owning thread won't unwrap (or worse yet, deallocate) the lock, just
before a thread tries an atomic operation on the lock?
The owning thread thus needs to somehow promise the other thread
that lock will stay consistent.
In VCC such a promise takes form of a \Def{claim}.
Later we'll see that claims are more powerful, but for
now consider the following to be the definition of a claim:

\begin{VCC}
_(ghost 
typedef struct {
  \ptrset claimed;
  _(invariant \forall \object o; o \in claimed ==> \consistent(o))
} \claim_struct, *\claim;
)
\end{VCC}

\noindent
Thus, a claim is an object, with an invariant stating that a number of other objects
(we call them \Def{claimed objects}) are consistent.
As this is stated in the invariant of the claim, it only needs to be true
as long as the claim itself stays consistent.

One will ask, how is this admissible
\todo{probably insert some back-reference to admissibility;
people will have forgotten it by here}
--- the claim doesn't own the claimed
objects, so how should it know it will stay consistent?
This is ensured by an implicit invariant, defined on all types
marked with \vcc{_(claimable)} attribute.
This invariant states that the object cannot be unwrapped when
when there are consistent claims on it.
More precisely, each claimable object keeps track of the count of outstanding
claims.
The number of outstanding claim on object \vcc{o} is given by
the \vcc{\claim_count(o)} function.

Now we getting back to our lock example, the trick is that there can be
multiple claims claiming the lock (note that this is orthogonal to
the fact that a single claim can claim multiple objects).
The thread that owns the lock will need to keep track of who's using
the lock.
The owner won't be able to destroy the lock (which requires unwrapping it),
before it makes sure there is no one using the lock.
Thus, we need to add \vcc{_(claimable)} attribute to our lock
definition, and change the contract on the functions operating
on the lock. As the changes are vary similar we'll only
show \vcc{Release()}.
\todo{ghost parameters should be explained earlier}

\vccInput[linerange={release-struct_data}]{c/08_lock_claimsobj.c}

\noindent
We pass a ghost parameter holding a claim.
The claim should be wrapped.
The function \vcc{\claims_obj(c, l)} is defined to be
\vcc{l \in c->claimed}, \ie that the claim claims the lock.
We also need to know that the claim is not the protected object,
otherwise we couldn't ensure that the claim is wrapped after the call.
This is the kind of weird corner case that VCC is very good catching
(even if it's bogus in this context).
Other than the contract, the only change is that we list the claim
as parameter to the atomic block.
\todo{our current syntax is horrible, we need something different}
Listing a normal object as parameter to the atomic makes VCC know you're
going to modify the object.
For claims, it is just a hint, that it should use this claim when trying
to prove that the object is consistent.

\subsection{Creating claims}
\label{sect:creating-claims}

When creating (or destroying) a claim one needs to list the claimed objects.
Let's have a look at an example.

\vccInput[linerange={create_claim-out}]{c/08_lock_claimsobj.c}

We initialize the lock, which leaves the lock wrapped (we'll get to that in a minute),
we create a claim on the lock, acquire it, release, destroy the claim, and unwrap the lock
(which allows it to be destroyed when the function activation record is popped off the stack).
The \vcc{\make_claim(...)} function takes the set of objects to be claimed
and a property (an invariant of the claim, we'll get to that in the next section).
Let us give desugaring of \vcc{\make_claim(...)} for a single object
in terms of the \vcc{\claim_struct} defined in the previous section.

\begin{VCC}
// c = \make_claim({o}, true) expands to
\claim_count(o) = \claim_count(o) + 1;
c = malloc(sizeof(\struct_claim));
c->claimed = {o};
_(wrap c);

// \destroy_claim(c, {o}) expands to
assert(o \in c->claimed);
\claim_count(o) = \claim_count(o) - 1;
_(unwrap c);
free(c);
\end{VCC}


Because creating or destroying a claim on \vcc{c} assigns to
\vcc{\claim_count(c)}, it requires write access to that memory ``location''.
One way to obtain such access is getting sequential write access to \vcc{c} itself:
in our example the lock is created on the stack and thus sequentially writable.
We can thus create a claim and immediately use it.
A more realistic scenario is when a thread creates an object, constructs
a number of claims on it, and stores the claims in some shared, possibly global, data-structures
(\eg a work-queue) where other threads can access them.

The \vcc{true} in \vcc{\make_claim(...)} is the claimed property (an invariant
of the claim), which will be explained in the next section.

The destruction can possibly leak claim counts, \ie one could say:
\begin{VCC}
\destroy_claim(c, {});
\end{VCC}
\noindent
and it would verify just fine.
This avoids the need to have write access to \vcc{o}, but on the other hand prevents
\vcc{o} from unwrapping forever (which might be actually fine if \vcc{o} is a ghost object).
%It seems clear why the claimed objects need to be listed when creating a claim, but
%why do we need them for destruction?

\subsection{Two-state invariants}
\label{sect:inv2}

Sometimes it is not only important what are the valid states of objects,
but also what are the allowed \emph{changes} to objects.
For example, let's take a counter keeping track of certain operations
since the beginning of the program.

\vccInput[linerange={counter-reading}]{c/09_counter.c}

\noindent
Its first invariant is plain single-state invariant -- for some reason
we decided to exclude zero as the valid count.
The second invariant says that for any atomic update of (consistent)
counter, \vcc{v} can either stay unchanged or increment by exactly one.
The syntax \vcc{\old(v)} is used to refer to value of \vcc{v} before
an atomic update, and plain \vcc{v} is used for the value of \vcc{v}
after the update.
That is, when checking that an atomic update preserves the invariant
of a counter \vcc{n}, we will take the state of the program right
before the update, the state right after the update, and check
that the invariant holds for that pair of states.

\begin{note}
In fact, it would be easy to prevent any changes to some field \vcc{f}, by
saying \vcc{_(invariant \old(f) == f)}.
This is roughly what happens under the hood when a field is
declared without the \vcc{volatile} modifier.
\end{note}

As we can see there is no syntactic distinction between single-
and two-state invariants in VCC.
The single-state invariants are just two-state invariants, which do not use
\vcc{\old(...)}.
However, we often need an interpretation of an object invariant in a single state \vcc{s}.
For that we use the \Def{stuttering} transition from \vcc{s} to \vcc{s} itself.
VCC enforces that all invariants are \Def{reflexive} that is if they hold
over a transition \vcc{s0, s1}, then they should hold in just \vcc{s1}
(\ie over \vcc{s1, s1}).
In practice,
this means that \vcc{\old(...)} should be only used to describe
how objects change, and not what are their proper values.
In particular,
all invariants of the form \vcc{\old(E) == E || ...}, for any expression \vcc{E}, are reflexive.
On the other hand, the invariant \vcc{\old(f) < 7} is not reflexive.

Let's now discuss where can you actually rely on invariants being preserved.

\begin{VCC}
void foo(struct Counter *n)
  _(requires \wrapped(n))
{
  int x, y;
  atomic(n) { x = n->v; }
  atomic(n) { y = n->v; }
}
\end{VCC}

\noindent
The question is what do we know about \vcc{x} and \vcc{y}
at the end of \vcc{foo()}.
If we knew that nobody is updating \vcc{n->v} while \vcc{foo()}
is running we would know \vcc{x == y}.
This would be the case if \vcc{n} was mutable not wrapped.
In our case, because \vcc{n} is consistent, other threads can update it,
while \vcc{foo()} is running, but they will need to
adhere to \vcc{n}'s invariant.
So we might guess that at end of \vcc{foo()} we know
\vcc{y == x || y == x + 1}.
But this is incorrect: \vcc{n->v} might get incremented
by more than one, in several steps.
The correct answer is thus \vcc{x <= y}.
Unfortunately, in general, such properties are very difficult to deduce
automatically, which is why we use common object invariants and admissibility
check to express such properties in VCC.

\begin{note}
An invariant is \emph{transitive} if it holds over states \vcc{s0, s2},
provided that it holds over \vcc{s0, s1} and \vcc{s1, s2}.
Transitive invariants could be assumed over arbitrary
pairs of states, provided that the object stays consistent
in between them. 
VCC does not require invariants to be transitive though.

Some invariants are naturally transitive (\eg we could say
\vcc{_(invariant old(x) <= x)} in \vcc{struct Counter},
and it would be almost as good our current invariant).
Some other invariants, especially the more complicated ones,
are more difficult to make transitive.
For example, an invariant on a reader-writer lock might say
\begin{VCC}
_(invariant writer_waiting ==> old(readers) >= readers)
\end{VCC}
\noindent
To make it transitive one needs to introduce version numbers.
Some invariants describing hardware (\eg a step of physical CPU)
are impossible to make transitive.
\end{note}

Consider the following structure definition:

\vccInput[linerange={reading-endreading}]{c/09_counter.c}

\noindent 
It is meant to represent a reading from a counter.
Let's consider its admissibility.
It has a pointer to the counter, and a owns a claim, which
claims the counter.
So far, so good.
It also states that the current value of the counter is no less than \vcc{r}.
Clearly, the reading doesn't own the counter, so our previous rule
that you can mention in your invariant everything that you own doesn't apply.
It would be tempting to extend that rule to say ``everything that you own
or have a claim on'', but VCC actually uses a more general rule.
In a nutshell, the rule says that every invariant should be preserved
under changes to other objects, provided that these other objects change
according to their invariants.
When we look at our \vcc{struct Reading}, its invariant cannot be broken when
its counter increments, which is the only change allowed by counters invariant.
On the other hand, an invariant like \vcc{r == n->v} or \vcc{r >= n->v}
could be broken by such a change.
But let us proceed with somewhat more precise definitions.

First, assume that every object invariant holds when the object is inconsistent.
This might sound counter-intuitive, but remember that consistency is controlled
by a field.
When that field is set to false, we want to \emph{effectively} disable the invariant,
which is the same as just forcing it to be true in that case.
Alternatively, you might try to think of all objects as being consistent for a while.

An atomic action, which updates state \vcc{s0} into \vcc{s1}, is \Def{legal} if and only if the invariants of
objects that have changed between \vcc{s0} and \vcc{s1} hold over \vcc{s0, s1}.
In other words, a legal action preservers invariants of updated objects.
This should not come as a surprise: this is exactly what VCC checks
for in atomic blocks.

An invariant is \Def{stable} if and only if it cannot be broken by legal updates.
More precisely, to prove that an invariant of \vcc{o} is stable,
one needs to ``simulate'' an arbitrary legal update:
\begin{itemize}
\item Take two arbitrary states \vcc{s0} and \vcc{s1}.
\item Assume that all invariants (including \vcc{o}'s) hold over \vcc{s0, s0}.
\item Assume that all fields of \vcc{o} are the same in \vcc{s0} and \vcc{s1}.
\item Assume that for all objects, fields of which are not the same in \vcc{s0} and \vcc{s1},
their invariants hold over \vcc{s0, s1}.
\item Check that invariant of \vcc{o} holds over \vcc{s0, s1}.
\end{itemize}
The first assumption comes from the fact that all invariants are reflexive.

An invariant is \Def{admissible} if and only if its stable and reflexive.

First, let's see how our previous notion of admissibility relates to this one.
If \vcc{o} owns \vcc{p}, then \vcc{p \in \owns(o)}.
By the second admissibility assumption, because \vcc{\owns(o)} resides
in a field of \vcc{o}, after the simulated action \vcc{o} still owns \vcc{p}.
Because of that, we know that \vcc{p} is consistent in both \vcc{s0} and \vcc{s1}.
Therefore its non-volatile fields do not change between \vcc{s0} and \vcc{s1}.
Additionally, if \vcc{p} owned \vcc{q} before the atomic action, and the owns set of \vcc{p} is non-volatile,
it will keep owning \vcc{q}, and thus non-volatile fields of \vcc{q}
will stay unchanged.
Thus our previous notion of admissibility is no stronger than this one.

Getting back to our \vcc{foo()} example, to deduce that \vcc{x <= y}, after
the first read one could create a ghost \vcc{Reading} object, and
use its invariant in the second action.
While we need to say that \vcc{x <= y} is what's required,
using a full-fledged object might seem like an overkill.
Luckily, definitions of claims themselves can specify additional invariants.

\subsection{Guaranteed properties in claims}
\label{sect:claim-props}

When constructing a claim, one can specify additional invariants to put on
the virtual definition of the claim structure.
Let's have a look at annotated version of our previous \vcc{foo()} function.

\vccInput[linerange={readtwice-endreadtwice}]{c/09_counter.c}

\noindent
First, a high-level description of what's going on.
Just after reading \vcc{n->v} we create a claim \vcc{r}, which guarantees
that in every state, where \vcc{r} is consistent,
the current value of \vcc{n->v} is no less than the value of \vcc{x}
at the time when \vcc{r} was created.
Then, after reading \vcc{n->v} for the second time, we tell VCC to
make use of \vcc{r}'s guaranteed property, by asserting that it is ``active''.
This makes VCC know \vcc{x <= n->v} in the current state, where also
\vcc{y == n->v}.
From these two facts VCC can conclude that \vcc{x <= y}.

The general syntax for constructing a claim is:

\begin{VCC}
_(ghost c = \make_claim(S, P))
\end{VCC}

\noindent
We already explained, that this requires that \vcc{\claim_count()}s of objects in the set \vcc{S} are writable.
As for the property \vcc{P}, we pretend it forms the invariant of the claim.
Because we're just constructing the claim, just like during regular object initialization,
the invariant has to hold initially (\ie at the moment when the claim is created,
that is wrapped).
Moreover, the invariant has to be admissible, under the condition
that all objects in \vcc{S} stay consistent as long as the claim itself
stays consistent.

But what about locals?
Normally, object invariants are not allowed to reference locals.
The idea is that when the claim is constructed, all the locals that the
claim references are copied into fields the claim.
The fields of the claim never change, once it is created.
Therefore an assignment \vcc{x = MAX_UINT;} in between the atomic
blocks would not invalidate the claim --- the claim would still
refer to the old value of \vcc{x}.
Of course, it would invalidate the final \vcc{x <= y} assert.

\begin{note}
For any expression \vcc{E} one can use \vcc{\at(\now, E)} in \vcc{P}
in order to have the value of \vcc{E} be evaluated in the state
when the claim is created, and stored in the field of the claim.
\end{note}

This copying business doesn't affect initial checking of the \vcc{P},
\vcc{P} should just hold at the point when the claim is created.
It does however affect the admissibility check for \vcc{P}:
\begin{itemize}
\item Consider an arbitrary legal action, from \vcc{s0} to \vcc{s1}.
\item Assume that all invariants hold over \vcc{s0, s0}, including assuming \vcc{P} in \vcc{s0}.
\item Assume that fields of \vcc{c} didn't change between \vcc{s0} and \vcc{s1}
(in particular locals referenced by the claim are the same as at the moment of its creation).
\item Assume all objects in \vcc{S} are consistent in both \vcc{s0} and \vcc{s1}.
\item Assume that for all objects, fields of which are not the same in \vcc{s0} and \vcc{s1},
their invariants hold over \vcc{s0, s1}.
\item Check that \vcc{P} holds in \vcc{s1}.
\end{itemize}

To prove \vcc{\active_claim(c)} one needs to prove \vcc{\consistent(c)} and that
the current state is a \Def{full-stop} state, \ie state where all invariants
are guaranteed to hold.
Any execution state outside of an atomic block is full-stop.
The state right at the beginning of an atomic block is also full-stop.
The states in the middle of it (\ie after some state updates) might not be
(this isn't a soundness problem because they are not observable by other threads).

The fact that \vcc{P} follows from \vcc{c}'s invariant after the construction
is expressed using \vcc{\claims(c, P)}.
\todo{explain \vcc@at()@ syntax before}
It is roughly equivalent to saying:
\begin{VCC}
\forall \state s {\at(s, \active_claim(c))};
  \at(s, \active_claim(c)) ==> \at(s, P)
\end{VCC}
Thus, after asserting \vcc{\active_claim(c)} in some state \vcc{s},
\vcc{\at(s, P)} will be assumed, which means VCC will
assume \vcc{P}, where all heap references are replaced by their values in
\vcc{s}, and all locals are replaced by the values at the point
when the claim was created.

\itodo{I think we need more examples about that at() business,
claim admissibility checks and so forth}

\subsection{Dynamic claim management}
\label{sect:dynamic-claims}

So far we have only considered the case of creating claims to wrapped objects.
In real systems some resources are managed dynamically:
threads ask for ``handles'' to resources, operate on them,
and give the handles back.
These handles are usually purely virtual --- asking for a handle amounts to incrementing
some counter.
Only after all handles are given back the resource can be disposed.
This is pretty much how claims work in VCC, and indeed they were modeled after this
real-world scenario. 
Below we have an example of prototypical reference counter.

\vccInput[linerange={refcnt-init}]{c/10_rundown.c}

\noindent
Thus, a \vcc{struct RefCnt} owns a resource, and makes sure that the number of outstanding
claims on the resource matches the physical counter stored in it.
\vcc{\claimable(p)} means that the type of object pointed to by \vcc{p} was marked
with \vcc{_(claimable)}.
The lowest bit is used to disable giving out of new references
(this is expressed in the last invariant).

\vccInput[linerange={init-incr}]{c/10_rundown.c}

\noindent
Initialization shouldn't be very surprising:
\vcc{\wrapped0(o)} means \vcc{\wrapped(o) && \claim_count(o) == 0},
and thus on initialization we require a resource without any outstanding
claims.
\todo{out parameters should be explained before}

\vccInput[linerange={incr-decr}]{c/10_rundown.c}

\noindent
First, let's have a look at the function contract.
The syntax \vcc{_(always c, P)} is equivalent to:
\begin{VCC}
  _(requires \wrapped(c) && \claims(c, P))
  _(ensures \wrapped(c))
\end{VCC}
Thus, instead of requiring \vcc{\claims_obj(c, r)}, we require that the claim
guarantees \vcc{\consistent(r)}.
One way of doing this is claiming \vcc{r}, but another is claiming the owner
of \vcc{r}, as we will see shortly.

As for the body, we assume our reference counter will never overflow.
This clearly depends on the running time of the system and usage patterns,
but in general it would be difficult to specify this, and thus we just
hand-wave it.

The new thing about the body is that we make a claim on the resource,
even though it's not wrapped.
There are two ways of obtaining write access to \vcc{\claim_count(p)}:
either having \vcc{p} writable sequentially and wrapped,
or in case \vcc{\owner(p)} is a non-thread object, checking
invariant of \vcc{\owner(p)}.
Thus, inside an atomic update on \vcc{\owner(p)} (which will check the invariant of \vcc{\owner(p)}) one can create
claims on \vcc{p}.
The same rule applies to claim destruction:

\vccInput[linerange={decr-use}]{c/10_rundown.c}

\noindent
A little tricky thing here, is that we need to make use of the \vcc{handle} claim
right after reading \vcc{r->cnt}. 
Because this claim is valid, we know that the claim count on the resource
is positive and therefore (by reference counter invariant) \vcc{v >= 2}.
Without using the \vcc{handle} claim to deduce it we would get a complaint
about overflow in \vcc{v - 2} in the second atomic block.

Finally, let's have a look at a possible use scenario of our reference counter.

\vccInput[linerange={use-enduse}]{c/10_rundown.c}

\noindent
The \vcc{struct B} contains a \vcc{struct A} governed by a reference counter.
It owns the reference counter, but not \vcc{struct A} (which is owned by the reference
counter).
A claim guaranteeing consistency of \vcc{struct B} also guarantees
consistency of its counter, so we can pass it to \vcc{try_incr()},
which gives us a handle on \vcc{struct A}.

Of course a question arises where one does get a claim on \vcc{struct B} from?
In real systems the top-level claims come either from global objects that are
always consistent, or from data passed when the thread is created.



\appendix
\section{Memory model}
\label{sect:memmodel}

In most situations in C the type of a pointer is statically known:
while at the machine code level the pointer is passed around as a type-less
word, at the C level, in places where it is used, we know its type.
VCC memory model makes this explicit: pointers are understood as pairs
of their type and address (an word or integer representing location in memory
understood as an array of bytes).
For any state of program execution, VCC maintains the set of \Def{proper pointers}.
\todo{we might want a better name}
Only proper pointers can be accessed (read or written).
There are rules on changing the proper pointer set --- \eg one can remove
a pointer \vcc{(T*)a}, and add pointers \vcc{(char*)a}, \vcc{(char*)(a+1)},
..., \vcc{(char*)(a+sizeof(T))}, or \textit{vice versa}.
These rules make sure that at any given time, representations of two
unrelated proper pointers do not overlap, which greatly simplifies reasoning.
Note that given a \vcc{struct SafeString *p}, when \vcc{\proper(p)}
we will also expect \vcc{\proper(&p->len)}.
That is, when a structure is proper, and thus safe to access, so should
be all its fields.
This is what ``unrelated'' means in the sentence above:
the representations overlap if and only if they pointer refer to a struct
and fields of that struct.
It is OK that fields overlap with their containing struct, but that
structs overlap each other.

\subsection{Reinterpretation}
\label{sect:reint}

\section{Bitvector reasoning}
\label{sect:bv}

Finally, 
to show something little less trivial, the implementation of \vcc{min()} can surprisingly be more involved.
For example the one below does not use a branch.

\vccInput[]{c/01_min5.c}

\noindent
The syntax:
\begin{VCC}
_( assert {bv} \forall int x; (x & -1) == x )
\end{VCC}

\noindent
VCC to prove the assertion using the fixed-length bit vector theory, a.k.a. machine integers.
This is then used as a lemma to prove the postcondition.





\bibliography{other}


\end{document}

% vim: spell
