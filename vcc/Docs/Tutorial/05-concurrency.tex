\section{Atomics}

Writing concurrent programs is generally considered to be harder than writing
sequential programs.
Similar opinions are held about verification.
Surprisingly, in VCC the leap from verifying sequential programs to
verifying fancy lock-free stuff is not that big.
This is because the verification in VCC is inherently based on invariants:
conditions that attached to data and need hold \emph{no matter which thread}
accesses it.

But let us move from words to actions, and verify a canonical example
of a lock-free algorithm, which is the implementation of a spin-lock itself.
The spin-lock data-structure is really simple -- it contains just a single
field, meant to be interpreted as a boolean stating whether the spin-lock
is currently acquired.
However, in VCC we would like to attach some formal meaning to this boolean.
We do that through ownership -- the spin-lock will protect some object,
and will own it whenever it is not acquired.
Thus, the following invariant should come as no surprise:

\vccInput[linerange={lock-init}]{c/08_lockw.c}

\noindent
We use a ghost field to hold a reference to the object meant to be protected
by this lock.
If one wishes to protect multiple objects with a single lock, one can make
the object referenced by \vcc{protected_obj} own them all.
The \vcc{locked} filed is annotated with \vcc{volatile}.
It has the usual meaning for the regular C compiler (\ie it makes the compiler
assume that the environment might write to that field, while the compiler
is not looking).
For VCC it means that the field can be written also when the object is
consistent.
The idea is that we will not unwrap the object, but write it atomically,
while preserving its invariant.
The attribute
\vcc{_(volatile_owns)} means that we want the \vcc{\owns} set
to be treated as a volatile field (\ie we want to be able to write
it while to object is consistent; normally this is not possible).

Now we can see how one operates on volatile fields.
We shall start with the function releasing the lock, as it is simpler.

\vccInput[linerange={release-out}]{c/08_lockw.c}

\noindent
First, let's have a look at the contract.
\vcc{Release()} requires the lock to be wrapped.
You might wonder how multiple threads can all own the lock (to have it
wrapped), we will fix that later.
Second, we require that the protected object is wrapped.
We need it to be consistent because we will want to make the lock own it, and
lock can only own consistent objects.
We need the current thread to own it, because ownership transfer can
only happen between the current thread and an object.
Third, we say we're going to write the protected object.
This allows for the transfer, and prevents the calling function from assuming
that the object stays wrapped after the call.
Note that this contract is much like the contract of the function
adding an object to a container data-structure (see TODO).

The \vcc{atomic} block is similar in spirit to the \vcc{unwrapping} block ---
it allows for modifications of listed objects and checks if their invariants
are preserved.
The difference is that the entire update happens instantaneously from the point
of view of other threads.
We needed the unwrapping operation because we wanted to mark that we temporarily
break the object invariants.
Here, there is no point in time, where other threads can observe that the invariants
are broken.
Invariants hold before the beginning of the atomic block (by our principal reasoning
rule, \secref{TODO}), and we check the invariant at the end of the atomic block.

The question arises, what guarantees that other threads won't interfere with the atomic
action?
VCC allows only one physical memory operation inside of an atomic block,
which is indeed atomic from the point of view of the hardware.
Here, that operation is writing to the \vcc{l->locked}.
Other possibilities include reading from a volatile field, or a performing
a primitive operation supported by the hardware, like interlocked
compare-and-exchange.
However, inside our atomic block we can also see the update of the owns set
(remember that \vcc{\union_with(\owns(A), B)} is a shorthand for \vcc{\set(\owns(A), \union(\owns(A), B))}).
\todo{need better explanation}
This is fine, because the ghost code is not executed by the actual hardware.
If we were to imagine a VCC machine, which would actually execute the 
ghost code, we could say that it blocks other threads when the ghost
code is executing.
Because the real machine doesn't execute any ghost code, it doesn't need to do
any such blocking.

It is not particularly difficult to see that this atomic operation preserves the
invariant of the lock.
But this isn't the only condition imposed by VCC here.
To transfer ownership of \vcc{l->protected_obj} to the lock, we also need
write permission to the object being transferred, and
we need to know its consistent.
For example, should we forget to mention \vcc{l->protected_obj}
in the writes clause VCC will complain about:

\vccInput[linerange={out-999}]{c/08_lockw_wrong.c}

\noindent
Should we forget to perform the ownership transfer inside of \vcc{Release()}, we'll get complain
about the invariant of lock.

\vccInput[linerange={out-999}]{c/08_lockw_wrong2.c}

Let's then move to \vcc{Acquire()}. 
The specification is not very surprising: it requires the lock to be wrapped,
and ensures that after the call the thread will own the protected object,
and moreover, that the thread didn't own it before.
\todo{is-fresh should be explained earlier}

\vccInput[linerange={acquire-release}]{c/08_lockw.c}

\noindent
The \vcc{InterlockedCompareAndExchange()} function is a compiler built-in,
which on the x86/x64 hardware translates to \vcc{cmpxchg} assembly instruction.
It takes a memory location and two values.
If the memory location contains the first value, then it is replaced with the second.
It returns the old value.
The entire operation is performed atomically (and is also a write barrier).

\subsection{Atomic inline}

VCC doesn't have all the primitives of all the C compilers predefined.
One can define them by suppling a body.
It is presented only to the VCC compiler (it is enclosed in
\vcc{_(atomic_inline ...)}) so that the normal compiler doesn't get confused
about it.

\vccInput[linerange={xchg-acquire}]{c/08_lockw.c}

\noindent
This is one of the places where one needs to be very careful,
as there is no way for VCC to know if the definition you provided matches
the semantics of your regular C compiler.
Make sure to check with the regular C compiler manual for exact semantics
of its built-in functions.

The header files coming with VCC provide a handful of popular operations,
you can just rename them to fit your compiler.

\subsection{Using claims}

The contracts of functions operating on the lock require that the lock
is wrapped.
This is because one can only perform atomic operations on objects
that are consistent. 
If object is inconsistent, then the owning thread is in full control of it.
However, wrapped means not only consistent, but also owned by the current thread,
which defeats the purpose of the lock --- it should be possible
for multiple threads to compete for the lock.
Let's then say, there is a thread which owns the lock.
Assume some other thread got to know that the lock is consistent.
How would it know that the owning thread won't unwrap (or worse yet, deallocate) the lock, just
before the thread tries an atomic operation on the lock?
The owning thread thus needs to somehow promise the other thread
that lock will stay consistent.
In VCC such a promise takes form of a \Def{claim}.
Later we'll see that claims are more powerful, but for
now consider the following to be the definition of a claim:

\begin{VCC}
_(ghost 
typedef struct {
  \object claimed;
  _(invariant \consistent(claimed))
} \claim_struct, *\claim;
)
\end{VCC}

\noindent
Thus, a claim is an object, with an invariant stating that some other object
(we call it \Def{claimed object}) is consistent.
As this is stated in the invariant of the claim, it only needs to be true
as long as the claim itself stays consistent.

One will ask, how is this admissible --- the claim doesn't own the claimed
object, so how should it know it will stay consistent?
This is ensured by an implicit invariant, defined on all types
marked with \vcc{_(claimable)} attribute.
This invariant states that the object cannot be unwrapped when
when there are consistent claims on it.
More precisely, each such object keeps track of the count of outstanding
claims.
The number of outstanding claim on object \vcc{o} is given by
the \vcc{\claim_count(o)} function.

Now we getting back to our lock example, the trick is that there can be
multiple claims claiming the lock.
The thread that owns the lock will need to keep track of who's using
the lock.
It won't be able to destroy the lock (which requires unwrapping it),
before it makes sure there is no one using the lock.
Thus, we need to add \vcc{_(claimable)} attribute to our lock
definition, and change the contract on the functions operating
on the lock. As the changes are vary similar we'll only
show \vcc{Release()}.
\todo{ghost parameters should be explained earlier}

\vccInput[linerange={release-out}]{c/08_lock_claimsobj.c}

\noindent
We pass a ghost parameter holding a claim.
The claim should be wrapped, and \vcc{c->claimed}
should point to \vcc{l} --- we use \vcc{\claims_obj(c, l)}
to express that (which is because in general claims can claim multiple objects).
We also need to know that the claim is not the protected object,
otherwise we couldn't ensure that the claim is wrapped after the call.
This is the kind of weird corner case that VCC is very good catching
(even if it's bogus in this context).
Other than the contract, the only change is that we list the claim
as parameter to the atomic block.
\todo{our current syntax is horrible, we need something different}
Listing a normal object as parameter to the atomic makes VCC know you're
going to modify the object.
For claims, it is just a hint, that it should use this claim when trying
to prove that the object is consistent.

\subsection{Creating claims}

\itodo{}

\subsection{Guaranteed properties}

\itodo{explain claims about properties -- possibly use counter as an example,
also for two-state invariants}

\subsection{Dynamic claim management}

\itodo{a rundown as a bigger example}

