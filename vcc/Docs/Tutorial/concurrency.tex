\section{Atomics}
\label{sect:concurrency}

Writing concurrent programs is generally considered to be harder than writing
sequential programs.
Similar opinions are held about verification.
Surprisingly, in VCC the leap from verifying sequential programs to
verifying fancy lock-free code is not that big.
This is because the verification in VCC is inherently based on invariants:
conditions that are attached to data and need to hold \emph{no matter which thread}
accesses it.

But let us move from words to actions, and verify a canonical example
of a lock-free algorithm, which is the implementation of a spin-lock itself.
The spin-lock data-structure is really simple -- it contains just a single
boolean field, meant to indicate whether the spin-lock
is currently acquired.
However, in VCC we would like to attach some formal meaning to this boolean.
We do that through ownership -- the spin-lock will protect some object,
and will own it whenever it is not acquired.
Thus, the following invariant should come as no surprise:

\vccInput[linerange={lock-init}]{c/08_lockw.c}

\noindent
We use a ghost field to hold a reference to the object meant to be protected
by this lock.
If you wish to protect multiple objects with a single lock, you can make
the object referenced by \vcc{protected_obj} own them all.
The \vcc{locked} field is annotated with \vcc{volatile}.
It has the usual meaning for the regular C compiler (\ie it makes the compiler
assume that the environment might write to that field, outside the knowledge
of the compiler).
For VCC it means that the field can be written also when the object is
closed (that is after wrapping it).
The idea is that we will not unwrap the object, but write it atomically,
while preserving its invariant.
The attribute
\vcc{_(volatile_owns)} means that we want the \vcc{\owns} set
to be treated as a volatile field (\ie we want to be able to write
it while the object is closed; normally this is not possible).

First, let's have a look at lock initialization:

\vccInput[linerange={init-xchg}]{c/08_lockw.c}

\noindent
One new thing there is the use of \Def{ghost parameter}.
The regular lock initialization function prototype does not say which
object the lock is supposed to protect, but our lock invariant requires it.
Thus, we introduce additional parameter for the purpose of verification.
A call to the initialization will look like \vcc{InitializeLock(&l _(ghost o))}.

Second, we require that the object to be protected is wrapped (recall that wrapped means closed
and owned by the current thread).
We need it to be closed because we will want to make the lock own it, and
lock can only own closed objects.
We need the current thread to own it, because ownership transfer can
only happen between the current thread and an object,
and not for example some other thread and an object.
Third, we say we're going to write the protected object.
This allows for the transfer, and prevents the calling function from assuming
that the object stays wrapped after the call.
Note that this contract is much like the contract of the function
adding an object to a container data-structure, like
\vcc{sc_add()} from \secref{dynamic-ownership}.

Now we can see how we operate on volatile fields.
We shall start with the function releasing the lock, as it is simpler,
than the one acquiring it.

\vccInput[linerange={release-out}]{c/08_lockw.c}

\noindent
First, let's have a look at the contract.
\vcc{Release()} requires the lock to be wrapped.%
\footnote{ You might wonder how multiple threads can all own the lock (to have it
wrapped), we will fix that later. }
The preconditions on the protected object are very similar to the
preconditions on the \vcc{InitializeLock()}.
Note that the \vcc{Release()} does not mention the lock in its writes clause,
this is because the write it performs is volatile.
Intuitively, VCC needs to assume such writes can happen at any time, so one additional
write from this function doesn't make a difference.

The \vcc{atomic} block is similar in spirit to the \vcc{unwrapping} block ---
it allows for modifications of listed objects and checks if their invariants
are preserved.
The difference is that the entire update happens instantaneously from the point
of view of other threads.
We needed the unwrapping operation because we wanted to mark that we temporarily
break the object invariants.
Here, there is no point in time where other threads can observe that the invariants
are broken.
Invariants hold before the beginning of the atomic block (by our principal reasoning
rule, \secref{wrap-unwrap}), and we check the invariant at the end of the atomic block.

The question arises, what guarantees that other threads won't interfere with the atomic
action?
VCC allows only one physical memory operation inside of an atomic block,
which is indeed atomic from the point of view of the hardware.
Here, that operation is writing to the \vcc{l->locked}.
Other possibilities include reading from a volatile field, or a performing
a primitive operation supported by the hardware, like interlocked
compare-and-exchange.
However, inside our atomic block we can also see the update of the owns set.
This is fine, because the ghost code is not executed by the actual hardware.

% I'm not sure if we need this...
\begin{note}
The reason we can use ghost code is a simulation relation between two machines.
Machine A executes the program with ghost code, and machine B executes the program
without ghost code.
Because ghost code cannot write physical data or influence the control flow of physical code
in any way, the contents of physical memory of machines A and B is the same.
Therefore any property we prove about physical memory of A also holds for B.
Now, if we imagine that both machines are multi-threaded, and the machine A blocks
other threads when it's executing ghost code, the same simulation property will still hold.
\end{note}

It is not particularly difficult to see that this atomic operation preserves the
invariant of the lock.
But this isn't the only condition imposed by VCC here.
To transfer ownership of \vcc{l->protected_obj} to the lock, we also need
write permission to the object being transferred, and
we need to know it is closed.
For example, should we forget to mention \vcc{l->protected_obj}
in the writes clause VCC will complain about:

\vccInput[linerange={out-999}]{c/08_lockw_wrong.c}

\noindent
As another example, should we forget to perform the ownership transfer inside of \vcc{Release()}, VCC will complain
about the invariant of the lock:

\vccInput[linerange={out-999}]{c/08_lockw_wrong2.c}

Let's then move to \vcc{Acquire()}. 
The specification is not very surprising: it requires the lock to be wrapped,
and ensures that after the call the thread will own the protected object,
and moreover, that the thread didn't directly own it before.
This is much like the postcondition on \vcc{sc_add()} function
from \secref{dynamic-ownership}.

\vccInput[linerange={acquire-release}]{c/08_lockw.c}

\noindent
The \vcc{InterlockedCompareAndExchange()} function is a compiler built-in,
which on the x86/x64 hardware translates to the \vcc{cmpxchg} assembly instruction.
It takes a memory location and two values.
If the memory location contains the first value, then it is replaced with the second.
It returns the old value.
The entire operation is performed atomically (and is also a write barrier).

\begin{note}
VCC doesn't have all the primitives of all the C compilers predefined.
One can define them by suppling a body.
It is presented only to the VCC compiler (it is enclosed in
\vcc{_(atomic_inline ...)}) so that the normal compiler doesn't get confused
about it.

\vccInput[linerange={xchg-acquire}]{c/08_lockw.c}

\noindent
This is one of the places where one needs to be very careful,
as there is no way for VCC to know if the definition you provided matches
the semantics of your regular C compiler.
Make sure to check with the regular C compiler manual for exact semantics
of its built-in functions.

We plan to include a header file with VCC containing a handful of popular operations,
so you can just rename them to fit your compiler. 
\end{note}

\subsection{Using claims}
\label{sect:using-claims}

The contracts of functions operating on the lock require that the lock
is wrapped.
This is because one can only perform atomic operations on objects
that are closed. 
If an object is open, then the owning thread is in full control of it.
However, wrapped means not only closed, but also owned by the current thread,
which defeats the purpose of the lock --- it should be possible
for multiple threads to compete for the lock.
Let's then say, there is a thread which owns the lock.
Assume some other thread \vcc|t| got to know that the lock is closed.
How would \vcc|t| know that the owning thread won't unwrap (or worse yet, deallocate) the lock, just
before \vcc|t| tries an atomic operation on the lock?
The owning thread thus needs to somehow promise \vcc|t|
that lock will stay closed.
In VCC such a promise takes the form of a \Def{claim}.
Later we'll see that claims are more powerful, but for
now consider the following to be the definition of a claim:

\begin{VCC}
_(ghost 
typedef struct {
  \ptrset claimed;
  _(invariant \forall \object o; o \in claimed ==> o->\closed)
} \claim_struct, *\claim;
)
\end{VCC}

\noindent
Thus, a claim is an object, with an invariant stating that a number of other objects
(we call them \Def{claimed objects}) are closed.
As this is stated in the invariant of the claim, it only needs to be true
as long as the claim itself stays closed.

Recall that what can be written in invariants is subject to the admissibility
condition, which we have seen partially explained in \secref{admissibility0}.
There we said that an invariant should talk only about things the object owns.
But here the claim doesn't own the claimed objects,
so how should the claim know the object will stay closed?
In general, an admissible invariant can depend on other objects invariants always being
preserved (we'll see the precise rule in \secref{inv2}).
So VCC adds an implicit invariant to all types
marked with \vcc{_(claimable)} attribute.
This invariant states that the object cannot be unwrapped when
there are closed claims on it.
More precisely, each claimable object keeps track of the count of outstanding
claims.
The number of outstanding claims on an object is stored in
\vcc{\claim_count} field.

Now, getting back to our lock example, the trick is that there can be
multiple claims claiming the lock (note that this is orthogonal to
the fact that a single claim can claim multiple objects).
The thread that owns the lock will need to keep track of who's using
the lock.
The owner won't be able to destroy the lock (which requires unwrapping it),
before it makes sure there is no one using the lock.
Thus, we need to add \vcc{_(claimable)} attribute to our lock
definition, and change the contract on the functions operating
on the lock. As the changes are very similar we'll only
show \vcc{Release()}.

\vccInput[linerange={release-struct_data}]{c/08_lock_claimsobj.c}

\noindent
We pass a ghost parameter holding a claim.
The claim should be wrapped.
The function \vcc{\claims_obj(c, l)} is defined to be
\vcc{l \in c->claimed}, \ie that the claim claims the lock.
We also need to know that the claim is not the protected object,
otherwise we couldn't ensure that the claim is wrapped after the call.
This is the kind of weird corner case that VCC is very good catching
(even if it's bogus in this context).
Other than the contract, the only change is that we list the claim
as parameter to the atomic block.
Listing a normal object as parameter to the atomic makes VCC know you're
going to modify the object.
For claims, it is just a hint, that it should use this claim when trying
to prove that the object is closed.

Additionally, the \vcc{InitializeLock()} needs to ensure \vcc{l->\claim_count} \vcc{== 0}
(\ie no claims on freshly initialized lock).
VCC even provides a syntax to say something is wrapped and has no claims: \vcc{\wrapped0(l)}.

\subsection{Creating claims}
\label{sect:creating-claims}

When creating (or destroying) a claim one needs to list the claimed objects.
Let's have a look at an example.

\vccInput[linerange={create_claim-out}]{c/08_lock_claimsobj.c}

This function tests that we can actually create a lock, create a claim on it,
use the lock, and then destroy it.
The \vcc{InitializeLock()} leaves the lock wrapped and writable by the current thread.
This allows for the creation of an appropriate claim, which is then passed to \vcc{Acquire()} and \vcc{Release()}.
Finally, we destroy the claim, which allows for unwrapping of the lock, and subsequently deallocating
it when the function activation record is popped off the stack.

The \vcc{\make_claim(...)} function takes the set of objects to be claimed
and a property (an invariant of the claim, we'll get to that in the next section).
Let us give desugaring of \vcc{\make_claim(...)} for a single object
in terms of the \vcc{\claim_struct} defined in the previous section.

\begin{VCC}
// c = \make_claim({o}, \true) expands to
o->\claim_count += 1;
c = malloc(sizeof(\claim_struct));
c->claimed = {o};
_(wrap c);

// \destroy_claim(c, {o}) expands to
assert(o \in c->claimed);
o->\claim_count -= 1;
_(unwrap c);
free(c);
\end{VCC}


Because creating or destroying a claim on \vcc{c} assigns to
\vcc{c->\claim_count}, it requires write access to that memory location.
One way to obtain such access is getting sequential write access to \vcc{c} itself:
in our example the lock is created on the stack and thus sequentially writable.
We can thus create a claim and immediately use it.
A more realistic claim management scenario is described in \secref{dynamic-claims}.
%when a thread creates an object, constructs
%a number of claims on it, and stores the claims in some shared, possibly global, data-structures
%(\eg a work-queue) where other threads can access them.

The \vcc{\true} in \vcc{\make_claim(...)} is the claimed property (an invariant
of the claim), which will be explained in the next section.

\begin{note}
The destruction can possibly leak claim counts, \ie one could say:
\begin{VCC}
\destroy_claim(c, {});
\end{VCC}
\noindent
and it would verify just fine.
This avoids the need to have write access to \vcc{p}, but on the other hand prevents
\vcc{p} from unwrapping forever (which might be actually fine if \vcc{p} is a ghost object).
%It seems clear why the claimed objects need to be listed when creating a claim, but
%why do we need them for destruction?
\end{note}

\subsection{Two-state invariants}
\label{sect:inv2}

Sometimes it is not only important what are the valid states of objects,
but also what are the allowed \emph{changes} to objects.
For example, let's take a counter keeping track of certain operations
since the beginning of the program.

\vccInput[linerange={counter-reading}]{c/09_counter.c}

\noindent
Its first invariant is a plain single-state invariant -- for some reason
we decided to exclude zero as the valid count.
The second invariant says that for any atomic update of (closed)
counter, \vcc{v} can either stay unchanged or increment by exactly one.
The syntax \vcc{\old(v)} is used to refer to value of \vcc{v} before
an atomic update, and plain \vcc{v} is used for the value of \vcc{v}
after the update.
(Note that the argument to \vcc{\old(...)} can be an arbitrary expression.)
That is, when checking that an atomic update preserves the invariant
of a counter, we will take the state of the program right
before the update, the state right after the update, and check
that the invariant holds for that pair of states.

\begin{note}
In fact, it would be easy to prevent any changes to some field \vcc{f}, by
saying \vcc{_(invariant \old(f) == f)}.
This is roughly what happens under the hood when a field is
declared without the \vcc{volatile} modifier.
\end{note}

As we can see the single- and two-state invariants are both defined
using the \vcc{_(invariant ...)} syntax.
The single-state invariants are just two-state invariants, which do not use
\vcc{\old(...)}.
However, we often need an interpretation of an object invariant in a single state \vcc{S}.
For that we use the \Def{stuttering transition} from \vcc{S} to \vcc{S} itself.
VCC enforces that all invariants are \Def{reflexive} that is if they hold
over a transition \vcc{S0, S1}, then they should hold in just \vcc{S1}
(\ie over \vcc{S1, S1}).
In practice,
this means that \vcc{\old(...)} should be only used to describe
how objects change, and not what are their proper values.
In particular,
all invariants which do not use \vcc{\old(...)} are reflexive, and so
are all invariants of the form \vcc{\old(E) == (E) || (P)}, for any expression \vcc{E} and condition \vcc{P}.
On the other hand, the invariants \vcc{\old(f) < 7} and \vcc{x == \old(x) + 1} are not reflexive.

Let's now discuss where can you actually rely on invariants being preserved.

\begin{VCC}
void foo(struct Counter *n)
  _(requires \wrapped(n))
{
  int x, y;
  atomic(n) { x = n->v; }
  atomic(n) { y = n->v; }
}
\end{VCC}

\noindent
The question is what do we know about \vcc{x} and \vcc{y}
at the end of \vcc{foo()}.
If we knew that nobody is updating \vcc{n->v} while \vcc{foo()}
is running we would know \vcc{x == y}.
This would be the case if \vcc{n} was unwrapped, but it is wrapped.
In our case, because \vcc{n} is closed, other threads can update it,
while \vcc{foo()} is running, but they will need to
adhere to \vcc{n}'s invariant.
So we might guess that at end of \vcc{foo()} we know
\vcc{y == x || y == x + 1}.
But this is incorrect: \vcc{n->v} might get incremented
by more than one, in several steps.
The correct answer is thus \vcc{x <= y}.
Unfortunately, in general, such properties are very difficult to deduce
automatically, which is why we use plain object invariants and admissibility
check to express such properties in VCC.

\begin{note}
An invariant is \Def{transitive} if it holds over states \vcc{S0, S2},
provided that it holds over \vcc{S0, S1} and \vcc{S1, S2}.
Transitive invariants could be assumed over arbitrary
pairs of states, provided that the object stays closed
in between them. 
VCC does not require invariants to be transitive, though.

Some invariants are naturally transitive (\eg we could say
\vcc{_(invariant \old(x) <= x)} in \vcc{struct Counter},
and it would be almost as good our current invariant).
Some other invariants, especially the more complicated ones,
are more difficult to make transitive.
For example, an invariant on a reader-writer lock might say
\begin{VCC}
_(invariant writer_waiting ==> old(readers) >= readers)
\end{VCC}
\noindent
To make it transitive one needs to introduce version numbers.
Some invariants describing hardware (\eg a step of physical CPU)
are impossible to make transitive.
\end{note}

Consider the following structure definition:

\vccInput[linerange={reading-endreading}]{c/09_counter.c}

\noindent 
It is meant to represent a reading from a counter.
Let's consider its admissibility.
It has a pointer to the counter, and a owns a claim, which
claims the counter.
So far, so good.
It also states that the current value of the counter is no less than \vcc{r}.
Clearly, the \vcc{Reading} doesn't own the counter, so our previous rule
from \secref{admissibility0}, which states
that you can mention in your invariant everything that you own, doesn't apply.
It would be tempting to extend that rule to say ``everything that you own
or have a claim on'', but VCC actually uses a more general rule.
In a nutshell, the rule says that every invariant should be preserved
under changes to other objects, provided that these other objects change
according to their invariants.
When we look at our \vcc{struct Reading}, its invariant cannot be broken when
its counter increments, which is the only change allowed by counters invariant.
On the other hand, an invariant like \vcc{r == n->v} or \vcc{r >= n->v}
could be broken by such a change.
But let us proceed with somewhat more precise definitions.

First, assume that every object invariant holds when the object is not closed.
This might sound counter-intuitive, but remember that closedness is controlled
by a field.
When that field is set to false, we want to \emph{effectively} disable the invariant,
which is the same as just forcing it to be true in that case.
Alternatively, you might try to think of all objects as being closed for a while.

An atomic action, which updates state \vcc{S0} into \vcc{S1}, is \Def{legal} if and only if the invariants of
objects that have changed between \vcc{S0} and \vcc{S1} hold over \vcc{S0, S1}.
In other words, a legal action preservers invariants of updated objects.
This should not come as a surprise: this is exactly what VCC checks
for in atomic blocks.

An invariant is \Def{stable} if and only if it cannot be broken by legal updates.
More precisely, to prove that an invariant of \vcc{p} is stable,
VCC needs to ``simulate'' an arbitrary legal update:
\begin{itemize}
\item Take two arbitrary states \vcc{S0} and \vcc{S1}.
\item Assume that all invariants (including \vcc{p}'s) hold over \vcc{S0, S0}.
\item Assume that for all objects, some fields of which are not the same in \vcc{S0} and \vcc{S1},
their invariants hold over \vcc{S0, S1}.
\item Assume that all fields of \vcc{p} are the same in \vcc{S0} and \vcc{S1}.
\item Check that invariant of \vcc{p} holds over \vcc{S0, S1}.
\end{itemize}
The first assumption comes from the fact that all invariants are reflexive.
The second assumption is legality.
The third assumption follows from the second (if \vcc{p} did change, its invariant would
automatically hold).

An invariant is \Def{admissible} if and only if it is stable and reflexive.

Let's see how our previous notion of admissibility relates to this one.
If \vcc{p} owns \vcc{q}, then \vcc{q \in p->\owns}.
By the third admissibility assumption, after the simulated action \vcc{p} still owns \vcc{q}.
By the rules of ownership (\secref{wrap-unwrap}), only threads can own
open objects, so we know that \vcc{q} is closed in both \vcc{S0}
and \vcc{S1}.
Therefore non-volatile fields of \vcc{q} do not change between \vcc{S0} and \vcc{S1},
and thus the invariant of \vcc{p} can freely talk about their values:
whatever property of them was true in \vcc{S0}, will also be true in \vcc{S1}.
Additionally, if \vcc{q} owned \vcc{r} before the atomic action, and the \vcc{q->\owns} is non-volatile,
it will keep owning \vcc{r}, and thus non-volatile fields of \vcc{r}
will stay unchanged.
Thus our previous notion of admissibility is a special case of this one.

Getting back to our \vcc{foo()} example, to deduce that \vcc{x <= y}, after
the first read we could create a ghost \vcc{Reading} object, and
use its invariant in the second action.
While we need to say that \vcc{x <= y} is what's required,
using a full-fledged object might seem like an overkill.
Luckily, definitions of claims themselves can specify additional invariants.

\begin{note}
The admissibility condition above is semantic: it will be checked by the theorem
prover. 
This allows construction of the derived concepts like claims and ownership,
and also escaping their limitations if needed.
It is therefore the most central concept of VCC verification methodology,
even if it doesn't look like much at the first sight.
\end{note}

\subsection{Guaranteed properties in claims}
\label{sect:claim-props}

When constructing a claim, you can specify additional invariants to put on
the imaginary definition of the claim structure.
Let's have a look at annotated version of our previous \vcc{foo()} function.

\vccInput[linerange={readtwice-endreadtwice}]{c/09_counter.c}

\noindent
Let's give a high-level description of what's going on.
Just after reading \vcc{n->v} we create a claim \vcc{r}, which guarantees
that in every state, where \vcc{r} is closed,
the current value of \vcc{n->v} is no less than the value of \vcc{x}
at the time when \vcc{r} was created.
Then, after reading \vcc{n->v} for the second time, we tell VCC to
make use of \vcc{r}'s guaranteed property, by asserting that it is ``active''.
This makes VCC know \vcc{x <= n->v} in the current state, where also
\vcc{y == n->v}.
From these two facts VCC can conclude that \vcc{x <= y}.

The general syntax for constructing a claim is:

\begin{VCC}
_(ghost c = \make_claim(S, P))
\end{VCC}

\noindent
We already explained, that this requires that \vcc{s->\claim_count} is writable for \vcc{s \in S}.
As for the property \vcc{P}, we pretend it forms the invariant of the claim.
Because we're just constructing the claim, just like during regular object initialization,
the invariant has to hold initially (\ie at the moment when the claim is created,
that is wrapped).
Moreover, the invariant has to be admissible, under the condition
that all objects in \vcc{S} stay closed as long as the claim itself
stays closed.
The claimed property cannot use \vcc{\old(...)}, and therefore it's automatically
reflexive, thus it only needs to be stable to guarantee admissibility.

But what about locals?
Normally, object invariants are not allowed to reference locals.
The idea is that when the claim is constructed, all the locals that the
claim references are copied into imaginary fields of the claim.
The fields of the claim never change, once it is created.
Therefore an assignment \vcc{x = UINT_MAX;} in between the atomic
blocks would not invalidate the claim --- the claim would still
refer to the old value of \vcc{x}.
Of course, it would invalidate the final \vcc{x <= y} assert.

\begin{note}
For any expression \vcc{E} you can use \vcc{\at(\now(), E)} in \vcc{P}
in order to have the value of \vcc{E} be evaluated in the state
when the claim is created, and stored in the field of the claim.
\end{note}

This copying business doesn't affect initial checking of the \vcc{P},
\vcc{P} should just hold at the point when the claim is created.
It does however affect the admissibility check for \vcc{P}:
\begin{itemize}
\item Consider an arbitrary legal action, from \vcc{S0} to \vcc{S1}.
\item Assume that all invariants hold over \vcc{S0, S0}, including assuming \vcc{P} in \vcc{S0}.
\item Assume that fields of \vcc{c} didn't change between \vcc{S0} and \vcc{S1}
(in particular locals referenced by the claim are the same as at the moment of its creation).
\item Assume all objects in \vcc{S} are closed in both \vcc{S0} and \vcc{S1}.
\item Assume that for all objects, fields of which are not the same in \vcc{S0} and \vcc{S1},
their invariants hold over \vcc{S0, S1}.
\item Check that \vcc{P} holds in \vcc{S1}.
\end{itemize}

To prove \vcc{\active_claim(c)} one needs to prove \vcc{c->\closed} and that
the current state is a \Def{full-stop} state, \ie state where all invariants
are guaranteed to hold.
Any execution state outside of an atomic block is full-stop.
The state right at the beginning of an atomic block is also full-stop.
The states in the middle of it (\ie after some state updates) might not be.

\begin{note}
Such middle-of-the-atomic states are not observable by other threads, and therefore
the fact that the invariants don't hold there does not create soundness problems.
\end{note}

The fact that \vcc{P} follows from \vcc{c}'s invariant after the construction
is expressed using \vcc{\claims(c, P)}.
It is roughly equivalent to saying:
\begin{VCC}
\forall \state s {\at(s, \active_claim(c))};
  \at(s, \active_claim(c)) ==> \at(s, P)
\end{VCC}
Thus, after asserting \vcc{\active_claim(c)} in some state \vcc{s},
\vcc{\at(s, P)} will be assumed, which means VCC will
assume \vcc{P}, where all heap references are replaced by their values in
\vcc{s}, and all locals are replaced by the values at the point
when the claim was created.

\itodo{I think we need more examples about that at() business,
claim admissibility checks and so forth}

\subsection{Dynamic claim management}
\label{sect:dynamic-claims}

So far we have only considered the case of creating claims to wrapped objects.
In real systems some resources are managed dynamically:
threads ask for ``handles'' to resources, operate on them,
and give the handles back.
These handles are usually purely virtual --- asking for a handle amounts to incrementing
some counter.
Only after all handles are given back the resource can be disposed.
This is pretty much how claims work in VCC, and indeed they were modeled after this
real-world scenario. 
Below we have an example of prototypical reference counter.

\vccInput[linerange={refcnt-init}]{c/10_rundown.c}

\noindent
Thus, a \vcc{struct RefCnt} owns a resource, and makes sure that the number of outstanding
claims on the resource matches the physical counter stored in it.
\vcc{\claimable(p)} means that the type of object pointed to by \vcc{p} was marked
with \vcc{_(claimable)}.
The lowest bit is used to disable giving out of new references
(this is expressed in the last invariant).

\vccInput[linerange={init-incr}]{c/10_rundown.c}

\noindent
Initialization shouldn't be very surprising:
\vcc{\wrapped0(o)} means \vcc{\wrapped(o) && o->\claim_count == 0},
and thus on initialization we require a resource without any outstanding
claims.

\vccInput[linerange={incr-decr}]{c/10_rundown.c}

\noindent
First, let's have a look at the function contract.
The syntax \vcc{_(always c, P)} is equivalent to:
\begin{VCC}
  _(requires \wrapped(c) && \claims(c, P))
  _(ensures \wrapped(c))
\end{VCC}
Thus, instead of requiring \vcc{\claims_obj(c, r)}, we require that the claim
guarantees \vcc{r->\closed}.
One way of doing this is claiming \vcc{r}, but another is claiming the owner
of \vcc{r}, as we will see shortly.

As for the body, we assume our reference counter will never overflow.
This clearly depends on the running time of the system and usage patterns,
but in general it would be difficult to specify this, and thus we just
hand-wave it.

The new thing about the body is that we make a claim on the resource,
even though it's not wrapped.
There are two ways of obtaining write access to \vcc{p->\claim_count}:
either having \vcc{p} writable sequentially and wrapped,
or in case \vcc{p->\owner} is a non-thread object, checking
invariant of \vcc{p->\owner}.
Thus, inside an atomic update on \vcc{p->\owner} (which will check the invariant of \vcc{p->\owner}) one can create
claims on \vcc{p}.
The same rule applies to claim destruction:

\vccInput[linerange={decr-use}]{c/10_rundown.c}

\noindent
A little tricky thing here, is that we need to make use of the \vcc{handle} claim
right after reading \vcc{r->cnt}. 
Because this claim is valid, we know that the claim count on the resource
is positive and therefore (by reference counter invariant) \vcc{v >= 2}.
Without using the \vcc{handle} claim to deduce it we would get a complaint
about overflow in \vcc{v - 2} in the second atomic block.

Finally, let's have a look at a possible use scenario of our reference counter.

\vccInput[linerange={use-enduse}]{c/10_rundown.c}

\noindent
The \vcc{struct B} contains a \vcc{struct A} governed by a reference counter.
It owns the reference counter, but not \vcc{struct A} (which is owned by the reference
counter).
A claim guaranteeing that \vcc{struct B} is closed also guarantees
that its counter is closed, so we can pass it to \vcc{try_incr()},
which gives us a handle on \vcc{struct A}.

Of course a question arises where one does get a claim on \vcc{struct B} from?
In real systems the top-level claims come either from global objects that are
always closed, or from data passed when the thread is created.

