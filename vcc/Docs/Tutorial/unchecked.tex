
%% Can the following section move to the appendix? -E
\section{Overflows and unchecked arithmetic}
\label{sect:overflows}

Consider the C expression \vcc{a+b}, when \vcc{a} and \vcc{b} are,
say, \vcc{unsigned int}s. This might represent one of two programmer
intentions. Most of the time, it is intended to mean ordinary
arithmetic addition on numbers; program correctness is then likely to
depend on this addition not causing an overflow. However, sometimes
the program is designed to cope with overflow, so the programmer means
\vcc{(a + b) % UINT_MAX+1}. It is always sound to use this
second interpretation, but VCC nevertheless assumes the first by
default, for several reasons:
\begin{itemize}
\item The first interpretation is much more common.
\item The second interpretation introduces an implicit \vcc{%}
  operator, turning linear arithmetic into nonlinear arithmetic and
  making subsequent reasoning much more difficult.
\item If the first interpretation is intended but the addition can in
  fact overflow, this potential error will only manifest later in the
  code, making the source of the error harder to track down.
\end{itemize}

Here is an example where the second interpretation is intended, but
VCC complains because it assumes the first:
\vccInput[linerange={begin-}]{c/02_hash_fail.c}

\noindent
VCC complains that the hash-computing operation might overflow.
To indicate that this possible overflow behavior is desired we use \vcc{_(unchecked)},
with syntax similar to a regular C type-cast.
This annotation applies to the following expression, and indicates that
you expect that there might be overflows in there.
Thus, replacing the body of the loop with the following
makes the program verify:

\vccInput[linerange={update-endupdate}]{c/02_hash.c}

Note that ``unchecked'' does not mean ``unsafe''.
The C standard mandates the second interpretation for unsigned overflows,
and signed overflows are usually implementation-defined to use two-complement.
It just means that VCC will loose information about the operation.
For example consider:
\begin{VCC}
int a, b;
// ...
a = b + 1;
_(assert a < b)
\end{VCC}
This will either complain about possible overflow of \vcc{b + 1} or succeed.
However, the following might complain about \vcc{a < b}, if VCC does not know
that \vcc{b + 1} doesn't overflow.
\begin{VCC}
int a, b;
// ...
a = _(unchecked)(b + 1);
_(assert a < b)
\end{VCC}
Think of \vcc{_(unchecked)E} as computing the expression using mathematical 
integers, which never overflow, and then casting the result to the desired range.
VCC knows that \vcc{_(unchecked)E == E} if \vcc{E} fits in the proper range,
and some other basic facts about \vcc{(unsigned)-1}.
If you need anything else, you will need to resort to bit-vector
reasoning (\secref{bv}).

%% I don't think the following example is worth the trouble -E
%% Let's have a look at another example:
%% \vccInput[linerange={begin-}]{c/02_rand.c}
%% \noindent
%% The reason we needed to use unchecked cast here, is that the C library
%% \vcc{rand()} function is specified to return a signed integer.

%% \begin{note}
%% In fact, the C standard, mandates the following specification:
%% \begin{VCC}
%% int rand(void)
%%   _(ensures 0 <= \result && \result <= RAND_MAX);
%% \end{VCC}
%% Thus, in principle the \vcc{_(unchecked)} shouldn't be required in the
%% example above. 
%% However,
%% VCC currently does not come with specifications for C standard library functions.
%% We plan to setup a open-source project, where you'll be able to contribute such
%% specifications.
%% It is indeed unclear why does it return a signed integer, only to ensure that the return value
%% is never negative.
%% \end{note}
