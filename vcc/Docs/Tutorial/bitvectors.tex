\section{Bitvector reasoning}
\label{sect:bv}

Remember our first \vcc{min()} example?  Surprisingly it can get more involved.
For example the one below does not use a branch.

\vccInput[linerange={begin-}]{c/01_min5.c}

\noindent
The syntax:
\begin{VCC}
_( assert {bv} \forall int x; (x & -1) == x )
\end{VCC}

\noindent
VCC to prove the assertion using the fixed-length bit vector theory, a.k.a. machine integers.
This is then used as a lemma to prove the postcondition.



