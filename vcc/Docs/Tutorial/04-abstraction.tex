\section{Abstraction}

Usually there are many ways of implementing a given data structure.
For example, a set might be implemented as a linked list, an array, linked list or a hash table.

The programmer using the data structure should not be concerned with implementation details
when reasoning about correctness of his program.
He should reason at somewhat higher, abstract level.
For example, when he uses a linked list as a representation of a set, he should not be concerned
with how the list nodes are laid out in memory.

In VCC one abstract over such details using ghost data: data-structures contain ghost fields with mathematical
abstractions of what they implement.
It is only rarely the case that a simple C type, say \vcc{unsigned int}, would be suitable
to store such abstraction (how would one store an unbounded set in some primitive C type?).
To that end, VCC provides \Def{map types}.
They can be used only in ghost code, and therefore the issue of memory consumption does not apply to them.
The syntax is similar to syntax of array types, \vcc{int m[T*]} defines a map \vcc{m} from \vcc{T*}
to \vcc{int}.
That is the type of expression \vcc{m[p]} is \vcc{int}, provided that \vcc{p} is a pointer
of type \vcc{T*}.
A map \vcc{T* a[unsigned]} is similar to an array of pointers of length $2^{32}$.
A map \vcc{bool s[int]} can be thought of as a set of \vcc{int}s: the operation
\vcc{s[k]} will return true if and only if the element \vcc{k} is in the set \vcc{s}.

Let's then have a look at an example of a list abstracted as a set:

\vccInput[linerange={types-init}]{c/07_list0.c}

\noindent
The expressions inside \vcc|{...}| after the quantified variables are hints for
the theorem prover called triggers, ignore them for a moment.
The invariant states that:
\begin{itemize}
\item the list owns the head node (if it's non-null)
\item if the list owns a node, it also owns the next node (provided it's non-null)
\item if the list owns a node, then its data is present in the set \vcc{val};
      this binds the values stored in the implementation to the abstract representation
\end{itemize}
A careful reader will note that the set \vcc{val} is under-specified: 
it might be that it has some elements not stored in the list.
We'll get back to this issue later.
Now let's have a look at the specification of a function adding a
node to the list:

\vccInput[linerange={add-endspec}]{c/07_list0.c}

\noindent
The writes-clause and contracts about the list being wrapped are similar to what
we've seen before.
Then, there are the contracts talking about the result value.
This function might fail because there is not enough memory to allocate list node,
in such case it will return a non-zero value (an error code perhaps),
and the contracts guarantee that the set represented by the list will not be changed.
However, if the function will succeed (and thus return zero), the contract specifies
that if we take an arbitrary integer \vcc{p}, then it is member of the new abstract
value if and only if it was already a member before or it is \vcc{k}.
In other words, the new value of \vcc{l->val} will be the union of the old
abstract value and the element \vcc{k}.

This contract is hopefully all that the caller will need to know about that function:
what kind of effect does it have on the abstract state. 
It doesn't specify if the node will be appended at the beginning, or in the middle
of the list.
It doesn't talk about possible duplicates or memory allocation.
Everything about implementation is completely abstracted away.
Still, we want a concrete implementation, and here it goes:

\vccInput[linerange={endspec-member}]{c/07_list0.c}

\noindent
We allocate the node, unwrap the list, initialize the new node,
wrap it (we want to wrap the list, and thus everything it is going to own will
need to be wrapped beforehand), and
prepend the node at the beginning of the list.
Then we update the owns set (we've also already seen that).
Finally, we update the abstract value using a \Def{lambda expression}.
%\subsection{Lambda expressions}
The expression \vcc{\lambda T x; E} returns a map, which for
any \vcc{x} returns the value of expression \vcc{E},
which can use \vcc{x}.
If the type of \vcc{E} is \vcc{S}, then the type of map is \vcc{S[T]}.
The expression is evaluated in the state as it was
when the lambda was first defined.
One can imagine, that when this lambda expression is defined,
VCC will iterate over all possible values of \vcc{x},
evaluate \vcc{E} for them, and create a giant mapping
with the results.
Lambda expressions are much like function values in functional
languages or delegates in C\#.

The body of our lambda expression shows similarity to the body
of the quantifier we have used in specification.
It doesn't, however, have to be the same, in particular the quantifier
may talk only about certain indices, while lambda expressions
are always total.

\subsection{Triggers}

SMT solvers, which are the underlaying VCC proving technology,
prove that the program is correct by looking at possible counterexamples,
or \Def{models}, where your program goes wrong (\eg by violating an assertion).
Usually the model is described only partially,
by a set of statements about the state of the program.
For example, the model description may say \vcc{x == 7}, \vcc{y > x}
and \vcc{*p == 12}.
So the solver considers more than one model at a time (\eg the description
above doesn't specify for example the value of variable \vcc{z}), and therefore
we can hope that after a while it will finally go through all \emph{all}
possible models, and say that your program is correct.

It is thus useful to think of the SMT solver as sitting there with
a possible model, and seeing what it can do to out rule it.
For example, if the description of the model says that \vcc{x > 7} and
\vcc{x < 3}, then the solver can apply rules of arithmetic, conclude this is
impossible, and move on to a next model.
The SMT solvers are usually very good about out ruling the models
where the statements describing them do not involve universal quantifiers.
With quantifiers things tend to get a bit tricky.

For example, let's say the model description states that the two
following facts are true:
\begin{VCC}
\forall unsigned i; i < 10 ==> a[i] > 7
a[4] == 3
\end{VCC}
The meaning of the universal quantifier is that it should hold
not matter what we substitute for \vcc{i}, for example
the universal quantifier above implies the following facts (which
are called instances of the quantifier):
\begin{VCC}
4 < 10 ==> a[4] > 7
11 < 10 ==> a[11] > 7
k < 10 ==> a[k] > 7
\end{VCC}
where \vcc{k} is a program variable of type \vcc{unsigned}.
However, there is potentially infinitely many such instances, and certainly too many
to enumerate them all.
Still, to prove that our model candidate is indeed contradictory
we only need the first one, not the other two.
Once the solver adds it to the model description,
it will simplify \vcc{4 < 10} to true,
and then see that \vcc{a[4] > 7} and \vcc{a[4] == 3} cannot hold
at the same time.

The question remains: how does the SMT solver decide that the first
instance is useful, and the other two are not?
This is done through triggers.
A trigger is a set of expressions, which together contain all variables
that the particular formula quantifies over.
For now we restrict our attention to single-element triggers.
In case of our formula the default trigger is \vcc|{a[i]}|.
The SMT solver will look at the model, and find all expressions
of the form \vcc|a[...]|, and will then use them for instantiating
the quantified formula.
That is, it will instantiate it with \vcc{i == 4}, because
if \vcc{i == 4} then \vcc{a[i] == a[4]}.
Triggers can be specified explicitly, or one can let the SMT solver
decide on triggers.
The explicit triggers are listed in \vcc|{...}|, after the quantified variables.

\subsubsection{Default trigger selection}

By default the SMT solver will select all smallest subexpressions, which
contain all quantified variables.
Each of these subexpressions will become a separate trigger.
Z3 will not select triggers containing arithmetic operators (like \vcc{<}
or \vcc{+}).

In future, it will likely be VCC (which has a lot more information available),
not the solver, who will decide on default triggers, but for now the SMT solver
choices might not be what you want.

\subsubsection{Matching loops}

Consider a model description
\begin{VCC}
\forall struct Node *n; {\mine(n)} \mine(n) ==> \mine(n->next)
\mine(a)
\end{VCC}
The SMT solver will instantiate the quantifier with \vcc{a}, yielding:
\begin{VCC}
\mine(a) ==> \mine(a->next)
\end{VCC}
It will now add \vcc{\mine(a->next)} to the set of facts
describing the model.
This however will lead to instantiating the quantifier again,
this time with \vcc{a->next}, and in turn again with
\vcc{a->next->next} and so forth.
Such situation is called a \emph{matching loop}.
The SMT solver would usually cut such loop at a certain depth,
but it might make the solver run out of time, memory, or both.

Note that, if you skip the explicit trigger annotation, by the definition
above, \vcc|{\mine(n)}| and \vcc|n->next| are still going to be default
triggers of the formula above.

To avoid matching loop we need to use a different, more restrictive,
trigger, for example \vcc|{\mine(n->next)}|.
This will cause the quantifier to be instantiated only when the current
model talks about \vcc{n->next} already.
It is thus more ``goal-oriented''.
This is what our example does, however instead of using the shorthand
notation, it uses explicit owns set membership.
\itodo{we need to fix that}

So the upside of the more restrictive explicit trigger is that it avoids
excessive quantifier instantiations.
The downside is that if the SMT solver is missing some instantiations,
it might fail to out rule a model and report a bogus counterexample.
This doesn't happen in for our list, but we'll see such problems
(but with solutions!) later.

\subsection{Improving the list}

At minimum the list should support adding elements and checking
for membership. For example, we would expect:

\vccInput[linerange={member-endspec}]{c/07_list.c}

\noindent
Our list invariant only allows for showing
\vcc{\result != 0 ==> l->val[k]}, because it only says
that if the list owns something, then it's in the \vcc{val}.
It also says that if something can be reached by following
the \vcc{next} field from the \vcc{head}, then it is owned.
What we want to say is that if something is in the \vcc{val}
set, then it can be reached from the \vcc{head}.
Unfortunately, such property is not expressible in first-order
logic (which is the underlaying logic of VCC specifications).
To workaround this problem we associate with each node
a set of values stored in that node, and all the following nodes.
In the invariant this is achieved by using a map from nodes
to maps from integers to booleans (this is conceptually close
to having a two-dimensional map).

\vccInput[linerange={type-init}]{c/07_list.c}

\noindent
\itodo{more explanations}
For now ignore the \vcc|{hint: ...}| thing.
All these changes in the invariant do not affect
the contract of \vcc{add()} function, and the only change in the body
is that we need to replace the update of \vcc{l->val} with the following:

\vccInput[linerange={specupdate-updateend}]{c/07_list.c}

\noindent
That is adding a node at the head only affect the followers set of the new head.
Now let us have a look at the \vcc{member()} function:

\vccInput[linerange={member-out}]{c/07_list.c}

\subsection{Hinting triggers}

